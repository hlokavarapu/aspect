 % Elbridge Gerry Puckett
% Department of Mathematics
% University of California, Davis
% Davis, California  95616

% Revision History
%   Saturday,  February 25, 2012 at 13:02 PST
%   Wednesday, October  31, 2012 at 18:11 PDT

% EGP's Macros for writing notes in general, especially notes associated with the book EGPPC.

% One of the LaTeX manuals cautions against using fiverm. However, perhaps CMR scaled by 50% is
% acceptable.  I use this font in order to make certain frequently used fractions (e.g., one-half)
% small so that they be placed in the text without disturbing the line spacing or so that they are
% better proportioned when placed in a subscript or a superscript.

\font\myfiverm=cmr5 scaled 500       % used for small sub/superscripts

\newcommand{\half}{{\myfiverm \frac{1}{2}}}

% Script font for denoting the Null Space of a matrix A as N(A)

\font\scriptten=cmsy10                           % scaled \magstep0	

% Script N for denoting the Null Space of a matrix A as N(A)

\newcommand{\Null}{\scriptten{\hbox{\char '116}}}    % Script N
\newcommand{\NullLap}{{\lap{\hbox{\char '116}}}}	 % Script N

% Boldface Roman N, which is the way Strang denotes the Null space of A; i.e., N(A). However Strang also
% uses N to denote the `Null' matrix of A.

\newcommand{\NullStrang}{{\bf N}}

% Defines \abs{arg1} to be the magnitude of arg1

%\newcommand{\abs}[1]{\, \vert \, #1 \, \vert \,}

% This macro is used to denote a Banach space

\newcommand{\Banach}{\mathcal{B}}

% The Complex numbers (Naoki's \C is defined the same way.)

\newcommand{\C}{\mathbb{C}}

% To denote the (i,j)th cell in the grid \Omega^h

\newcommand{\Cell}{\mathcal{C}}

% We use \Csp to denote the space C^p([a,b]) of p times continuously differentiable functions on
% the interval [a,b].

\newcommand{\Csp}{{\mathit{C}}}

% From Naoki's "AMS_Math_Symbol_Macros.tex" to provide a macro for d^n f / d n.

\newcommand{\difh}[3]{{\displaystyle\frac{\mathrm{d}^{#1}{#2}}{\mathrm{d}\,{#3}^{#1}}}}

% Use this symbol to indicate that the quantity on the left-hand side of the equal sign is defined by the
% quantity on the right-hand side of the equal sign. CAUTION! The command "\text" is only available in
% AMS-TeX, so this will not work unless you are using the "amsart" class file or "\usepackage{amsmath}"
% or something similar. One puzzling piece of behavior is that sometimes I get the error "\define already
% defined ...". There was a \define command in AMS-TeX but it has been superseded by "\newcommand". What is
% puzzling is that - as near as I can tell - this error does not always occur.

% On Thursday, February 07 at 17:24 PST EGP changed \define to \Define.  However, I am keeping \define until
% I can eliminate all occurrences of it from my LaTeX documents.

\newcommand{\Define}{\; \stackrel{\text{\tiny def}}{=} \;}
\newcommand{\define}{\; \stackrel{\text{\tiny def}}{=} \;}

% Distance between two gridpoints in the x and y directions

\newcommand{\dx}{{\Delta x}}
\newcommand{\dy}{{\Delta y}}

% \newcommand{\dx}{{h}}
% \newcommand{\dy}{{h}}

% These macros make typesetting \dx / 2 and \dy / 2 easier.

\newcommand{\dxhalf}{\frac{\dx}{2}}
\newcommand{\dyhalf}{\frac{\dy}{2}}

% These macros are used to create a little more space before the dx, dy and dt in an integral.

\newcommand{\Ds}{\, ds}
\newcommand{\DS}{\, dS}
\newcommand{\Dt}{\, dt}
\newcommand{\Dx}{\, dx}
\newcommand{\Dy}{\, dy}

% This macro is used to create a little more space before the d \xvec in an integral in two or more
% space dimensions

\newcommand{\Dxvec}{\; d \xvec}

% The time step

\newcommand{\dt}{{\Delta t}}
\newcommand{\dthalf}{\frac{\dt}{2}}
\newcommand{\dtprime}{{{\Delta t}^\prime}}

% One of the packages I am using defines \sp as an alternate way to create a superscript.  However, I never
% use \sp for this purpose, so I have defined it be the smallest unit of space one can place between two
% characters.

\renewcommand{\sp}{{\mspace{1mu}}}

\newcommand{\eq}{\; = \;}
\newcommand{\Eq}{\; = \;}
\newcommand{\EQ}{\; = \;}

\newcommand{\gt}{\;  >  \;}
\newcommand{\GT}{\;  >  \;}
\newcommand{\GE}{\; \ge \;}

\newcommand{\lt} {\;  <   \;}
\newcommand{\LT} {\;  <   \;}
\newcommand{\LE} {\; \le  \;}
\newcommand{\LEQ}{\; \leq \;}

\newcommand{\NE} {\; \neq \;}
\newcommand{\NEQ}{\; \neq \;}

% If and only if ...

\renewcommand{\iff}{\Longleftrightarrow}

% Plus and minus signs with more space on either side.

\newcommand{\minus}{\; - \;}
\newcommand{\plus }{\; + \;}

\newcommand{\intersect}{\cup}
\newcommand{\union    }{\cap}

% We use \iu = \iota for the imaginary unit i rather than a lowercase i so that we can use the
% lowercase i as as subscript for indexing cells and edges in space.

\newcommand{\iu}{\iota}

% These macros are used to create more space between the argument and the vertical lines.

\newcommand{\abs} {\, \vert \,}
\newcommand{\labs}{   \vert \,}
\newcommand{\rabs}{\, \vert   }

% This symbol may also be used to denote the definition of some mathematical quantity.

\newcommand{\defined}{\, \equiv \,}

\newcommand{\norm}{{\| \cdot \|}}    % Use this for a generic norm
\newcommand{\lnorm}{{\| \,}}
\newcommand{\rnorm}{{\, \|}}

% A little more space in the inner product <f,g>

\newcommand{\lang}{\langle \,}
\newcommand{\rang}{\, \rangle}

% Various symbols for the real numbers

\newcommand{\R}{\boldsymbol{R}}         % Apparently Strang uses {\boldsymbol{R}} for the real mumbers

\newcommand{\Reals}{\mathbb{R}}         % The real mumbers (\mathbb stands for ``Math Black Board'')

\newcommand{\Real }{{I \! \! R}}	    % EGP's homemade version of the symbol for the real numbers

% The Real Numbers


% EGP's `homemade' notation for the real numbers

% \newcommand{\Reals}{{I \! \! R}}

% I used \Rtilde to denote the remainder in a Taylor Series approximation to the function \scalar.

\newcommand{\Rtilde}{\tilde{R}}

% The Integers

\newcommand{\Z}{\mathbb{Z}}

% Defines \ip{arg1}{arg2} to mean inner product

\newcommand{\innerp}[2]{\langle \, #1 , \, #2 \, \rangle}

% Norms ...

\newcommand{\supnorm}[1]{\| \, #1 \, \|_\infty}

% Fourier coefficients of the functions f and g
% (Or the Fourier Transform of f and g.)

\newcommand{\fhat}{{\hat f}}
\newcommand{\ghat}{{\hat g}}

% For the modified Fourier transform of the function f

%\newcommand{\ftilde}{{\tilde f}}

% From http://www.physicsforums.com/showthread.php?t=185826 on Wednesday, February 06, 2013 at 20:20 PST
%
%   Timo suggested \mathbf, but that doesn't work on some things.
%   If you have access to the AMSLaTeX macros, (and you should), use \boldsymbol.

%   For fluxes on the edges of cells

\newcommand{\Fvec}{\boldsymbol{F}}

% The spectrally accurate approximation to f

\newcommand{\fS}{f_S}

% Used to denote the matrix corresponding to the standard five point Laplacian in two dimensions multiplied
% by h^2.

\newcommand{\Atilde}{{\tilde A}}

% In EGP's second and third CAMCoS papers I use this symbol to denote the 5 X 5 block of cells centered on
% the ijth cell C_{ij}.

\newcommand{\Btilde}{\widetilde B}

% Used to denote constants

\newcommand{\Cbar}{\widebar C}
\newcommand{\Ctilde}{\widetilde C}

% In order to denote the discrete Fourier coefficients of f

\newcommand{\ftilde}{{\tilde f}}

% In order to denote the M interpolation points on the grid or vectors in C^M
% (In Part One EGP is not using bold faced symbols to denote vectors.)

\newcommand{\fvec}{{f}}
\newcommand{\gvec}{{g}}

% The complex conjugates of f and g

\newcommand{\fbar}{{\bar{f}}}
\newcommand{\gbar}{{\bar{g}}}

\newcommand{\grad}{\nabla}

% \hhalf = h/2 is half a cell width, or, for cell-centered grids, it denotes the
% center of the first cell on the periodic domain [0,1].

\newcommand{\hhalf}{{\myfiverm \frac{h}{2}}}

% For indices running over the Fourier coefficients and the discrete Fourier coefficients

\newcommand{\jpr}{{j^\prime}}
\newcommand{\kpr}{{k^\prime}}

% Upper limit on the truncated Fourier series for f

\newcommand{\Mhalf}{{\myfiverm \frac{M}{2}}}
\newcommand{\Nhalf}{{\myfiverm \frac{N}{2}}}

\newcommand{\thalf}{{\myfiverm \frac{3}{2}}}
\newcommand{\Thalf}{{\myfiverm \frac{3 \sp M}{2}}}

% The discrete Fourier coefficients of the algebraic error \delta^l = \utilde - u

\newcommand{\dhat}{{\hat{\delta}}}

% Vector of the (j_x, j_y) component of wvec^(\kvec)

\newcommand{\jvec}{{\boldsymbol j}}

% Vector of Fourier wave numbers \kvec = (k_x, k_y)

\newcommand{\kvec}{{\boldsymbol k}}

% The discrete Fourier coefficient of \phi(x) defined using the values of \phi on
% the grid with N points, [ \phi(x_0), \phi(x_1), \ldots , \phi(x_{N-1}) ]

\newcommand{\phihat}{{\hat{\phi}}}

% The macro \scalar is used to denote a scalar quantity that is being advected in the velocity
% field \uvec(\xvec,t)

\newcommand{\scalar}{\mathfrak{s}}

% The macro \stilde is used to distinguish the second-order accurate advection algorithm from the
% the first-order accurate advection algorithm; i.e., CTU.

\newcommand{\stilde}{{\tilde{\mathfrak{s}}}}

% We use this macro to write "the $(i,j)^{th}$ cell in a more succinct form: "the $(\ij)^\th$ cell"

\renewcommand{\th}{{th}}

% The complex conjugate of w

\newcommand{\wbar}{{\bar{w}}}

% Computed solutions to on a grid with % \dx = h at time t^n and t^{n+1}.

\newcommand{\uc}{u}
\newcommand{\ucn}{\uc^{n}}
\newcommand{\ucnp}{\uc^{n+1}}

% Exact solution to the model linear advection equation (1.1a,b).

\newcommand{\ue}{v}
\newcommand{\uen}{v^n}
\newcommand{\uenp}{v^{n+1}}

% EGP used \utilde to denote the compute approximation to the solution u of a conservation law in
% the note "Second-Order Accurate Conservative Finite Difference Update"

\newcommand{\utilde}{{\tilde{u}}}
\newcommand{\utilden}{{\tilde{u}^n}}
\newcommand{\utildenp}{{\tilde{u}^{n+1}}}

% Boldfaced u, which is typically used to denote a velocity in two or more space dimensions

\newcommand{\uvec}{\boldsymbol{u}}

% A generic vector space V

\newcommand{\Vspace}{\mathscr{V}}
% \newcommand{\Vspace}{\mathpzc{V}}
% \newcommand{\Vspace}{\mathcal{V}}

\newcommand{\Avec}{{\boldsymbol A}}

% For von Neumann analysis and the Discrete Fourier Transform basis vectors

\newcommand{\Wvec}{{W}}
\newcommand{\wvec}{{\boldsymbol w}}
%\newcommand{\wvec}{{w}}

% The vector (0,0) or (0,0,0)

\newcommand{\zero}{\boldsymbol{0}}

% The vector b

\newcommand{\bvec}{\boldsymbol{b}}

% Useful abbreviations for subscripts and superscripts

\renewcommand{\ij}{{i,j}}

% Indices for the components of Fourier functions such as \wvec^(\kvec) in two
% dimensions

\newcommand{\jxjy}{{j_x,j_y}}

\newcommand{\im}{{i-1}}
\newcommand{\ip}{{i+1}}
\newcommand{\iph}{{i + \half}}
\newcommand{\imh}{{i - \half}}

\newcommand{\jm}{{j - 1}}
\newcommand{\jp}{{j + 1}}
\newcommand{\jph}{{j + \half}}
\newcommand{\jmh}{{j - \half}}

\newcommand{\imj }{{i-1,j  }}
\newcommand{\ipj }{{i+1,j  }}
\newcommand{\ijm }{{i  ,j-1}}
\newcommand{\ijp }{{i  ,j+1}}
\newcommand{\imjm}{{i-1,j-1}}

\newcommand{\ijmh}{{i  ,j-\half}}
\newcommand{\ijph}{{i,j+\half}}
\newcommand{\imhj}{{i-\half,j}}
\newcommand{\iphj}{{i+\half,j}}

\newcommand{\np}{{n + 1}}
\newcommand{\nph}{{n + \half}}
\newcommand{\nmh}{{n - \half}}

\newcommand{\atilde}{\tilde a}
\newcommand{\btilde}{\tilde b}

\newcommand{\cbar}{\bar c}
\newcommand{\ctilde}{\tilde c} % comparison circle

\newcommand{\dr}{\; dr}
\newcommand{\ds}{\; ds}

\newcommand{\eps}{{\epsilon}}

\newcommand{\gtilde}{\tilde g}
\newcommand{\Gtilde}{\tilde G}

%\newcommand{\implies}{\Rightarrow}
\newcommand{\kappatilde}{{{\tilde \kappa}}}

\newcommand{\Lambdatilde}{{\tilde {\Lambda}}}
\newcommand{\mtilde}{{\tilde {m}}}

\newcommand{\ntilde}{\tilde \boldsymbol{n}}
\newcommand{\nvec}{\boldsymbol{n}}
\newcommand{\Nvec}{\boldsymbol{N}}

\newcommand{\ptilde}{{\tilde p}}

% Finite Element Types P_{-s}

\newcommand{\Prm}{\mathrm{P}}

% Finite Element Types Q_{-s}

\newcommand{\Q}{\mathrm{Q}}

% \newcommand{\stilde}{\tilde s}
\newcommand{\sstar}{s^{*}}

\newcommand{\thetatilde}{{\tilde \theta}}

\newcommand{\taudot}{{\dot \boldsymbol{T}}}
\newcommand{\tauvec}{{\boldsymbol{T}}}
\newcommand{\tauvecdot}{{\dot\boldsymbol{T}}}

\newcommand{\Tdot}{{\dot \boldsymbol{T}}}
\newcommand{\Tvec}{{\boldsymbol{T}}}
\newcommand{\Tvecdot}{{\dot\boldsymbol{T}}}

\newcommand{\vvec}{\mathbf{v}}
\newcommand{\V}{\mathbf{V}}

\newcommand{\xbar}{\bar x}

\newcommand{\xdot}{\dot x}
\newcommand{\ydot}{\dot y}

\newcommand{\xddot}{\ddot x}
\newcommand{\yddot}{\ddot y}

\newcommand{\xhat}{{\hat x}}
\newcommand{\xitilde}{{\tilde \xi}}

\newcommand{\xstar}{{x}^{*}}
\newcommand{\ystar}{{y}^{*}}

\newcommand{\xtilde}{{\tilde x}}
\newcommand{\ytilde}{{\tilde y}}

\newcommand{\xdtilde}{{\dot{\tilde x}}}
\newcommand{\ydtilde}{{\dot{\tilde y}}}

\newcommand{\xddtilde}{{\ddot{\tilde x}}}
\newcommand{\yddtilde}{{\ddot{\tilde y}}}


\newcommand{\xvec}{\boldsymbol{x}}
\newcommand{\xvecdot}{{\dot \boldsymbol{x}}}
\newcommand{\xvecddot}{{\ddot \boldsymbol{x}}}

\newcommand{\Xvec}{\boldsymbol{X}}

\newcommand{\zetatilde}{\boldsymbol{{\tilde \zeta}}}

\newcommand{\zvec}{\boldsymbol{z}}
\newcommand{\ztilde}{\boldsymbol{{\tilde z}}} 