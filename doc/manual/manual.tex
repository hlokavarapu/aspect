\documentclass{article}
\usepackage[pdftex]{graphicx,color}
\usepackage[T1]{fontenc}
\usepackage{lmodern}
\usepackage{amsmath}
\usepackage{amsfonts}
\usepackage{subfigure}
\usepackage{textpos}

% use a larger page size; otherwise, it is difficult to have complete
% code listings and output on a single page
\usepackage{fullpage}

% have an index. we use the imakeidx' replacement of the 'multind' package so
% that we can have an index of all run-time parameters separate from other
% items (if we ever wanted one)
\usepackage{imakeidx}
\makeindex[name=prmindex, title=Index of run-time parameter entries]
\makeindex[name=prmindexfull, title=Index of run-time parameters with section names]

% be able to use \note environments with a box around the text
\usepackage{fancybox}
\newcommand{\note}[1]{
{\parindent0pt
  \begin{center}
    \shadowbox{
      \begin{minipage}[c]{0.9\linewidth}
        \textbf{Note:} #1
      \end{minipage}
    }
  \end{center}
}}

% use the listings package for code snippets. define keywords for prm files
% and for gnuplot
\usepackage{listings}
\lstset{
  language=C++,
  basicstyle=\small\ttfamily,
  columns=fullflexible,
  keepspaces=true,
  frame=single,
  breaklines=true,
  postbreak=\raisebox{0ex}[0ex][0ex]{\hspace{5em}\ensuremath{\color{red}\hookrightarrow\space}}
}
\lstdefinelanguage{prmfile}{morekeywords={set,subsection,end},
                            morecomment=[l]{\#},escapeinside={\%\%}{\%},}
\lstdefinelanguage{gnuplot}{morekeywords={plot,using,title,with,set,replot},
                            morecomment=[l]{\#},}


% use the hyperref package; set the base for relative links to
% the top-level aspect directory so that we can link to
% files in the aspect tree without having to specify the
% location relative to the directory where the pdf actually
% resides
\usepackage[colorlinks,linkcolor=blue,urlcolor=blue,citecolor=blue,baseurl=../]{hyperref}

\newcommand{\dealii}{{\textsc{deal.II}}}
\newcommand{\pfrst}{{\normalfont\textsc{p4est}}}
\newcommand{\trilinos}{{\textsc{Trilinos}}}
\newcommand{\aspect}{\textsc{ASPECT}}

\begin{document}

%%%%%%%%%%%%%%%%%%%%%%%%%%%%%%
%%% START OF CIG MANUAL COVER TEMPLATE %%%
%%%%%%%%%%%%%%%%%%%%%%%%%%%%%%
% This should be pasted at the start of manuals and appropriate strings entered at locations indicated with FILL.
% Be sure the TeX file includes the following packages.
% \usepackage{graphicx}
% \usepackage{times}
% \usepackage{textpos}

\definecolor{dark_grey}{gray}{0.3}
\definecolor{aspect_blue}{rgb}{0.3125,0.6875,0.9375}

%LINE 1%
{
\renewcommand{\familydefault}{\sfdefault}

\pagenumbering{gobble}
\begin{center}
\resizebox{\textwidth}{!}{\textcolor{dark_grey}{\fontfamily{\sfdefault}\selectfont
COMPUTATIONAL INFRASTRUCTURE FOR GEODYNAMICS (CIG)
}}

\hrule

%LINE 2%
\color{dark_grey}
\rule{\textwidth}{2pt}

%LINE 3%
\color{dark_grey}
% FILL: additional organizations
% e.g.: {\Large Organization 1\\Organization 2}
{\Large }
\end{center}

%COLOR AND CODENAME BLOCK%
\begin{center}
\resizebox{\textwidth}{!}{\colorbox
% FILL: color of code name text box
% e.g. blue
{aspect_blue}{\fontfamily{\rmdefault}\selectfont \textcolor{white} {
% FILL: name of the code
% You may want to add \hspace to both sides of the codename to better center it, such as:
% \newcommand{\codename}{\hspace{0.1in}CodeName\hspace{0.1in}}
\hspace{0.1in}\aspect{}\hspace{0.1in}
}}}
\\[12pt]
{\Large Advanced Solver for Problems in Earth's ConvecTion}
\end{center}

%MAIN PICTURE%
\begin{textblock*}{0in}(1.5in,0.3in)
% FILL: image height
% e.g. height=6.5in
\begin{center}
\vspace{1em}
\includegraphics[height=3.5in]
% FILL: image file name
% e.g. cover_image.png
{mesh-2d.png}
\hspace{5em}
\end{center}
\end{textblock*}

%USER MANUAL%
\color{dark_grey}
\hfill{\Huge \fontfamily{\sfdefault}\selectfont User Manual \\
\raggedleft \huge \fontfamily{\sfdefault}\selectfont Version
% keep the following line as is so that we can replace this using a script:
1.5.0-pre %VERSION-INFO%
\\\large(generated \today)\\
{\Large Wolfgang Bangerth\\Juliane Dannberg\\Rene Gassm{\"o}ller\\Timo Heister\\}
}

%AUTHOR(S) & WEBSITE%
\null
\vspace{18em}
\color{dark_grey}
{\fontfamily{\sfdefault}\selectfont
% FILL: author list
% e.g. Author One\\Author Two\\Author Three\\
% be sure to have a newline (\\) after the final author
\large
\noindent with contributions by: \\
    Jacqueline Austermann,
    Markus B{\"u}rg,
    Samuel Cox,
    William Durkin,
    Grant Euen,
    Menno Fraters,
    Thomas Geenen,
    Anne Glerum,
    Ryan Grove,
    Eric Heien,
    Scott King,
    Martin Kronbichler,
    Shangxin Liu,
    Elvira Mulyukova,
    Jonathan Perry-Houts,
    Tahiry Rajaonarison,
    Ian Rose,
    D.~Sarah Stamps,
    Cedric Thieulot,
    Iris van Zelst,
    Siqi Zhang \\
\vspace{1em}
}

{\fontfamily{\sfdefault}\selectfont \href{https://geodynamics.org}{geodynamics.org}}


%LINE%
\color{dark_grey}
\rule{\textwidth}{2pt}

}

\pagebreak
\pagenumbering{arabic}

%%%%%%%%%%%%%%%%%%%%%%%%%%%%%%
%%%   END OF CIG MANUAL COVER TEMPLATE    %%%
%%%%%%%%%%%%%%%%%%%%%%%%%%%%%%

\pagebreak

\tableofcontents

\pagebreak

\section{Introduction}

\aspect{} --- short for Advanced Solver for Problems in Earth's ConvecTion ---
is a code intended to solve the equations that describe thermally driven
convection with a focus on doing so in the context of convection in the earth
mantle. It is primarily developed by computational scientists at Texas A\&M
University based on the following principles:
\begin{itemize}
\item \textit{Usability and extensibility:} Simulating mantle convection is a
  difficult problem characterized not only by complicated and nonlinear
  material models but, more generally, by a lack of understanding which parts
  of a much more complicated model are really necessary to simulate the
  defining features of the problem. To name just a few examples:
  \begin{itemize}
  \item Mantle convection is often solved in a spherical shell geometry, but
    the earth is not a sphere -- its true shape on the longest length scales is
    dominated by polar oblateness, but deviations from spherical shape
    relevant to convection patterns may go down to the length scales of
    mountain belts, mid-ocean ridges or subduction trenches. Furthermore,
    processes outside the mantle like crustal depression during glaciations
    can change the geometry as well.
  \item Rocks in the mantle flow on long time scales, but on shorter time
    scales they behave more like a visco-elasto-plastic material as they break
    and as their crystalline structure heals again. The mathematical models
    discussed in Section~\ref{sec:models} can therefore only be
    approximations.
    \item If pressures are low and temperatures high enough, rocks melt,
      leading to all sorts of new and interesting behavior.
  \end{itemize}
  This uncertainty in what problem one actually wants to solve requires a code
  that is easy to extend by users to support the community in determining what
  the essential features of convection in the earth mantle are. Achieving this
  goal also opens up possibilities outside the original scope, such as the
  simulation of convection in exoplanets or the icy satellites of the gas
  giant planets in our solar system.

\item \textit{Modern numerical methods:} We build \aspect{} on numerical
  methods that are at the forefront of research in all areas -- adaptive mesh
  refinement, linear and nonlinear solvers, stabilization of
  transport-dominated processes. This implies complexity in our algorithms,
  but also guarantees highly accurate solutions while remaining efficient in
  the number of unknowns and with CPU and memory resources.

\item \textit{Parallelism:} Many convection processes of interest are
  characterized by small features in large domains -- for example, mantle
  plumes of a few tens of kilometers diameter in a mantle almost 3,000 km
  deep. Such problems can not be solved on a single computer but require
  dozens or hundreds of processors to work together. \aspect{} is designed
  from the start to support this level of parallelism.

\item \textit{Building on others' work:} Building a code that satisfies above
  criteria from scratch would likely require several 100,000 lines of
  code. This is outside what any one group can achieve on academic time
  scales. Fortunately, most of the functionality we need is already available
  in the form of widely used, actively maintained, and well tested and
  documented libraries, and we leverage these to make \aspect{} a much smaller
  and easier to understand system. Specifically, \aspect{} builds immediately
  on top of the \dealii{} library (see \url{https://www.dealii.org/}) for
  everything that has to do with finite elements, geometries, meshes, etc.;
  and, through \dealii{} on Trilinos (see \url{http://trilinos.org/})
  for parallel linear algebra and on \pfrst{} (see
  \url{http://www.p4est.org/}) for parallel mesh handling.

\item \textit{Community:} We believe that a large project like \aspect{} can
  only be successful as a community project. Every contribution is welcome and
  we want to help you so we can improve \aspect{} together.

\end{itemize}

Combining all of these aspects into one code makes for an interesting
challenge. We hope to have achieved our goal of providing a useful tool to the
geodynamics community and beyond!


\note{\aspect{} is a community project. As such, we encourage contributions
  from the community to improve this code over time. Natural candidates for
  such contributions are implementations of new plugins as discussed in
  Section~\ref{sec:plugins-concrete} since they are typically self-contained and do not
  require much knowledge of the details of the remaining code. Obviously,
  however, we also encourage contributions to the core functionality in any
  form! If you have something that might be of general interest, please
  contact us.}

\note{\aspect{} will only solve problems relevant to the community if we get
  feedback from the community on things that are missing or necessary for what
  you want to do. Let us know by personal email to the developers, or the
  mantle convection or \texttt{aspect-devel} mailing lists hosted at
  \url{http://lists.geodynamics.org/cgi-bin/mailman/listinfo/aspect-devel}!}

\subsection{Referencing \aspect{}}

As with all scientific work, funding agencies have a reasonable expectation
that if we ask for continued funding for this work, we need to demonstrate
relevance. 
In addition, many have contributed to the development of \aspect{} and deserve credit
for their work.
To this end, we ask that if you publish results that were obtained
to some part using \aspect{}, cite the following, canonical reference for
this software:
\begin{lstlisting}[frame=single,language=tex]
@Article{KHB12,
  author =       {M. Kronbichler and T. Heister and W. Bangerth},
  title =        {High Accuracy Mantle Convection Simulation through Modern Numerical Methods},
  journal =      {Geophysics Journal International},
  year =         2012,
  volume =       191,
  pages =        {12--29}
}
\end{lstlisting}
Please also cite the User's Manual:
\begin{lstlisting}[frame=single,language=tex]
@Manual{aspectmanual,
  title =        {\textsc{ASPECT}: {Advanced Solver for Problems in Earth's
                  ConvecTion} v1.4.0}, 
  author =       {W. Bangerth and T. Heister and others},
  organization = {Computational Infrastructure for Geodynamics},
  year =         2016
}
\end{lstlisting}

%@Manual{aspectmanual,
%  title =        {\textsc{ASPECT}: Advanced Solver for Problems in Earth's
%                  ConvecTion v1.5.0}, 
%  author =       {W. Bangerth and J. Dannberg and 
%                  R. Gassm{\"o}ller and T. Heister and others},
%  organization = {Computational Infrastructure for Geodynamics},
%  year =         2016
%}



Updated citation information can also be found on the \aspect{}
website at \url{https://aspect.dealii.org/}.



\subsection{Acknowledgments}

The development of \aspect{} has been funded
through a variety of grants to the authors. Most immediately, it has been
supported through the Computational Infrastructure in Geodynamics (CIG-II)
grant (National Science Foundation Award No. EAR-0949446, via The University
of California -- Davis) but the initial portions have also been supported
by the original CIG grant (National Science Foundation Award No. EAR-0426271,
via The California Institute of Technology). In addition, the libraries upon
which \aspect{} builds heavily have been supported through many other grants
that are equally gratefully acknowledged.

Please acknowledge CIG as follows:
{\parindent0pt
  \begin{center}
    \shadowbox{
      \begin{minipage}[c]{0.9\linewidth}
ASPECT is hosted by the Computational Infrastructure for Geodynamics (CIG)
which is supported by the National Science Foundation award NSF-094946.
      \end{minipage}
    }
  \end{center}
}

\section{Equations, models, coefficients}
\label{sec:models}

\subsection{Basic equations}
\label{sec:equations}

\aspect{} solves a system of equations in a $d=2$- or $d=3$-dimensional
domain $\Omega$ that describes the motion of a highly viscous fluid driven
by differences in the gravitational force due to a density that depends on
the temperature. In the following, we largely follow the exposition of this
material in Schubert, Turcotte and Olson \cite{STO01}.

Specifically, we consider the following set of equations for velocity $\mathbf
u$, pressure $p$ and temperature $T$, as well as a set of advected quantities
$c_i$ that we call \textit{compositional fields}:
\begin{align}
  \label{eq:stokes-1}
  -\nabla \cdot \left[2\eta \left(\varepsilon(\mathbf u)
                                  - \frac{1}{3}(\nabla \cdot \mathbf u)\mathbf 1\right)
                \right] + \nabla p &=
  \rho \mathbf g
  &
  & \textrm{in $\Omega$},
  \\
  \label{eq:stokes-2}
  \nabla \cdot (\rho \mathbf u) &= 0
  &
  & \textrm{in $\Omega$},
  \\
  \label{eq:temperature}
  \rho C_p \left(\frac{\partial T}{\partial t} + \mathbf u\cdot\nabla T\right)
  - \nabla\cdot k\nabla T
  &=
  \rho H
  \notag
  \\
  &\quad
  +
  2\eta
  \left(\varepsilon(\mathbf u) - \frac{1}{3}(\nabla \cdot \mathbf u)\mathbf 1\right)
  :
  \left(\varepsilon(\mathbf u) - \frac{1}{3}(\nabla \cdot \mathbf u)\mathbf 1\right)
  \\
  &\quad
  +\alpha T \left( \mathbf u \cdot \nabla p \right)
  \notag
  \\
  &\quad
  + \rho T \Delta S \left(\frac{\partial X}{\partial t} + \mathbf u\cdot\nabla X\right)
  &
  & \textrm{in $\Omega$},
  \notag
  \\
  \label{eq:compositional}
  \frac{\partial c_i}{\partial t} + \mathbf u\cdot\nabla c_i
  &=
  q_i
  &
  & \textrm{in $\Omega$},
  i=1\ldots C
\end{align}
where $\varepsilon(\mathbf u) = \frac{1}{2}(\nabla \mathbf u + \nabla\mathbf
u^T)$ is the symmetric gradient of the velocity (often called the
\textit{strain rate}).%
\footnote{There is no consensus in the sciences on the notation used
  for strain and strain rate. The symbols $\varepsilon$,
  $\dot\varepsilon$,  $\varepsilon(\mathbf u)$, and
  $\dot\varepsilon(\mathbf u)$, can all be found. In this manual, and
  in the code, we will consistently use $\varepsilon$ as an
  \textit{operator}, i.e., the symbol is not used on its own but only
  as applied to a field. In other words, if $\mathbf u$ is the
  velocity field, then $\varepsilon(\mathbf u) = \frac{1}{2}(\nabla
  \mathbf u + \nabla\mathbf u^T)$ will denote the strain rate. On the
  other hand, if $\mathbf d$ is the
  displacement field, then $\varepsilon(\mathbf d) = \frac{1}{2}(\nabla
  \mathbf d + \nabla\mathbf d^T)$ will denote the strain.}


In this set of equations, \eqref{eq:stokes-1} and \eqref{eq:stokes-2}
represent the compressible Stokes equations in which $\mathbf u=\mathbf
u(\mathbf x,t)$ is the velocity field and $p=p(\mathbf x,t)$ the pressure
field. Both fields depend on space $\mathbf x$ and time $t$. Fluid flow is
driven by the gravity force that acts on the fluid and that is proportional to
both the density of the fluid and the strength of the gravitational pull.

Coupled to this Stokes system is equation \eqref{eq:temperature} for the
temperature field $T=T(\mathbf x,t)$ that contains heat conduction terms as
well as advection with the flow velocity $\mathbf u$. The right hand side
terms of this equation correspond to
\begin{itemize}
\item internal heat production for example due to radioactive
  decay;
\item friction heating;
\item adiabatic compression of material;
\item phase change.
\end{itemize}
The last term of the temperature equation corresponds to
the latent heat generated or consumed in the process of phase change of material. The latent heat release
is proportional to changes in the fraction of material $X$ that has already
undergone the phase transition (also called phase function) and the change
of entropy $\Delta S$. This process applies both
to solid-state phase transitions and to melting/solidification.
Here, $\Delta S$ is positive for exothermic phase
transitions. As the phase of the material, for a given composition, depends
on the temperature and pressure, the latent heat term can be reformulated:
\begin{gather*}
\frac{\partial X}{\partial t} + \mathbf u\cdot\nabla X
=
\frac{DX}{Dt} 
= 
\frac{\partial X}{\partial T} \frac{DT}{Dt}
 + \frac{\partial X}{\partial p} \frac{Dp}{Dt}
= 
\frac{\partial X}{\partial T} 
\left(\frac{\partial T}{\partial t} + \mathbf u\cdot\nabla T
\right)
 + \frac{\partial X}{\partial p} \mathbf u\cdot\nabla p.
\end{gather*}
The last transformation results from the assumption that the flow field is
always in equilibrium and consequently $\partial p/\partial t=0$ (this is the
same assumption that underlies the fact that equation \eqref{eq:stokes-1}
does not have a term $\partial \mathbf u / \partial t$). With this
reformulation, we can rewrite \eqref{eq:temperature} in the following way in
which it is in fact implemented:
\begin{align}
  \label{eq:temperature-reformulated}
  \left(\rho C_p - \rho T \Delta S \frac{\partial X}{\partial T}\right) 
  \left(\frac{\partial T}{\partial t} + \mathbf u\cdot\nabla
  T\right) - \nabla\cdot k\nabla T
  &=
  \rho H
  \notag
  \\
  &\quad
  +
  2\eta
  \left(\varepsilon(\mathbf u) - \frac{1}{3}(\nabla \cdot \mathbf u)\mathbf 1\right)
  :
  \left(\varepsilon(\mathbf u) - \frac{1}{3}(\nabla \cdot \mathbf u)\mathbf 1\right)
  \\
  &\quad
  +\alpha T \left( \mathbf u \cdot \nabla p \right)
  \notag
  \\
  &\quad
  + \rho T \Delta S \frac{\partial X}{\partial p} \mathbf u\cdot\nabla p
  & \quad & \textrm{in $\Omega$}.
  \notag
\end{align}

The last of the equations above, equation~\eqref{eq:compositional}, describes
the evolution of additional fields that are transported along with the
velocity field $\mathbf u$ and may react with each other and react to other
features of the solution, but that do not diffuse. We call these fields $c_i$
\textit{compositional fields}, although they can also be used for other
purposes than just tracking chemical compositions. We will discuss this
equation in more detail in Section~\ref{sec:compositional}.

\subsubsection{A comment on adiabatic heating}
Other codes and texts sometimes make a simplification to the adiabatic heating
term in the previous equation. If you assume the vertical component of the
gradient of the \textit{dynamic} pressure to be small compared to the gradient
of the \textit{total} pressure (in other words, the gradient is dominated by
the gradient of the hydrostatic pressure), then $ -\rho \mathbf g \approx
\nabla \mathbf{p} $, and we have the following relation (the negative sign is
due to $\mathbf g$ pointing downwards) 
\begin{align*}
\alpha T \left( \mathbf u \cdot \nabla \mathbf p \right)
  & \approx -\alpha \rho T \mathbf u \cdot \mathbf g.
\end{align*}
While this simplification is possible, it is not necessary if you have access
to the total pressure. \aspect{} therefore implements the original term
without this simplification.

\subsubsection{Boundary conditions}
Having discussed \eqref{eq:temperature}, let us come to the last one of the
original set of equations, \eqref{eq:compositional}. It describes the
motion of a set of advected quantities $c_i(\mathbf x,t),i=1\ldots C$. We call these
\textit{compositional fields} because we think of them as spatially and
temporally varying concentrations of different elements, minerals, or other
constituents of the composition of the material that convects. As such, these
fields participate actively in determining the values of the various
coefficients of these equations. On the other hand, \aspect{} also allows the
definition of material models that are independent of these compositional
fields, making them passively advected quantities. Several of the cookbooks in
Section~\ref{sec:cookbooks} consider compositional fields in this way, i.e.,
essentially as tracer quantities that only keep track of where material came
from.

These equations are
augmented by boundary conditions that can either be of Dirichlet, Neumann, or
tangential type on subsets of the boundary $\Gamma=\partial\Omega$:
\begin{align}
  \mathbf u &= 0 & \qquad &\textrm{on $\Gamma_{0,\mathbf u}$},
  \\
  \mathbf u &= \mathbf u_\text{prescribed} & \qquad &\textrm{on
  $\Gamma_{\text{prescribed},\mathbf u}$},
  \\
  \mathbf n \cdot \mathbf u &= 0 & \qquad &\textrm{on $\Gamma_{\parallel,\mathbf
  u}$},
  \\
  (2\eta \varepsilon(\mathbf u) -p I)\mathbf n  &= \mathbf t & \qquad
  &\textrm{on $\Gamma_{\text{traction},\mathbf u}$},
  \\
  T &= T_{\text{prescribed}}
   & \qquad &\textrm{on $\Gamma_{D,T}$},
  \\
  \mathbf n \cdot k\nabla T &= 0
   & \qquad &\textrm{on $\Gamma_{N,T}$}.
  \\
  \label{eq:gamma-in-composition}
  c_i &= c_{i,\text{prescribed}}
   & \qquad &\textrm{on $\Gamma_{\text{in}}=\{\mathbf x: \mathbf
   u\cdot\mathbf n<0\}$}.
\end{align}
Here, the boundary conditions for velocity and temperature are subdivided into
disjoint parts:
\begin{itemize}
  \item $\Gamma_{0,\mathbf u}$ corresponds to parts of the boundary on
which the velocity is fixed to be zero.
  \item $\Gamma_{\text{prescribed},\mathbf u}$ corresponds to parts of the
  boundary on which the velocity is prescribed to some value (which could also
  be zero). It is possible to restrict prescribing the velocity to only certain
  components of the velocity vector.
  \item $\Gamma_{\parallel,\mathbf u}$ corresponds to parts of the boundary on
  which the velocity may be nonzero but must be parallel to the boundary, with the
tangential component undetermined.
  \item $\Gamma_{\text{traction},\mathbf u}$ corresponds to parts of the
  boundary on which the traction is prescribed to some surface force density (a
  common application being $\mathbf t=-p\mathbf n$ if one
  just wants to prescribe a pressure component). It is possible to restrict
  prescribing the traction to only certain vector components.
  \item $\Gamma_{D,T}$ corresponds to places where the temperature is prescribed
  (for example at the inner and outer boundaries of the earth mantle).
  \item $\Gamma_{N,T}$ corresponds to places where the temperature is unknown
  but the heat flux across the boundary is zero (for example on symmetry surfaces if only a part
of the shell that constitutes the domain the Earth mantle occupies is
simulated).
\end{itemize}
We require that one of these boundary conditions hold at each
point for both velocity and temperature, i.e.,
$\Gamma_{0,\mathbf u}\cup\Gamma_{\text{prescribed},\mathbf
  u}\cup\Gamma_{\parallel,\mathbf u}\cup\Gamma_{\text{traction},\mathbf
  u}=\Gamma$ and
$\Gamma_{D,T}\cup\Gamma_{N,T}=\Gamma$. 

Boundary conditions have to be imposed for the compositional fields only
at those parts of the boundary where flow points inward, see equation
\eqref{eq:gamma-in-composition}, but not where it is either tangential
to the boundary or points outward. The difference in treatment between
temperature and compositional boundary conditions is due to the fact
that the temperature equation contains a (possibly small) diffusion
component, whereas the compositional equations do not.


\subsubsection{Comments on the final set of equations}
\aspect{} solves these equations in essentially the form stated. In
particular, the form given in \eqref{eq:stokes-1} implies that the pressure
$p$ we compute is in fact the \textit{total pressure}, i.e., the sum of
hydrostatic pressure and dynamic pressure (however, see
Section~\ref{sec:pressure-static-dyn} for more information on this, as well as
the extensive discussion of this issue in \cite{KHB12}).
Consequently, it allows the direct use of this pressure when looking up
pressure dependent material parameters.


\subsection{Coefficients}
\label{sec:coefficients}

The equations above contain a significant number of coefficients that we will
discuss in the following. In the most general form, many of these coefficients
depend nonlinearly on the solution variables pressure $p$, temperature $T$
and, in the case of the viscosity, on the strain rate $\varepsilon(\mathbf
u)$. If compositional fields $\mathfrak c=\{c_1,\ldots,c_C\}$ are present (i.e.,
if $C>0$), coefficients may also depend on them. Alternatively, they may be
parameterized as a function
of the spatial variable $\mathbf x$. \aspect{} allows both kinds of
parameterizations.

\note{One of the next versions of \aspect{} will actually iterate out
  nonlinearities in the material description. However, in the current version,
  we simply evaluate all nonlinear dependence of coefficients at the solution
  variables from the previous time step or a solution suitably extrapolated from
  the previous time steps.}

Note that below we will discuss examples of the dependence of coefficients on
other quantities; which dependence is actually implemented in the code is a
different matter. As we will discuss in Sections~\ref{sec:parameters} and
\ref{sec:extending}, some versions of these models are already implemented and
can be selected from the input parameter file; others are easy to add to
\aspect{} by providing self-contained descriptions of a set of coefficients
that the rest of the code can then use without a need for further
modifications.

Concretely, we consider the following coefficients and dependencies:
\begin{itemize}
\item \textit{The viscosity $\eta=\eta(p,T,\varepsilon(\mathbf u),\mathfrak
c,\mathbf x)$:} Units $\textrm{Pa}\cdot \textrm{s} =
  \textrm{kg}\frac{1}{\textrm{m}\cdot\textrm{s}}$.

  The viscosity is the proportionality factor that relates total forces
  (external gravity minus pressure gradients) and fluid velocities $\mathbf
  u$. The simplest models assume that $\eta$ is constant, with the constant
  often chosen to be on the order of $10^{21} \textrm{Pa}\;\textrm{s}$.

  More complex (and more realistic) models assume that the viscosity depends
  on pressure, temperature and strain rate. Since this dependence is often
  difficult to quantify, one modeling approach is to make $\eta$ spatially
  dependent.

\item \textit{The density $\rho=\rho(p,T,\mathfrak c,\mathbf x)$:} Units
  $\frac{\textrm{kg}}{\textrm{m}^3}$.

  In general, the density depends on pressure and temperature, both through
  pressure compression, thermal expansion, and phase changes the material may
  undergo as it moves through the pressure-temperature phase diagram.

  The simplest parameterization for the density is to assume a linear
  dependence on temperature, yielding the form
  $\rho(T)=\rho_{\text{ref}}[1-\beta (T-T_{\text{ref}})]$ where
  $\rho_{\text{ref}}$ is the reference density at temperature $T_{\text{ref}}$
  and $\beta$ is the linear thermal expansion coefficient. For the earth
  mantle, typical values for this parameterization would be
  $\rho_{\text{ref}}=3300\frac{\textrm{kg}}{\textrm{m}^3}$,
  $T_{\text{ref}}=293 \textrm{K}$, $\beta=2\cdot 10^{-5}
  \frac{1}{\mathrm{K}}$.

\item \textit{The gravity vector $\mathbf g=\mathbf g(\mathbf x)$:} Units
  $\frac{\textrm{m}}{\textrm{s}^2}$.

  Simple models assume a radially inward gravity vector of constant magnitude
  (e.g., the surface gravity of Earth, $9.81 \frac{\textrm{m}}{\textrm{s}^2}$),
  or one that can be computed analytically assuming a homogeneous mantle
  density.

  A physically self-consistent model would compute the gravity vector as
  $\mathbf g = -\nabla \varphi$ with a gravity potential $\varphi$ that
  satisfies $-\Delta\varphi=4\pi G\rho$ with the density $\rho$ from above and
  $G$ the universal constant of gravity. This would provide a gravity vector
  that changes as a function of time. Such a model is not currently
  implemented.

\item \textit{The specific heat capacity $C_p=C_p(p,T,\mathfrak c,\mathbf x)$:}
Units $\frac{\textrm{J}}{\textrm{kg}\cdot\textrm{K}} =
  \frac{\textrm{m}^2}{\textrm{s}^2\cdot\textrm{K}}$.

  The specific heat capacity denotes the amount of energy needed to increase
  the temperature of one kilogram of material by one degree. Wikipedia lists a
  value of 790 $\frac{\textrm{J}}{\textrm{kg}\cdot\textrm{K}}$ for granite%
  \footnote{See \url{http://en.wikipedia.org/wiki/Specific_heat}.}
  For the earth mantle, a value of 1250
  $\frac{\textrm{J}}{\textrm{kg}\cdot\textrm{K}}$ is within the range
  suggested by the literature.


\item \textit{The thermal conductivity $k=k(p,T,\mathfrak c,\mathbf x)$:} Units
  $\frac{\textrm{W}}{\textrm{m}\cdot\textrm{K}}=\frac{\textrm{kg}\cdot\textrm{m}}{\textrm{s}^3\cdot\textrm{K}}$.

  The thermal conductivity denotes the amount of thermal energy flowing
  through a unit area for a given temperature gradient. It depends on the
  material and as such will from a physical perspective depend on pressure and
  temperature due to phase changes of the material as well as through
  different mechanisms for heat transport (see, for example, the partial
  transparency of perovskite, the most abundant
  material in the earth mantle, at pressures above around 120 GPa
  \cite{BRVMFG04}).

  As a rule of thumb for its
  order of magnitude, Wikipedia quotes values of
  $1.83$--$2.90\frac{\textrm{W}}{\textrm{m}\cdot\textrm{K}}$ for sandstone and
  $1.73$--$3.98\frac{\textrm{W}}{\textrm{m}\cdot\textrm{K}}$ for granite.%
  \footnote{See \url{http://en.wikipedia.org/wiki/Thermal_conductivity} and
    \url{http://en.wikipedia.org/wiki/List_of_thermal_conductivities}.} The
  values in the mantle are almost certainly higher than this though probably
  not by much. The exact value is not really all that important: heat
  transport through convection is several orders of magnitude more important
  than through thermal conduction.

  The thermal conductivity $k$ is often expressed in terms of the
  \textit{thermal diffusivity} $\kappa$ using the relation $k = \rho C_p \kappa$.

\item \textit{The intrinsic specific heat production $H=H(\mathbf x)$:} Units
  $\frac{\textrm{W}}{\textrm{kg}}=\frac{\textrm{m}^2}{\textrm{s}^3}$.

  This term denotes the intrinsic heating of the material, for example due to
  the decay of radioactive material. As such, it depends not on pressure or
  temperature, but may depend on the location due to different chemical
  composition of material in the earth mantle. The literature suggests a value
  of $\gamma=7.4\cdot 10^{-12}\frac{\textrm{W}}{\textrm{kg}}$.

\item \textit{The change of entropy $\Delta S$ at a
  phase transition together with the derivatives of the phase function
  $X=X(p,T,\mathfrak c,\mathbf x)$ with regard to temperature and pressure:} Units
  $\frac{\textrm{J}}{\textrm{kg}\textrm{K}^2}$ ($-\Delta S \frac{\partial X}{\partial T}$) and
  $\frac{\textrm{m}^3}{\textrm{kg}\textrm{K}}$ ($\Delta S \frac{\partial X}{\partial p}$).

  When material undergoes a phase transition, the entropy changes due to
  release or consumption of latent heat. However, phase transitions occur
  gradually and for a given chemical composition it depends on temperature
  and pressure which phase prevails. Thus, the latent heat release can
  be calculated from the change of entropy $\Delta S$ and the derivatives
  of the phase function $\frac{\partial X}{\partial T}$ and
  $\frac{\partial X}{\partial p}$. These values have to be provided by
  the material model, separately for the coefficient
  $-\Delta S \frac{\partial X}{\partial T}$ on the left-hand side and
  $\Delta S \frac{\partial X}{\partial p}$ on the right-hand side of the
  temperature equation. However, they may be either approximated with the help
  of an analytic phase function, employing data from a thermodynamic database
  or in any other way that seems appropriate to the user.
\end{itemize}


\subsection{Dimensional or non-dimensionalized equations?}
\label{sec:non-dimensional}

Equations \eqref{eq:stokes-1}--\eqref{eq:temperature} are stated in the
physically correct form. One would usually interpret them in a way that the
various coefficients such as the viscosity, density and thermal conductivity
$\eta,\rho,\kappa$ are given in their correct physical units, typically
expressed in a system such as the meter, kilogram, second (MKS) system that is
part of the \href{http://en.wikipedia.org/wiki/SI}{SI} system.
This is certainly how we envision \aspect{} to be used: with geometries,
material models, boundary conditions and initial values to be given in their correct
physical units. As a consequence, when \aspect{} prints information about the
simulation onto the screen, it typically does so by using a postfix such as
\texttt{m/s} to indicate a velocity or \texttt{W/m\^{}2} to indicate a heat
flux.

\note{For convenience, output quantities are sometimes provided
  in units meters per \textit{year} instead of meters per \textit{second}
  (velocities) or in \textit{years} instead of \textit{seconds} (the current
  time, the time step size); this
  conversion happens at the time output is generated, and is not part of the
  solution process. Whether this conversion should happen is determined by the
  flag ``\texttt{Use years in output instead of seconds}'' in the input file,
\index[prmindex]{Use years in output instead of seconds}
\index[prmindexfull]{Use years in output instead of seconds}
  see Section~\ref{parameters:global}. Obviously, this conversion from seconds
  to years only makes sense if the model is described in physical units rather
  than in non-dimensionalized form, see below.}

That said, in reality, \aspect{} has no preferred system of
units as long as every material constant, geometry, time, etc., are all
expressed in the same system. In other words, it is entirely legitimate to
implement geometry and material models in which the dimension of the domain is
one, density and viscosity are one, and the density variation as a function of
temperature is scaled by the Rayleigh number -- i.e., to use the usual
non-dimensionalization of the Boussinesq equations. Some of the cookbooks in
Section~\ref{sec:cookbooks} use this non-dimensional form; for example,
the simplest cookbook in Section~\ref{sec:cookbooks-simple-box} as well as
the SolCx, SolKz and inclusion benchmarks in Sections~\ref{sec:benchmark-solcx},
are such cases. Whenever this is the case, output showing units \texttt{m/s} or
\texttt{W/m\^{}2} clearly no longer have a literal meaning. Rather, the unit postfix must in this case simply
be interpreted to mean that the number that precedes the first is a velocity and
a heat flux in the second case.

In other words, whether a computation uses physical or non-dimensional units
really depends on the geometry, material, initial and boundary condition
description of the particular case under consideration -- \aspect{} will simply
use whatever it is given. Whether one or the other is the more appropriate
description is a decision we purposefully leave to the user. There are of
course good reasons to use non-dimensional descriptions of realistic problems,
rather than to use the original form in which all coefficients remain in their
physical units. On the other hand, there are also downsides:
\begin{itemize}
  \item Non-dimensional descriptions, such as when using the
  \href{http://en.wikipedia.org/wiki/Rayleigh_number}{Rayleigh} number to
  indicate the relative strength of convective to diffusive thermal transport,
  have the advantage that they allow to reduce a system to its essence. For
  example, it is clear that we get the same behavior if one increases both the
  viscosity and the thermal expansion coefficient by a factor of two because the
  resulting Rayleigh number; similarly, if we were to increase the size of the
  domain by a factor of 2 and thermal diffusion coefficient by a factor of 8. In both of
  these cases, the non-dimensional equations are exactly the same. On the other
  hand, the equations in their physical unit form are different and one may not
  see that the result of this variations in coefficients will be exactly the
  same as before. Using non-dimensional variables therefore reduces the space of
  independent parameters one may have to consider when doing parameter studies.

  \item From a practical perspective, equations
  \eqref{eq:stokes-1}--\eqref{eq:temperature} are often ill-conditioned in
  their original form: the two sides of each equation have physical units
  different from those of the other equations, and their numerical values are
  often vastly different.%
  \footnote{To illustrate this, consider convection in the Earth as a
  back-of-the-envelope example.
  With the length scale of the mantle $L=3\cdot 10^6\;\text{m}$, viscosity
  $\eta=10^{24} \; \text{kg}/\text{m}/\text{s}$, density $\rho=3\cdot 10^3 \; \text{kg}/\text{m}^3$ and a typical
  velocity of $U=0.1\;\text{m}/\text{year}=3\cdot 10^{-9}\; \text{m}/\text{s}$, we get that the friction
  term in \eqref{eq:stokes-1} has size $\eta U/L^2 \approx 3\cdot 10^2 \;
  \text{kg}/\text{m}^2/\text{s}^2$. On the other hand, the term $\nabla\cdot(\rho u)$ in the
  continuity equation \eqref{eq:stokes-2} has size $\rho U/L\approx 3\cdot
  10^{-12} \; \text{kg}/\text{s}/\text{m}^3$. In other words, their \textit{numerical values} are 14
  orders of magnitude apart.}
  Of course, these values can not be compared: they have different physical
  units, and the ratios between these values depends on whether we choose to
  measure lengths in meters or kilometers, for example. Nevertheless, when
  implementing these equations in software, at one point or another, we have to
  work with numbers and at this point the physical units are lost. If one does
  not take care at this point, it is easy to get software in which all accuracy
  is lost due to round-off errors. On the other hand, non-dimensionalization
  typically avoids this since it normalizes all quantities so that values that
  appear in computations are typically on the order of one.

  \item On the downside, the numbers non-dimensionalized equations produce are
  not immediately comparable to ones we know from physical experiments. This is
  of little concern if all we have to do is convert every output number of our
  program back to physical units. On the other hand, it is more difficult and a
  source of many errors if this has to be done inside the program, for example,
  when looking up the viscosity as a pressure-, temperature- and
  strain-rate-dependent function: one first has to convert pressure,
  temperature and strain rate from non-dimensional to physical units, look up
  the corresponding viscosity in a table, and then convert the viscosity back to
  non-dimensional quantities. Getting this right at every one of the dozens or
  hundreds of places inside a program and using the correct (but distinct)
  conversion factors for each of these quantities is both a challenge and a possible source
  of errors.

  \item From a mathematical viewpoint, it is typically clear how an equation
  needs to be non-dimensionalized if all coefficients are constant. However, how
  is one to normalize the equations if, as is the case in the earth mantle, the
  viscosity varies by several orders of magnitude? In cases like these, one has
  to choose a reference viscosity, density, etc. While the resulting
  non-dimensionalization retains the universality of parameters in the
  equations, as discussed above, it is not entirely clear that this would also
  retain the numerical stability if the reference values are poorly chosen.
\end{itemize}

As a consequence of such considerations, most codes in the past have used
non-dimensionalized models. This was aided by the fact that until recently and
with notable exceptions, many models had constant coefficients and the
difficulties associated with variable coefficients were not a concern. On the
other hand, our goal with \aspect{} is for it to be a code that solves realistic
problems using complex models and that is easy to use. Thus, we allow users to
input models in physical or non-dimensional units, at their discretion. We
believe that this makes the description of realistic models simpler. On
the other hand, ensuring numerical stability is not something users should have
to be concerned about, and is taken care of in the implementation of \aspect{}'s
core (see the corresponding section in \cite{KHB12}).



\subsection{Static or dynamic pressure?}
\label{sec:pressure-static-dyn}

One could reformulate equation \eqref{eq:stokes-1} somewhat. To this end, let us
say that we would want to represent the pressure $p$ as the sum of two parts
that we will call static and dynamic, $p=p_s+p_d$. If we assume that $p_s$ is
already given, then we can replace \eqref{eq:stokes-1} by
\begin{gather*}
  -\nabla \cdot 2\eta
  \nabla \mathbf u + \nabla p_d =
  \rho\mathbf g - \nabla p_s.
\end{gather*}
One typically chooses $p_s$ as the pressure one would get if the whole medium
were at rest -- i.e., as the hydrostatic pressure. This pressure can be
computed noting that \eqref{eq:stokes-1} reduces to
\begin{gather*}
  \nabla p_s = \rho(p_s,T_s,\mathbf x)\mathbf g
\end{gather*}
in the absence of any motion where $T_s$ is some static temperature field (see
also Section~\ref{sec:adiabatic}). This, our rewritten version of
\eqref{eq:stokes-1} would look like this:
\begin{gather*}
  -\nabla \cdot 2\eta
  \nabla \mathbf u + \nabla p_d =
  \left[\rho(p,T,\mathbf x)-\rho(p_s,T_s,\mathbf x)\right]\mathbf g.
\end{gather*}
In this
formulation, it is clear that the quantity that drives the fluid flow is in
fact the \textit{buoyancy} caused by the \textit{variation} of densities,
not the density itself.

This reformulation has a number of advantages and disadvantages:
\begin{itemize}
\item One can notice that in many realistic cases, the dynamic component $p_d$
  of the pressure is orders of magnitude smaller than the static component
  $p_s$. For example, in the earth, the two are separated by around 6 orders
  of magnitude at the bottom of the earth mantle. Consequently, if one wants
  to solve the linear system that arises from discretization of the original
  equations, one has to solve it a significant degree of accuracy (6--7
  digits) to get the dynamic part of the pressure correct to even one
  digit. This entails a very significant numerical effort, and one that is not
  necessary if we can split the pressure in a way so that the pre-computed
  static pressure $p_s$ (or, rather, the density using the static pressure and
  temperature from which $p_s$ results) absorbs the dominant part and one only
  has to compute the remaining, dynamic pressure to 2 or 3 digits of accuracy,
  rather than the corresponding 7--8 for the total pressure.

\item On the other hand, the pressure $p_d$ one computes this way is not immediately
  comparable to quantities that we use to look up pressure-dependent
  quantities such as the density. Rather, one needs to first find the static
  pressure as well (see Section~\ref{sec:adiabatic}) and add the two together
  before they can be used to look up material properties or to compare them with
  experimental results. Consequently, if the pressure a program outputs
  (either for visualization, or in the internal interfaces to parts of the
  code where users can implement pressure- and temperature-dependent material
  properties) is only the dynamic component, then all of the consumers of this
  information need to convert it into the total pressure when comparing with
  physical experiments. Since any code implementing realistic material models
  has a great many of these places, there is a large potential for inadvertent
  errors and bugs.

\item Finally, the definition of a reference density $\rho(p_s,T_s,\mathbf x)$
  derived from static pressures and temperatures
  is only simple if we have incompressible models and under the assumption
  that the temperature-induced density variations are small compared to the
  overall density. In this case, we can choose $\rho(p_s,T_s,\mathbf
  x)=\rho_0$ with a constant reference density $\rho_0$. On the other hand,
  for more complicated models, it is not a priori
  clear which density to choose since we first need to compute static
  pressures and temperatures -- quantities that satisfy equations that
  introduce boundary layers, may include phase changes releasing latent heat,
  and where the density may have discontinuities at certain depths, see
  Section~\ref{sec:adiabatic}.

  Thus, if we compute adiabatic pressures and
  temperatures $\bar p_s,\bar T_s$ under the assumption of a thermal boundary layer
  worth 900 Kelvin at the top, and we get a corresponding density profile
  $\bar\rho=\rho(\bar p_s,\bar T_s, \mathbf x)$, but after running for a few
  million years the temperature turns out to be so that the top boundary layer
  has a jump of only 800 Kelvin with corresponding adiabatic pressures and
  temperatures $\hat p_s,\hat T_s$, then a more appropriate density profile
  would be $\hat\rho=\rho(\hat p_s,\hat T_s, \mathbf x)$.

  The problem is that it may well be that the erroneously computed density
  profile $\hat \rho$ does \textit{not} lead to a separation where
  $|p_d|\ll|p_s|$ because, especially if the material undergoes phase changes,
  there will be entire areas of the computational domain in which $|\rho-\hat
  \rho_s|\ll |\rho|$ but $|\rho-\bar
  \rho_s|\not\ll |\rho|$. Consequently the benefits of lesser requirements on the
  iterative linear solver would not be realized.
\end{itemize}

We do note that most of the codes available today and that we are aware of
split the pressure into static and dynamic parts nevertheless, either
internally or require the user to specify the density profile as the
difference between the true and the hydrostatic density. This may, in part, be
due to the fact that historically most codes were written to solve problems
in which the medium was considered incompressible, i.e., where the definition
of a static density was simple.

On the other hand, we intend \aspect{} to be a code that can solve more
general models for which this definition is not as simple. As a consequence, we
have chosen to solve the equations as stated originally -- i.e., we solve for
the \textit{full} pressure rather than just its \textit{dynamic} component. With
most traditional methods, this would lead to a catastrophic loss of accuracy in the
dynamic pressure since it is many orders of magnitude smaller than the total
pressure at the bottom of the earth mantle. We avoid this problem in \aspect{}
by using a cleverly chosen iterative solver that ensures that the full pressure
we compute is accurate enough so that the dynamic pressure can be extracted from
it with the same accuracy one would get if one were to solve for only the
dynamic component. The methods that ensure this are described in detail in
\cite{KHB12} and in particular in the appendix of that paper.


\subsection{Pressure normalization}
\label{sec:pressure}

The equations described above, \eqref{eq:stokes-1}--\eqref{eq:temperature},
only determine the pressure $p$ up to an additive constant. On the other hand,
since the pressure appears in the definition of many of the coefficients, we
need a pressure that has some sort of \textit{absolute} definition. A
physically useful definition would be to normalize the pressure in such a way
that the average pressure along the ``surface'' has a prescribed value where
the geometry description (see Section~\ref{sec:geometry-models}) has to
determine which part of the boundary of the domain is the ``surface'' (we call
a part of the boundary the ``surface'' if its depth is ``close to zero'').

Typically, one will choose this average pressure to be zero, but there is a
parameter ``\texttt{Surface pressure}''
\index[prmindex]{Surface pressure}
\index[prmindexfull]{Surface pressure}
in the input file (see Section~\ref{parameters:global}) to set it to
a different value. One may want to do that, for example, if one wants to
simulate the earth mantle without the overlying lithosphere. In that case, the
``surface'' would be the interface between mantle and lithosphere, and the
average pressure at the surface to which the solution of the equations will be
normalized should in this case be the hydrostatic pressure at the bottom of
the lithosphere.

An alternative is to normalize the pressure in such a way that the
\textit{average} pressure throughout the domain is zero or some constant
value. This is not a useful approach for most geodynamics applications but is
common in benchmarks for which analytic solutions are available. Which kind of
normalization is chosen is determined by the ``\texttt{Pressure
  normalization}'' flag in the input file,
\index[prmindex]{Pressure normalization}
\index[prmindexfull]{Pressure normalization}
see Section~\ref{parameters:global}.


\subsection{Initial conditions and the adiabatic pressure/temperature}
\label{sec:adiabatic}

Equations \eqref{eq:stokes-1}--\eqref{eq:temperature} require us to
pose initial conditions for the temperature, and this is done by
selecting one of the existing models for initial conditions in the
input parameter file, see
Section~\ref{parameters:Initial_20conditions}. The equations
themselves do not require that initial conditions are specified for
the velocity and pressure variables (since there are no time
derivatives on these variables in the model).

Nevertheless, a nonlinear solver will have difficulty converging to
the correct solution if we start with a completely unphysical pressure
for models in which coefficients such as density $\rho$ and viscosity
$\eta$ depend on the pressure and temperature. To this end, \aspect{} computes
pressure and temperature fields $p_{\textrm{ad}}(z),
T_{\textrm{ad}}(z)$ that satisfy adiabatic conditions:
\begin{align}
  \rho C_p \frac{\textrm{d}}{\textrm{d}z} T_{\textrm{ad}}(z)
  &=
  \frac{\partial\rho}{\partial T} T_{\textrm{ad}}(z) g_z,
\\
  \frac{\textrm{d}}{\textrm{d}z} p_{\textrm{ad}}(z)
  &=
  \rho g_z,
\end{align}
where strictly speaking $g_z$ is the magnitude of the vertical
component of the gravity vector field, but in practice we take the
magnitude of the entire gravity vector.

These equations can be integrated numerically starting at $z=0$, using
the depth dependent gravity field and values of the coefficients
$\rho=\rho(p,T,z), C_p=C_p(p,T,z)$. As starting conditions at $z=0$ we
choose a pressure $p_{\textrm{ad}}(0)$ equal to the average surface
pressure (often chosen to be zero, see Section~\ref{sec:pressure}),
and an adiabatic surface temperature $T_{\textrm{ad}}(0)$ that is
\index[prmindex]{Adiabatic surface temperature}
\index[prmindexfull]{Adiabatic surface temperature}
also selected in the input parameter file.

\note{The adiabatic surface temperature is often chosen significantly
  higher than the actual surface temperature. For example, on earth,
  the actual surface temperature is on the order of 290 K, whereas a
  reasonable adiabatic surface temperature is maybe 1200 K. The reason
  is that the bulk of the mantle is more or less in thermal equilibrium
  with a thermal profile that corresponds to the latter temperature,
  whereas the very low actual surface temperature and the very high
  bottom temperature at the core-mantle boundary simply induce a
  thermal boundary layer. Since the temperature and pressure profile
  we compute using the equations above are simply meant to be good
  starting points for nonlinear solvers, it is important to choose
  this profile in such a way that it covers most of the mantle well;
  choosing an adiabatic surface temperature of 290 K would yield a
  temperature and pressure profile that is wrong almost throughout the
  entire mantle.}



\subsection{Compositional fields}
\label{sec:compositional}

The last of the basic equations, \eqref{eq:compositional}, describes the
evolution of a set of variables $c_i(\mathbf x, t), i=1\ldots C$ that we
typically call \textit{compositional fields} and that we often aggregate into
a vector $\mathfrak c$.

Compositional fields were originally intended to track what their name
suggest, namely the chemical composition of the convecting medium. In this
interpretation, the composition is a quantity that is simply advected along
passively, i.e., it would satisfy the equation
\begin{align*}
  \frac{\partial \mathfrak c}{\partial t} + \mathbf u \cdot \nabla \mathfrak c
  = 0.
\end{align*}
However, the compositional fields may also participate in determining the values of
the various coefficients as discussed in
Section~\ref{sec:coefficients}, and in this sense the equation above
describes a composition that is \textit{passively advected}, but an
\textit{active participant} in the equations.

That said, over time compositional fields have shown to be a much more useful
tool than originally intended. For example, they can be used to track where
material comes from and goes to (see Section~\ref{sec:cookbooks-composition})
and, if one allows for a reaction rate $\mathfrak q$ on the right hand side,
\begin{align*}
  \frac{\partial \mathfrak c}{\partial t} + \mathbf u \cdot \nabla \mathfrak c
  = \mathfrak q,
\end{align*}
then one can also model interaction between species -- for example to simulate
phase changes where one compositional field, indicating a particular phase,
transforms into another phase depending on pressure and temperature, or where
several phases combine to other phases. Another example of using a
right hand side -- quite outside what the original term
\textit{compositional field} was supposed to indicate -- is to track
the accumulation of finite strain, see Section~\ref{sec:finite-strain}.

In actual practice, one finds that it is often useful to allow
$\mathfrak q$ to be a function that has both a smooth (say,
continuous) in time component, and one that is singular in time (i.e.,
contains Dirac delta, or ``impulse'' functions). Typical time
integrators require the evaluation of the right hand side at specific
points in time, but this would preclude the use of delta
functions. Consequently, the integrators in \aspect{} only require
material models to provide an \textit{integrated} value
$\int_t^{t+\Delta t} \mathfrak q(\tau) \;
\text{d}\tau$ through the {\tt reaction\_term} output
variable. Implementations often approximate this as $\triangle t \cdot
\mathfrak q(t)$, or similar formulas.

A second application for only providing integrated right hand sides
comes from the fact that
modeling reactions between different compositional fields often involves
finding an equilibrium state between different fields because
chemical reactions happen on a much faster time scale than transport. In other
words, one then often assumes that there is a $\mathfrak c^\ast(p,T)$ so that
\begin{align*}
  \mathfrak q(p,T,\varepsilon(\mathbf u),\mathfrak c^\ast(p,T)) = 0.
\end{align*}
Consequently, the material model methods that deal with source terms for the
compositional fields need to compute an \textit{increment} $\Delta\mathfrak c$
to the previous value of the compositional fields so that the sum of the
previous values and the increment equals $\mathfrak c^\ast$. This
corresponds to an \textit{impulse change} in the compositions at every
time step, as opposed
to the usual approach of evaluating the right hand side term
$\mathfrak q$ as a continuous function in time,
which corresponds to a \textit{rate}.

On the other hand, there are other uses of compositional fields that do not
actually have anything to do with quantities that can be considered related to
compositions. For example, one may define a field that tracks the grain size
of rocks. If the strain rate is high, then the grain size decreases as the
rocks break. If the temperature is high enough, then grains heal and their size
increases again. Such ``damage'' models would then call for an equation of the
form (assuming one uses only a single compositional field)
\begin{align*}
  \frac{\partial c}{\partial t} + \mathbf u \cdot \nabla c
  = q(T,c),
\end{align*}
where in the simplest case one could postulate
\begin{align*}
  q(T,c) = -\alpha c + \beta \max\{T-T_\text{healing},0\} c.
\end{align*}
One would then use this compositional field in the definition of the viscosity
of the material: more damage means lower viscosity because the rocks are weaker.

In cases like this, there is only a single compositional field and it is not
in permanent equilibrium. Consequently, the increment implementations of
material models in \aspect{} need to compute is typically the rate $q(T,c)$
times the time step.  In other words, if you compute a reaction rate inside the material model you need to multiply it by the time step size before returning the value.

Compositional fields have proven to be surprisingly versatile tools to model
all sorts of components of models that go beyond the simple Stokes plus
temperature set of equations. Play with them!


\subsection{Constitutive laws}

Equation \eqref{eq:stokes-1} describes buoyancy-driven flow in an isotropic
fluid where strain rate is related to stress by a scalar (possibly spatially variable)
multiplier, $\eta$. For some material models it is useful to generalize this
relationship to anisotropic materials, or other exotic constitutive laws.
For these cases \aspect{} can optionally include a generalized, fourth-order
tensor field as a material model state variable which changes equation
\eqref{eq:stokes-1} to
\begin{align}
  \label{eq:stokes-1-anisotropic}
  -\nabla \cdot \left[2\eta \left(C \varepsilon(\mathbf u)
                                  - \frac{1}{3}(tr(C \varepsilon(\mathbf u)))\mathbf 1\right)
                \right] + \nabla p &=
  \rho \mathbf g
  & \qquad
  & \textrm{in $\Omega$}
\end{align}
and the shear heating term in equation \eqref{eq:temperature} to
\begin{align}
  \label {eq:temperature-anisotropic}
  \dots
  \notag
  \\
  + 2 \eta
  \left(C \varepsilon(\mathbf u) - \frac{1}{3}(tr(C \varepsilon(\mathbf u)))\mathbf 1\right)
  :
  \left(\varepsilon(\mathbf u) - \frac{1}{3}(\nabla \cdot \mathbf u)\mathbf 1\right)
  \\
  \dots
  \notag
\end{align}
where $C = C_{ijkl}$ is defined by the material model. For physical reasons, $C$ needs
to be a symmetric rank-4 tensor: i.e., when multiplied by a symmetric (strain rate)
tensor of rank 2 it needs to return another symmetric tensor of rank 2. In mathematical
terms, this means that $C_{ijkl}=C_{jikl}=C_{ijlk}=C_{jilk}$. Energy considerations
also require that $C$ is positive definite: i.e., for any $\varepsilon \neq 0$, the
scalar $\varepsilon : (C \varepsilon)$ must be positive.

This functionality can be optionally invoked by any material model that chooses to
define a $C$ field, and falls back to the default case ($C=\mathbb I$) if no such
field is defined. It should be noted that $\eta$ still appears in equations
\eqref{eq:stokes-1-anisotropic} and \eqref{eq:temperature-anisotropic}. $C$ is
therefore intended to be thought of as a ``director'' tensor rather than a
replacement for the viscosity field, although in practice either interpretation
is okay.


\subsection{Numerical methods}

There is no shortage in the literature for methods to solve the equations
outlined above. The methods used by \aspect{} use the following,
interconnected set of strategies in the implementation of numerical
algorithms:
\begin{itemize}
\item \textit{Mesh adaptation:} Mantle convection problems are characterized
  by widely disparate length scales (from plate boundaries on the order of
  kilometers or even smaller, to the scale of the entire earth). Uniform
  meshes can not resolve the smallest length scale without an intractable
  number of unknowns.  Fully adaptive meshes allow resolving local features of
  the flow field without the need to refine the mesh globally. Since the
  location of plumes that require high resolution change and move with time,
  meshes also need to be adapted every few time steps.
\item \textit{Accurate discretizations:} The Boussinesq problem upon which
  most models for the earth mantle are based
  has a number of intricacies that make the choice of discretization
  non-trivial. In particular, the finite elements chosen for velocity and
  pressure need to satisfy the usual compatibility condition for saddle point
  problems. This can be worked around using pressure stabilization schemes for
  low-order discretizations, but high-order methods can yield better accuracy
  with fewer unknowns and offer more reliability. Equally important is the choice of
  a stabilization method for the highly advection-dominated temperature
  equation. \aspect{} uses a nonlinear artificial diffusion method for the latter.
\item \textit{Efficient linear solvers:} The major obstacle in solving the
  Boussinesq system is the saddle-point nature of the Stokes equations. Simple
  linear solvers and preconditioners can not efficiently solve this system in
  the presence of strong heterogeneities or when the size of the system
  becomes very large. \aspect{} uses an efficient solution strategy based on a
  block triangular preconditioner utilizing an algebraic multigrid that
  provides optimal complexity even up to problems with hundreds of millions of
  unknowns.
\item \textit{Parallelization of all of the steps above:} Global mantle convection
  problems frequently require extremely large numbers of unknowns for
  adequate resolution in three dimensional simulations. The only realistic way to solve such problems lies in
  parallelizing computations over hundreds or thousands of processors. This is
  made more complicated by the use of dynamically changing meshes, and it
  needs to take into account that we want to retain the optimal complexity of
  linear solvers and all other operations in the program.
\item \textit{Modularity of the code:} A code that implements all of these
  methods from \textit{scratch} will be unwieldy, unreadable and unusable as a community
  resource. To avoid this, we build our implementation on widely used and well
  tested libraries that can provide researchers interested in extending it
  with the support of a large user community. Specifically, we use the
  \dealii{} library \cite{BHK07,BK99m} for meshes, finite
  elements and everything discretization related; the \trilinos{} library
  \cite{trilinos,trilinos-web-page} for scalable and parallel linear algebra;
  and \pfrst{} \cite{p4est} for distributed, adaptive meshes. As a
  consequence, our code is freed of the mundane tasks of defining finite
  element shape functions or dealing with the data structures of linear algebra,
  can focus on the high-level description of what is supposed to happen, and
  remains relatively compact. The code will also
  automatically benefit from improvements to the underlying libraries with
  their much larger development communities. \aspect{} is extensively
  documented to enable other researchers to understand, test, use, and extend it.
\end{itemize}

Rather than detailing the various techniques upon which \aspect{} is built, we
refer to the paper by Kronbichler, Heister and Bangerth \cite{KHB12} that
gives a detailed description and rationale for the various building blocks.


\subsection{Simplifications of the basic equations}

There are two common variations to equations
\eqref{eq:stokes-1}--\eqref{eq:temperature} that are frequently used and that
make the system much simpler to solve and analyze: assuming that the fluid is
incompressible (the Boussinesq approximation) and a linear dependence of the
density on the temperature with constants that are otherwise independent of
the solution variables. These are
discussed in the following; \aspect{} has
run-time parameters that allow both of these simpler models to be used.

\subsubsection{The Boussinesq approximation: Incompressibility}
\label{sec:boussinesq}

The original Boussinesq approximation assumes that the density can be
considered constant in all occurrences in the equations with the exception of
the buoyancy term on the right hand side of \eqref{eq:stokes-1}. The primary
result of this assumption is that the continuity equation \eqref{eq:stokes-2}
will now read
\begin{gather*}
  \nabla \cdot \mathbf u = 0.
\end{gather*}
This makes the equations \textit{much} simpler to solve: First, because the
divergence operation in this equation is the transpose of the gradient of the
pressure in the momentum equation \eqref{eq:stokes-1}, making the system of
these two equations symmetric. And secondly, because the two equations are now
linear in pressure and velocity (assuming that the viscosity $\eta$ and the
density $\rho$ are considered fixed). In addition, one can drop all terms
involving $\nabla \cdot \mathbf u$ from the left hand side of the momentum
equation \eqref{eq:stokes-1} as well as from the shear heating term on the
right hand side of \eqref{eq:temperature}; while dropping these terms does not
affect the solution of the equations, it makes assembly of linear systems
faster. In addition, in the incompressible case, one needs to neglect the
adiabatic heating term $\frac{\partial \rho}{\partial T} T \mathbf u \cdot
\mathbf g$ on the right hand side of \eqref{eq:temperature}.

From a physical perspective, the assumption that the density is constant in
the continuity equation but variable in the momentum equation is of course
inconsistent. However, it is justified if the variation is small since the
momentum equation can be rewritten to read
\begin{gather*}
  -\nabla \cdot 2\eta \varepsilon(\mathbf u) + \nabla p_d =
  (\rho-\rho_0) \mathbf g,
\end{gather*}
where $p_d$ is the \textit{dynamic} pressure and $\rho_0$ is the constant
reference density. This makes it clear that the true driver of motion is in
fact the \textit{deviation} of the density from its background value, however
small this value is: the resulting velocities are simply proportional to the
density variation, not to the absolute magnitude of the density.

As such, the Boussinesq approximation can be justified. On the other hand,
given the real pressures and temperatures at the bottom of the earth mantle,
it is arguable whether the density can be considered to be almost
constant. Most realistic models predict that the density of mantle rocks
increases from somewhere around 3300 at the surface to over 5000 kilogram per
cubic meters at the core mantle boundary, due to the increasing lithostatic
pressure. While this appears to be a large variability, if the density changes
slowly with depth, this is not in itself an indication that the Boussinesq
approximation will be wrong. To this end, consider that the continuity
equation can be rewritten as $\frac 1\rho \nabla \cdot (\rho \mathbf u)=0$,
which we can multiply out to obtain
\begin{gather*}
  \nabla \cdot \mathbf u
  +
  \frac 1\rho \mathbf u \cdot \nabla \rho
  = 0.
\end{gather*}
The question whether the Boussinesq approximation is valid is then whether the
second term (the one omitted in the Boussinesq model) is small compared to the
first. To this end, consider that the velocity can change completely over length
scales of maybe 10 km, so that $\nabla \cdot\mathbf u \approx \|u\| /
10\text{km}$. On the other hand, given a smooth dependence of density on pressure,
the length scale for variation of the density is the entire earth mantle,
i.e., $\frac 1\rho \mathbf u \cdot \nabla\rho \approx \|u\| 0.5 / 3000 \text{km}$
(given a variation between minimal and maximal density of 0.5 times the
density itself). In other words, for a smooth variation, the contribution of
the compressibility to the continuity equation is very small. This may be
different, however, for models in which the density changes rather abruptly,
for example due to phase changes at mantle discontinuities.

In summary, models that use the approximation of incompressibility solve the
following set of equations instead of \eqref{eq:stokes-1}--\eqref{eq:temperature}:
\begin{align}
  \label{eq:stokes-1-boussinesq}
  -\nabla \cdot \left[2\eta \varepsilon(\mathbf u)
                \right] + \nabla p &=
  \rho \mathbf g
  & \qquad
  & \textrm{in $\Omega$},
  \\
  \label{eq:stokes-2-boussinesq}
  \nabla \cdot \mathbf u &= 0
  & \qquad
  & \textrm{in $\Omega$},
  \\
  \label{eq:temperature-boussinesq}
  \rho C_p \left(\frac{\partial T}{\partial t} + \mathbf u\cdot\nabla T\right)
  - \nabla\cdot k\nabla T
  &=
  \rho H
  +
  2\eta
  \varepsilon(\mathbf u)
  :
  \varepsilon(\mathbf u)
  & \quad
  & \textrm{in $\Omega$},
\end{align}
where the coefficients $\eta,\rho,\mathbf g,C_p$ may possible depend on the
solution variables.

\note{As we will see in Section~\ref{sec:extending}, it is easy to add new material
models to \aspect. Each model can decide whether it wants to use the
Boussinesq approximation or not. The description of the models in
Section~\ref{parameters:Material_20model} also gives an answer which of the
models already implemented uses the approximation or considers the material
sufficiently compressible to go with the fully compressible continuity equation.}


\subsubsection{Almost linear models}

A further simplification can be obtained if one assumes that all coefficients
with the exception of the density do not depend on the solution variables but
are, in fact, constant. In such models, one typically assumes that the density
satisfies a relationship of the form $\rho=\rho(T)=\rho_0(1-\beta(T-T_0))$
with a small thermal expansion coefficient $\beta$ and a reference density
$\rho_0$ that is attained at temperature $T_0$. Since the thermal expansion is
considered small, this naturally leads to the following variant of the Boussinesq
model discussed above:
\begin{align*}
  -\nabla \cdot \left[2\eta \varepsilon(\mathbf u)
                \right] + \nabla p &=
  \rho_0 (1-\beta (T-T_0)) \mathbf g
  & \qquad
  & \textrm{in $\Omega$},
  \\
  \nabla \cdot \mathbf u &= 0
  & \qquad
  & \textrm{in $\Omega$},
  \\
  \rho C_p \left(\frac{\partial T}{\partial t} + \mathbf u\cdot\nabla T\right)
  - \nabla\cdot k\nabla T
  &=
  \rho H
  +
  2\eta
  \varepsilon(\mathbf u)
  :
  \varepsilon(\mathbf u)
  & \quad
  & \textrm{in $\Omega$},
\end{align*}
If the gravitational acceleration $\mathbf g$ results from a gravity potential
$\varphi$ via $\mathbf g = -\nabla \varphi$, then one can rewrite the
equations above in the following, commonly used form:%
\footnote{Note, however, that \aspect{} does not solve the equations in the
  form given in
  \eqref{eq:stokes-1-boussinesq-linear}--\eqref{eq:temperature-boussinesq-linear}. Rather,
  it takes the original form with the real density, not the variation of
  the density. That said, you can use the formulation
  \eqref{eq:stokes-1-boussinesq-linear}--\eqref{eq:temperature-boussinesq-linear}
  by implementing a material model (see Section~\ref{sec:material-models}) in
  which the density in fact has the form $\rho(T)=\beta \rho_0 T$ even though
  this is not physical.}
\begin{align}
  \label{eq:stokes-1-boussinesq-linear}
  -\nabla \cdot \left[2\eta \varepsilon(\mathbf u)
                \right] + \nabla p_d &=
  -\beta\rho_0 T \mathbf g
  & \qquad
  & \textrm{in $\Omega$},
  \\
  \label{eq:stokes-2-boussinesq-linear}
  \nabla \cdot \mathbf u &= 0
  & \qquad
  & \textrm{in $\Omega$},
  \\
  \label{eq:temperature-boussinesq-linear}
  \rho C_p \left(\frac{\partial T}{\partial t} + \mathbf u\cdot\nabla T\right)
  - \nabla\cdot k\nabla T
  &=
  \rho H
  +
  2\eta
  \varepsilon(\mathbf u)
  :
  \varepsilon(\mathbf u)
  & \quad
  & \textrm{in $\Omega$},
\end{align}
where $p_d=p+\rho_0(1+\beta T_0)\varphi$ is the dynamic pressure, as opposed
to the total pressure $p=p_d+p_s$ that also includes the hydrostatic pressure
$p_s=-\rho_0(1+\beta T_0)\varphi$. Note that the right hand side forcing term
in \eqref{eq:stokes-1-boussinesq-linear} is now only the deviation of the
gravitational force from the force that would act if the material were at
temperature $T_0$.

Under the assumption that all other coefficients are constant, one then
arrives at equations in which the only nonlinear terms are the advection term,
$\mathbf u \cdot \nabla T$, and the shear friction, $2\eta\varepsilon(\mathbf
u):\varepsilon(\mathbf u)$, in the temperature equation
\eqref{eq:temperature-boussinesq-linear}. This facilitates the use of a
particular class of time stepping scheme in which one does not solve the whole
set of equations at once, iterating out nonlinearities as necessary, but
instead in each time step solves first the Stokes system with the previous
time step's temperature, and then uses the so-computed velocity to solve the
temperature equation. These kind of time stepping schemes are often referred
to as \textit{operator splitting} methods. A particular operator
splitting method, used in
earlier \aspect{} versions, solves first the Stokes equations and then
uses a semi-explicit time stepping method for the temperature equation
where diffusion is handled implicitly and advection explicitly;
this algorithm is often called \textit{IMPES} (it originated in the
porous media flow 
community, where the acronym stands for \textit{Im}plicit \textit{P}ressure,
\textit{E}xplicit \textit{S}aturation) and is explained in more detail
in \cite{KHB12}. However, since then the algorithm in \aspect{} has
been rewritten to use an implicit time stepping algorithm also for the
temperature equation because this allows to use larger time steps.



\subsubsection{Compressible models}
In the compressible case, the conservation of mass equation in
equation~\eqref{eq:stokes-2} becomes $\nabla 
\cdot \left( \rho \textbf{u} \right)= 0$ instead of $\nabla \cdot \textbf{u} =
0$, which is nonlinear and no longer symmetric to the $\nabla p$ term in the
force balance equation \eqref{eq:stokes-1}, making solving and preconditioning
the resulting linear and nonlinear systems difficult. To make this work in
\aspect{}, we consequently reformulate this equation. Dividing by $\rho$ and
applying the product rule of differentiation gives
\begin{equation*}
\frac{1}{\rho} \nabla \cdot \left( \rho \textbf{u} \right) = \nabla \cdot \textbf{u} + \frac{1}{\rho} \nabla \rho \cdot  \textbf{u}.
\end{equation*}
We will now make two basic assumptions: First, the variation of the density
$\rho(p,T,\mathbf x, \mathfrak c)$ is dominated by the dependence on the
(total) pressure; in other words, $\nabla \rho \approx \frac{\partial \rho}{\partial
  p}\nabla p$. This assumption is primarily justified by the fact that, in the
Earth mantle, the density increases by at least 50\% between Earth crust and
the core-mantle boundary due to larger pressure there. Secondly, we assume
that the pressure is dominated by the static pressure, which can be written as
$\nabla p \approx \nabla p_s \approx \rho \textbf{g}$. This is essentially
motivated by the slowness of the movement in the Earth or, alternatively,
based on the fact that the viscosity is so large.
This finally allows us to write
\begin{equation*}
\frac{1}{\rho} \nabla \rho \cdot \textbf{u} \approx \frac{1}{\rho} \frac{\partial \rho}{\partial p} \nabla p \cdot \textbf{u} \approx \frac{1}{\rho} \frac{\partial \rho}{\partial p} \nabla p_s \cdot \textbf{u} \approx \frac{1}{\rho} \frac{\partial \rho}{\partial p} \rho \textbf{g} \cdot \textbf{u} 
\end{equation*}
so we get
\begin{equation}
\label{eq:stokes-2-compressible}
\nabla \cdot \textbf{u} = -\frac{1}{\rho} \frac{\partial \rho}{\partial p} \rho \textbf{g} \cdot \textbf{u}
\end{equation}
where $\frac{1}{\rho} \frac{\partial \rho}{\partial p}$ is often referred to
as the compressibility.

In the implementation used in \aspect{}, this equation replaces
\eqref{eq:stokes-2}. It has the advantage that it retains the symmetry of the
Stokes equations if we can treat the right hand side of
\eqref{eq:stokes-2-compressible} as known. We do so by evaluating $\rho$ and
$\mathbf u$ using the solution from the last time step (or values extrapolated
from previous time steps).



\subsection{Free surface calculations}
\label{sec:freesurface}

In reality the boundary conditions of a convecting Earth are not no-slip or 
free slip (i.e., no normal velocity).  Instead, we expect that a free surface
is a more realistic approximation, since air and water should not prevent the
flow of rock upward or downward.  This means that we require zero stress on the 
boundary, or $\sigma \cdot \textbf{n} = 0$, where $\sigma = 2 \eta \varepsilon (\textbf{u})$. 
In general there will be flow across the boundary with this boundary condition.  
To conserve mass we must then advect the boundary of the domain in the direction 
of fluid flow.  Thus, using a free surface necessitates that the mesh be dynamically deformable.  

\subsubsection{Arbitrary Lagrangian-Eulerian implementation}

The question of how to handle the motion of the mesh with a free surface is
challenging.  Eulerian meshes are well behaved, but they do not move with the 
fluid motions, which makes them difficult for use with free surfaces. 
Lagrangian meshes do move with the fluid, but they quickly become so 
distorted that remeshing is required. \aspect{} implements an Arbitrary 
Lagrangian-Eulerian (ALE) framework for handling motion of the mesh.  The ALE 
approach tries to retain the benefits of both the Lagrangian and the Eulerian
approaches by allowing the mesh motion $\textbf{u}_m$ to be largely independent of 
the fluid. The mass conservation condition requires that 
$\textbf{u}_m \cdot \textbf{n} = \textbf{u} \cdot \textbf{n}$ on the free 
surface, but otherwise the mesh motion is unconstrained, and should be chosen
to keep the mesh as well behaved as possible.

\aspect{} uses a Laplacian scheme for calculating the mesh velocity.  The mesh
velocity is calculated by solving

\begin{align}
-\Delta \textbf{u}_m &= 0 & \qquad & \textrm{in } \Omega, \\ 
\textbf{u}_m &= \left( \textbf{u} \cdot \textbf{n} \right) \textbf{n} & \qquad & \textrm{on } \partial \Omega_{\textrm{free surface}}, \\
\textbf{u}_m \cdot \textbf{n} &= 0 & \qquad & \textrm{on } \partial \Omega_{\textrm{free slip}}, \\
\textbf{u}_m &= 0 & \qquad & \textrm{on } \partial \Omega_{\textrm{Dirichlet}}.
\end{align}
After this mesh velocity is calculated, the mesh vertices are time-stepped explicitly.
This scheme has the effect of choosing a minimally distorting perturbation to the mesh.
Because the mesh velocity is no longer zero in the ALE approach, we must then correct
the Eulerian advection terms in the advection system with the mesh velocity (see, e.g.
\cite{DHPR2004}).  For instance, the temperature equation \eqref{eq:temperature-boussinesq-linear}
becomes

\begin{equation*}
  \rho C_p \left(\frac{\partial T}{\partial t} + \left(\mathbf u - \mathbf u_m \right) \cdot\nabla T\right)
  - \nabla\cdot k\nabla T
  =
  \rho H
   \quad
   \textrm{in $\Omega$}.
\end{equation*}

\subsubsection{Free surface stabilization}

Small disequilibria in the location of a free surface can cause instabilities in
the surface position and result in a ``sloshing'' instability.  This may be countered with a
quasi-implicit free surface integration scheme described in \cite{KMM2010}.
This scheme enters the governing equations as a small stabilizing surface
traction that prevents the free surface advection from overshooting its
true position at the next time step.  \aspect{} implements this stabilization,
the details of which may be found in \cite{KMM2010}.

An example of a simple model which uses a free surface may be found in Section \ref{sec:cookbooks-freesurface}.

\subsection{Calculations with melt transport}
\label{sec:melt_transport}

The original formulation of the equations in Section~\ref{sec:equations} describes the movement of solid mantle material. These computations also allow for taking into account how partially molten material changes the material properties and the energy balance through the release of latent heat. However, this will not consider melt extraction or any relative movement between melt and solid and there might be problems where the transport of melt is of interest. Thus, \aspect{} allows for solving additional equations describing the behavior of silicate melt percolating through and interacting with a viscously deforming host rock. This requires 
the advection of a compositional field representing the volume fraction of melt present at any given time (the porosity $\phi$), 
and also a change of the mechanical part of the system. The latter is implemented using the approach of \cite{KMK2013} and changes 
the Stokes system to

\begin{align}
  \label{eq:stokes-1-melt}
  -\nabla \cdot \left[2\eta \left(\varepsilon(\mathbf{u}_s)
                                  - \frac{1}{3}(\nabla \cdot \mathbf{u}_s)\mathbf 1\right)
                \right] + \nabla p_f + \nabla p_c  &=
  \rho \mathbf g
  & \qquad
  & \textrm{in $\Omega$},
  \\
  \label{eq:stokes-2-melt}
  \nabla \cdot \mathbf{u}_s - \nabla \cdot K_D \nabla p_f 
  - K_D \nabla p_f \cdot \frac{\nabla \rho_f}{\rho_f}
  &= 
  - \nabla \cdot K_D \rho_f \mathbf g
  \notag
  \\
  &\quad
  + \Gamma \left( \frac{1}{\rho_f} - \frac{1}{\rho_s} \right)
  \\
  &\quad
  - \frac{\phi }{\rho_f} \mathbf{u}_s \cdot \nabla\rho_f 
  - \frac{1 - \phi }{\rho_s} \mathbf{u}_s \cdot \nabla\rho_s
  \notag
  \\
  &\quad
  - K_D \mathbf g \cdot \nabla \rho_f 
  & \qquad
  & \textrm{in $\Omega$},
  \notag
  \\
  \label{eq:stokes-3-melt}
  \nabla \cdot \mathbf{u}_s + \frac{p_c}{\xi} 
  &=
  0.
\end{align}

We use the indices $s$ to indicate properties of the solid and $f$ for the properties of the fluid. 
The equations are solved for the solid velocity $\mathbf{u}_s$, the fluid pressure $p_f$, and an additional 
variable, the compaction pressure $p_c$, which is related to the fluid and solid pressure through the relation 
$p_c = (1-\phi) (p_s-p_f)$. $K_D$ is the Darcy coefficient, which is defined as the quotient of the permeability 
and the fluid viscosity and $\Gamma$ is the melting rate. $\eta$ and $\xi$ are the shear and compaction viscosities 
and can depend on the porosity, temperature, pressure, strain rate and composition. However, there are various 
laws for these quantities and so they are implemented in the material model. Common formulations for the dependence 
on porosity are $\eta = (1-\phi) \eta_0 e^{-\alpha_\phi \phi}$ with $\alpha_\phi \approx 25...30$ and 
$\xi = \eta_0 \phi^{-n}$ with $n \approx 1$.

To avoid the density gradients in Equation~\eqref{eq:stokes-2-melt}, which would have to be specified individually 
for each material model by the user, we can use the same method as for the mass conservation (described in Section~\ref{sec:boussinesq}) and assume the change in solid density is dominated by the change in static pressure, 
which can be written as
$\nabla p_s \approx \nabla p_\text{static} \approx \rho_s \textbf{g}$.
This finally allows us to write
\begin{equation*}
\frac{1}{\rho_s} \nabla \rho_s
\approx \frac{1}{\rho_s} \frac{\partial \rho_s}{\partial p_s} \nabla p_s
\approx \frac{1}{\rho_s} \frac{\partial \rho_s}{\partial p_s} \nabla p_s
\approx \frac{1}{\rho_s} \frac{\partial \rho_s}{\partial p_s} \rho_s \textbf{g}
\approx \kappa_s \rho_s \textbf{g}. 
\end{equation*}
For the fluid pressure, choosing a good approximation depends on the model parameters and setup (see \cite{DH2015}). 
Hence, we make $\nabla \rho_{f}$ a model input parameter, which can be adapted based on the forces that are expected 
to be dominant in the model. 
We can then replace the second equation by
\begin{align*}
\nabla \cdot \mathbf{u}_s - \nabla \cdot K_D \nabla p_f 
  - K_D \nabla p_f \cdot \frac{\nabla \rho_f}{\rho_f}
  &= 
  - \nabla \cdot (K_D\rho_f \mathbf g)
  \\
  &\quad
  + \Gamma \left( \frac{1}{\rho_f} - \frac{1}{\rho_s} \right)
  \notag
  \\
  &\quad
  - \frac{\phi }{\rho_f} \mathbf{u}_s \cdot \nabla\rho_f
  - (\mathbf{u}_s \cdot \mathbf g ) (1 - \phi) \kappa_s \rho_s
  \notag
  \\
  &\quad
  - K_D \mathbf g \cdot \nabla \rho_f .
  \notag
\end{align*}
%
The melt velocity is computed as
\[
 \mathbf{u}_f =  \mathbf{u}_s - \frac{K_D}{\phi} (\nabla p_f - \rho_f g),
\]
but is only used for postprocessing purposes and for computing the time step length.  

\note{Here, we do not use the visco-elasto-plastic rheology of the \cite{KMK2013} formulation. 
Hence, we do not consider the elastic deformation terms that would appear on the right hand side of Equation 
\eqref{eq:stokes-1-melt} and Equation~\eqref{eq:stokes-3-melt} and that include the elastic and compaction stress 
evolution parameters $\xi_\tau$ and $\xi_p$. Moreover, our viscosity parameters $\eta$ and $\xi$ only cover viscous 
deformation instead of combining visco-elasticity and plastic failure. This would require a modification of the rheologic 
law using effective shear and compaction viscosities $\eta_\text{eff}$ and $\xi_\text{eff}$ combining a failure criterion 
and shear and compaction visco-elasticities.}

Moreover, melt transport requires an advection equation for the porosity field $\phi$:
\begin{align}
  \label{eq:porosity}
  \rho_s \frac{\partial (1 - \phi)}{\partial t} + \nabla \cdot \left[ \rho_s (1 - \phi) \mathbf{u}_s \right]
  &=
  - \Gamma
  & \quad
  & \textrm{in $\Omega$},
  i=1\ldots C
\end{align}

In order to solve this equation in the same way as the other advection equations, we replace the second term of the equation by: 

\begin{equation*}
\nabla \cdot \left[ \rho_s (1 - \phi) \mathbf{u}_s \right]
= \left( 1-\phi \right) \left( \rho_s \nabla \cdot \mathbf{u}_s 
+ \nabla \rho_s \cdot \mathbf{u}_s \right)
- \nabla \phi \cdot \rho_s \mathbf{u}_s 
\end{equation*}
Then we use the same method as described above and assume again that the change in density is dominated by the change in static pressure
\begin{equation*}
\frac{1}{\rho_s} \nabla \rho_s \cdot \mathbf{u}_s 
\approx \kappa_s \rho_s \textbf{g} \cdot \mathbf{u}_s 
\end{equation*}
so we get
\begin{equation*}
\frac{\partial \phi}{\partial t} + \mathbf{u}_s \cdot \nabla \phi
= \frac{\Gamma}{\rho_s}
+ (1 - \phi) (\nabla \cdot \mathbf{u}_s + \kappa_s \rho_s \textbf{g} \cdot \mathbf{u}_s ).
\end{equation*}

More details on the implementation can be found in \cite{DH2015}. A benchmark case demonstrating the propagation of solitary waves can be found in Section~\ref{sec:benchmark-solitary_wave}.

\subsection{Nullspace removal}

The Stokes equation (\ref{eq:stokes-1}) only involves symmetric gradients of the velocity, and as such 
the velocity is determined only up to rigid-body motions (that is to say, translations and rotations).
For many simulations the boundary conditions will fully specify the velocity solution, but for some 
combinations of geometries and boundary conditions the solution will still be underdetermined.
In the language of linear algebra, the Stokes system may have a nullspace.

Usually the user will be able to determine beforehand whether their problem has a nullspace.  For instance, 
a model in a spherical shell geometry with free-slip boundary conditions at the top and bottom will 
have a rigid-body rotation in its nullspace (but not translations, as the boundary conditions do not 
allow flow through them).  That is to say, the solver may be able to come up with a solution to 
the Stokes operator, but that solution plus an arbitrary rotation is also an equally valid solution.

Another example is a model in a Cartesian box with periodic boundary conditions in the $x$-direction, 
and free slip boundaries on the top and bottom. This setup has arbitrary translations along the $x$-axis 
in its nullspace, so any solution plus an arbitrary $x$-translation is also a solution.

A solution with some small power in these nullspace modes should not affect the physics of the simulation. 
However, the timestepping of the model is based on evaluating the maximum velocities in the solution, 
and having unnecessary motions can severely shorten the time steps that \aspect{} takes. 
Furthermore, rigid body motions can make postprocessing calculations and visualization more 
difficult to interpret.  

\aspect{} allows the user to specify if their model has a nullspace. If so, any power in the nullspace 
is calculated and removed from the solution after every timestep.
There are two varieties of nullspace removal implemented: removing net linear/angular momentum, and 
removing net translations/rotations. 

For removing linear momentum we search for a constant velocity vector $\bf c$ such that 
\begin{equation*}
\int_\Omega \rho ({\bf u - c}) = 0
\end{equation*}

This may be solved by realizing that $\int_\Omega \rho {\bf u} = {\bf p}$, the linear momentum, and 
$\int_\Omega \rho = M$, the total mass of the model.  Then we find 
\begin{equation*}
{\bf c} = {\bf p}/M
\end{equation*}
which is subtracted off of the velocity solution.
 
Removing the angular momentum is similar, though a bit more complicated. 
We search for a rotation vector $\mathbf \omega$ such that 
\begin{equation*}
\int_\Omega \rho ( {\bf x \times (u - {\mathbf \omega} \times x) } ) = 0
\end{equation*}

Recognizing that $\int_\Omega \rho {\bf x \times u} = {\bf H}$, the angular momentum, 
and $\int_\Omega \rho {\bf x \times {\mathbf \omega} \times x} = {\bf I \cdot {\mathbf \omega} }$, 
the moment of inertia dotted into the sought-after vector, we can solve for ${\mathbf \omega}$: 
\begin{equation*}
{\mathbf \omega} = {\bf I^{-1} \cdot H}
\end{equation*}
A rotation about the rotation vector $\omega$ is then subtracted from the velocity solution.

Removing the net translations/rotations are identical to their momentum counterparts, but for those the 
density is dropped from the formulae. For most applications the density should not vary so wildly 
that there will be an appreciable difference between the two varieties, 
though removing linear/angular momentum is more physically motivated.

The user can flag the nullspace for removal by setting the \texttt{Remove nullspace} option,
as described in Section~\ref{parameters:Model_20settings}.
Figure~\ref{fig:rigid_rotation} shows the result of removing angular momentum from a convection 
model in a 2D annulus with free-slip velocity boundary conditions. 

\begin{figure}[tbp]
  \centering
  \includegraphics[width=0.8\textwidth]{rigid_rotation}
  \caption{\it Example of nullspace removal. 
On the left the nullspace (a rigid rotation) is removed, and the velocity vectors accurately 
show the mantle flow. On the right there is a significant clockwise rotation to the velocity 
solution which is making the more interesting flow features difficult to see. }
  \label{fig:rigid_rotation}
\end{figure}

\section{Installation}
\label{sec:installation}

This is a brief explanation of how to install all the required software and
\aspect{} itself.

\subsection{System prerequisites}

In order to install \aspect{} and its dependencies, you should
have your system set up to be able to compile and link programs. Additionally,
\aspect{} needs a number of widely used libraries that are available
for most operating systems. Therefore, you will need compilers for C, C++ and
Fortran, the GNU make system, the CMake build system, and the libraries and
header files of BLAS, LAPACK and zlib, which is used for compressing
the output data. To use more than one process for your computations
you will need to install a MPI library, its headers and the
necessary executables to run MPI programs. There are some optional packages
for additional features, like the HDF5 libraries for additional output formats,
PETSC for alternative solvers, and Numdiff for checking \aspect{}'s test
results with reasonable accuracy, but these are not strictly required, and in
some operating systems they are not available as packages but need to be
compiled from scratch.
Finally, for obtaining a recent development version of \aspect{} you will
need the git version control system.

An exemplary command to obtain all required packages on Ubuntu 14.04 would be:
\begin{verbatim}
sudo apt-get install build-essential \
                     cmake \
                     gcc \
                     g++ \
                     gfortran \
                     git \
                     libblas-dev \
                     liblapack-dev \
                     libopenmpi-dev \
                     numdiff \
                     openmpi-bin \
                     zlib1g-dev
\end{verbatim}

\subsection{Software prerequisites}

\aspect{} builds on a few other libraries that are widely used in the
computational science area and that provide most of the lower-level
functionality such as finite element descriptions or parallel linear
algebra. Specifically, it builds on \dealii{} which in turn uses Trilinos and
\pfrst{}. These need to be installed first before you can compile and run
\aspect{}. All of these libraries can readily be installed in a user's home
directory, without the need to modify the overall system directories.

The following steps should guide you through the installation of these
prerequisites:
\begin{enumerate}
\item \textit{Trilinos:} Trilinos can be downloaded from
  \href{http://trilinos.org/download/}{http://trilinos.org/download/}. At
  the current time we recommend Trilinos Version 11.4.x.%
  \footnote{Other versions of Trilinos like 10.6.x
  and 10.8.x have bugs that make these versions unusable for our purpose. The
  \dealii{} ReadMe file provides a list of versions that are known to work
  without bugs with \dealii{}.} For installation instructions see
  \href{https://www.dealii.org/developer/external-libs/trilinos.html}{the deal.II README file on installing Trilinos}. Note that you have
  to configure with MPI by using
\begin{verbatim}
 TPL_ENABLE_MPI:BOOL=ON
\end{verbatim}
  in the call to cmake. After that, run {\tt{make install}}.

\item \textit{\pfrst{}:} Download and install \pfrst{} as described in the
  \href{https://www.dealii.org/developer/external-libs/p4est.html}{deal.II
    p4est installation instructions}. This is done using the
  {\tt{p4est-setup.sh}}; do not use the \pfrst{} stand-alone installation
  instructions.

%\item  \textit{\dealii{}:}
%  The current version of \aspect{} requires features of \dealii{} that have
%  been developed after the 7.1 release. Since at the time of writing,
%  \dealii{} 7.2 has not yet appears, we currently require the development
%  version of \dealii{}, which can be obtained by running
%\begin{verbatim}
% svn checkout https://svn.dealii.org/trunk/deal.II
%\end{verbatim}
%  Once \dealii{} 7.2 is available, this will suffice as well.

\item \textit{\dealii{}:}
  The current version of \aspect{} requires \dealii{} version 8.4.0 or later.
  This version can be downloaded and installed from
  \url{https://www.dealii.org/download.html}.

\item \textit{Configuring and compiling \dealii:} Now it is time to configure
  \dealii.
  To this end, follow the \dealii{}
  \href{https://www.dealii.org/developer/readme.html}{installation
    instructions}. Note that \dealii{} recently made the switch to cmake,
    so the configuration changed. Make sure you enable MPI.
    A typical command line would look like this:
\begin{verbatim}
mkdir build
cd build
cmake -DDEAL_II_WITH_MPI=ON \
      -DCMAKE_INSTALL_PREFIX=/u/username/deal-installed/ \
      -DTRILINOS_DIR=/u/username/trilinos-11.4.1/ \
      -DP4EST_DIR=/u/username/p4est-0.3.4.1/ \
      ../deal.II
\end{verbatim}
  if the Trilinos and \pfrst{} packages have been installed in the
  subdirectory \texttt{/u/username/}.
  Make sure the configuration succeeds and detects the MPI compilers
  correctly. For more information see the documentation of \dealii.

  Now you are ready to compile \dealii{} by running {\tt{make install}}. If you
  have multiple processor cores, feel free to do {\tt{make install -jN}} where
  \texttt{N} is the number of processors in your machine to accelerate the
  process.

\item \textit{Testing your installation:} Test that your installation works
  by running the {\tt{step-32}} example that you can find in
  {\tt{\$DEAL\_II\_DIR/examples/step-32}}. Compile by running {\tt{make}} and run
  with {\tt{mpirun -n 2 ./step-32}}.

\item  You may now want to set the environment variable\footnote{For bash
    this would be adding the line {\tt{export DEAL\_II\_DIR=/path/to/deal-installed/}} to
    the file {\tt{\~{}/.bashrc}}. Then close the terminal and open it again to activate the change.}
{\tt{DEAL\_II\_DIR}} to the directory where you \textit{installed} \dealii.

\end{enumerate}


\subsection{Obtaining \aspect{} and initial configuration}

The development version of \aspect{} can be downloaded by executing the command
\begin{verbatim}
 git clone https://github.com/geodynamics/aspect.git
\end{verbatim}
If {\tt{\$DEAL\_II\_DIR}} points to your \dealii{} installation, you can configure
\aspect{} by running
\begin{verbatim}
 cmake .
\end{verbatim}
in the {\tt{aspect}} directory created by the {\tt{git clone}} command above.
If you did not set {\tt{\$DEAL\_II\_DIR}} you have to supply cmake with the location:
\begin{verbatim}
 cmake -DDEAL_II_DIR=/u/username/deal-installed/ .
\end{verbatim}

An alternative would be to configure \aspect{} as an out-of-source build. Similar to
the configuration of \dealii{}, you would need to create a build directory and
specify an install directory using -DCMAKE\_INSTALL\_PREFIX. The
instructions in the following sections assume an in-source build, so you need
to adapt the location of the \aspect{} binary.

\subsection{Compiling \aspect{} and generating documentation}
\label{sec:compiling}

After downloading \aspect{} and having built the libraries it builds on, you
can compile it by typing
\begin{verbatim}
  make
\end{verbatim}
on the command line (or \texttt{make -jN} if you have multiple processors in
your machine, where \texttt{N} is the number of processors). This builds the
\aspect{} executable which will reside in
the main directory and will be named \texttt{./aspect}. If you intend to
modify \aspect{} for your own experiments, you may want to also generate
documentation about the source code. This can be done using the command
\begin{verbatim}
  cd doc; make
\end{verbatim}
which assumes that you have the \texttt{doxygen} documentation generation tool
installed. Most Linux distributions have packages for \texttt{doxygen}. The
result will be the file \url{doc/doxygen/index.html} that is the starting
point for exploring the documentation.

\subsection{Compiling a static \aspect{} executable}
\aspect{} defaults to a dynamically linked executable, which saves disk space and build time. In some circumstances however, it is preferred to generate a statically linked executable that incorporates all used libraries. This need may arise on large clusters on which libraries and loaded modules/variables on login nodes may be different from the ones available on compute nodes. The general build procedure in such a case equals the above explained instructions with the following differences:
\begin{enumerate}
\item \textit{Trilinos}: Add the following lines to your cmake call:
\begin{verbatim}
 -DBUILD_SHARED_LIBS=OFF
 -DTPL_FIND_SHARED_LIBS=OFF
\end{verbatim}
\item \textit{\pfrst{}:} Change items "--enable-shared" to "--enable-static" in p4est-setup.sh lines 83 and 97.
\item \textit{\dealii{}:} Add the following lines to your call to cmake:
\begin{verbatim}
 -DDEAL_II_STATIC_EXECUTABLE=ON
\end{verbatim}
\item \textit{\aspect{}:} If everything above is set up correctly, there is no need for any configuration change to \aspect{}'s build procedure. You should see the following cmake output from \aspect{}:
\begin{verbatim}
-- Creating a statically linked executable
-- Disabling dynamic loading of plugins from the input file
\end{verbatim}
\end{enumerate}

The here mentioned build was tested on a Cray XC30 cluster, which was set up for default static compiling and linking. On machines that default to dynamic linking additional compiler and/or linker flags may be required (e.g. "-fPIC" / "--static"). In case of questions send an email to the mailing list.

\subsection{Installing and running \aspect{} on Mac OS X}
OS X has some eccentricities which can complicate installation of \aspect{}. Currently the easiest and most reliable way to run \aspect{} under Mac OS X Mavericks (10.9), Yosemite (10.10) and El Capitan (10.11) is to install and run the binary package for \dealii{}. The step-by-step process is described in detail, with screenshots, here: \url{https://wiki.geodynamics.org/_media/software:aspect:aspect_yosemite_20150529.pdf}

\begin{enumerate}
\item Install Xcode from the app store. You might need to install the command line tools. Open a terminal and make sure ``clang'' does not report ``command not found''. You might need to run
\begin{verbatim}
  xcode-select --install
\end{verbatim}
to install the tools.

\item Install Cmake.

CMake is a cross-platform, open-source build system that can be downloaded from \url{http://www.cmake.org}.  After installation of CMake.app, the terminal command for cmake will be
\begin{verbatim}
  /Applications/CMake.app/Contents/bin/cmake
\end{verbatim}

\item Download and install the parallel \dealii{}. This is the binary package for Mac OS .dmg file.
\begin{verbatim}
  open  https://github.com/dealii/dealii/releases/download/v8.4.1/dealii-8.4.1.dmg
\end{verbatim}

Open the downloaded disk image, and drag \dealii{}.app into the Applications folder.
 To start the \dealii{} app, double click the icon in the Applications folder or use the open command:
\begin{verbatim}
  open /Applications/deal.II.app
\end{verbatim}
\dealii{} app opens a terminal window and displays a \dealii{} message.   \dealii{}.app will install all required libraries for \aspect{} (p4est, parMeTiS, and Trilinos) and will include the environment variables needed to use these libraries.

To ensure the correct compilers are picked up when configuring \aspect{}, add the following lines to your \texttt{\textasciitilde/.profile} or \texttt{\textasciitilde/.bash\_profile}:

\begin{verbatim}
  export OMPI_CXX=clang++
  export OMPI_CC=clang
\end{verbatim}

\item Download  the  \aspect{} source from the git repository.

\begin{verbatim}
  cd $HOME/src
  git clone https://github.com/geodynamics/aspect.git
\end{verbatim}

If you wish to use Xcode as your IDE it is recommended to skip to step 7 below. Otherwise, proceed with steps 5 and 6 below.

\item Build \aspect{}

Note: if you are not using Xcode (see step 7 below), you {\bf MUST} \underline{build} and \underline{run} \aspect{} in the terminal window started by \dealii{}, rather than in the Terminal app.

\begin{enumerate}
\item Go to the  directory where you wish to install \aspect{}  and run the following commands:

{\footnotesize
\begin{verbatim}
  cmake .
\end{verbatim}}

This should display something like:

{\footnotesize
\begin{verbatim}
  Project aspect set up with  deal.II-8.4.1  found at /Applications/deal.II.app/Contents/Resources
\end{verbatim}}

\item Run make:

{\footnotesize
\begin{verbatim}
  bash-3.2$ make
  Scanning dependencies of target aspect
  [  0%] Building CXX object CMakeFiles/aspect.dir/source/adiabatic_conditions/initial_profile.cc.o
  [  0%] Building CXX object CMakeFiles/aspect.dir/source/adiabatic_conditions/interface.cc.o
  ...
  Linking CXX executable aspect
  [100%] Built target aspect
  bash-3.2$ ls -l aspect
  -rwxr-xr-x  1 <name>  staff  19131292 May  7 15:02 aspect
\end{verbatim}
}
There may be  warnings from the compiler, but if the \aspect{} target is created then it was successful.

\item By default,  \aspect{}  compiles the debug version of the code.
To compile the optimized version:
{\footnotesize\begin{verbatim}
  make release 
\end{verbatim}}

\item Run make test

{\footnotesize
\begin{verbatim}
  make test
\end{verbatim}}


\end {enumerate}



\item Run \aspect{}.

A reminder: you {\bf must} run  \aspect{}  on the terminal window which is opened by \dealii{}.app.

To start \aspect{} using MPI for parallelization, from the directory where you installed \aspect{}:
\begin{verbatim}
  mpirun -np <# of processes> ./aspect <parameter file>
\end{verbatim}

To check quickly whether you are running  \aspect{} on the  \dealii{}.app terminal, check the location of the \verb|mpirun| command:
\begin{verbatim}
  bash-3.2$ which mpirun
  /Applications/deal.II.app/Contents/Resources/opt/openmpi-1.6.5/bin/mpirun
\end{verbatim}

\item Build \aspect{} as an Xcode project.

After completing step 4 above, do the following:
\begin{enumerate}
\item Go to the directory where you wish to install \aspect{} and run the following command:

\begin{verbatim}
  cmake -G Xcode .
\end{verbatim}

Now open the resulting \texttt{aspect.xcodeproj} in Xcode.

\item Open Product\textgreater Scheme\textgreater Manage Schemes, then select ``aspect'' and click ``Edit''. Go to Run\textgreater Arguments, and under "Arguments Passed On Launch" add the filepath to the desired parameter file. Click "Close" then select Product\textgreater Run to run \aspect{} with this parameter file.

\item If you want to run in parallel in Xcode, the easiest way is to create a new scheme. Open Product\textgreater Scheme\textgreater Manage Schemes, select ``aspect'' then select the gear icon, then ``Duplicate scheme''. Rename the new scheme to ``aspect parallel''. Go to Run\textgreater Info, and in Executable choose ``Other...''. Navigate to 
\begin{verbatim}
  /Applications/deal.II.app/Contents/Resources/opt/openmpi-1.10.2/bin
\end{verbatim}
and select ``orterun'', then click ``Choose''.

\item Now go to the Arguments tab, and add 3 arguments: first, add \texttt{-np4} (for 4 processors -- you may use however many your machine has). Then add a path to your aspect executable -- this should be within your aspect folder, or the subfolders Debug or Release. Finally, ensure you have a path to your parameter file as above. Ensure these are in the correct order, or reorder them if necessary. Click ``Close'' then select Product\textgreater Run to run \aspect{} in parallel with this parameter file.
\end{enumerate}



\end{enumerate}


%%%%%%%%%%%%%%%%%%%%%%

\section{Running \aspect}
\label{sec:running}

\subsection{Overview}
\label{sec:running-overview}

After compiling \aspect{} as described above, you should have an executable
file in the main directory. It can be called as follows:
\begin{verbatim}
  ./aspect parameter-file.prm
\end{verbatim}
or, if you want to run the program in parallel, using something like
\begin{verbatim}
  mpirun -np 32 ./aspect parameter-file.prm
\end{verbatim}
to run with 32 processors. In either case, the argument denotes the (path and)
name of a file that contains input parameters.%
\footnote{As a special case, if you call \aspect{} with an argument that
consists of two dashes, ``\texttt{--}'', then the arguments will be read from
the standard input stream of the program. In other words, you could type the
input parameters into your shell window in this case (though that would be
cumbersome, \aspect{} would seem to hang until you finish typing all of your
input into the window and then terminating the input stream by typing
\texttt{Ctrl-D}). A more common case would be to use Unix pipes so that the
default 
input of \aspect{} is the output of another program, as in a command like
\texttt{cat parameter-file.prm.in | mypreprocessor | ./aspect --}, where
\texttt{mypreprocessor} would be a program of your choice that somehow
transforms the file \texttt{parameter-file.prm.in} into a valid input file,
for example to systematically vary one of the input parameters.

If you want to run \aspect{} in parallel, you can do something like
\texttt{cat parameter-file.prm.in | mypreprocessor | mpirun -np 4 ./aspect
  --}. In cases like this, \texttt{mpirun} only forwards the output of
\texttt{mypreprocessor} to the first of the four MPI processes, which then
sends the text to all other processors.}
When you download \aspect{}, there are a number of sample input files in the
\texttt{cookbooks} directory, corresponding to the examples discussed in
Section~\ref{sec:cookbooks}, and input files for some of the benchmarks discussed
in Section~\ref{sec:cookbooks-benchmarks} are located in the \texttt{benchmarks}
directory. A full description of all parameters one can specify in these files
is given in Section~\ref{sec:parameters}.

Running \aspect{} with an input file will produce output that will look
something like this (numbers will all be different, of course):
\begin{lstlisting}[frame=single,language=ksh]
Number of active cells: 1,536 (on 5 levels)
Number of degrees of freedom: 20,756 (12,738+1,649+6,369)

*** Timestep 0:  t=0 years

   Rebuilding Stokes preconditioner...
   Solving Stokes system... 30+3 iterations.
   Solving temperature system... 8 iterations.

Number of active cells: 2,379 (on 6 levels)
Number of degrees of freedom: 33,859 (20,786+2,680+10,393)

*** Timestep 0:  t=0 years

   Rebuilding Stokes preconditioner...
   Solving Stokes system... 30+4 iterations.
   Solving temperature system... 8 iterations.

   Postprocessing:
     Writing graphical output: output/solution-00000
     RMS, max velocity:        0.0946 cm/year, 0.183 cm/year
     Temperature min/avg/max:  300 K, 3007 K, 6300 K
     Inner/outer heat fluxes:  1.076e+05 W, 1.967e+05 W

*** Timestep 1:  t=1.99135e+07 years

   Solving Stokes system... 30+3 iterations.
   Solving temperature system... 8 iterations.

   Postprocessing:
     Writing graphical output: output/solution-00001
     RMS, max velocity:        0.104 cm/year, 0.217 cm/year
     Temperature min/avg/max:  300 K, 3008 K, 6300 K
     Inner/outer heat fluxes:  1.079e+05 W, 1.988e+05 W

*** Timestep 2:  t=3.98271e+07 years

   Solving Stokes system... 30+3 iterations.
   Solving temperature system... 8 iterations.

   Postprocessing:
     RMS, max velocity:       0.111 cm/year, 0.231 cm/year
     Temperature min/avg/max: 300 K, 3008 K, 6300 K
     Inner/outer heat fluxes: 1.083e+05 W, 2.01e+05 W

*** Timestep 3:  t=5.97406e+07 years

...
\end{lstlisting}

This output was produced by a parameter file that, among other settings,
contained the following values (we will discuss many such input files in
Section~\ref{sec:cookbooks}:
\lstinputlisting[language=prmfile]{cookbooks/overview/simple.prm.out}

In other words, these run-time parameters specify that we should start with a
geometry that represents a spherical shell (see
Sections~\ref{parameters:Geometry_20model} and
\ref{parameters:Geometry_20model/Spherical_20shell} for details). The coarsest
mesh is refined 4 times globally, i.e., every cell is refined into four
children (or eight, in 3d) 4 times. This yields the initial number of 1,536
cells on a mesh hierarchy that is 5 levels deep. We then solve the problem
there once and, based on the number of adaptive refinement steps at the
initial time set in the parameter file, use the solution so computed to refine
the mesh once adaptively (yielding 2,379 cells on 6 levels) on which we start
the computation over at time $t=0$.

Within each time step, the output indicates the number of iterations performed
by the linear solvers, and we generate a number of lines of output by the
postprocessors that were selected (see
\index[prmindex]{List of postprocessors}
\index[prmindexfull]{Postprocess!List of postprocessors}
Section~\ref{parameters:Postprocess}). Here, we have selected to run all
postprocessors that are currently implemented in \aspect{} which includes the
ones that evaluate properties of the velocity, temperature, and heat flux as
well as a postprocessor that generates graphical output for visualization.

While the screen output is useful to monitor the progress of a simulation,
its lack of a structured output makes it not useful for later plotting things
like the evolution of heat flux through the core-mantle boundary. To this end,
\aspect{} creates additional files in the output directory selected in the
input parameter file
\index[prmindex]{Output directory}
\index[prmindexfull]{Output directory}
(here, the \texttt{output/} directory relative to the
directory in which \aspect{} runs). In a simple case, this will look as
follows:
\begin{lstlisting}[frame=single,language=ksh]
aspect> ls -l output/
total 780
-rw------- 1 b   9863 Dec  1 15:13 parameters.prm
-rw------- 1 b 306562 Dec  1 15:13 solution-00000.0000.vtu
-rw------- 1 b  97057 Nov 30 05:58 solution-00000.0001.vtu
...
-rw------- 1 b   1061 Dec  1 15:13 solution-00000.pvtu
-rw------- 1 b     35 Dec  1 15:13 solution-00000.visit
-rw------- 1 b 306530 Dec  1 15:13 solution-00001.0000.vtu
-rw------- 1 b   1061 Dec  1 15:13 solution-00001.pvtu
-rw------- 1 b     35 Dec  1 15:13 solution-00001.visit
...
-rw-r--r-- 1 b    997 Dec  1 15:13 solution.pvd
-rw-r--r-- 1 b    997 Dec  1 15:13 solution.visit
-rw------- 1 b    924 Dec  1 15:13 statistics
\end{lstlisting}
The purpose of these files is as follows:
\begin{itemize}
\item \textit{A listing of all run-time parameters:} The
  \texttt{output/parameters.prm} file contains a complete listing of all
  run-time parameters. In particular, this includes the one that have been
  specified in the input parameter file passed on the command line, but it
  also includes those parameters for which defaults have been used. It is
  often useful to save this file together with simulation data to allow for
  the easy reproduction of computations later on.

\item \textit{Graphical output files:} One of the postprocessors you select
  when you say ``all'' in the parameter files is the one that generates output
  files that represent the solution at certain time steps. The screen output
  indicates that it has run at time step 0, producing output files of the form
  \texttt{output/solution-00000}. At the current time, the default is that \aspect{} generates
  this output in VTK format%
  \footnote{The output is in fact in the VTU version of the VTK file
    format. This is the XML-based version of this file format in which
    contents are compressed. Given that typical file sizes for 3d simulation
    are substantial, the compression saves a significant amount of disk
    space.}  as that is widely used by a number of excellent visualization
  packages and also supports parallel visualization.%
  \footnote{The underlying \dealii{} package actually supports output in
    around a dozen different formats, but most of them are not very useful for
    large-scale, 3d, parallel simulations. If you need a different format than
    VTK, you can select this using the run-time parameters discussed in
    Section~\ref{parameters:Postprocess/Visualization}.}  If
  the program has been run with multiple MPI processes, then the list of
  output files will look as shown above, with the base \texttt{solution-x.y}
  denoting that this the \texttt{x}th time we create output files and that the
  file was generated by the \texttt{y}th processor.

  VTK files can be visualized by many of the large visualization packages. In
  particular, the
  \href{https://visit.llnl.gov}{Visit} and
  \href{http://www.paraview.org/}{ParaView} programs, both
  widely used, can read the files so created. However, while VTK has become a
  de-facto standard for data visualization in scientific computing, there
  doesn't appear to be an agreed upon way to describe which files jointly make
  up for the simulation data of a single time step (i.e., all files with the
  same \texttt{x} but different \texttt{y} in the example above). Visit and
  Paraview both have their method of doing things, through \texttt{.pvtu} and
  \texttt{.visit} files. To make it easy for you to view data, \aspect{}
  simply creates both kinds of files in each time step in which graphical data
  is produced.

  The final two files of this kind, \texttt{solution.pvd} and
  \texttt{solution.visit}, are files that
  describes to Paraview and Visit, respectively, which
  \texttt{solution-xxxx.pvtu} and \texttt{solution-xxxx.yyyy.vtu} jointly form
  a complete simulation. In the former case, the file lists the \texttt{.pvtu}
  files of all
  timesteps together with the simulation time to which they correspond. In the
  latter case, it actually lists all \texttt{.vtu} that belong to one
  simulation, grouped by the timestep they correspond to.
  To visualize an entire simulation, not just a single time step, it is
  therefore simplest to just load one of these files, depending on whether you
  use Paraview or Visit.%
  \footnote{At the time of writing this, current versions of Visit (starting
    with version 2.5.1) actually have a bug that prevents them from
    successfully reading the \texttt{solution.visit} or
    \texttt{solution-xxxx.visit} files -- Visit believes that each of these
    files corresponds to an individual time step, rather than that a whole
    group of files together form one time step. This bug is not fixed in Visit
    2.6.3, but may be fixed in later versions.}

  For more on visualization, see also Section~\ref{sec:viz}.

\item \textit{A statistics file:} The \texttt{output/statistics} file contains
  statistics collected during each time step, both from within the simulator
  (e.g., the current time for a time step, the time step length, etc.) as well
  as from the postprocessors that run at the end of each time step. The file
  is essentially a table that allows for the simple production of time
  trends. In the example above, it looks like this:
  \begin{lstlisting}[frame=single,language=ksh]
# 1: Time step number
# 2: Time (years)
# 3: Iterations for Stokes solver
# 4: Time step size (year)
# 5: Iterations for temperature solver
# 6: Visualization file name
# 7: RMS velocity (m/year)
# 8: Max. velocity (m/year)
# 9: Minimal temperature (K)
# 10: Average temperature (K)
# 11: Maximal temperature (K)
# 12: Average nondimensional temperature (K)
# 13: Core-mantle heat flux (W)
# 14: Surface heat flux (W)
0 0.0000e+00 33 2.9543e+07 8                    "" 0.0000 0.0000   0.0000    0.0000 ...
0 0.0000e+00 34 1.9914e+07 8 output/solution-00000 0.0946 0.1829 300.0000 3007.2519 ...
1 1.9914e+07 33 1.9914e+07 8 output/solution-00001 0.1040 0.2172 300.0000 3007.8406 ...
2 3.9827e+07 33 1.9914e+07 8                    "" 0.1114 0.2306 300.0000 3008.3939 ...
  \end{lstlisting}
  The actual columns you have in your statistics file may differ from the ones above,
  but the format of this file should be obvious. Since the hash mark is a comment
  marker in many programs (for example, \texttt{gnuplot} ignores lines in text
  files that start with a hash mark), it is simple to plot these columns as time
  series. Alternatively, the data can be imported into a spreadsheet and
  plotted there.
\note{As noted in Section~\ref{sec:non-dimensional}, \aspect{} can be
  thought of as using the meter-kilogram-second (MKS, or SI) system. Unless otherwise noted,
  the quantities in the output file are therefore also in MKS units.}

  A simple way to plot the contents of this file is shown in Section~\ref{sec:viz-stat}.

\item \textit{Output files generated by other postprocessors:} Similar to the
  \texttt{output/statistics} file, several of the existing
  postprocessors one can select from the parameter file generate their
  data in their own files in the output directory. For example, \aspect{} can
  write depth-average statistics into \texttt{output/depth\_average.gnuplot}.
  This is done by the ``depth average'' postprocessor and the user can control
  how often this file is updated, as well as what graphical file format to use
  (if anything other than \texttt{gnuplot} is desired).

  By default, the data is written in text format that can be easily displayed
  by e.g. gnuplot. For an example, see Figure~\ref{fig:depthaverage}. The plot
  shows how an initially linear temperature profile forms upper and lower
  boundary layers.

\begin{figure}[tbp]
  \centering
  \includegraphics[width=0.6\textwidth]{depthaverage2}
  \caption{\it Example output for depth average statistics. On the left axis are 13 time
  steps, on the right is the depth (from the top at 0 to the bottom of the mantle on the
  far right), and the upwards pointing axis is the average temperature. This
  plot is generated by gnuplot, but the depth averages can be written in many
  other output formats as well, if preferred (see
  Section~\ref{parameters:Postprocess/Depth_20average}).}
  \label{fig:depthaverage}
\end{figure}

\end{itemize}



\subsection{Selecting between 2d and 3d runs}
\label{sec:2d-vs-3d}

\aspect{} can solve both two- and three-dimensional problems. You
select which one you want by putting a line like the following into
\index[prmindex]{Dimension}
\index[prmindexfull]{Dimension}
the parameter file (see Section~\ref{sec:parameters}):
\lstinputlisting[language=prmfile]{cookbooks/overview/dim.part.prm.out}

Internally, dealing with the dimension builds on a feature in
\dealii{}, upon which \aspect{} is based, that is called
\textit{dimension-independent programming}. In essence, what this does is that
you write your code only once in a way so that the space dimension is a
variable (or, in fact, a template parameter) and you can compile the code for
either 2d or 3d. The advantage is that codes can be tested and debugged in 2d
where simulations are relatively cheap, and the same code can then be
re-compiled and executed in 3d where simulations would otherwise be
prohibitively expensive for finding bugs; it is also a useful feature when
scoping out whether certain parameter settings will have the desired effect by
testing them in 2d first, before running them in 3d. This feature is discussed
in detail in the
\href{https://www.dealii.org/developer/doxygen/deal.II/step_4.html}{\dealii{}
  tutorial program step-4}.
Like there, all the functions and classes in
\aspect{} are compiled for both 2d and 3d. Which dimension is actually
called internally depends on what you have set in the input file, but
in either case, the machine code generated for 2d and 3d results from
the same source code and should, thus, contain the same set of
features and bugs. Running in 2d and 3d should therefore yield
comparable results. Be prepared to wait much longer for
computations to finish in the latter case, however.


\subsection{Debug or optimized mode}
\label{sec:debug-mode}

\aspect{} utilizes a \dealii{} feature called \textit{debug
  mode}. By default, \aspect{} uses debug mode, i.e., it calls a version of
the \dealii{} library that contain lots of checks for the correctness of
function arguments, the consistency of the internal state of data structure,
etc. If you program with \dealii{}, for example to extend \aspect{}, it has
been our experience over the years that, by number, most programming errors are of the
kind where one forgets to initialize a vector, one accesses data that has not
been updated, one tries to write into a vector that has ghost elements,
etc. If not caught, the result of these bugs is that parts of the program use
invalid data (data written into ghost elements is not communicated to other
processors), that operations simply make no sense (adding vectors of different
length), that memory is corrupted (writing past the end of an array) or, in
rare and fortunate cases, that the program simply crashes.

Debug mode is designed to catch most of these errors: It enables some 7,300
assertions (as of late 2011) in \dealii{} where we check for errors like the
above and, if the condition is violated, abort the program with a detailed
message that shows the failed check, the location in the source code, and a
stacktrace how the program got there. The downside of debug mode is, of
course, that it makes the program much slower -- depending on application by a
factor of 4--10. An example of the speedup one can get is shown in
Section~\ref{sec:cookbooks-simple-box}.

\aspect{} by default uses debug mode because most users will want to play with
the source code, and because it is also a way to verify that the compilation
process worked correctly. If you have verified that the program runs correctly
with your input parameters, for example by letting it run for the first 10
time steps, then you can switch to optimized mode by compiling \aspect{}
with the command\footnote{Note that this procedure also changed with the switch to cmake.}
\begin{verbatim}
 make release
\end{verbatim}
and then compile using
\begin{verbatim}
 make
\end{verbatim}
To switch back to debug mode type:
\begin{verbatim}
 make debug
\end{verbatim}

\note{It goes without saying that if you make significant modifications to the
  program, you should do the first runs in debug mode to verify that your
  program still works as expected.}


\subsection{Visualizing results}
\label{sec:viz}

Among the postprocessors that can be selected in the input parameter file (see
Sections~\ref{sec:running-overview} and
\ref{parameters:Postprocess/Visualization}) are some that can produce files in
a format that can later be used to generate a graphical visualization of the
solution variables $\mathbf u, p$ and $T$ at select time steps, or of
quantities derived from these variables (for the latter, see
Section~\ref{sec:viz-postpostprocessors}).

By default, the files that are generated are in VTU format, i.e., the
XML-based, compressed format defined by the VTK library, see
\url{http://public.kitware.com/VTK/}. This file format has become a broadly
accepted pseudo-standard that many visualization program support, including
two of the visualization programs used most widely in computational science:
Visit (see \url{https://visit.llnl.gov/}) and ParaView (see
\url{http://www.paraview.org/}). The VTU format has a number of
advantages beyond being widely distributed:
\begin{itemize}
\item It allows for compression, keeping files relatively small even for
  sizeable computations.
\item It is a structured XML format, allowing other programs to read it
  without too much trouble.
\item It has a degree of support for parallel computations where every
  processor would only write that part of the data to a file that this
  processor in fact owns, avoiding the need to communicate all data to a
  single processor that then generates a single file. This requires a master
  file for each time step that then contains a reference to the individual
  files that together make up the output of a single time step. Unfortunately,
  there doesn't appear to be a standard for these master records; however,
  both ParaView and Visit have defined a format that each of these programs
  understand and that requires placing a file with ending \texttt{.pvtu} or
  \texttt{.visit} into the same directory as the output files from each
  processor. Section~\ref{sec:running-overview} gives an example of what can
  be found in the output directory.
\end{itemize}

\note{You can select other formats for output than VTU, see the run-time
  parameters in Section~\ref{parameters:Postprocess/Visualization}. However,
  none of the numerous formats currently implemented in \dealii{} other than
  the VTK/VTU formats allows for splitting up data over multiple files in case
  of parallel computations, thus making subsequent visualization of the entire
  volume impossible. Furthermore, given the amount of data \aspect{} can
  produce, the compression that is part of the VTU format is an important part
  of keeping data manageable.
\index[prmindex]{Output format}
\index[prmindexfull]{Postprocess!Visualization!Output format}
}

\subsubsection{Visualization the graphical output using \textit{Visit}}
In the following, let us discuss the process of visualizing a 2d computation
using Visit. The steps necessary for other visualization programs will
obviously differ but are, in principle, similar.

To this end, let us consider a simulation of convection in a box-shaped, 2d
region (see the ``cookbooks'' section, Section~\ref{sec:cookbooks}, and in
particular Section~\ref{sec:cookbooks-simple-box} for
the input file for this particular model). We can run the program with 4 processors using
\begin{verbatim}
  mpirun -np 4 ./aspect cookbooks/convection-box.prm
\end{verbatim}
Letting the program run for a while will result in several output files as
discussed in Section~\ref{sec:running-overview} above.

In order to visualize one time step, follow these steps:%
\footnote{The instructions and screenshots were generated with Visit
  2.1. Later versions of Visit differ slightly in the arrangement of
  components of the graphical user interface, but the workflow and general
  idea remains unchanged.}

\begin{figure}[tbp]
  \phantom{.}
  \hfill
  \subfigure[]{
    \includegraphics[width=0.24\textwidth]{viz/visit/visit-1}
    \label{fig:visit-1:a}
  }
  \hfill
  \subfigure[]{
    \includegraphics[width=0.24\textwidth]{viz/visit/visit-2}
    \label{fig:visit-1:b}
  }
  \hfill
  \subfigure[]{
    \includegraphics[width=0.24\textwidth]{viz/visit/visit-3}
    \label{fig:visit-1:c}
  }
  \hfill
  \phantom{.}
  \caption{\it Main window of Visit, illustrating the different steps of
    adding content to a visualization.}
  \label{fig:visit-1}
\end{figure}

\begin{figure}[tbp]
  \phantom{.}
  \hfill
  \subfigure[]{
    \includegraphics[width=0.48\textwidth]{viz/visit/visit-4}
    \label{fig:visit-2:a}
  }
  \hfill
  \subfigure[]{
    \includegraphics[width=0.48\textwidth]{viz/visit/visit-5}
    \label{fig:visit-2:b}
  }
  \hfill
  \phantom{.}
  \caption{\it Display window of Visit, showing a single plot and one where
    different data is overlaid.}
  \label{fig:visit-2}
\end{figure}

\begin{itemize}
\item \textit{Selecting input files:} As mentioned above, in parallel
  computations we usually generate one output file per processor in each time
  step for which visualization data is produced (see, however,
  Section~\ref{sec:viz-data}). To tell Visit which files together make up one
  time step, \aspect{} creates a \texttt{solution-NNNNN.visit} file in the
  output directory. To open it, start Visit, click on the ``Open'' button in
  the ``Sources'' area of
  its main window (see Fig.~\ref{fig:visit-1:a}) and select the file you
  want. Alternatively, you can also select files using the ``File $>$ Open''
  menu item, or hit the corresponding keyboard short-cut. After adding an
  input source, the ``Sources'' area of the main window should list the
  selected file name.

\item \textit{Selecting what to plot:} \aspect{} outputs all sorts of
  quantities that characterize the solution, such as temperature, pressure,
  velocity, and many others on demand (see
  Section~\ref{parameters:Postprocess/Visualization}). Once an input file has
  been opened, you will want to add graphical representations of some of this
  data to the still empty canvas. To this end, click on the ``Add'' button of
  the ``Plots'' area. The resulting menu provides a number of different kinds
  of plots. The most important for our purpose are: (i) ``Pseudocolor'' allows
  the visualization of a scalar field (e.g., temperature, pressure, density)
  by using a color field. (ii) ``Vector'' displays a vector-valued field
  (e.g., velocity) using arrows. (iii) ``Mesh'' displays the mesh. The
  ``Contour'', ``Streamline'' and ``Volume'' options are also frequently
  useful, in particular in 3d.

  Let us choose the ``Pseudocolor'' item and select the temperature field as
  the quantity to plot. Your main window should now look as shown in
  Fig.~\ref{fig:visit-1:b}. Then hit the ``Draw'' button to make Visit generate
  data for the selected plots. This will yield a picture such as shown in
  Fig.~\ref{fig:visit-2:a} in the display window of Visit.

\item \textit{Overlaying data:} Visit can overlay multiple plots in the same
  view. To this end, add another plot to the view using again the ``Add''
  button to obtain the menu of possible plots, then the ``Draw'' button to
  actually draw things. For example, if we add velocity vectors and the mesh,
  the main window looks as in Fig.~\ref{fig:visit-1:c} and the main view as in
  Fig.~\ref{fig:visit-2:b}.

\item \textit{Adjusting how data is displayed:} Without going into too much
  detail, if you double click onto the name of a plot in the ``Plots'' window,
  you get a dialog in which many of the properties of this plot can be
  adjusted. Further details can be changed by using ``Operators'' on a plot.

\item \textit{Making the output prettier:} As can be seen in
  Fig.~\ref{fig:visit-2}, Visit by default puts a lot of clutter around the
  figure -- the name of the user, the name of the input file, color bars, axes
  labels and ticks, etc. This may be useful to explore data in the beginning
  but does not yield good pictures for presentations or publications. To
  reduce the amount of information displayed, go to the ``Controls $>$
  Annotations'' menu item to get a dialog in which all of these displays can
  be selectively switched on and off.

\item \textit{Saving figures:} To save a visualization into a file that can
  then be included into presentations and publications, go to the menu item
  ``File $>$ Save window''. This will create successively numbered files in
  the directory from which Visit was started each time a view is saved. Things
  like the format used for these files can be chosen using the ``File $>$ Set
  save options'' menu item. We have found that one can often get better
  looking pictures by selecting the ``Screenshot'' method in this dialog.
\end{itemize}

More information on all of these topics can be found in the Visit
documentation, see \url{https://visit.llnl.gov/}. We have also recorded
video lectures demonstrating this process interactively at
\url{http://www.youtube.com/watch?v=3ChnUxqtt08} for Visit, and at
\url{http://www.youtube.com/watch?v=w-65jufR-bc} for Paraview.


\subsubsection{Visualizing statistical data}
\label{sec:viz-stat}

In addition to the graphical output discussed above, \aspect{} produces a
statistics file that collects information produced during each time step.
For the remainder of this section, let us assume that we have run \aspect{}
with the input file discussed in Section~\ref{sec:cookbooks-simple-box},
simulating convection in a box. After running \aspect{}, you will find
a file called \texttt{statistics} in the output directory that, at the time
of writing this, looked like this:
This file has a structure that looks (at the time of writing this section)
like this:
\begin{lstlisting}[frame=single,language=ksh]
# 1: Time step number
# 2: Time (seconds)
# 3: Number of mesh cells
# 4: Number of Stokes degrees of freedom
# 5: Number of temperature degrees of freedom
# 6: Iterations for temperature solver
# 7: Iterations for Stokes solver
# 8: Velocity iterations in Stokes preconditioner
# 9: Schur complement iterations in Stokes preconditioner
# 10: Time step size (seconds)
# 11: RMS velocity (m/s)
# 12: Max. velocity (m/s)
# 13: Minimal temperature (K)
# 14: Average temperature (K)
# 15: Maximal temperature (K)
# 16: Average nondimensional temperature (K)
# 17: Outward heat flux through boundary with indicator 0 ("left") (W)
# 18: Outward heat flux through boundary with indicator 1 ("right") (W)
# 19: Outward heat flux through boundary with indicator 2 ("bottom") (W)
# 20: Outward heat flux through boundary with indicator 3 ("top") (W)
# 21: Visualization file name
 0 0.0000e+00 256 2467 1089  0 29 30 29 1.2268e-02 1.79026783e+00 2.54322608e+00
 1 1.2268e-02 256 2467 1089 32 29 30 30 3.7388e-03 5.89844152e+00 8.35160076e+00
 2 1.6007e-02 256 2467 1089 20 28 29 29 2.0239e-03 1.09071922e+01 1.54298908e+01
 3 1.8031e-02 256 2467 1089 15 27 28 28 1.3644e-03 1.61759153e+01 2.28931189e+01
 4 1.9395e-02 256 2467 1089 13 26 27 27 1.0284e-03 2.14465789e+01 3.03731397e+01
 5 2.0424e-02 256 2467 1089 11 25 26 26 8.2812e-04 2.66110761e+01 3.77180480e+01
 \end{lstlisting}

In other words, it first lists what the individual columns mean with a hash
mark at the beginning of the line and then has one line for each time step
in which the individual columns list what has been explained above.%
\footnote{With
  input files that ask for initial adaptive refinement, the first time step may
  appear twice because we solve on a mesh
  that is globally refined and we then start the entire computation
  over again on a once adaptively refined mesh (see the parameters in
  Section~\ref{parameters:Mesh_20refinement} for how to do that).}

This file is easy to visualize. For example, one can import it as a whitespace
separated file into a spreadsheet such as Microsoft Excel or OpenOffice/LibreOffice
Calc and then generate graphs of one column against another. Or, maybe simpler,
there is a multitude of simple graphing programs that do not need the overhead
of a full fledged spreadsheet engine and simply plot graphs. One that is
particularly simple to use and available on every major platform is \texttt{Gnuplot}.
It is extensively documented at \url{http://www.gnuplot.info/}.

\texttt{Gnuplot} is a command line program in which you enter commands that
plot data or modify the way data is plotted. When you call it, you will first
get a screen that looks like this:
\begin{lstlisting}[frame=single]
/home/user/aspect/output gnuplot

        G N U P L O T
        Version 4.6 patchlevel 0    last modified 2012-03-04
        Build System: Linux x86_64

        Copyright (C) 1986-1993, 1998, 2004, 2007-2012
        Thomas Williams, Colin Kelley and many others

        gnuplot home:     http://www.gnuplot.info
        faq, bugs, etc:   type "help FAQ"
        immediate help:   type "help"  (plot window: hit 'h')

Terminal type set to 'qt'
gnuplot>
\end{lstlisting}
At the prompt on the last line, you can then enter commands. Given the
description of the individual columns given above, let us first try to
plot the heat flux through boundary 2 (the bottom
boundary of the box), i.e., column 19, as a function of time (column 2).
This can be achieved using the following command:
\begin{lstlisting}[frame=single,language=gnuplot]
  plot "statistics" using 2:19
\end{lstlisting}
The left panel of Fig.~\ref{fig:viz-gnuplot-1} shows what \texttt{Gnuplot}
will display in its output window. There are many things one can
configure in these plots (see the \texttt{Gnuplot} manual referenced above).
For example, let us assume that we want to add labels to the $x$- and $y$-axes,
use not just points but lines and points for the curves,
restrict the time axis to the range $[0,0.2]$ and the heat flux axis to
$[-10:10]$,
plot not only the flux through the bottom but also through the top boundary
(column 20) and finally add a key to the figure, then the following
commands achieve this:
\begin{lstlisting}[frame=single,language=gnuplot]
  set xlabel "Time"
  set ylabel "Heat flux"
  set style data linespoints
  plot [0:0.2][-10:10] "statistics" using 2:19 title "Bottom boundary", \
                       "statistics" using 2:20 title "Top boundary"
\end{lstlisting}
If a line gets too long, you can continue it by ending it in a backslash as
above. This is rarely used on the command line but useful when writing the
commands above into a script file, see below. We have done it here to get
the entire command into the width of the page.

\begin{figure}
  \centering
  \phantom.
  \hfill
  \includegraphics[width=0.4\textwidth]{viz/statistics/1}
  \hfill
  \includegraphics[width=0.4\textwidth]{viz/statistics/2}
  \hfill
  \phantom.
  \caption{\it Visualizing the statistics file obtained from the example in
    Section~\ref{sec:cookbooks-simple-box} using \texttt{Gnuplot}: Output
    using simple commands.}
  \label{fig:viz-gnuplot-1}
\end{figure}

For those who are lazy, \texttt{Gnuplot} allows to abbreviate things in many
different ways. For example, one can abbreviate most commands. Furthermore,
one does not need to repeat the name of an input file if it is the same
as the previous one in a plot command. Thus, instead of the commands above,
the following abbreviated form would have achieved the same effect:
\begin{lstlisting}[frame=single,language=gnuplot]
  se xl "Time"
  se yl "Heat flux"
  se sty da lp
  pl [:0.2][-10:10] "statistics" us 2:19 t "Bottom boundary", "" us 2:20 t "Top boundary"
\end{lstlisting}
This is of course unreadable at first but becomes useful once you become
more familiar with the commands offered by this program.

Once you have gotten the commands that create the plot you want right, you probably
want to save it into a file. \texttt{Gnuplot} can write output in many
different formats. For inclusion in publications, either \texttt{eps} or
\texttt{png} are the most common. In the latter case, the commands to
achieve this are
\begin{lstlisting}[frame=single,language=gnuplot]
  set terminal png
  set output "heatflux.png"
  replot
\end{lstlisting}
The last command will simply generate the same plot again but this time
into the given file. The result is a graphics file similar to the one
shown in Fig.~\ref{fig:convection-box-stats} on page \pageref{fig:convection-box-stats}.

\note{After setting output to a file, \textit{all} following plot commands will
  want to write to this file. Thus, if you want to create more plots after
  the one just created, you need to reset output back to the screen. On Linux,
  this is done using the command \texttt{set terminal X11}. You can then
  continue experimenting with plots and when you have the next plot ready,
  switch back to output to a file.}

What makes \texttt{Gnuplot} so useful is that it doesn't just allow entering
all these commands at the prompt. Rather, one can write them all into a file,
say \texttt{plot-heatflux.gnuplot}, and then, on the command line, call
\begin{lstlisting}[frame=single,language=ksh]
  gnuplot plot-heatflux.gnuplot
\end{lstlisting}
to generate the \texttt{heatflux.png} file. This comes in handy if one wants
to create the same plot for multiple simulations while playing with parameters
of the physical setup. It is also a very useful tool if one wants to generate
the same kind of plot again later with a different data set, for example when
a reviewer requested additional computations to be made for a paper or if one
realizes that one has forgotten or misspelled an axis label in a plot.%
\footnote{In my own work, I usually save the \aspect{} input file, the
  \texttt{statistics} output file and the \texttt{Gnuplot} script along with
  the actual figure I want to include in a paper. This way, it is easy to
  either re-run an entire simulation, or just tweak the graphic at a later
  time. Speaking from experience, you will not believe how often one wants
  to tweak a figure long after it was first created. In such situations it is
  outstandingly helpful if one still has both the actual data as well as the script
  that generated the graphic.}

\texttt{Gnuplot} has many many more features we have not even touched upon. For
example, it is equally happy to produce three-dimensional graphics, and it also
has statistics modules that can do things like curve fits, statistical regression,
and many more operations on the data you provide in the columns of an input file.
We will not try to cover them here but instead refer to the manual at
\url{http://www.gnuplot.info/}. You can also get a good amount of information
by typing \texttt{help} at the prompt, or a command like \texttt{help plot} to
get help on the \texttt{plot} command.


\subsubsection{Large data issues for parallel computations}
\label{sec:viz-data}

Among the challenges in visualizing the results of parallel computations is
dealing with the large amount of data. The first bottleneck this presents is
during run-time when \aspect{} wants to write the visualization data of a time
step to disk. Using the compressed VTU format, \aspect{} generates on the
order of 10 bytes of output for each degree of freedom in 2d and more in 3d;
thus, output of a single time step can run into the range of gigabytes that
somehow have to get from compute nodes to disk. This stresses both the cluster
interconnect as well as the data storage array.
\index[prmindex]{Number of grouped files}
\index[prmindexfull]{Postprocess!Visualization!Number of grouped files}


There are essentially two strategies supported by \aspect{} for this scenario:
\begin{itemize}
\item If your cluster has a fast interconnect, for example Infiniband, and if
  your cluster has a fast, distributed file system, then \aspect{} can produce
  output files that are already located in the correct output directory (see
  the options in Section~\ref{parameters:global}) on the global file
  system. \aspect{} uses MPI I/O calls to this end, ensuring that the local
  machines do not have to access these files using slow NFS-mounted global
  file systems.

\item If your cluster has a slow interconnect, e.g., if it is simply a
  collection of machines connected via Ethernet, then writing data to a
  central file server may block the rest of the program for a while. On the
  other hand, if your machines have fast local storage for temporary file
  systems, then \aspect{} can write data first into such a file and then move
  it in the background to its final destination while already continuing
  computations. To select this mode, set the appropriate variables discussed
  in Section~\ref{parameters:Postprocess/Visualization}. Note, however, that
  this scheme only makes sense if every machine on which MPI processes run has
  fast local disk space for temporary storage.
\end{itemize}

\note{An alternative would be if every processor directly writes its own files
  into the global output directory (possibly in the background), without the
  intermediate step of the temporary file. In our experience, file servers are
  quickly overwhelmed when encountering a few hundred machines wanting to
  open, fill, flush and close their own file via NFS mounted file system
  calls, sometimes completely blocking the entire cluster environment for
  extended periods of time.}

\subsection{Checkpoint/restart support}
\label{sec:checkpoint-restart}

If you do long runs, especially when using parallel computations, there are a
number of reasons to periodically save the state of the program:
\begin{itemize}
\item If the program crashes for whatever reason, the entire computation may
  be lost. A typical reason is that a program has exceeded the requested
  wallclock time allocated by a batch scheduler on a cluster.
\item Most of the time, no realistic initial conditions for strongly
  convecting flow are available. Consequently, one typically starts with a
  somewhat artificial state and simply waits for a long while till the
  convective state enters the phase where it shows its long-term
  behavior. However, getting there may take a good amount of CPU time and it
  would be silly to always start from scratch for each different parameter
  setting. Rather, one would like to start such parameter studies with a saved
  state that has already passed this initial, unphysical, transient stage.
\end{itemize}

To this end, \aspect{} creates a set of files in the output directory
\index[prmindex]{Output directory}
\index[prmindexfull]{Output directory}
(selected in the parameter file) every N time steps (controlled by the number
of steps or wall time as specified in \texttt{subsection Checkpointing}, see
Section~\ref{parameters:Checkpointing}) in which the entire state of the
program is saved so that a simulation can later be continued at this
point. The previous checkpoint files will then be deleted. To resume
operations from the last saved state, you need to set the \texttt{Resume
  computation} flag in the input parameter file to \texttt{true}, see
\index[prmindex]{Resume computation}
\index[prmindexfull]{Resume computation}
Section~\ref{parameters:Resume computation}.

\note{It is not imperative that the parameters selected in the input file are
  exactly the same when resuming a program from a saved state than what they
  were at the time when this state was saved. For example, one may want to
  choose a different parametrization of the material law, or add or remove
  postprocessors that should be run at the end of each time step. Likewise,
  the end time, the times at which some additional mesh refinement steps
  should happen, etc., can be different.

  Yet, it is
  clear that some other things can't be changed: For example, the geometry
  model that was used to generate the coarse mesh and describe the boundary
  must be the same before and after resuming a computation. Likewise, you can
  not currently restart a computation with a different number of processors
  than initially used to checkpoint the simulation.
  Not all invalid
  combinations are easy to detect, and \aspect{} may not always realize
  immediate what is going on if you change a setting that can't be
  changed. However, you will almost invariably get nonsensical results after
  some time.}


\subsection{Making \aspect{} run faster}

When developing \aspect{}, we are guided by the principle that the default for
all settings should be \textit{safe}. In particular, this means that you should
get errors when something goes wrong, the program should not let you choose an
input file parameter so that it doesn't make any sense, and we should solve the
equations to best ability without cutting corners. The goal is that when you
start working with \aspect{} that we give you the best answer we can. The
downside is that this also makes \aspect{} run slower than may be possible. This
section describes ways of making \aspect{} run faster -- assuming that you know
what you are doing and are making conscious decisions.

\subsubsection{Debug vs.~optimized mode}
Both \dealii{} and \aspect{} by default have a great deal of internal checking
to make sure that the code's state is valid. For example, if you write a new
postprocessing plugin (see Section~\ref{sec:plugins})) in which you need to
access the solution vector, then \dealii{}'s \texttt{Vector} class will make
sure that you are only accessing elements of the vector that actually exist and
are available on the current machine if this is a parallel computation. We do so
because it turns out that by far the most bugs one introduces in programs are of
the kind where one tries to do something that obviously doesn't make sense
(such as accessing vector element 101 when it only has 100 elements). These
kinds of bugs are more frequent than implementing a wrong algorithm, but they
are fortunately easy to find if you have a sufficient number of assertions in
your code. The downside is that assertions cost run time.

As mentioned above, the default is to have all of these assertions in the code
to catch those places where we may otherwise silently access invalid memory
locations. However, once you have a plugin running and verified that your input
file runs without problems, you can switch off all of these checks by switching
from debug to optimized mode. This means re-compiling \aspect{} and linking
against a version of the \dealii{} library without all of these internal checks.
Because this is the first thing you will likely want to do, we have already
discussed how to do all of this in Section~\ref{sec:debug-mode}.

\subsubsection{Adjusting solver tolerances} At the heart of every time step
lies the solution of linear systems for the Stokes equations, the temperature
field, and possibly for compositional fields. In essence, each of these steps
requires us to solve a linear system of the form $Ax=b$ which we do through
iterative solvers, i.e., we try to find a sequence of approximations $x^{(k)}$
where $x^{(k)}\rightarrow x=A^{-1}b$. This iteration is terminated at iteration
$k$ if the approximation is ``close enough'' to the exact solution. The solvers
we use this determine this by testing after every iteration whether the
\textit{residual}, $r^{(k)}=A(x-x^{(k)})=b-Ax^{(k)}$, satisfies
$\|r^{(k)}\|\le\varepsilon\|r^{(0)}\|$ where $\varepsilon$ is called the
(relative) \textit{tolerance}.

Obviously, the smaller we choose $\varepsilon$, the more accurate the
approximation $x^{(k)}$ will be. On the other hand, it will also take more
iterations and, consequently, more CPU time to reach the stopping criterion with
a smaller tolerance. The default value of these tolerances are chosen so that
the approximation is typically sufficient. You can make \aspect{} run faster if
you choose these tolerances larger.
The parameters you can adjust are all listed in
Section~\ref{parameters:global} and are located at the top level of the input
file. In particular, the parameters you want to look at are \texttt{Linear
solver tolerance}, \texttt{Temperature solver tolerance} and
\texttt{Composition solver tolerance}.
\index[prmindex]{Composition solver tolerance}
\index[prmindexfull]{Composition solver tolerance}
\index[prmindex]{Linear solver tolerance}
\index[prmindexfull]{Linear solver tolerance}
\index[prmindex]{Temperature solver tolerance}
\index[prmindexfull]{Temperature solver tolerance}

All this said, it is important to understand the consequences of choosing
tolerances larger. In particular, if you choose tolerances too large, then the
difference between the exact solution of a linear system $x$ and the
approximation $x^{(k)}$ may become so large that you do not get an accurate
output of your model any more. A rule of thumb in choosing tolerances is to
start with a small value and then increase the tolerance until you come to a
point where the output quantities start to change significantly. This is the
point where you will want to stop.

\subsubsection{Adjusting solver preconditioner tolerances} To solve the Stokes 
equations it is necessary to lower the condition number of the
Stokes matrix by preconditioning  it. In \aspect{} a right preconditioner $Y^{-1} = 
\begin{pmatrix}
\widetilde{A^{-1}} & -\widetilde{A^{-1}}B^{T}\widetilde{S^{-1}} \\
0 & \widetilde{S^{-1}}
\end{pmatrix}$ is used to precondition the system, where $\widetilde{A^{-1}}$ is 
the approximate inverse of the A block and $\widetilde{S^{-1}}$ is the approximate 
inverse of the Schur complement matrix. Matrix $\widetilde{A^{-1}}$ and 
$\widetilde{S^{-1}}$ are calculated through a CG solve, which requires a tolerance 
to be set. In comparison with the solver tolerances of the previous section, these 
parameters are relatively safe to use, since they only change the preconditioner, 
but can speed up or slow down solving the Stokes system considerably. 

In practice $\widetilde{A^{-1}}$ takes by far the most time to compute, but is 
also very important in conditioning the system. The accuracy of the computation 
of $\widetilde{A^{-1}}$ is controlled by the parameter \texttt{Linear solver A 
block tolerance} which has a default value of $1e-2$. Setting this tolerance 
to a less strict value will result in more outer iterations, since the 
preconditioner is not as good, but the amount of time to compute 
$\widetilde{A^{-1}}$ can drop significantly resulting in a reduced total solve 
time. The cookbook crustal deformation (Section 
\ref{sec:cookbooks-crustal-deformation}) for example can be computed much faster 
by setting the \texttt{Linear solver A block tolerance} to $5e-1$. The 
calculation of $\widetilde{S^{-1}}$ is usually much faster and the 
conditioning of the system is less sensitive to the parameter \texttt{Linear 
solver S block tolerance}, but for some problems it might be worth it to 
investigate.
\index[prmindex]{Linear solver A block tolerance}
\index[prmindexfull]{Linear solver A block tolerance}
\index[prmindex]{Linear solver S block tolerance}
\index[prmindexfull]{Linear solver S block tolerance}

\subsubsection{Using lower order elements for the temperature/compositional discretization}
The default settings of \aspect{} use quadratic finite elements for the
velocity. Given that the temperature and compositional fields essentially (up
to material parameters) satisfy advection equations of the kind $\partial_t T +
\mathbf u \cdot \nabla T = \ldots$, it seems appropriate to also use quadratic
finite elemen shape functions for the temperature and compositional fields.

However, this is not mandatory. If you do not care about high accuracy in these
fields and are mostly interested in the velocity or pressure field, you can
select lower-order finite elements in the input file. The polynomial degrees are
controlled with the parameters in the \textit{discretization} section of the
input file, see Section~\ref{parameters:Discretization}, in particular by
\texttt{Temperature polynomial degree} and
\texttt{Composition polynomial degree}.
\index[prmindex]{Temperature polynomial degree}
\index[prmindexfull]{Discretization!Temperature polynomial degree}
\index[prmindex]{Composition polynomial degree}
\index[prmindexfull]{Discretization!Composition polynomial degree}

As with the other parameters discussed above and below, it is worthwhile
comparing the results you get with different values of these parameters when
making a decision whether you want to save on accuracy in order to reduce
compute time. An example of how this choice affects the accuracy you get is
discussed in Section~\ref{sec:cookbooks-simple-box}.



\subsubsection{Limiting postprocessing}
\aspect{} has a lot of postprocessing capabilities, from generating graphical
output to computing average temperatures or temperature fluxes. To see what all
is possible, take a look at the \texttt{List of postprocessors} parameter that
can be set in the input file, see Section~\ref{parameters:Postprocess}.
\index[prmindex]{List of postprocessors}
\index[prmindexfull]{Postprocess!List of postprocessors}

Many of these postprocessors take a non-negligible amount of time. How much they
collectively use can be inferred from the timing report \aspect{} prints
periodically among its output, see for example the output shown in
Section~\ref{sec:cookbooks-simple-box}. So, if your computations take too long,
consider limiting which postprocessors you run to those you really need. Some
postprocessors -- for example those that generate graphical output, see
Section~\ref{parameters:Postprocess/Visualization} -- also allow you to run them
only once every once in a while, rather than at every time step.


\subsubsection{Switching off pressure normalization}
In most practically relevant cases, the Stokes equations
\eqref{eq:stokes-1}--\eqref{eq:stokes-2} only determine the pressure up to a
constant because only the pressure gradient appears in the equations, not the
actual value of it. However, unlike this ``mathematical'' pressure, we have a
very specific notion of the ``physical'' pressure: namely a well-defined
quantity that at the surface of Earth equals the air pressure, which compared to
the hydrostatic pressure inside Earth is essentially zero.

As a consequence, the default in \aspect{} is to normalize the computed
``mathematical'' pressure in such a way that either the mean pressure at the
surface is zero (where the geometry model describes where the ``surface'' is,
see Section~\ref{sec:geometry-models}), or that the mean pressure in the domain
is zero. This normalization is important if your model describes densities,
viscosities and other quantities in dependence of the pressure -- because you
almost certainly had the ``physical'' pressure in mind, not some unspecified
``mathematical'' one. On the other hand, if you have a material model in which
the pressure does not enter, then you don't need to normalize the pressure at
all -- simply go with whatever the solver provides. In that case, you can switch
off pressure normalization by looking at the \texttt{Pressure normalization}
parameter at the top level of the input file, see
Section~\ref{parameters:global}.
\index[prmindex]{Pressure normalization}
\index[prmindexfull]{Pressure normalization}


\subsubsection{Regularizing models with large coefficient variation}
Models with large jumps in viscosity and other coefficients present
significant challenges to both discretizations and solvers. In particular,
they can lead to very long solver
times. Section~\ref{sec:sinker-with-averaging} presents parameters that can
help regularize models and these typically also include significant
improvements in run-time.
\index[prmindex]{Material averaging}
\index[prmindexfull]{Material model!Material averaging}



\subsection{Input parameter files}
\label{sec:parameters-overview}

What \aspect{} computes is driven by two things:
\begin{itemize}
\item The models implemented in \aspect{}. This includes the geometries, the
  material laws, or the initial conditions currently supported. Which of these
  models are currently implemented is discussed below;
  Section~\ref{sec:extending} discusses in great detail the process of
  implementing additional models.

\item Which of the implemented models is selected, and what their run-time
  parameters are. For example, you could select a model that prescribes
  constant coefficients throughout the domain from all the material models
  currently implemented; you could then select appropriate values for all of
  these constants. Both of these selections happen from a parameter file that
  is read at run time and whose name is specified on the command line. (See
  also Section~\ref{sec:running-overview}.)
\end{itemize}
In this section, let us give an overview of what can be selected in the
parameter file. Specific parameters, their default values, and allowed values
for these parameters are documented in Section~\ref{sec:parameters}. An index
with page numbers for all run-time parameters can be found on
page~\pageref{sec:runtime-parameter-index}.

\subsubsection{The structure of parameter files}

Most of the run-time behavior of \aspect{} is driven by a parameter file that
looks in essence like this:
\lstinputlisting[language=prmfile]{cookbooks/overview/structure.part.prm.out}

Some parameters live at the top level, but most parameters are grouped into
subsections. An input parameter file is therefore much like a file system: a
few files live in the root directory; others are in a nested hierarchy of
sub-directories. And just as with files, parameters have both a name (the
thing to the left of the equals sign) and a content (what's to the right).

All parameters you can list in this input file have been \textit{declared} in
\aspect. What this means is that you can't just list anything in the input
file, and expect that entries that are unknown are simply ignored.
Rather, if your input file contains a line setting a parameter that is unknown, you
will get an error message. Likewise, all declared parameters have a
description of possible values associated with them -- for example, some
parameters must be non-negative integers (the number of initial refinement
steps), can either be true or false (whether the computation should be resumed
from a saved state), or can only be a single element from a selection (the
name of the material model). If an entry in your input file doesn't satisfy
these constraints, it will be rejected at the time of reading the file (and
not when a part of the program actually accesses the value and the programmer
has taken the time to also implement some error checking at this location).
Finally, because parameters have been declared, you do not \textit{need} to
specify a parameter in the input file: if a parameter isn't listed, then the
program will simply use the default provided when declaring the parameter.

\note{In cases where a parameter requires a significant amount of text, you can
end a line in the input file with a backslash. This indicates that the
following line will simply continue to be part of the text of the current line,
in the same way as the C/C++ preprocessor expands lines that end in
backslashes. The underlying implementation always eats whitespace at
the beginning of each continuing line, but not before the
backslash. This means that the parameter file \\
\hspace*{.25cm} \texttt{set Some parameter = abc}$\backslash$ \\
\hspace*{.25cm} \texttt{\phantom{set Some parameter = }def}\\
is equivalent to \\
\hspace*{.25cm} \texttt{set Some parameter = abcdef}\\
that is, with no space between \texttt{abc} and \texttt{def} despite
the leading whitespace at the beginning of the second line. If you
do want space between these two parts, you need to add it before the
backslash in the first of the two lines.
}

\subsubsection{Categories of parameters}

The parameters that can be provided in the input file can roughly be
categorized into the following groups:
\begin{itemize}
\item Global parameters (see Section~\ref{parameters:global}): These
  parameters determine the overall behavior of the program. Primarily they
  describe things like the output directory, the end time of the simulation,
  or whether the computation should be resumed from a previously saved state.

\item Parameters for certain aspects of the numerical algorithm: These
  describe, for example, the specifics of the spatial discretization. In
  particular, this is the case for parameters concerning
  the polynomial degree of the finite element approximation
  (Section~\ref{parameters:Discretization}), some details about the
  stabilization
  (Section~\ref{parameters:Discretization/Stabilization_20parameters}), and
  how adaptive mesh refinement is supposed to work
  (Section~\ref{parameters:Mesh_20refinement}).

\item Parameters that describe certain global aspects of the equations to be
  solved: This includes, for example, a description if certain terms in the
  model should be omitted or not. See
  Section~\ref{parameters:Model_20settings} for the list of parameters in this
  category.

\item Parameters that characterize plugins: Certain behaviors of
  \aspect{} are described by what we call \textit{plugins} -- self-contained
  parts of the code that describe one particular aspect of the simulation. An
  example would be which of the implemented material models to use, and the
  specifics of this material model. The sample parameter file above gives an
  indication of how this works: within a subsection of the file that pertains
  to the material models, one can select one out of several plugins (or, in
  the case of the postprocessors, any number, including none, of the available
  plugins), and one can then specify the specifics of this model in a
  sub-subsection dedicated to this particular model.

  A number of components of \aspect{} are implemented via plugins. Some of
  these, together with the sections in which their parameters are declared, are
  the following:
  \begin{itemize}
  \item The material model:
    Sections~\ref{parameters:Material_20model} and following.
  \item The geometry:
    Sections~\ref{parameters:Geometry_20model} and following.
  \item The gravity description:
    Sections~\ref{parameters:Gravity_20model} and following.
  \item Initial conditions for the temperature:
    Sections~\ref{parameters:Initial_20conditions} and following.
  \item Temperature boundary conditions:
    Sections~\ref{parameters:Boundary_20temperature_20model} and following.
  \item Postprocessors:
    Sections~\ref{parameters:Postprocess} and following for most postprocessors,
    section \ref{parameters:Postprocess/Visualization} and following for
    postprocessors related to visualization.
  \end{itemize}
\end{itemize}

The details of parameters in each of these categories can be found in the
sections linked to above. Some of them will also be used in the cookbooks in
Section~\ref{sec:cookbooks}.


\subsubsection{A note on the syntax of formulas in input files}
\label{sec:muparser-format}

Input files have different ways of describing certain things to \aspect{}. For
example, you could select a plugin for the temperature initial values that
prescribes a constant temperature, or a
plugin that implements a particular formula for these initial conditions in
C++ in the code of the plugin, or a
plugin that allows you to describe this formula in a symbolic way in the input file
(see Section~\ref{parameters:Initial_20conditions}). An example of this latter
case is this snippet of code discussed in
Section~\ref{sec:cookbooks-simple-box-3d}:
%
\lstinputlisting[language=prmfile]{cookbooks/convection-box-3d/initial.part.prm.out}
%
The formulas you can enter here need to use a syntax that is understood by the
functions and classes that interpret what you write. Internally, this is done
using the muparser library, see \url{http://muparser.beltoforion.de/}. The
syntax is mostly self-explanatory in that it allows to use the usual symbols
\texttt{x}, \texttt{y} and \texttt{z} to reference coordinates (unless a
particular plugin uses different variables, such as the depth), the symbol
\texttt{t} for time in many situations, and allows you to use all of the
typical mathematical functions such as sine and cosine. Another common case is
an if-statement that has the general form
\texttt{if(condition, true-expression, false-expression)}. For more examples of
the syntax understood, reference the documentation of the muparser library
linked to above.



\section{Cookbooks}
\label{sec:cookbooks}

In this section, let us present a number of ``cookbooks'' -- examples of how
to use \aspect{} in typical or less typical ways. As discussed in
Sections~\ref{sec:running} and \ref{sec:parameters}, \aspect{} is driven by
run-time parameter files, and so setting up a particular situation primarily
comes down to creating a parameter file that has the right entries. Thus, the
subsections below will discuss in detail what parameters to set and to what
values. Note that parameter files need not specify \textit{all} parameters --
of which there is a bewildering number -- but only those that are relevant to
the particular situation we would like to model. All parameters not listed
explicitly in the input file are simply left at their default value (the
default values are also documented in Section~\ref{sec:parameters}).

Of course, there are situations where what you want to do is not covered by
the models already implemented. Specifically, you may want to try a different
geometry, a different material or gravity model, or different boundary
conditions. In such cases, you will need to implement these extensions in the
actual source code. Section~\ref{sec:extending} provides information on how to
do that.

The remainder of this section shows a number of applications of
\aspect{}. They are grouped into three categories: Simple setups of examples
that show thermal convection (Section~\ref{sec:cookbooks-simple}), setups
that try to model geophysical situations (Section~\ref{sec:cookbooks-geophysical})
and setups that are used to benchmark \aspect{} to ensure correctness or to test accuracy
of our solvers (Section~\ref{sec:cookbooks-benchmarks}). Before we get there,
however, we will review how one usually approaches setting up computations in
Section~\ref{sec:cookbooks-overview}.

\note{The input files discussed in the following sections can generally be
  found in the \texttt{cookbooks/} directory of your \aspect{} installation.}


\subsection{How to set up computations}
\label{sec:cookbooks-overview}

\aspect{}'s computations are controlled by input parameter files such as those
we will discuss in the following sections.%
\footnote{You can also extend \aspect{} using plugins -- i.e., pieces of code
you compile separately and either link into the \aspect{} executable itself, or
reference from the input file. This is discussed in
Section~\ref{sec:extending}.}
Basically, these are just regular text files you can edit with programs like
\texttt{gedit}, \texttt{kwrite} or \texttt{kate} when working on Linux, or
something as simple as \texttt{NotePad} on Windows. When setting up these input
files, you basically have to describe everything that characterizes the
computation you want to do. In particular, this includes the following:
\begin{itemize}
  \item What internal forces act on the medium (the equation)?
  \item What external forces do we have (the right hand side)
  \item What is the domain (geometry)?
  \item What happens at the boundary for each variable involved (boundary
 conditions)?
  \item How did it look at the beginning (initial conditions)?
\end{itemize}
For each of these questions, there are one or more input parameters (sometimes
grouped into sections) that allow you to specify what you want. For example, to
choose a geometry, you will typically have a block like this in your input file:
%
\lstinputlisting[language=prmfile]{cookbooks/overview/geometry.part.prm.out}
%
This indicates that you want to do a computation in 2d, using a rectangular
geometry (a ``box'') with edge length equal to one in both the $x$- and
$y$-directions. Of course, there are other geometries you can choose from for
the \texttt{Model name} parameter, and consequently other subsections that
specify the details of these geometries.

Similarly, you describe boundary conditions using parameters such as this:
%
\lstinputlisting[language=prmfile]{cookbooks/overview/boundary-conditions.part.prm}
%
This snippet describes which of the four boundaries of the two-dimensional box
we have selected above should have a prescribed temperature or an insulating
boundary, and at which parts of the boundary we want zero, tangential or
prescribed velocities.%
\footnote{Internally, the geometry models \aspect{} uses label every part of
  the boundary with what is called a \textit{boundary indicator} -- a number
  that identifies pieces of the boundary. If you know which number each piece
  has, you can list these numbers on the right hand sides of the assignments
  of boundary types above. For example, the left boundary of the box has
  boundary indicator zero (see Section~\ref{parameters:Geometry_20model}), and
  using this number instead of the \texttt{left} would have been equally
  valid. However, numbers are far more difficult to remember than names, and
  consequently every geometry model provides string aliases such as
  ``\texttt{left}'' for each boundary indicator describing parts of the
  boundary. These symbolic aliases are specific to the geometry -- for the
  box, they are ``\texttt{left}'', ``\texttt{right}'', ``\texttt{bottom}'',
  etc., whereas for a spherical shell they are ``\texttt{inner}'' and
  ``\texttt{outer}'' -- but are described in the documentation of every
  geometry model, see Section~\ref{parameters:Geometry_20model}.}

If you go down the list of questions about the setup above, you have already
done the majority of the work describing your computation. The remaining
parameters you will typically want to specify have to do with the computation
itself. For example, what variables do you want to output and how often? What
statistics do you want to compute. The following sections will give ample
examples for all of this, but using the questions above as a guideline is
already a good first step.

\note{It is of course possible to set up input files for computations
completely from scratch. However, in practice, it is often simpler to go
through the list of cookbooks already provided and find one that comes close to
what you want to do. You would then modify this cookbook until it does what you
want to do. The advantage is that you can start with something you already know
works, and you can inspect how each change you make -- changing the
details of the geometry, changing the material model, or changing what is being
computed at the end of each time step -- affects what you get.}


\subsection{Simple setups}
\label{sec:cookbooks-simple}

\subsubsection{Convection in a 2d box}
\label{sec:cookbooks-simple-box}

In this first example, let us consider a simple situation: a 2d box of dimensions
$[0,1]\times [0,1]$ that is heated from below, insulated at the left and right,
and cooled from the top. We will also consider the simplest model, the
incompressible Boussinesq approximation with constant coefficients
$\eta,\rho_0,\mathbf g,C_p, k$, for this testcase. Furthermore, we
assume that the medium expands linearly with
temperature. This leads to the following set of equations:
\begin{align}
  -\nabla \cdot \left[2\eta \varepsilon(\mathbf u)
                \right] + \nabla p &=
  \rho_0 (1-\alpha (T-T_0)) \mathbf g
  & \qquad
  & \textrm{in $\Omega$},
  \\
  \nabla \cdot \mathbf u &= 0
  & \qquad
  & \textrm{in $\Omega$},
  \\
  \rho_0 (1-\alpha (T-T_0)) C_p \left(\frac{\partial T}{\partial t} + \mathbf
  u\cdot\nabla T\right) - \nabla\cdot k\nabla T
  &=
  0
  & \qquad
  & \textrm{in $\Omega$}.
\end{align}
It is well known that we can non-dimensionalize this set of equations by
introducing the Rayleigh number $Ra=\frac{\rho_0 g \alpha \Delta T h^3}{\eta \kappa}$, 
where $h$ is the height of the box, $\kappa = \frac{k}{\rho C_p}$ is the thermal diffusivity
and $\Delta T$ is the temperature difference between top and bottom of the box. Formally,
we can obtain the non-dimensionalized equations by using the above form and
setting coefficients in the following way:
\begin{align*}
  \rho_0=C_p=\kappa=\alpha=\eta=h=\Delta T=1, \qquad T_0=0, \qquad g=Ra,
\end{align*}
where $\mathbf g=-g \mathbf e_z$ is the gravity vector in negative
$z$-direction. While this would be a valid description of the problem, it is not
what one typically finds in the literature because there the density in the
temperature equation is chosen as reference density $\rho_0$ rather than the 
full density $\rho(1-\alpha(T-T_0))$
as used by \aspect{}. However, we can mimic this by choosing a very small value
for $\alpha$ -- small enough to ensure that for all reasonable temperatures,
the density used here is equal to $\rho_0$ for all practical purposes --, and
instead making $g$ correspondingly larger.
Consequently, in this cookbook we will use the following set of parameters:
\begin{align*}
  \rho_0=C_p=\kappa=\eta=h=\Delta T=1, \qquad T_0=0, \qquad \alpha=10^{-10}, \qquad
  g=10^{10} Ra.
\end{align*}
We will see all of these values again in the input file discussed below.
The problem is completed by stating the velocity boundary conditions: tangential
flow along all four of the boundaries of the box.

This situation describes a well-known benchmark problems for which a lot is
known and against which we can compare our results. For example, the following
is well understood:
\begin{itemize}
  \item For values of the Rayleigh number less than a critical number
  $Ra_c\approx 780$, thermal diffusion dominates convective heat transport and
  any movement in the fluid is damped exponentially. If the Rayleigh number is moderately larger
  than this threshold then a stable convection pattern forms that transports
  heat from the bottom to the top boundaries. The simulations we will set up
  operates in this regime. Specifically, we will choose $Ra=10^4$.

  On the other hand, if the Rayleigh number becomes even larger, a serious of
  period doublings starts that makes the system become more and more unstable.
  We will investigate some of this behavior at the end of this section.

  \item For certain values of the Rayleigh number, very accurate values for the
  heat flux through the bottom and top boundaries are available in the literate.
  For example, Blankenbach \textit{et al.} report a non-dimensional heat flux of
  $4.884409 \pm 0.00001$, see \cite{BBC89}. We will compare our results against
  this value below.
\end{itemize}

With this said, let us consider how to represent this situation in practice.


\paragraph{The input file.}
The verbal description of this problem can be translated into an \aspect{}
input file in the following way (see Section~\ref{sec:parameters} for a
description of all of the parameters that appear in the following input file,
and the indices at the end of this manual if you want to find a particular
parameter; you can find the input file to run this cookbook example in
\url{cookbooks/convection-box.prm}):

\lstinputlisting[language=prmfile]{cookbooks/convection-box/box.prm.out}


\paragraph{Running the program.}
When you run this program for the first time, you are probably still running
\aspect{} in debug mode (see Section~\ref{sec:debug-mode}) and you will get
output like the following:

\begin{lstlisting}[frame=single,language=ksh]
Number of active cells: 256 (on 5 levels)
Number of degrees of freedom: 3,556 (2,178+289+1,089)

*** Timestep 0:  t=0 seconds
   Solving temperature system... 0 iterations.
   Rebuilding Stokes preconditioner...
   Solving Stokes system... 30+5 iterations.

[... ...]

*** Timestep 1077:  t=0.499901 seconds
   Solving temperature system... 9 iterations.
   Solving Stokes system... 5 iterations.

   Postprocessing:
     RMS, max velocity:                  43.1 m/s, 69.8 m/s
     Temperature min/avg/max:            0 K, 0.5 K, 1 K
     Heat fluxes through boundary parts: 0.02056 W, -0.02061 W, -4.931 W, 4.931 W



+---------------------------------------------+------------+------------+
| Total wallclock time elapsed since start    |       454s |            |
|                                             |            |            |
| Section                         | no. calls |  wall time | % of total |
+---------------------------------+-----------+------------+------------+
| Assemble Stokes system          |      1078 |      19.2s |       4.2% |
| Assemble temperature system     |      1078 |       329s |        72% |
| Build Stokes preconditioner     |         1 |    0.0995s |     0.022% |
| Build temperature preconditioner|      1078 |      5.84s |       1.3% |
| Solve Stokes system             |      1078 |      15.6s |       3.4% |
| Solve temperature system        |      1078 |      3.72s |      0.82% |
| Initialization                  |         2 |    0.0474s |      0.01% |
| Postprocessing                  |      1078 |      61.9s |        14% |
| Setup dof systems               |         1 |     0.221s |     0.049% |
+---------------------------------+-----------+------------+------------+
\end{lstlisting}

If you've read up on the difference between debug and optimized mode (and you
should before you switch!) then consider disabling debug mode. If you run the
program again, every number should look exactly the same (and it does, in fact,
as I am writing this) except for the timing information printed every hundred
time steps and at the end of the program:

\begin{lstlisting}[frame=single,language=ksh]
+---------------------------------------------+------------+------------+
| Total wallclock time elapsed since start    |      48.3s |            |
|                                             |            |            |
| Section                         | no. calls |  wall time | % of total |
+---------------------------------+-----------+------------+------------+
| Assemble Stokes system          |      1078 |      1.68s |       3.5% |
| Assemble temperature system     |      1078 |      26.3s |        54% |
| Build Stokes preconditioner     |         1 |    0.0401s |     0.083% |
| Build temperature preconditioner|      1078 |      4.87s |        10% |
| Solve Stokes system             |      1078 |      6.76s |        14% |
| Solve temperature system        |      1078 |      1.76s |       3.7% |
| Initialization                  |         2 |    0.0241s |      0.05% |
| Postprocessing                  |      1078 |      4.99s |        10% |
| Setup dof systems               |         1 |    0.0394s |     0.082% |
+---------------------------------+-----------+------------+------------+
\end{lstlisting}

In other words, the program ran about 10 times faster than before. Not all
operations became faster to the same degree: assembly, for example, is an area
that traverses a lot of code both in \aspect{} and in \dealii{} and so
encounters a lot of verification code in debug mode. On the other hand, solving
linear systems primarily requires lots of matrix vector operations. Overall, the
fact that in this example, assembling linear systems and preconditioners takes
so much time compared to actually solving them is primarily a reflection of how
simple the problem is that we solve in this example. This can also be seen in
the fact that the number of iterations necessary to solve the Stokes and
temperature equations is so low. For more complex problems with non-constant
coefficients such as the viscosity, as well as in 3d, we have to spend much more
work solving linear systems whereas the effort to assemble linear systems
remains the same.

\paragraph{Visualizing results.}
Having run the program, we now want to visualize the numerical results we got.
\aspect{} can generate graphical output in formats understood by pretty much any
visualization program (see the parameters described in
Section~\ref{parameters:Postprocess/Visualization}) but we will here follow the
discussion in Section~\ref{sec:viz} and use the default VTU output format to
visualize using the Visit program.

In the parameter file we have specified that graphical output should be
generated every 0.01 time units. Looking through these output files, we find
that the flow and temperature fields quickly converge to a stationary state.
Fig.~\ref{fig:convection-box-fields} shows the initial and final states of this
simulation.

\begin{figure}
\phantom.
\hfill
\includegraphics[width=0.4\textwidth]{cookbooks/convection-box/visit0000}
\hfill
\includegraphics[width=0.4\textwidth]{cookbooks/convection-box/visit0001}
\hfill
\phantom.
\caption{\it Convection in a box: Initial temperature and velocity field (left)
and final state (right).}
\label{fig:convection-box-fields}
\end{figure}

There are many other things we can learn from the output files generated by
\aspect{}, specifically from the statistics file that contains information
collected at every time step and that has been discussed in
Section~\ref{sec:viz-stat}. In particular, in our input file, we have selected
that we would like to compute velocity, temperature, and heat flux statistics.
These statistics, among others, are listed in the statistics file whose head
looks like this for the current input file:
\begin{lstlisting}[frame=single,language=prmfile]
# 1: Time step number
# 2: Time (seconds)
# 3: Number of mesh cells
# 4: Number of Stokes degrees of freedom
# 5: Number of temperature degrees of freedom
# 6: Iterations for temperature solver
# 7: Iterations for Stokes solver
# 8: Time step size (seconds)
# 9: RMS velocity (m/s)
# 10: Max. velocity (m/s)
# 11: Minimal temperature (K)
# 12: Average temperature (K)
# 13: Maximal temperature (K)
# 14: Average nondimensional temperature (K)
# 15: Outward heat flux through boundary with indicator 0 (W)
# 16: Outward heat flux through boundary with indicator 1 (W)
# 17: Outward heat flux through boundary with indicator 2 (W)
# 18: Outward heat flux through boundary with indicator 3 (W)
# 19: Visualization file name
... lots of numbers arranged in columns ...
\end{lstlisting}

Fig.~\ref{fig:convection-box-stats} shows the results of visualizing the data
that can be found in columns 2 (the time) plotted against columns 9 and 10
(root mean square and maximal velocities). Plots of this kind can be generated with
\texttt{Gnuplot} by typing (see Section~\ref{sec:viz-stat} for a more thorough
discussion):
\begin{verbatim}
  plot "output/statistics" using 2:9 with lines
\end{verbatim}
Fig.~\ref{fig:convection-box-stats} shows clearly that the simulation
enters a steady state after about $t\approx 0.1$ and then changes very little. This can also be observed using the
graphical output files from which we have generated
Fig.~\ref{fig:convection-box-fields}. One can look further into this data to
find that the flux through the top and bottom boundaries is not exactly the same
(up to the obvious difference in sign, given that at the bottom boundary heat
flows into the domain and at the top boundary out of it) at the beginning of the
simulation until the fluid has attained its equilibrium. However, after
$t\approx 0.2$, the fluxes differ by only $5\cdot 10^{-5}$, i.e., by less than
0.001\% of their magnitude.%
\footnote{This difference is far smaller than the numerical error in the heat
flux on the mesh this data is computed on.}
The flux we get at the last time step, 4.931, is less than 1\% away from the
value reported in \cite{BBC89} although we compute on a $16\times 16$ mesh and
the values reported by Blankenbach are extrapolated from meshes of size up to
$72\times 72$. This shows the accuracy that can be obtained using a higher order
finite element. Secondly, the fluxes through the left and right boundary are not
exactly zero but small. Of course, we have prescribed boundary conditions of the
form $\frac{\partial T}{\partial \mathbf n}=0$ along these boundaries, but this
is subject to discretization errors. It is easy to verify that the heat flux
through these two boundaries disappears as we refine the mesh further.

\begin{figure}
\phantom.
\hfill
\includegraphics[width=0.4\textwidth]{cookbooks/convection-box/velocity}
\hfill
\includegraphics[width=0.4\textwidth]{cookbooks/convection-box/heatflux}
\hfill
\phantom.
\caption{\it Convection in a box: Root mean square and maximal velocity as a
function of simulation time (left). Heat flux through the four boundaries of
the box (right).}
\label{fig:convection-box-stats}
\end{figure}


Furthermore, \aspect{} automatically also collects statistics about many of its
internal workings. Fig.~\ref{fig:convection-box-iterations} shows the number of
iterations required to solve the Stokes and temperature linear systems in each
time step. It is easy to see that these are more difficult to solve in the
beginning when the solution still changes significant from time step to time
step. However, after some time, the solution remains mostly the same and solvers
then only need 9 or 10 iterations for the temperature equation and 4 or 5
iterations for the Stokes equations because the starting guess for the linear
solver -- the previous time step's solution -- is already pretty good. If you
look at any of the more complex cookbooks, you will find that one needs many
more iterations to solve these equations.

\begin{figure}
\phantom.
\hfill
\includegraphics[width=0.4\textwidth]{cookbooks/convection-box/iterations}
\hfill
\phantom.
\caption{\it Convection in a box: Number of linear iterations required to solve
the Stokes and temperature equations in each time step.}
\label{fig:convection-box-iterations}
\end{figure}


\paragraph{Play time 1: Different Rayleigh numbers.} After showing you results
for the input file as it can be found in \url{cookbooks/convection-box.prm}, let us
end this section with a few ideas on how to play with it and what to explore.
The first direction one could take this example is certainly to consider
different Rayleigh numbers. As mentioned above, for the value $Ra=10^4$ for
which the results above have been produced, one gets a stable convection
pattern. On the other hand, for values $Ra<Ra_c\approx 780$, any movement of
the fluid dies down exponentially and we end up with a situation where the fluid
doesn't move and heat is transported from the bottom to the top only through
heat conduction. This can be explained by considering that the Rayleigh number
in a box of unit extent is defined as $Ra=\frac{g\alpha}{\eta k}$. A small
Rayleigh number means that the viscosity is too large (i.e., the buoyancy given
by the product of the magnitude of gravity times the thermal expansion
coefficient is not strong enough to overcome friction forces within the fluid).

On the other hand, if the Rayleigh number is large (i.e., the viscosity is
small or the buoyancy large) then the fluid develops an unsteady convection
period. As we consider fluids with larger and larger $Ra$, this pattern goes
through a sequence of period-doubling events until flow finally becomes chaotic.
The structures of the flow pattern also become smaller and smaller.

\begin{figure}
\phantom.
\hfill
\includegraphics[width=0.4\textwidth]{cookbooks/convection-box/ra_1e2_visit0000}
\hfill
\includegraphics[width=0.4\textwidth]{cookbooks/convection-box/ra_1e6_visit0001}
\hfill
\phantom.
\caption{\it Convection in a box: Temperature fields at the end of a
simulation for $Ra=10^2$ where thermal diffusion dominates (left) and $Ra=10^6$
where convective heat transport dominates (right).
The mesh on the right is clearly too coarse to resolve the structure of the solution.}
\label{fig:convection-box-fields-different-Ra}
\end{figure}

\begin{figure}
\phantom.
\hfill
\includegraphics[width=0.4\textwidth]{cookbooks/convection-box/ra_1e6_velocity}
\hfill
\includegraphics[width=0.4\textwidth]{cookbooks/convection-box/ra_1e6_heatflux}
\hfill
\phantom.
\caption{\it Convection in a box: Velocities (left) and heat flux across the
top and bottom boundaries (right) as a function of time at $Ra=10^6$.}
\label{fig:convection-box-stats-different-Ra}
\end{figure}

We illustrate these situations in
Fig.s~\ref{fig:convection-box-fields-different-Ra} and
\ref{fig:convection-box-stats-different-Ra}. The first shows the temperature
field at the end of a simulation for $Ra=10^2$ (below $Ra_c$) and at $Ra=10^6$.
Obviously, for the right picture, the mesh is not fine enough to accurately
resolve the features of the flow field and we would have to refine it more. The
second of the figures shows the velocity and heatflux statistics for the
computation with $Ra=10^6$; it is obvious here that the flow no longer settles
into a steady state but has a periodic behavior. This can also be seen by
looking at movies of the solution.

To generate these results, remember that we have chosen $\alpha=10^{-10}$ and
$g=10^{10}Ra$ in our input file. In other words, changing the input file to
contain the parameter setting
%
\lstinputlisting[language=prmfile]{cookbooks/convection-box/gravity.part.prm.out}
%
will achieve the desired effect of computing with $Ra=10^6$.


\paragraph{Play time 2: Thinking about finer meshes.}
In our computations for $Ra=10^4$ we used a $16\times 16$ mesh and obtained a
value for the heat flux that differed from the generally accepted value from
Blankenbach \textit{et al.} \cite{BBC89} by less than 1\%. However, it may be
interesting to think about computing even more accurately. This is easily done
by using a finer mesh, for example. In the parameter file above, we have chosen
the mesh setting as follows:
%
\lstinputlisting[language=prmfile]{cookbooks/convection-box/refine.part.prm.out}
%
We start out with a box geometry consisting of a single cell that is refined
four times. Each time we split each cell into its 4 children, obtaining the
$16\times 16$ mesh already mentioned. The other settings indicate that we do not
want to refine the mesh adaptively at all in the first time step, and a setting
of zero for the last parameter means that we also never want to adapt the mesh
again at a later time. Let us stick with the never-changing, globally refined
mesh for now (we will come back to adaptive mesh refinement again at a later
time) and only vary the initial global refinement. In particular, we could
choose the parameter \texttt{Initial global refinement} to be 5, 6, or even
larger. This will get us closer to the exact solution albeit at the expense of a
significantly increased computational time.

A better strategy is to realize that for $Ra=10^4$, the flow enters a steady
state after settling in during the first part of the simulation (see, for
example, the graphs in Fig.~\ref{fig:convection-box-stats}). Since we are not
particularly interested in this initial transient process, there is really no
reason to spend CPU time using a fine mesh and correspondingly small time
steps during this part of the simulation (remember that each refinement results
in four times as many cells in 2d and a time step half as long, making reaching
a particular time at least 8 times as expensive, assuming that all solvers in
\aspect{} scale perfectly with the number of cells). Rather, we can use a
parameter in the \aspect{} input file that let's us increase the mesh resolution
at later times. To this end, let us use the following snippet for the input
file:
\lstinputlisting[language=prmfile]{cookbooks/convection-box/refine2.part.prm.out}

What this does is the following: We start with an $8\times 8$ mesh (3 times
globally refined) but then at times $t=0.2,0.3$ and $0.4$ we refine the mesh
using the default refinement indicator (which one this is is not important
because of the next statement). Each time, we refine, we refine a fraction 1.0
of the cells, i.e., \textit{all} cells and we coarsen a fraction of 0.0 of the
cells, i.e. no cells at all. In effect, at these additional refinement times, we
do another global refinement, bringing us to refinement levels 4, 5 and finally
6.

\begin{figure}
\phantom.
\hfill
\includegraphics[width=0.4\textwidth]{cookbooks/convection-box/steps_unknowns}
\hfill
\includegraphics[width=0.4\textwidth]{cookbooks/convection-box/steps_heatflux}
\hfill
\phantom.
\caption{\it Convection in a box: Refinement in stages. Total number
of unknowns in each time step, including all velocity, pressure and
temperature unknowns (left) and heat flux across the top boundary (right).}
\label{fig:convection-box-stats-steps}
\end{figure}


Fig.~\ref{fig:convection-box-stats-steps} shows the results. In the left panel,
we see how the number of unknowns grows over time (note the logscale for the
$y$-axis). The right panel displays the heat flux. The jumps in the number of
cells is clearly visible in this picture as well. This may be surprising at
first but remember that the mesh is clearly too coarse in the beginning to
really resolve the flow and so we should expect that the solution changes
significantly if the mesh is refined. This effect becomes smaller with every
additional refinement and is barely visible at the last time this happens,
indicating that the mesh before this refinement step may already have been fine
enough to resolve the majority of the dynamics.

In any case, we can compare the heat fluxes we obtain at the end of these
computations: With a globally four times refined mesh, we get a value of 4.931
(an error of approximately 1\% against the accepted value from Blankenbach,
$4.884409\pm 0.00001$). With a globally five times refined mesh we get 4.914 (an
error of 0.6\%) and with the mesh generated using the procedure above we get
4.895 with the four digits printed on the screen%
\footnote{The statistics file gives this
value to more digits: 4.89488768. However, these are clearly more digits than
the result is accurate.}
(corresponding to an error of 0.2\%). In other words, our
simple procedure of refining the mesh during the simulation run yields an
accuracy of three times smaller than using the globally refined approach even
though the compute time is not much larger than that necessary for the 5 times
globally refined mesh.


\paragraph{Play time 3: Changing the finite element in use.}
Another way to increase the accuracy of a finite element computation is to use a
higher polynomial degree for the finite element shape functions. By default,
\aspect{} uses quadratic shape functions for the velocity and the temperature
and linear ones for the pressure. However, this can be changed with a single
number in the input file.

Before doing so, let us consider some aspects of such a change. First, looking
at the pictures of the solution in Fig.~\ref{fig:convection-box-fields}, one
could surmise that the quadratic elements should be able to resolve the velocity
field reasonably well given that it is rather smooth. On the other hand, the
temperature field has a boundary layer at the top and bottom. One could
conjecture that the temperature polynomial degree is therefore the limiting
factor and not the polynomial degree for the flow variables. We will test this
conjecture below. Secondly, given the nature of the equations, increasing the
polynomial degree of the flow variables increases the cost to solve these
equations by a factor of $\frac{22}{9}$ in 2d (you can get this factor by
counting the number of degrees of freedom uniquely associated with each cell) but leaves
the time step size and the cost of solving the temperature system unchanged. On
the other hand, increasing the polynomial degree of the temperature variable
from 2 to 3 requires $\frac 94$ times as many degrees of freedom for the
temperature and also requires us to reduce the size of the time step by a factor
of $\frac 23$. Because solving the temperature system is not a dominant factor
in each time step (see the timing results shown at the end of the screen output
above), the reduction in time step is the only important factor. Overall,
increasing the polynomial degree of the temperature variable turns out to be the
cheaper of the two options.

Following these considerations, let us add the following section to the
parameter file:
\lstinputlisting[language=prmfile]{cookbooks/convection-box/disc.part.prm.out}

This leaves the velocity and pressure shape functions at quadratic and linear
polynomial degree but increases the polynomial degree of the temperature from
quadratic to cubic. Using the original, four times globally refined mesh, we
then get the following output:
\begin{lstlisting}[frame=single,language=ksh]
Number of active cells: 256 (on 5 levels)
Number of degrees of freedom: 4,868 (2,178+289+2,401)

*** Timestep 0:  t=0 seconds
   Solving temperature system... 0 iterations.
   Rebuilding Stokes preconditioner...
   Solving Stokes system... 30+5 iterations.

[... ...]

*** Timestep 1619:  t=0.499807 seconds
   Solving temperature system... 8 iterations.
   Solving Stokes system... 5 iterations.

   Postprocessing:
     RMS, max velocity:                  42.9 m/s, 69.5 m/s
     Temperature min/avg/max:            0 K, 0.5 K, 1 K
     Heat fluxes through boundary parts: -0.004622 W, 0.004624 W, -4.878 W, 4.878 W


+---------------------------------------------+------------+------------+
| Total wallclock time elapsed since start    |       127s |            |
|                                             |            |            |
| Section                         | no. calls |  wall time | % of total |
+---------------------------------+-----------+------------+------------+
| Assemble Stokes system          |      1620 |      3.03s |       2.4% |
| Assemble temperature system     |      1620 |      75.7s |        60% |
| Build Stokes preconditioner     |         1 |    0.0422s |     0.033% |
| Build temperature preconditioner|      1620 |      21.7s |        17% |
| Solve Stokes system             |      1620 |      10.3s |       8.1% |
| Solve temperature system        |      1620 |       4.9s |       3.8% |
| Initialization                  |         2 |    0.0246s |     0.019% |
| Postprocessing                  |      1620 |      8.05s |       6.3% |
| Setup dof systems               |         1 |    0.0438s |     0.034% |
+---------------------------------+-----------+------------+------------+
\end{lstlisting}

Note here that the heat flux through the top and bottom boundaries is now
computed as 4.878, an error of 0.13\%. This is 4 times more accurate than the
once more globally refined mesh with the original quadratic elements, at a cost
significantly smaller. Furthermore, we can of course combine this with the mesh
that is gradually refined as simulation time progresses, and we then get a heat
flux that is equal to 4.8843, only 0.002\% away from the accepted value!

As a final remark, to test our hypothesis that it was indeed the temperature
polynomial degree that was the limiting factor, we can increase the Stokes
polynomial degree to 3 while leaving the temperature polynomial degree at 2. A
quick computation shows that in that case we get a heat flux of 4.931 -- exactly
the same value as we got initially with the lower order Stokes element. In other
words, at least for this testcase, it really was the temperature variable that
limits the accuracy.


\subsubsection{Convection in a 3d box}
\label{sec:cookbooks-simple-box-3d}

The world is not two-dimensional. While the previous section introduced a number
of the knobs one can play with with \aspect{}, things only really become
interesting once one goes to 3d. The setup from the previous section is easily
adjusted to this and in the following, let us walk through some of the changes
we have to consider when going from 2d to 3d. The full input file that
contains these modifications and that was used for the simulations we will show
subsequently can be found at \url{cookbooks/convection-box-3d.prm}.

The first set of changes has to do with the geometry: it is three-dimensional,
and we will have to address the fact that a box in 3d has 6 sides, not the 4 we
had previously. The documentation of the ``box'' geometry
(see Section~\ref{parameters:Geometry_20model}) states that these sides are
numbered as follows: ``\textit{in 3d, boundary indicators 0 through 5 indicate
left, right, front, back, bottom and top boundaries}.'' Recalling that we want
tangential flow all around and want to fix the temperature to known values at
the bottom and top, the following will make sense:
\lstinputlisting[language=prmfile]{cookbooks/convection-box-3d/start.part.prm.out}


The next step is to describe the initial conditions. As before, we will use an
instably layered medium but the perturbation now needs to be both in $x$- and
$y$-direction
\lstinputlisting[language=prmfile]{cookbooks/convection-box-3d/initial.part.prm.out}

The third issue we need to address is that we can likely not afford a mesh as
fine as in 2d. We choose a mesh that is refined 3 times globally at the
beginning, then 3 times adaptively, and is then adapted every 15 time steps. We
also allow one additional mesh refinement in the first time step following
$t=0.003$ once the initial instability has given way to a more stable pattern:
\lstinputlisting[language=prmfile]{cookbooks/convection-box-3d/amr.part.prm.out}

Finally, as we have seen in the previous section, a computation with $Ra=10^4$
does not lead to a simulation that is exactly exciting. Let us choose $Ra=10^6$
instead (the mesh chosen above with up to 7 refinement levels after some time
is fine enough to resolve this). We can achieve this in the same way as in the
previous section by choosing $\alpha=10^{-10}$ and setting
\lstinputlisting[language=prmfile]{cookbooks/convection-box-3d/gravity.part.prm.out}
This has some interesting implications. First, a higher Rayleigh number makes
time scales correspondingly smaller; where we generated graphical output only
once every 0.01 time units before, we now need to choose the corresponding
increment smaller by a factor of 100:
\lstinputlisting[language=prmfile]{cookbooks/convection-box-3d/postprocess.part.prm.out}
Secondly, a simulation like this -- in 3d, with a significant number of cells,
and for a significant number of time steps -- will likely take a good amount of
time. The computations for which we show results below was run using 64
processors by running it using the command
{\tt{mpirun -n 64 ./aspect convection-box-3d.prm}}. If the machine should crash
during such a run, a significant amount of compute time would be lost if we had
to run everything from the start. However, we can avoid this by periodically
checkpointing the state of the computation:
\lstinputlisting[language=prmfile]{cookbooks/convection-box-3d/checkpoint.part.prm.out}
If the computation does crash (or if a computation runs out of the time limit
imposed by a scheduling system), then it can be restarted from such
checkpointing files (see the parameter {\tt Resume computation}
in Section~\ref{parameters:global}).
\index[prmindex]{Resume computation}
\index[prmindexfull]{Resume computation}

Running with this input file requires a bit of patience%
\footnote{For computations of this size, one should test a few time steps in
  debug mode but then, of course, switch to running the actual computation in
  optimized mode -- see Section~\ref{sec:debug-mode}.}
since the number of
degrees of freedom is just so large: it starts with a bit over 330,000\ldots
\begin{lstlisting}[frame=single,language=ksh]
Running with 64 MPI tasks.
Number of active cells: 512 (on 4 levels)
Number of degrees of freedom: 20,381 (14,739+729+4,913)

*** Timestep 0:  t=0 seconds
   Solving temperature system... 0 iterations.
   Rebuilding Stokes preconditioner...
   Solving Stokes system... 18 iterations.

Number of active cells: 1,576 (on 5 levels)
Number of degrees of freedom: 63,391 (45,909+2,179+15,303)

*** Timestep 0:  t=0 seconds
   Solving temperature system... 0 iterations.
   Rebuilding Stokes preconditioner...
   Solving Stokes system... 19 iterations.

Number of active cells: 3,249 (on 5 levels)
Number of degrees of freedom: 122,066 (88,500+4,066+29,500)

*** Timestep 0:  t=0 seconds
   Solving temperature system... 0 iterations.
   Rebuilding Stokes preconditioner...
   Solving Stokes system... 20 iterations.

Number of active cells: 8,968 (on 5 levels)
Number of degrees of freedom: 331,696 (240,624+10,864+80,208)

*** Timestep 0:  t=0 seconds
   Solving temperature system... 0 iterations.
   Rebuilding Stokes preconditioner...
   Solving Stokes system... 21 iterations.
[...]
\end{lstlisting}
\ldots{}but then increases quickly to around 2 million as the solution develops
some structure and, after time $t=0.003$ where we allow for an additional
refinement, increases to over 10 million where it then hovers between 8 and 14
million with a maximum of 15,147,534. Clearly, even on a reasonably quick
machine, this will take some time: running this on a machine bought in 2011,
doing the 10,000 time steps to get to $t=0.0219$ takes approximately 484,000
seconds (about five and a half days).

The structure or the solution is easiest to grasp by looking at isosurfaces of
the temperature. This is shown in Fig.~\ref{fig:box-3d-solution} and you can
find a movie of the motion that ensues from the heating at the bottom at
\url{http://www.youtube.com/watch?v=_bKqU_P4j48}. The simulation uses adaptively
changing meshes that are fine in rising plumes and sinking blobs and are coarse
where nothing much happens. This is most easily seen in the movie at
\url{http://www.youtube.com/watch?v=CzCKYyR-cmg}. Fig.~\ref{fig:box-3d-mesh}
shows some of these meshes, though still pictures do not do the evolving nature
of the mesh much justice. The effect of increasing the Rayleigh number is
apparent when comparing the size of features with, for example, the picture at
the right of Fig.~\ref{fig:convection-box-fields}. In contrast to that picture,
the simulation is also clearly non-stationary.

\begin{figure}
  \centering
  \includegraphics[width=0.3\textwidth]{cookbooks/convection-box-3d/movie0010.png}
  \hfill
  \includegraphics[width=0.3\textwidth]{cookbooks/convection-box-3d/movie0040.png}
  \hfill
  \includegraphics[width=0.3\textwidth]{cookbooks/convection-box-3d/movie0060.png}
  \\
  \includegraphics[width=0.3\textwidth]{cookbooks/convection-box-3d/movie0100.png}
  \hfill
  \includegraphics[width=0.3\textwidth]{cookbooks/convection-box-3d/movie0130.png}
  \hfill
  \includegraphics[width=0.3\textwidth]{cookbooks/convection-box-3d/movie0180.png}
  \caption{\it Convection in a 3d box: Temperature isocontours and some
  velocity vectors at the first time step after times $t=0.001, 0.004, 0.006$
  (top row, left to right) an $t=0.01, 0.013, 0.018$ (bottom row).}
  \label{fig:box-3d-solution}
\end{figure}


\begin{figure}
  \centering
  \includegraphics[width=0.3\textwidth]{cookbooks/convection-box-3d/mesh0060.png}
  \hfill
  \includegraphics[width=0.3\textwidth]{cookbooks/convection-box-3d/mesh0100.png}
  \hfill
  \includegraphics[width=0.3\textwidth]{cookbooks/convection-box-3d/mesh0180.png}
  \caption{\it Convection in a 3d box: Meshes and large-scale velocity field
  for the third, fourth and sixth of the snapshots shown in
  Fig.~\ref{fig:box-3d-solution}.}
  \label{fig:box-3d-mesh}
\end{figure}

As before, we could analyze all sorts of data from the statistics file but we
will leave this to those interested in specific data. Rather,
Fig.~\ref{fig:box-3d-heat-flux} only shows the upward heat flux through the
bottom and top boundaries of the domain as a function of time.%
\footnote{Note that the statistics file actually contains the \textit{outward}
heat flux for each of the six boundaries, which corresponds to the
\textit{negative} of upward flux for the bottom boundary. The figure therefore
shows the negative of the values available in the statistics file.}
The figure reinforces a pattern that can also be seen by watching the movie of
the temperature field referenced above, namely that the simulation can be
subdivided into three distinct phases. The first phase corresponds to the
initial overturning of the instable layering of the temperature field and is
associated with a large spike in heat flux as well as large velocities (not
shown here). The second phase, until approximately $t=0.01$ corresponds to a
relative lull: some plumes rise up, but not very fast because the medium is now
stably layered but not fully mixed. This can be seen in the relatively low heat
fluxes, but also in the fact that there are almost horizontal temperature
isosurfaces in the second of the pictures in Fig.~\ref{fig:box-3d-solution}.
After that, the general structure of the temperature field is that the interior
of the domain is well mixed with a mostly constant average temperature and thin
thermal boundary layers at the top and bottom from which plumes rise and sink.
In this regime, the average heat flux is larger but also more variable depending
on the number of plumes currently active. Many other analyses would be possible
by using what is in the statistics file or by enabling additional
postprocessors.

\begin{figure}
  \centering
  \includegraphics[width=0.6\textwidth]{cookbooks/convection-box-3d/heat-flux.png}
  \caption{\it Convection in a 3d box: Upward heat flux through the bottom and
  top boundaries as a function of time.}
  \label{fig:box-3d-heat-flux}
\end{figure}

\subsubsection{Convection in a box with prescribed, variable velocity boundary conditions}
\label{sec:cookbooks-platelike}

A similarly simple setup to the ones considered in the previous subsections is
to equip the model we had with a different set of boundary conditions. There, we used slip boundary
conditions, i.e., the fluid can flow tangentially along the four sides of our
box but this tangential velocity is unspecified. On the other hand, in many
situations, one would like to actually prescribe the tangential flow velocity as
well. A typical application would be to use boundary conditions at the top that
describe experimentally determined velocities of plates. This cookbook shows a
simple version of something like this. To make it slightly more interesting, we
choose a $2\times 1$ domain in 2d.

Like for many other things, \aspect{} has a set of plugins for prescribed
velocity boundary values (see
Sections~\ref{parameters:Boundary_20velocity_20model} and
\ref{sec:prescribed-velocity-boundary-conditions}). These plugins allow one to
write sophisticated models for the boundary velocity on parts or all of the
boundary, but there is also one simple implementation that just takes a formula
for the components of the velocity.

To illustrate this, let us consider the \url{cookbooks/platelike-boundary.prm}
input file. It essentially extends the input file considered in the previous example.
The part of this file that we are particularly interested in in the current
context is the selection of the kind of boundary conditions on the four
sides of the box geometry, which we do using a section like this:
\lstinputlisting[language=prmfile]{cookbooks/platelike-boundary/modelsettings.part.prm.out}

Following the convention for numbering boundaries described in the previous
section, this means that we prescribe a fixed temperature at the bottom and top sides of the box (boundary
numbers two and three). We use tangential flow at boundaries zero, one and two
(left, right and bottom).
Finally, the last entry above is a comma separated list (here with only a single element) of pairs consisting of the
number of a boundary and the name of the prescribed velocity boundary model to
be used on this boundary. Here, we use the \texttt{function} boundary model,
which allows us to provide a function-like notation for the components of the
velocity vector at the boundary.

The second part we need is that we actually describe the function that sets the
velocity. We do this as follows:
\lstinputlisting[language=prmfile]{cookbooks/platelike-boundary/boundary.part.prm.out}
The first of these gives names to the components of the position vector (here,
we are in 2d and we use $x$ and $z$ as spatial variable names) and the time.
We could have left this entry at its default, \texttt{x,y,t}, but since we
often think in terms of ``depth'' as the vertical direction, let us use
\texttt{z} for the second coordinate.
In the second parameter we define symbolic constants that can be used
in the formula for the velocity that is specified in the last parameter. This
formula needs to have as many components as there are space dimensions,
separated by semicolons. As stated, this means that we prescribe the
(horizontal) $x$-velocity and set the vertical velocity to zero. The horizontal
component is here either $1$ or $-1$, depending on whether we are to the right
or the left of the point $1+\sin(\pi t/2)$ that is moving back and forth with
time once every four time units. The \texttt{if} statement understood by the
parser we use for these formulas has the syntax
\texttt{if(condition, value-if-true, value-if-false)}.

\note{While you can enter most any expression into the parser for these
velocity boundary conditions, not all make sense. In particular, if you use an
incompressible medium like we do here, then you need to make sure that either
the flow you prescribe is indeed tangential, or that at least the flow into and
out of the boundary this function applies to is balanced so that in sum the
amount of material in the domain stays constant.

It is in general not possible for \aspect{} to verify that a given input is
sensible. However, you will quickly find out if it isn't: The linear solver for
the Stokes equations will simply not converge. For example, if your function
expression in the input file above read \\
\hspace*{.25cm} \texttt{if(x>1+sin(0.5*pi*t), 1, -1); 1}\\
then at the time of writing this you would get the following error message: \\
\hspace*{.25cm}\texttt{*** Timestep 0:  t=0 seconds} \\
\hspace*{.25cm}\texttt{   Solving temperature system... 0 iterations.} \\
\hspace*{.25cm}\texttt{   Rebuilding Stokes preconditioner...} \\
\hspace*{.25cm}\texttt{   Solving Stokes system... } \\
\\
\hspace*{.25cm}\texttt{\ldots some timing output \ldots} \\
\\
\\
\hspace*{.25cm}\texttt{----------------------------------------------------} \\
\hspace*{.25cm}\texttt{Exception on processing: } \\
\hspace*{.25cm}\texttt{Iterative method reported convergence failure in step
9539 with residual 6.0552} \\
\hspace*{.25cm}\texttt{Aborting!} \\
\hspace*{.25cm}\texttt{----------------------------------------------------}

The reason is, of course, that there is no incompressible (divergence free) flow
field that allows for a constant vertical outflow component along the top
boundary without corresponding inflow anywhere else.}


The remainder of the setup is described in the following, complete input file:
\lstinputlisting[language=prmfile]{cookbooks/platelike-boundary/platelike.prm.out}


This model description yields a setup with a Rayleigh number of 200 (taking
into account that the domain has size 2). It would, thus, be dominated by heat
conduction rather than convection if the prescribed velocity boundary conditions
did not provide a stirring action. Visualizing the results of this simulation%
\footnote{In fact, the pictures are generated using a twice more refined mesh
to provide adequate resolution. We keep the default setting of five
global refinements in the parameter file as documented above to keep compute
time reasonable when using the default settings.}
yields images like the ones shown in Fig.~\ref{fig:platelike}.

\begin{figure}
  \centering
  \includegraphics[width=0.3\textwidth]{cookbooks/platelike-boundary/visit0000.png}
  \hfill
  \includegraphics[width=0.3\textwidth]{cookbooks/platelike-boundary/visit0001.png}
  \hfill
  \includegraphics[width=0.3\textwidth]{cookbooks/platelike-boundary/visit0003.png}
  \\
  \includegraphics[width=0.3\textwidth]{cookbooks/platelike-boundary/visit0004.png}
  \hfill
  \includegraphics[width=0.3\textwidth]{cookbooks/platelike-boundary/visit0005.png}
  \hfill
  \includegraphics[width=0.3\textwidth]{cookbooks/platelike-boundary/visit0006.png}
  \caption{\it Variable velocity boundary conditions: Temperature and velocity
  fields at the initial time (top left) and at various other points in time during the
  simulation.}
  \label{fig:platelike}
\end{figure}


\subsubsection{Using passive and active compositional fields}
\label{sec:cookbooks-composition}

One frequently wants to track where material goes, either because one simply
wants to see where stuff ends up (e.g., to determine if a particular model
yields mixing between the lower and upper mantle) or because the material model
in fact depends not only pressure, temperature and location but also on the
mass fractions of certain chemical or other species. We will refer to the first
case as \textit{passive} and the latter as \textit{active} to indicate the role
of the additional quantities whose distribution we want to track. We refer to
the whole process as \textit{compositional} since we consider quantities that
have the flavor of something that denotes the composition of the material at any
given point.

There are basically two ways to achieve this: one can advect a set of
particles (``tracers'') along with the velocity field, or one can advect along a
field. In the first case, where the closest particle came from indicates the
value of the concentration at any given position. In the latter case, the
concentration(s) at any given position is simply given by the value of the
field(s) at this location.

\aspect{} implements both strategies, at least to a certain degree. In this
cookbook, we will follow the route of advected fields.

\paragraph{The passive case.}
We will consider the
exact same situation as in the previous section but we will ask where the
material that started in the bottom 20\% of the domain
ends up, as well as the material that started in the top 20\%. For the moment,
let us assume that there is no material between the materials at the bottom, the
top, and the middle. The way to describe this situation is to simply add the
following block of definitions to the parameter file (you can find the full
parameter file in \url{cookbooks/composition-passive.prm}:

\lstinputlisting[language=prmfile]{cookbooks/composition-passive/passive.part.prm.out}

Running this simulation yields results such as the ones shown in
Fig.~\ref{fig:compositional-passive} where we show the values of the functions
$c_1(\mathbf x,t)$ and $c_2(\mathbf x,t)$ at various times in the simulation.
Because these fields were one only inside the lowermost and uppermost parts of
the domain, zero everywhere else, and because they have simply been advected
along with the flow field, the places where they are larger than one half
indicate where material has been transported to so far.%
\footnote{Of course, this interpretation suggests that we could have achieved
the same goal by encoding everything into a single function -- that would, for
example, have had initial values one, zero and minus one in the three parts of
the domain we are interested in.}

\begin{figure}
  \centering
  \includegraphics[width=0.3\textwidth]{cookbooks/composition-passive/visit0007.png}
  \hfill
  \includegraphics[width=0.3\textwidth]{cookbooks/composition-passive/visit0008.png}
  \hfill
  \includegraphics[width=0.3\textwidth]{cookbooks/composition-passive/visit0009.png}
  \\
  \includegraphics[width=0.3\textwidth]{cookbooks/composition-passive/visit0010.png}
  \hfill
  \includegraphics[width=0.3\textwidth]{cookbooks/composition-passive/visit0012.png}
  \hfill
  \includegraphics[width=0.3\textwidth]{cookbooks/composition-passive/visit0014.png}
  \caption{\it Passive compositional fields: The figures show, at
    different times in the simulation, the velocity field along with
    those locations where the first compositional field is larger than
    0.5 (in red, indicating the locations where material from the bottom
    of the domain has gone) as well as where the second compositional
    field is larger than 0.5 (in blue, indicating material from the top
    of the domain. The results were obtained with two more global
    refinement steps compared to the
    \url{cookbooks/composition-passive.prm} input file.}
  \label{fig:compositional-passive}
\end{figure}

\begin{figure}
  \centering
  \includegraphics[height=0.3\textwidth]{cookbooks/composition-passive/visit0015.png}
  \hfill
  \includegraphics[height=0.3\textwidth]{cookbooks/composition-passive/visit0017.png}
  \caption{\it Passive compositional fields: A later image of the simulation
    corresponding to the sequence shown in
    Fig.~\ref{fig:compositional-passive} (left) and zoom-in on the
    center, also showing the mesh (right).}
  \label{fig:compositional-passive-zoom}
\end{figure}


Fig.~\ref{fig:compositional-passive} shows one aspect of compositional
fields that occasionally makes them difficult to use for very long
time computations. The simulation shown here runs for 20 time units,
where every cycle of the spreading center at the top moving left and
right takes 4 time units, for a total of 5 such cycles. While this is
certainly no short-term simulation, it is obviously visible in the
figure that the interface between the materials has diffused over
time. Fig.~\ref{fig:compositional-passive-zoom} shows a zoom into the
center of the domain at the final time of the simulation. The
figure only shows values that are larger than 0.5, and it looks like
the transition from red or blue to the edge of the shown region is no
wider than 3 cells. This means that the computation is not overly
diffusive but it is nevertheless true that this method has difficulty
following long and thin filaments.%
\footnote{We note that this is no different for tracers where the
  position of tracers has to be integrated over time and is subject to
  numerical error. In simulations, their location is therefore not the
  exact one but also subject to a diffusive process resulting from
  numerical inaccuracies. Furthermore, in long thin filaments, the
  number of tracers per cell often becomes too small and new tracers
  have to be inserted; their properties are then interpolated from the
  surrounding tracers, a process that also incurs a smoothing penalty.}
This is an area in which \aspect{} may see improvements in the future.


\begin{figure}
  \centering
  \includegraphics[width=0.4\textwidth]{cookbooks/composition-passive/mass-composition-1.png}
  \caption{\it Passive compositional fields: Minimum and maximum of the first
  compositional variable over time, as well as the mass $m_1(t)=\int_\Omega c_1(\mathbf x,t)$ stored in this variable.}
  \label{fig:compositional-passive-mass}
\end{figure}

A different way of looking at the quality of compositional fields as opposed to
tracers is to ask whether they conserve mass. In the current context, the
mass contained in the $i$th compositional field is $m_i(t)=\int_\Omega c_i(\mathbf x,t)$.
This can easily be achieve in the following way, by adding the \texttt{composition statistics}
postprocessor:

\lstinputlisting[language=prmfile]{cookbooks/composition-passive/postprocess.part.prm.out}

While the scheme we use to advect the compositional fields is not strictly
conservative, it is almost perfectly so in practice. For example, in
the computations shown in this section (using two additional global mesh
refinements over the settings in the parameter file
\url{cookbooks/composition-passive.prm}), Fig.~\ref{fig:compositional-passive-mass}
shows the maximal and minimal values of the first compositional fields over time,
along with the mass $m_1(t)$ (these are all tabulated in columns of the
statistics file, see Sections~\ref{sec:running-overview} and \ref{sec:viz-stat}). While
the maximum and minimum fluctuate slightly due to the instability of the finite element
method in resolving discontinuous functions,
the mass appears stable at a value of 0.403646 (the exact value, namely the
area that was initially filled by each material, is 0.4; the difference is a
result of the fact that we can't exactly represent the step function on our
mesh with the finite element space). In fact, the maximal difference in this
value between time steps 1 and 500 is only $1.1\cdot 10^{-6}$. In other words,
these numbers show that the compositional field approach is almost exactly mass conservative.


\paragraph{The active case.} The next step, of course, is to make the flow
actually depend on the composition. After all, compositional fields are not only
intended to indicate where material come from, but also to indicate the
properties of this material. In general, the way to achieve this is to write
material models where the density, viscosity, and other parameters depend on the
composition, taking into account what the compositional fields actually denote
(e.g., if they simply indicate the origin of material, or the concentration of
things like olivine, perovskite, \ldots). The construction of material models is
discussed in much greater detail in Section~\ref{sec:material-models}; we do not
want to revisit this issue here and instead choose -- once again -- the simplest
material model that is implemented in \aspect{}: the \texttt{simple} model.

The place where we are going to hook in a compositional dependence is the
density. In the \texttt{simple} model, the density is fundamentally described by
a material that expands linearly with the temperature; for small density
variations, this corresponds to a density model of the form
$\rho(T)=\rho_0(1-\alpha(T-T_0))$. This is, by virtue of its simplicity, the
most often considered density model. But the \texttt{simple} model also has a
hook to make the density depend on the first compositional field $c_1(\mathbf
x,t)$, yielding a dependence of the form
$\rho(T)=\rho_0(1-\alpha(T-T_0))+\gamma c_1$. Here, let us choose $\rho_0=1,
\alpha=0.01, T_0=0, \gamma=100$. The rest of our model setup will be as
in the passive case above. Because the temperature will be between zero and one,
the temperature induced density variations will be restricted to 0.01, whereas
the density variation by origin of the material is 100. This should make sure
that dense material remains at the bottom despite the fact that it is hotter
than the surrounding material.%
\footnote{The actual values do not matter as much here. They are chosen in such
a way that the system -- previously driven primarily by the velocity boundary
conditions at the top -- now also feels the impact of the density variations.
To have an effect, the buoyancy induced by the density difference between
materials must be strong enough to balance or at least approach the forces
exerted by whatever is driving the velocity at the top.}

This setup of the problem can be described using an input file that is almost
completely unchanged from the passive case. The only difference is the use of
the following section (the complete input file can be found in
\url{cookbooks/composition-active.prm}:


\lstinputlisting[language=prmfile]{cookbooks/composition-active/active.part.prm.out}

To debug the model, we will also want to visualize the density in our
graphical output files. This is done using the following addition to the
postprocessing section, using the \texttt{density} visualization plugin:

\lstinputlisting[language=prmfile]{cookbooks/composition-active/postprocess.part.prm.out}

\begin{figure}
  \centering
  \centering
  \includegraphics[width=0.3\textwidth]{cookbooks/composition-active/visit0007.png}
  \hfill
  \includegraphics[width=0.3\textwidth]{cookbooks/composition-active/visit0009.png}
  \hfill
  \includegraphics[width=0.3\textwidth]{cookbooks/composition-active/visit0008.png}
  \caption{\it Active compositional fields: Compositional field 1 at the time
    $t=0, 10, 20$. Compared to the results shown in
    Fig.~\ref{fig:compositional-passive} it is clear that the heavy material
    stays at the bottom of the domain now. The effect of the density on the
    velocity field is also clearly visible by noting that at all three times
    the spreading center at the top boundary is in exactly the same position;
    this would result in exactly the same velocity field if the density and
    temperature were constant.}
  \label{fig:composition-active-composition}
\end{figure}

\begin{figure}
  \centering
  \includegraphics[width=0.3\textwidth]{cookbooks/composition-active/visit0000.png}
  \hfill
  \includegraphics[width=0.3\textwidth]{cookbooks/composition-active/visit0001.png}
  \hfill
  \includegraphics[width=0.3\textwidth]{cookbooks/composition-active/visit0002.png}
  \\
  \includegraphics[width=0.3\textwidth]{cookbooks/composition-active/visit0003.png}
  \hfill
  \includegraphics[width=0.3\textwidth]{cookbooks/composition-active/visit0004.png}
  \hfill
  \includegraphics[width=0.3\textwidth]{cookbooks/composition-active/visit0006.png}
  \caption{\it Active compositional fields: Temperature fields at $t=0, 2, 4, 8,
  12, 20$. The black line is the isocontour line $c_1(\mathbf x,t)=0.5$
    delineating the position of the dense material at the bottom.}
  \label{fig:composition-active-temperature}
\end{figure}

Results of this model are visualized in
Fig.s~\ref{fig:composition-active-composition} and \ref{fig:composition-active-temperature}. What is visible is
that over the course of the simulation, the material that starts at the bottom
of the domain remains there. This can only happen if the circulation is
significantly affected by the high density material once the interface starts
to become non-horizontal, and this is
indeed visible in the velocity vectors. As a second consequence, if the
material at the bottom does not move away, then there needs to be a different
way for the heat provided at the bottom to get through the bottom layer:
either there must be a secondary convection system in the bottom layer, or
heat is simply conducted. The pictures in the figure seem to suggest
that the latter is the case.

It is easy, using the
outline above, to play with the various factors that drive this system, namely:
\begin{itemize}
  \item The magnitude of the velocity prescribed at the top.
  \item The magnitude of the velocities induced by thermal buoyancy, as
  resulting from the magnitude of gravity and the thermal expansion coefficient.
  \item The magnitude of the velocities induced by compositional variability, as
  described by the coefficient $\gamma$ and the magnitude of gravity.
\end{itemize}
Using the coefficients involved in these considerations, it is trivially
possible to map out the parameter space to find which of these effects is
dominant. As mentioned in discussing the values in the input file, what is
important is the \textit{relative} size of these parameters, not the fact
that currently the density in the material at the bottom is 100 times larger
than in the rest of the domain, an effect that from a physical perspective
clearly makes no sense at all.


\paragraph{The active case with reactions.}

\textit{This section was contributed by Juliane Dannberg and Ren{\'e} Ga{\ss}m{\"o}ller}.

In addition, there are setups where one wants the compositional fields to interact with each other. One example would be material upwelling at a mid-ocean ridge and changing the composition to that of oceanic crust when it reaches a certain depth. In this cookbook, we will describe how this kind of behaviour can be achieved by using the \texttt{composition reaction} function of the material model. 

We will consider the exact same setup as in the previous paragraphs, except for the initial conditions and properties of the two compositional fields. There is one material that initially fills the bottom half of the domain and is less dense than the material above. In addition, there is another material that only gets created when the first material reaches the uppermost 20\% of the domain, and that has a higher density. This should cause the first material to move upwards, get partially converted to the second material, which then sinks down again. This means we want to change the initial conditions for the compositional fields: 

\lstinputlisting[language=prmfile]{cookbooks/composition-reaction/initial.part.prm.out}


Moreover, instead of the \texttt{simple} material model, we will use the \texttt{composition reaction} material model, which basically behaves in the same way, but can handle two active compositional field and a reaction between those two fields. In the input file, the user defines a depth and above this \texttt{reaction depth} the first compositional fields is converted to the second field. This can be done by changing the following section (the complete input file can be found in \url{cookbooks/composition-reaction.prm}). 

\lstinputlisting[language=prmfile]{cookbooks/composition-reaction/material.part.prm.out}

\begin{figure}
  \centering
  \includegraphics[width=0.3\textwidth]{cookbooks/composition-reaction/0.png}
  \hfill
  \includegraphics[width=0.3\textwidth]{cookbooks/composition-reaction/2.png}
  \hfill
  \includegraphics[width=0.3\textwidth]{cookbooks/composition-reaction/4.png}
  \\
  \includegraphics[width=0.3\textwidth]{cookbooks/composition-reaction/8.png}
  \hfill
  \includegraphics[width=0.3\textwidth]{cookbooks/composition-reaction/12.png}
  \hfill
  \includegraphics[width=0.3\textwidth]{cookbooks/composition-reaction/20.png}
  \caption{\it Reaction between compositional fields: Temperature fields at $t=0, 2, 4, 8,
  12, 20$. The black line is the isocontour line $c_1(\mathbf x,t)=0.5$
    delineating the position of the material starting at the bottom and the white line is the    isocontour line $c_2(\mathbf x,t)=0.5$
    delineating the position of the material that is created by the reaction.}
  \label{fig:composition-reaction}
\end{figure}

Results of this model are visualized in
Fig~\ref{fig:composition-reaction}. What is visible is
that over the course of the simulation, the material that starts at the bottom
of the domain ascends, reaches the reaction depth and gets converted to the second material, which starts to sink down.



\subsubsection{Using tracer particles}

Using compositional fields to trace where material has come from or is going to
has many advantages from a computational point of view. For example, the
numerical methods to advect along fields are well developed and we can do so at
a cost that is equivalent to one temperature solve for each of the compositional
fields. Unless you have many compositional fields, this cost is therefore
relatively small compared to the overall cost of a time step. Another advantage
is that the value of a compositional field is well defined at every point within
the domain. On the other hand, compositional fields over time diffuse initially
sharp interfaces, as we have seen in the images of the previous section.

At the same time, the geodynamics community has a history of using tracers for
this purpose. Historically, this may have been because it is conceptually
simpler to advect along individual particles rather than whole fields, since it
only requires an ODE integrator rather than the stabilization techniques
necessary to advect fields. They also provide the appearance of no diffusion,
though this is arguable. Leaving aside the debate whether fields or particles are the
way to go, \aspect{} supports both: using fields and using tracers.

In order to advect tracer particles along with the flow field, one just needs to
add the \texttt{tracers} postprocessor to the list of postprocessors and specify
a few parameters. We do so in the
\url{cookbooks/composition-passive-tracers.prm} input file, which is otherwise
just a minor variation of the \url{cookbooks/composition-passive.prm} case
discussed in the previous Section~\ref{sec:cookbooks-composition}. In
particular, the postprocess section now looks like this:

\index[prmindex]{Number of tracers}
\index[prmindexfull]{Postprocess!Tracers!Number of tracers}

\lstinputlisting[language=prmfile]{cookbooks/composition-passive-tracers/tracer.part.prm.out}

The 1000 particles we are asking here are initially uniformly distributed
throughout the domain and are, at the end of each time step, advected along with
the velocity field just computed. (There are a number of options to decide which
method to use for advecting particles, see
Section~\ref{parameters:Postprocess/Tracers}.) We can visualize them by opening
both the field-based output files and the ones that correspond to particles
(for example, \texttt{output/solution-00072.visit} and
\texttt{output/particles-00072.visit}) and using a pseudo-color plot for the
particles, selecting the ``id'' of particles to color each particle. This
results in a plot like the one shown in
Fig.~\ref{fig:composition-passive-tracers}.

\begin{figure}
  \centering
  \includegraphics[width=0.5\textwidth]{cookbooks/composition-passive-tracers/solution-00072.png}
  \caption{\it Passively advected quantities visualized through both a
  compositional field and a set of 1,000 particles, at $t=7.2$.}
  \label{fig:composition-passive-tracers}
\end{figure}

The particles shown here are not too impressive in still pictures since they are
colorized by their particle number, which -- since particles were initially
randomly distributed -- is essentially a random number. The purpose of using the
particle id to colorize becomes more apparent if you use it when viewing an
animation of time steps. There, the different colors of adjacent particles come
in handy because they allow the eye to follow the motion of a single particle.
This makes it rather intuitive to understand a flow field, but it can of course
not be reproduced in a static medium such as this manual.

\paragraph{Using tracer properties.}

The particles in the above example only fulfill the purpose of
visualizing the convection pattern. A more meaningful use for
particles is to attach ``properties'' to them. A property consists of
one or more numbers (or vectors or tensors) that may either be set at
the beginning of the model run and stay constant, or are updated
during the model runtime. These properties can then be used for many
applications, e.g., storing an initial property (like the position, or
initial composition), evaluating a property at a defined particle path (like the pressure-temperature evolution of a certain piece of rock), or by integrating a quantity along a particle parth (like the integrated strain a certain domain has experienced).  We illustrate these properties in the cookbook \url{cookbooks/composition-passive-tracers-properties.prm}, in which we add the following lines to the \texttt{Tracers} subsection.

\lstinputlisting[language=prmfile]{cookbooks/composition-passive-tracers/tracer-properties.part.prm.out}

These commands make sure that every tracer will carry four different
properties (\texttt{function}, \texttt{pT path}, \texttt{initial
  position} and \texttt{initial composition}), some of which may be
scalars and others can have multiple components. (A full list of
particle properties that can currently be selected can be found in
Section~\ref{parameters:Postprocess/Tracers}, and new tracer
properties can be added as plugins as described in
Section~\ref{sec:write-plugin}.) The properties selected above do the following:

\begin{itemize}
\item \texttt{initial position:} This particle property simply stores
  the initial position of the particle and then never changes it. This
  can be useful to compare the final position of a particle with its initial position and therefore determine how far certain domains traveled during the model runtime.
\item \texttt{initial composition:} This property uses the same
  method to initialize particle properties as is used to initialize
  the corresponding compositional fields. Using this, it stores the
  compositional field initialization values at the location where the particle
  started, and again never changes them. This is useful in the same
  context as shown for the field-based example in
  Section~\ref{sec:cookbooks-composition} where we would like to
  figure where materials ends up. In this case, one would set the
  initial composition to an indicator function for certain parts of
  the domain, and then set the initial composition property for the
  particles to match this composition. Letting the particles advect
  and at a later time visualizing this particle property will then
  show where particles came from. In cases where compositional
  variables undergo changes, e.g., by describing phase changes or
  chemical reactions, the ``initial composition'' property can also be
  useful to compare the final composition of a particle with its
  initial composition and therefore determine which regions underwent
  reactions such as those described in Section~\ref{sec:cookbooks-composition},
  and where the material that underwent this reaction got transported
  to. 
\item \texttt{function:} This particle property can be used to assign
  to each particle values that are described based on a function of
  space. It provides an alternative way to set initial values if you
  don't want to first set a compositional field's initial values based
  on a function, and then copy these values via the ``initial
  composition'' property to particles. In the example above, we use
  the same function as for the compositional initial composition of
  field number one in
  Section~\ref{sec:cookbooks-composition}. Therefore, this property
  should behave identical to the compositional field (except that the
  compositional field may have a reaction term that this particle
  property does not), although the two are of course advected using
  very different methods. This allows to compare the error in tracer
  position to the numerical diffusion of the compositional field.
\item \texttt{pT path:} This property is interesting in that the
  particle property's values always exactly mirror the pressure and
  temperature at the particle's current location. This does not seem
  to be very useful since the information is already
  avaiable. However, because each particle has a unique id, one can
  select a particular tracer particle and output its properties
  (including pressure and temperature based on the \texttt{pT path}
  property) at all time steps. This allows for the creation of a pressure-temperature curve of a certain piece of rock. This property is interesting in many lithosphere and crustal scale models, because it is determining the metamorphic reactions that happen during deformation processes (e.g., in a subduction zone).
\end{itemize}

\paragraph{Using active tracers.}
In the examples above, tracer properties passively track distinct
model properties.  These tracer properties, however, may also be used
to actively influence the model as it runs.  For instance, a
composition-dependent material model may use particles' initial
composition rather than an advected compositional field. To make this
work -- i.e., to get information from particles that are located at
unpredictable locations, to the quadrature
points at which material models and other parts of the code need to
evaluate these properties -- we need to somehow get the values from
particles back to fields that can then be evaluated at any point where
this is necessary.
A slightly modified version of the active-composition cookbook (\url{cookbooks/composition-active.prm}) illustrates how to use `active tracers' in this manner.

This cookbook, \url{cookbooks/composition-active-tracers.prm}, modifies two sections of the input file.  First, tracers are added under the \texttt{Postprocess} section: 

\lstinputlisting[language=prmfile]{cookbooks/composition-active-tracers/tracers.part.prm.out}
Here, each particle will carry the \texttt{velocity} and
\texttt{initial composition} properties.  In order to use the tracer initial composition value to modify the flow through the material model, we now modify the \texttt{Composition} section:

\lstinputlisting[language=prmfile]{cookbooks/composition-active-tracers/composition.part.prm.out}
 
What this does is the following: It says that there will be two
compositional fields, called \texttt{lower} and \texttt{upper}
(because we will use them to indicate material that comes from either
the lower or upper part of the domain). Next, the
\texttt{Compositional field methods} states that each of these fields
will be computed by interpolation from the particles (if we had left
this parameter at its default value, \texttt{field}, for each field,
then it would have solved an advection PDE in each time step, as we
have done in all previous examples).

In this case, we specify that both of the compositional fields are in
fact interpolated from particle properties in each time step. How this
is done is described in the fourth line. To understand it, it is
important to realize that particles and fields have matching names: We
have named the fields \texttt{lower} and \texttt{upper}, whereas the
properties that result from the \texttt{initial composition} entry in
the particles section are called \texttt{initial lower} and
\texttt{initial upper}, since they inherit the names of the fields.

The syntax for interpolation from particles to fields then
states that the \texttt{lower} field will be set to the interpolated
value of the \texttt{initial lower} particle property at the end of
each time step, and similarly
for the \texttt{upper} field. In turn, the
\texttt{initial composition} particle property was using the same
method that one would have used for the compositional field
initialization if these fields were actually advected along in each
time step.
 
In this model the given global refinement level (5), associated number of cells (1024) and 100,000 total particles produces an average particle-per-cell count slightly below 100.  While on the high end compared to most geodynamic studies using active particles, increasing the number of particles per cell further may alter the solution.  As with the numerical resolution, any study using active particles should systematically vary the number of particles per cell in order to determine this parameter's influence on the simulation.

\note{\aspect{}'s tracer implementation is in a preliminary state. While the accuracy and scalability of the implementation is benchmarked, other limitations remain. This in particular means that it is not optimized for performance, and more than a few thousand tracers per process can slow down a model significantly. Moreover, models with a highly adaptive mesh and many tracers do encounter a significant slowdown, because \aspect{} only considers the number of degrees of freedom for load balancing across processes and not the number of tracers. Therefore processes that compute the solution for coarse-grid regions have to process many more tracers than other processes. Additionally, the checkpoint/restart functionality for tracers is only implemented in models with a constant number of processes before and after the checkpoint and when the selected tracer properties do not change. These limitations might be removed over time, but for current models the user should be aware of them.}


\subsubsection{Using a free surface}
\label{sec:cookbooks-freesurface}
\textit{This section was contributed by Ian Rose}.

Free surfaces are numerically challenging but can be useful for self consistently
tracking dynamic topography and may be quite important as a boundary condition
for tectonic processes like subduction.  The parameter file \url{cookbooks/free-surface.prm} 
provides a simple example of how to set up a model with a free surface, as well 
as demonstrates some of the challenges associated with doing so.

\aspect{} supports models with a free surface using an Arbitrary Lagrangian-Eulerian 
framework (see Section~\ref{sec:freesurface}).  Most of this is done internally, so you do not need to worry about the
details to run this cookbook.  Here we demonstrate the evolution of surface topography 
that results when a blob of hot material rises in the mantle, pushing up the free
surface as it does.  Usually the amplitude of free surface topography 
will be small enough that it is difficult to see with the naked eye in visualizations,
but the \texttt{topography} postprocessor can help by outputting the maximum and minumum 
topography on the free surface at every time step. 

The bulk of the parameter file for this cookbook is similar to previous ones in this manual.
We use initial temperature conditions that set up a hot blob of rock in the center of the 
domain. In the \texttt{Model settings} secion you need to give \aspect{} a comma 
separated list of the free surface boundary indicators.  In this case, we are 
dealing with the top boundary of a box in 2D, corresponding to boundary indicator 3.  

The main addition is the \texttt{Free surface} subsection.  There is one main
parameter that needs to be set here: the value for the stabilization parameter ``theta''.
If this parameter is zero, then there is no stabilization, and you are likely to
see instabilities develop in the free surface.  If this parameter is one then it
will do a good job of stabilizing the free surface, but it may overly damp its 
motions.  The default value is 0.5.

Also worth mentioning is the change to the CFL number. Stability concerns typically 
mean that when making a model with a free surface you will want to take smaller 
time steps.  In general just how much smaller will depend on the problem at hand
as well as the desired accuracy.  

Following are the sections in the input file specific to this testcase.  The full parameter
file may be found at \url{cookbooks/free-surface.prm}.

\lstinputlisting[language=prmfile]{cookbooks/free_surface/freesurface.part.prm.out}

Running this input file will produce results like those in Figure~\ref{fig:freesurface}.
The model starts with a single hot blob of rock which rises in the domain.  As it 
rises, it pushes up the free surface in the middle, creating a topographic high there.
This is similar to the kind of dynamic topography that you might see above a mantle 
plume on Earth.  As the blob rises and diffuses, it loses the buoyancy to push up 
the boundary, and the surface begins to relax.

After running the cookbook, you may modify it in a number of ways:
\begin{itemize}
\item Add a more complicated initial temperature field to see how that affects topography.
\item Add a high-viscosity lithosphere to the top using a compositional field to tamp down on topography.
\item Explore different values for the stabilization theta and the CFL number to understand the nature of when and why stabilization is necessary.
\item Try a model in a different geometry, such as spherical shells.
\end{itemize}

\begin{figure}
  \centering
  \includegraphics[height=0.25\textwidth]{cookbooks/free_surface/free_surface_blob.png}
  \hfill
  \includegraphics[height=0.25\textwidth]{cookbooks/free_surface/free_surface_topography.png}
  \caption{\it Evolution of surface topography due to a rising blob.  On the left is a 
           snapshot of the model setup.  The right shows the value of the highest 
           topography in the domain over 18 Myr of model time.  The topography peaks
           at 165 meters after 5.2 Myr.  This cookbook may be run with the
           \url{cookbooks/free-surface.prm} input file.}
  \label{fig:freesurface}
\end{figure}


\subsubsection{Using a free surface in a model with a crust}
\label{sec:cookbooks-freesurfaceWC}

\textit{This section was contributed by William Durkin}.

This cookbook is a modification of the previous example that explores changes in the way topography develops when a 
highly viscous crust is added.  
In this cookbook, we use a material model in which the material changes from low
viscosity mantle to high viscosity crust at $z = z_j = \texttt{jump height}$,
i.e., the piecewise viscosity function is defined as
\begin{align*}
  \eta(z) = \left\{
    \begin{matrix}
      \eta_U & \text{for}\ z > z_j, \\
      \eta_L & \text{for}\ z  \le z_j.
    \end{matrix}
  \right.
\end{align*}
where $\eta_U$ and $\eta_L$ are the viscosities of the upper and lower layers,
respectively. This viscosity model can be implemented by creating a plugin that
is a small modification of the \texttt{simpler} material model (from which it
is otherwise simply copied). We call this material model ``SimplerWithCrust''.
In particular, what is necessary is an evaluation function that looks like this:
\begin{lstlisting}[frame=single,language=C++] 
    template <int dim>
    void
    SimplerWithCrust<dim>::
    evaluate(const typename Interface<dim>::MaterialModelInputs &in, 
              typename Interface<dim>::MaterialModelOutputs &out ) const
    {
      for (unsigned int i=0; i<in.position.size(); ++i)
        { 
          const double z = in.position[i][1];

          if (z>jump_height)
            out.viscosities[i] = eta_U;
          else
            out.viscosities[i] = eta_L;
                     
          out.densities[i] = reference_rho*(1.0-thermal_alpha*(in.temperature[i]-reference_T));
          out.thermal_expansion_coefficients[i] = thermal_alpha;
          out.specific_heat[i] = reference_specific_heat;
          out.thermal_conductivities[i] = k_value;
          out.compressibilities[i] = 0.0;
        }
    }
\end{lstlisting}
Additional changes make the new parameters \texttt{Jump height}, \texttt{Lower
viscosity}, and \texttt{Upper viscosity} available to the input parameter file,
and corresponding variables available in the class and used in the code snippet
above. The entire code can be found in
\url{cookbooks/free-surface-with-crust/plugin/simpler-with-crust.cc}. Refer to
Section~\ref{sec:plugins} for more information about writing and running
plugins.

The following changes are necessary compared to the input file from the
cookbook shown in Section~\ref{sec:cookbooks-freesurface} to include a crust:
\begin{itemize}
  \item Load the plugin implementing the new material model:
  \lstinputlisting[language=prmfile]{cookbooks/free_surface_with_crust/free-surface-wc.part1.prm.out}
  
  \item Declare values for the new parameters:
  \lstinputlisting[language=prmfile]{cookbooks/free_surface_with_crust/free-surface-wc.part2.prm.out}
  Note that the height of the interface at 170km is interpreted in the
  coordinate system in which the box geometry of this cookbook lives. The box
  has dimensions $500\text{km}\times 200\text{km}$, so an interface height of
  170km implies a depth of 30km.
\end{itemize}

The entire script is located in
\url{cookbooks/free-surface-with-crust/free-surface-with-crust.prm}.

Running this input file yields a
crust that is 30km thick and 1000 times as viscous as the lower layer.
Figure~\ref{fig:freesurfaceWC} shows that adding a crust to the model causes the maximum topography to both decrease and occur at a later time.
Heat flows through the system primarily by advection until the temperature anomaly reaches the base of the
crustal layer (approximately at the time for which Fig~\ref{fig:freesurfaceWC}
shows the temperature profile).
The crust's high viscosity reduces the temperature anomaly's velocity
substantially, causing it to affect the surface topography at a later time. Just
as the cookbook shown in Section~\ref{sec:cookbooks-freesurface}, the
topography returns to zero after some time.

\begin{figure}
  \centering
  \includegraphics[height=0.25\textwidth]{cookbooks/free_surface_with_crust/free-surfaceWC.png}
  \hfill
  \includegraphics[height=0.25\textwidth]{cookbooks/free_surface_with_crust/Topography.png}
  \caption{\it Adding a viscous crust to a model with surface topography. The
  thermal anomaly spreads horizontally as it collides with the highly viscous crust (left). The addition of a crustal layer both dampens and delays the appearance of the topographic maximum and minimum (right). }
  \label{fig:freesurfaceWC}
\end{figure}


\subsubsection{Averaging material properties}
\label{sec:sinker-with-averaging}

\textit{The original motivation for the functionality discussed here, as well
  as the setup of the input file, were provided by Cedric Thieulot.}

Geophysical models are often characterized by abrupt and large jumps in material
properties, in particular in the viscosity. An example is a subducting, cold
slab surrounded by the hot mantle: Here, the strong
temperature-dependence of the viscosity will lead to a sudden jump in the
viscosity between mantle and slab. The length scale over which this jump happens
will be a few or a few tens of kilometers. Such length scales cannot be
adequately resolved in three-dimensional computations with typical meshes for
global computations.

Having large viscosity variations in models poses a variety of problems to
numerical computations. First, you will find that they lead to very long compute
times because our solvers and preconditioners break down. This may be
acceptable if it would at least lead to accurate solution, but large viscosity
gradients lead also to large pressure gradients, and this in turn leads to over-
and undershots in the numerical approximation of the gradient. We will 
demonstrate both of these issues experimentally below.

One of the solution to such problems is the realization that one can mitigate
some of the effects by averaging material properties on each cell somehow
(see, for example, \cite{Bab08,Deu08,DMGT11,Thi15,TMK14}).
Before going into detail, it is important to realize that if we choose material
properties not per quadrature point when doing the integrals for forming the
finite element matrix, but per cell, then we will lose accuracy in the solution
in those cases where the solution is smooth. More specifically, we will likely
lose one or more orders of convergence. In other words, it would be a bad idea
to do this averging unconditionally. On the other hand, if the solution has
essentially discontinuous gradients and kinks in the velocity field, then at
least at these locations we cannot expect a particularly high convergence order
anyway, and the averaging will not hurt very much either. In cases where
features of the solution that are due to strongly varying viscosities or other
parameters, dominate, we may then as well do the averaging per cell.

To support such cases, \aspect{} supports an operation where we evaluate the
material model at every quadrature point, given the temperature, pressure,
strain rate, and compositions at this point, and then either (i) use these
values, (ii) replace the values by their arithmetic average $\bar x = \frac 1N
\sum_{i=1}^N x_i$, (iii) replace the values by their harmonic average $\bar x
= \left(\frac 1N \sum_{i=1}^N \frac{1}{x_i}\right)^{-1}$, (iv) replace the
values by their geometric average $\bar x 
= \left(\prod_{i=1}^N \frac{1}{x_i}\right)^{-1/N}$, 
or (v) replace the
values by the largest value over all quadrature points on this cell. Option
(vi) is to project the values from the quadrature points to a bi- (in 2d) or
trilinear (in 3d) $Q_1$ finite element space on every cell, and then evaluate this
finite element representation again at the quadrature points. Unlike the other
five operations, the values we get at the quadrature points are not all the
same here.

We do this operation for all quantities that the material model computes,
i.e., in particular, the viscosity, the density, the compressibility, and the
various thermal and thermodynamic properties. In the first 4 cases, the
operation guarantees that the resulting material properties are bounded below
and above by the minimum and maximum of the original data set. In the last
case, the situation is a bit more complicated: The nodal values of the $Q_1$
projection are not necessarily bounded by the minimal or maximal original
values at the quadrature points, and then neither are the output values after
re-interpolation to the quadrature points. Consequently, after projection, we
limit the nodal values of the projection to the minimal and maximal original
values, and only then interpolate back to the quadrature points.

We demonstrate the effect of all of this with the ``sinker'' benchmark. This
benchmark is defined by a high-viscosity, heavy sphere at the center of a
two-dimensional box. This is achieved by defining a compositional field that is
one inside and zero outside the sphere, and assigning a compositional dependence
to the viscosity and density. We run only a single time step for this benchmark.
This is all modeled in the following input file that can also be found in
\url{cookbooks/sinker-with-averaging/sinker-with-averaging.prm}:
\lstinputlisting[language=prmfile]{cookbooks/sinker-with-averaging/full.prm.out}

The type of averaging on each cell is chosen using this part of the input file:
\lstinputlisting[language=prmfile]{cookbooks/sinker-with-averaging/harmonic.prm.out}
For the various different averaging options, and for different levels of mesh
refinement, Fig.~\ref{fig:sinker-with-averaging-pressure} shows
pressure plots that illustrate the problem with oscillations of the discrete
pressure. The important part of these plots is not that the solution looks
discontinuous -- in fact, the exact solution is discontinuous at the edge of the
circle\footnote{This is also easy to try experimentally -- use the input file
from above and select 5 global and 10 adaptive refinement steps, with the
refinement criteria set to \texttt{density}, then visualize the solution.} --
but the spikes that go far above and below the ``cliff'' in the pressure along 
the edge of the circle. Without averaging, these spikes are obviously orders 
of magnitude larger than the actual jump height. The spikes do not disappear 
under mesh refinement nor averaging, but they become far less pronounced with
averaging. The results shown in the figure do not really allow to draw
conclusions as to which averaging approach is the best; a discussion of this
question can also be found in \cite{Bab08,Deu08,DMGT11,TMK14}).

A very pleasant side effect of averaging is that not only does the solution
become better, but it also becomes cheaper to compute.
Table~\ref{tab:sinker-with-averaging-iteration-counts} shows the
number of outer GMRES iterations when solving the Stokes
equations~\eqref{eq:stokes-1}--\eqref{eq:stokes-2}.%
\footnote{The outer iterations are only part of the problem. As discussed in
  \cite{KHB12}, each GMRES iteration requires solving a linear system with the
  elliptic operator $-\nabla \cdot 2 \eta \varepsilon(\cdot)$. For highly
  heterogeneous models, such as the one discussed in the current section, this
  may require a lot of Conjugate Gradient iterations. For example, for 8
  global refinement steps, the 30+188 outer iterations without averaging shown
  in Table~\ref{tab:sinker-with-averaging-iteration-counts} require a total of
  22,096 inner CG iterations for the elliptic block (and a total of 837 for the
  aproximate Schur complement). Using harmonic averaging, the 30+26 outer
  iterations require only 1258 iterations on the elliptic block (and 84 on the
  Schur complement). In other words, the number of inner iterations per outer
  iteration (taking into account the split into ``cheap'' and ``expensive''
  outer iterations, see \cite{KHB12}) is reduced from 117 to 47 for the
  elliptic block and from 3.8 to 1.5 for the Schur complement.}
The implication of these results is that the averaging gives us a solution
that not only reduces the degree of pressure over- and undershots, but is also
significantly faster to compute: for example, the total run time for 8 global
refinement steps is reduced from 5,250s for no averaging to 358s for harmonic
averaging.


\begin{figure}[htb]
  \centering
  \begin{tabular}{cccccc}
    \includegraphics[width=0.14\textwidth]{cookbooks/sinker-with-averaging/q2q1/sinker-7-none.png}
    &
    \includegraphics[width=0.14\textwidth]{cookbooks/sinker-with-averaging/q2q1/sinker-7-arithmetic.png}
    &
    \includegraphics[width=0.14\textwidth]{cookbooks/sinker-with-averaging/q2q1/sinker-7-harmonic.png}
    &
    \includegraphics[width=0.14\textwidth]{cookbooks/sinker-with-averaging/q2q1/sinker-7-geometric.png}
    &
    \includegraphics[width=0.14\textwidth]{cookbooks/sinker-with-averaging/q2q1/sinker-7-largest.png}
    &
    \includegraphics[width=0.14\textwidth]{cookbooks/sinker-with-averaging/q2q1/sinker-7-project.png}
    \\
    $[-45.2,45.2]$
    &
    $[-2.67,2.67]$
    &
    $[-3.58,3.58]$
    &
    $[-3.57,3.57]$
    &
    $[-1.80,1.80]$
    &
    $[-2.77,2.77]$
    \\
    \\
    \includegraphics[width=0.14\textwidth]{cookbooks/sinker-with-averaging/q2q1/sinker-8-none.png}
    &
    \includegraphics[width=0.14\textwidth]{cookbooks/sinker-with-averaging/q2q1/sinker-8-arithmetic.png}
    &
    \includegraphics[width=0.14\textwidth]{cookbooks/sinker-with-averaging/q2q1/sinker-8-harmonic.png}
    &
    \includegraphics[width=0.14\textwidth]{cookbooks/sinker-with-averaging/q2q1/sinker-8-geometric.png}
    &
    \includegraphics[width=0.14\textwidth]{cookbooks/sinker-with-averaging/q2q1/sinker-8-largest.png}
    &
    \includegraphics[width=0.14\textwidth]{cookbooks/sinker-with-averaging/q2q1/sinker-8-project.png}
    \\
    $[-44.5,44.5]$
    &
    $[-5.18,5.18]$
    &
    $[-5.09,5.09]$
    &
    $[-5.18,5.18]$
    &
    $[-5.20,5.20]$
    &
    $[-7.99,7.99]$
  \end{tabular}
  \caption{\it Visualization of the pressure field for the ``sinker''
    problem. Left to right: No averaging, arithmetic averaging, harmonic
    averaging, geometric averaging, pick largest, project to $Q_1$. Top: 7
    global refinement steps. Bottom: 8 global refinement steps. The minimal and maximal pressure
    values are indicated below every picture. This range is symmetric because
    we enforce that the average of the pressure equals zero. The color scale
    is adjusted to show only values between $p=-3$ and $p=3$.}
  \label{fig:sinker-with-averaging-pressure}
\end{figure}

\begin{table}[htb]
  \center
  \begin{tabular}{|c|cccccc|}
    \hline
    \# of global & no averaging & arithmetic & harmonic & geometric
    & pick & project \\
    refinement steps & & averaging & averaging &
    averaging & largest & to $Q_1$
    \\ \hline
    4          & 30+64   & 30+13      & 30+10    & 30+12 & 30+13 & 30+15 \\
    5          & 30+87   & 30+14      & 30+13    & 30+14 & 30+14 & 30+16 \\
    6          & 30+171  & 30+14      & 30+15    & 30+14 & 30+15 & 30+17 \\
    7          & 30+143  & 30+27      & 30+28    & 30+26 & 30+26 & 30+28 \\
    8          & 30+188  & 30+27      & 30+26    & 30+27 & 30+28 & 30+28 \\ \hline
  \end{tabular}
  \caption{\it Number of outer GMRES iterations to solve the Stokes equations
  for various numbers of global mesh refinement steps and for different
  material averaging operations. The GMRES solver first tries to run 30
  iterations with a cheaper preconditioner before switching to a more expensive
  preconditioner (see Section~\ref{parameters:Nonlinear solver tolerance}).}
  \label{tab:sinker-with-averaging-iteration-counts}
\end{table}
Such improvements carry over to more complex and realistic models. For
example, in a simulation of flow under the East African Rift by Sarah Stamps,
using approximately 17 million unknowns and run on 64 processors, the number
of outer and inner iterations is reduced from 169 and 114,482 without
averaging to 77 and 23,180 with harmonic averaging, respectively.
This translates into a reduction of run-time from 145 hours to 17
hours. Assessing the accuracy of the answers is of course more complicated in
such cases because we do not know the exact solution. However, the results
without and with averaging do not differ in any significant way.

A final comment is in order. First, one may think that the results should be
better in cases of discontinuous pressures if the numerical approximation
actually allowed for discontinuous pressures. This is in fact possible: We can
use a finite element in which the pressure space contains piecewise constants
(see Section~\ref{parameters:Discretization}). To do so, one simply needs to add
the following piece to the input file:
\lstinputlisting[language=prmfile]{cookbooks/sinker-with-averaging/conservative.prm.out}
Disappointingly, however, this makes no real difference: the pressure
oscillations are no better (maybe even worse) than for the standard Stokes
element we use, as shown in
Fig.~\ref{fig:sinker-with-averaging-pressure-q2q1iso} and
Table~\ref{tab:sinker-with-averaging-max-pressure-q2q1iso}. Furthermore, as
shown in Table~\ref{tab:sinker-with-averaging-iteration-counts-q2q1iso}, the
iteration numbers are also largely unaffected if any kind of averaging is used
-- though they are far worse using the locally conservative discretization if no
averaging has been selected. On the positive side, the visualization of the
discontinuous pressure finite element solution makes it much easier to see
that the true pressure is in fact discontinuous along the edge of the circle.

\begin{figure}[htb]
  \centering
  \begin{tabular}{cccccc}
    \includegraphics[width=0.14\textwidth]{cookbooks/sinker-with-averaging/q2q1plus/sinker-7-none.png}
    &
    \includegraphics[width=0.14\textwidth]{cookbooks/sinker-with-averaging/q2q1plus/sinker-7-arithmetic.png}
    &
    \includegraphics[width=0.14\textwidth]{cookbooks/sinker-with-averaging/q2q1plus/sinker-7-harmonic.png}
    &
    \includegraphics[width=0.14\textwidth]{cookbooks/sinker-with-averaging/q2q1plus/sinker-7-geometric.png}
    &
    \includegraphics[width=0.14\textwidth]{cookbooks/sinker-with-averaging/q2q1plus/sinker-7-pick-largest.png}
    &
    \includegraphics[width=0.14\textwidth]{cookbooks/sinker-with-averaging/q2q1plus/sinker-7-project-to-Q1.png}
    \\
    \includegraphics[width=0.14\textwidth]{cookbooks/sinker-with-averaging/q2q1plus/sinker-8-none.png}
    &
    \includegraphics[width=0.14\textwidth]{cookbooks/sinker-with-averaging/q2q1plus/sinker-8-arithmetic.png}
    &
    \includegraphics[width=0.14\textwidth]{cookbooks/sinker-with-averaging/q2q1plus/sinker-8-harmonic.png}
    &
    \includegraphics[width=0.14\textwidth]{cookbooks/sinker-with-averaging/q2q1plus/sinker-8-geometric.png}
    &
    \includegraphics[width=0.14\textwidth]{cookbooks/sinker-with-averaging/q2q1plus/sinker-8-pick-largest.png}
    &
    \includegraphics[width=0.14\textwidth]{cookbooks/sinker-with-averaging/q2q1plus/sinker-8-project-to-Q1.png}
  \end{tabular}
  \caption{\it Visualization of the pressure field for the ``sinker''
    problem. Like Fig.~\ref{fig:sinker-with-averaging-pressure} but using the
    locally conservative, enriched Stokes element. Pressure values are shown
    in Table~\ref{tab:sinker-with-averaging-max-pressure-q2q1iso}.}
  \label{fig:sinker-with-averaging-pressure-q2q1iso}
\end{figure}


\begin{table}[htb]
  \center
  \begin{tabular}{|c|cccccc|}
    \hline
    \# of global & no averaging & arithmetic & harmonic & geometric
    & pick & project \\
    refinement steps & & averaging & averaging &
    averaging & largest & to $Q_1$
    \\ \hline
    4 & 66.32 & 2.66 & 2.893 & 1.869 & 3.412 & 3.073 \\
    5 & 81.06 & 3.537 & 4.131 & 3.997 & 3.885 & 3.991 \\
    6 & 75.98 & 4.596 & 4.184 & 4.618 & 4.568 & 5.093 \\
    7 & 84.36 & 4.677 & 5.286 & 4.362 & 4.635 & 5.145 \\
    8 & 83.96 & 5.701 & 5.664 & 4.686 & 5.524 & 6.42 \\ \hline
  \end{tabular}
  \caption{\it Maximal pressure values for the ``sinker'' benchmark, using the
  locally conservative, enriched Stokes element. The corresponding
  pressure solutions are shown in
  Fig.~\ref{fig:sinker-with-averaging-pressure-q2q1iso}.}
  \label{tab:sinker-with-averaging-max-pressure-q2q1iso}
\end{table}


\begin{table}[htb]
  \center
  \begin{tabular}{|c|cccccc|}
    \hline
    \# of global & no averaging & arithmetic & harmonic & geometric
    & pick & project \\
    refinement steps & & averaging & averaging &
    averaging & largest & to $Q_1$
    \\ \hline
    4 & 30+376 & 30+16 & 30+12 & 30+14 & 30+14 & 30+17 \\
    5 & 30+484 & 30+16 & 30+14 & 30+14 & 30+14 & 30+16 \\
    6 & 30+583 & 30+16 & 30+17 & 30+14 & 30+17 & 30+17 \\
    7 & 30+1319 & 30+27 & 30+28 & 30+26 & 30+28 & 30+28 \\
    8 & 30+1507 & 30+28 & 30+27 & 30+28 & 30+28 & 30+29  \\ \hline
  \end{tabular}
  \caption{\it Like Table~\ref{tab:sinker-with-averaging-iteration-counts}, but
  using the locally conservative, enriched Stokes element.}
  \label{tab:sinker-with-averaging-iteration-counts-q2q1iso}
\end{table}


\subsubsection{Prescribed internal velocity constraints}
\label{sec:prescribed-velocities}
\textit{This section was contributed by Jonathan Perry-Houts}

In cases where it is desirable to investigate the behavior of one part of the model
domain but the controlling physics of another part is difficult to capture,
such as corner flow in subduction zones, it may be useful to force the desired
behavior in some parts of the model domain and solve for the resulting flow
everywhere else. This is possible through the use of \aspect{}'s ``signal'' mechanism,
as documented in Section~\ref{sec:extending-signals}.

Internally, \aspect{} adds ``constraints'' to the finite element system for boundary
conditions and hanging nodes. These are places in the finite element system where
certain solution variables are required to match some prescribed value. Although it
is somewhat mathematically inadmissible to prescribe constraints on nodes inside
the model domain, $\Omega$, it is nevertheless possible so long as the prescribed
velocity field fits in to the finite element's solution space, and satisfies the
other constraints (i.e., is divergence free).

Using \aspect{}'s signals mechanism, we write a shared library which provides a
``slot'' that listens for the signal which is triggered after the regular model
constraints are set, but before they are ``distributed.''

As an example of this functionality, below is a plugin which allows the user to prescribe
internal velocities with functions in a parameter file:
\lstinputlisting[language=C++]{../../cookbooks/prescribed_velocity/prescribed_velocity.cc}

The above plugin can be compiled with \texttt{cmake . \&\& make} in the
\url{cookbooks/prescribed_velocity} directory. It can be loaded in a parameter file
as an ``Additional shared library.'' By setting parameters like those shown below,
it is possible to produce many interesting flow fields such as the ones visualized in
(Figure~\ref{fig:prescribed-velocity}).
\lstinputlisting[language=prmfile]{cookbooks/prescribed_velocity/minimal.prm.out}

\begin{figure}
    \centering
  \subfigure[]{
    \includegraphics[width=.48\textwidth]{cookbooks/prescribed_velocity/corner_flow}}
  ~
  \subfigure[]{
    \includegraphics[width=.48\textwidth]{cookbooks/prescribed_velocity/circle}}
    \caption{Examples of flows with prescribed internal velocities, as described in Section \ref{sec:prescribed-velocities}.}
    \label{fig:prescribed-velocity}
\end{figure}

\subsubsection{Artificial viscosity smoothing}
\label{sec:artificial-viscosity-smoothing}
\textit{This section was contributed by Ryan Grove}

Standard finite element discretizations of advection-diffusion equations introduce unphysical oscillations around steep gradients. Therefore, stabilization must be added to the discrete formulation to obtain correct solutions. In ASPECT, we use the Entropy Viscosity scheme developed by Guermond et al.~in the paper \cite{guer11}. In this scheme, an artificial viscosity is calculated on every cell and used to try to combat these oscillations that cause unwanted overshoot and undershoot.  More information about how \aspect{} does this is located at \url{https://dealii.org/developer/doxygen/deal.II/step_31.html}.  

Instead of just looking at an individual cell's artificial viscosity, improvements in the minimizing of the oscillations can be made by smoothing.  Smoothing is the act of finding the maximum artificial viscosity taken over a cell $T$ and the neighboring cells across the faces of $T$, i.e.,
\begin{equation*}
\bar{v_h}(T) = \max_{K \in N(T)} v_h(K)
\end{equation*}
where $N(T)$ is the set containing $T$ and the neighbors across the faces of $T$.

This feature can be turned on by setting the \hyperref[parameters:Discretization/Stabilization parameters/Use artificial viscosity smoothing]{Use artificial viscosity smoothing} flag inside the \hyperref[parameters:Discretization/Stabilization_20parameters]{Stabilization} subsection inside the \hyperref[parameters:Discretization]{Discretization} subsection in your parameter file.

To show how this can be used in practice, let us consider the simple convection in a quarter of a 2d annulus cookbook in Section \ref{sec:shell-simple-2d}, a radial compositional field was added to help show the advantages of using the artificial viscosity smoothing feature.
  
By applying the following changes shown below to the parameters of the already existing file \begin{verbatim}cookbooks/shell_simple_2d.prm, \end{verbatim} 
\lstinputlisting[language=prmfile]{cookbooks/shell_simple_2d_smoothing/shell_simple_2d_smoothing.part.prm.out}
it is possible to produce pictures of the simple convection in a quarter of a 2d annulus such as the ones visualized in
Figure~\ref{fig:smoothing}.

\begin{figure}
    \centering
  \subfigure[]{
    \includegraphics[width=.48\textwidth]{cookbooks/shell_simple_2d_smoothing/with_smoothing.png}}
  ~
  \subfigure[]{
    \includegraphics[width=.48\textwidth]{cookbooks/shell_simple_2d_smoothing/without_smoothing.png}}
    \caption{Artificial viscosity smoothing: Example of the output of two similar runs.  The run on the left has the artificial viscosity smoothing turned on and the run on the right does not, as described in Section \ref{sec:artificial-viscosity-smoothing}.}
    \label{fig:smoothing}
\end{figure}



\subsubsection{Tracking finite strain}
\label{sec:finite-strain}
\textit{This section was contributed by Juliane Dannberg}

\note{In this section, we denote the strain rate tensor as $\varepsilon (\mathbf u)$, where 
$\varepsilon(\mathbf u)_{ij} = \frac{1}{2}\left(\frac{\partial u_i}{\partial x_j} + \frac{\partial u_j}{\partial x_i}\right)$ 
and $\mathbf u$ the velocity. We denote the strain tensor by $e$, where 
$e_{ij} = \frac{1}{2}\left(\frac{\partial d_i}{\partial x_j} + \frac{\partial d_j}{\partial x_i}\right)$ 
with the displacement $\mathbf d$ the difference between a point's
position at time $t$ and at time zero.}

In many geophysical settings, material properties, and in particular the rheology, do not only depend
on the current temperature, pressure and strain rate, but also on the history of the system. 
This can be incorporated in \aspect{} models by tracking history variables through compositional fields. 
In this cookbook, we will show how to do this by tracking the strain that accumulates over time at every 
(Lagrangian) point in the model.

Here, we use a material model plugin that defines the compositional fields as the components of the strain 
tensor $e_{ij}$, and modifies the right-hand side of 
the corresponding advection equations to accumulate strain over time. This is done by adjusting the 
\verb!out.reaction_terms! variable:
\lstinputlisting[language=C++]{cookbooks/finite_strain/finite_strain.cc}

Let us denote the accumulated strain at time step $n$ as $e^n$. We can express it 
as the sum of the strain $e^{n-1}$ at the previous time step, rotated by the rotational 
component of the velocity field, plus the strain increment $\varepsilon(\mathbf u^n) \Delta t^n$ 
accumulated over the current time step.
Hence, the right-hand side term of the advection equation for the accumulated strain consists of two parts:
The first one, $R e^{n-1} R^T$, rotates $e^{n-1}$, the accumulated strain from all the previous time steps;
and the second part adds the strain of the current time step. 

The full plugin can be found in \url{cookbooks/finite_strain/finite_strain.cc} and can be compiled 
with \texttt{cmake . \&\& make} in the \url{cookbooks/finite_strain} directory. 
It can be loaded in a parameter file as an ``Additional shared library'', and selected as material
model. As it is derived from the ``simple'' material model, all input parameters for the material
properties are read in from the subsection \texttt{Simple model}. 
\lstinputlisting[language=prmfile]{cookbooks/finite_strain/finite_strain_part.prm.out}

\begin{figure}
    \centering
    \includegraphics[width=0.5\textwidth]{cookbooks/finite_strain/finite_strain.pdf}
    \caption{Accumulated finite strain in an example convection model, as described in Section 
             \ref{sec:finite-strain}.}
    \label{fig:finite_strain}
\end{figure}

We will demonstrate its use at the example of a 2D Cartesian convection model (Figure~\ref{fig:finite_strain}):
Heating from the bottom leads to the the ascent of plumes from the bottom boundary layer, and the
associated deformation is visible in the components of the accumulated finite strain. 
Material moves to the sides at the top of the plume head, so that it is shortened in vertical 
direction (blue areas in the $yy$ component) and stretched in horizontal direction (red areas in 
the $xx$ component). The sides of the plume head show the opposite effect. Shear occurs mostly at the 
edges of the plume head and in the plume tail (blue and red areas in the $xy$ component). 

The example used here shows how history variables can be integrated up over the model evolution. 
While we do not use these variables actively in the computation so far (in our example, there is no 
influence of the accumulated strain on the rheology or any other material property), it would be trivial 
to extend this material model in a way that material properties depend on the integrated strain: 
Because the values of the compositional fields are part of what the material model gets as inputs, 
they can easily be used for computing material model outputs such as the viscosity.  


\subsubsection{Reading in compositional initial composition files generated with geomIO}
\label{sec:geomio}
\textit{This section was contributed by Juliane Dannberg}

Many geophysical setups require initial conditions with several different materials and complex geometries. 
Hence, sometimes it would be easier to generate the initial geometries of the materials as a drawing instead 
of by writing code. The MATLAB-based library geomIO (\url{http://geomio.bitbucket.org/}) provides a convenient tool
to convert a drawing generated with the vector graphics editor Inkscape (\url{https://inkscape.org/en/}) to a data 
file that can be read into \aspect{}. Here, we will demonstrate how this can be done for a 2D setup for a model
with one compositional field, but geomIO also has the capability to create 3D volumes based on a series of 2D vector 
drawings using any number of different materials. Similarly, initial conditions defined in this way can also be used with particles 
instead of compositional fields. 

To obtain the developer version of geomIO, you can clone the bitbucket repository by executing the command
\begin{verbatim}
 git clone https://bitbucket.org/geomio/geomio.git
\end{verbatim}
or you can download geomIO \href{https://bitbucket.org/geomio/geomio/downloads}{here}. 
You will then need to add the geomIO source folders to your MATLAB path by running the file located in
\texttt{/path/to/geomio/installation/InstallGeomIO.m}.
An extensive documentation for how to use geomIO can be found \href{http://geomio-doc.bitbucket.org/}{here}.
Among other things, it explains \href{http://geomio-doc.bitbucket.org/tuto2D.html#drawing}{how to generate drawings in Inkscape}
that can be read in by geomIO, which involves assigning new attributes to paths in Inkscape's XML editor. 
In particular, a new property `phase' has to be added to each path, and set to a value corresponding to the 
index of the material that should be present in this region in the initial condition of the geodynamic model.   

\note{geomIO currently only supports the latest stable version of Inkscape (0.91), and other versions might not
  work with geomIO or cause errors. Moreover, geomIO currently does not support grouping paths (paths can still 
  be combined using \texttt{Path$\rightarrow$Union}, \texttt{Path$\rightarrow$Difference} or similar commands), 
  and only the outermost closed contour of a path will be considered. This means that, for example, for modeling a spherical 
  annulus, you would have to draw two circles, and assign the inner one the same phase as the background of your drawing.}

We will here use a drawing of a jellyfish located in \url{cookbooks/geomio/jellyfish.svg}, where different phases 
have already been assigned to each path (Figure~\ref{fig:jelly-picture}). 
\begin{figure}[tb]
    \centering
    \includegraphics[width=0.2\textwidth]{cookbooks/geomio/jellyfish.pdf}
    \caption{Vector drawing of a jellyfish.}
    \label{fig:jelly-picture}
\end{figure}
\note{The page of your drawing in Inkscape should already have the extents (in px) that you later want to use in your model (in m).}

After geomIO is initialized in MATLAB, we \href{http://geomio-doc.bitbucket.org/tuto2D.html#assigning-phase-to-markers}
{run geomIO as desribed in the documentation}, loading the default options and then specifying all the option we want to 
change, such as the path to the input file, or the resolution: 
\lstinputlisting[language=matlab]{cookbooks/geomio/run_geomio.part1.m}
You can view all of the options available by typing \texttt{opt} in MATLAB. 

In the next step we create the grid that is used for the coordinates in the ascii data initial conditions file 
and assign a phase to each grid point: 
\lstinputlisting[language=matlab]{cookbooks/geomio/run_geomio.part2.m}
You can plot the \texttt{Phase} variable in MATLAB to see if the drawing was read in and all phases are assigned correctly 
(Figure~\ref{fig:jelly-plot}). 
\begin{figure}[tb]
    \centering
    \includegraphics[width=0.45\textwidth]{cookbooks/geomio/jelly.png}
    \caption{Plot of the \texttt{Phase} variable in MATLAB.}
    \label{fig:jelly-plot}
\end{figure}
Finally, we want to write output in a format that can be read in by \aspect{}'s \texttt{ascii data} compositional
initial conditions plugin. We write the data into the file \texttt{jelly.txt}:
\lstinputlisting[language=matlab]{cookbooks/geomio/save_file_as_txt.m}

To read in the file we just created (a copy is located in \aspect{}'s data directory), 
we set up a model with a box geometry with the same extents we specified for the drawing in px 
and one compositional field. We choose the ascii data compositional initial conditions and specify that we 
want to read in our jellyfish. The relevant parts of the input file are listed below:
\lstinputlisting[language=prmfile]{cookbooks/geomio/geomIO.prm}

If we look at the output in \texttt{paraview}, we can see our jellyfish, with the mesh refined at the 
boundaries between the different phases (Figure~\ref{fig:jelly-paraview}). 
\begin{figure}[tb]
    \centering
    \includegraphics[width=0.55\textwidth]{cookbooks/geomio/jelly-paraview.pdf}
    \caption{\aspect{} model output of the jellyfish and corresponding mesh in paraview.}
    \label{fig:jelly-paraview}
\end{figure}

For a geophysical setup, the MATLAB code could be extended to write out the phases into several different columns 
of the ascii data file (corresponding to different compositional fields). This initial conditions file could then be 
used in \aspect{} with a material model such as the \texttt{multicomponent} model, assigning each phase different 
material properties. 

An animation of a model using the jellyfish as initial condition and assigning it a higher viscosity can be found here: \url{https://www.youtube.com/watch?v=YzNTubNG83Q}. 

\subsection{Geophysical setups}
\label{sec:cookbooks-geophysical}

Having gone through the ways in which one can set up problems in rectangular
geometries, let us now move on to situations that are directed more towards the
kinds of things we want to use \aspect{} for: the simulation of convection in
the rocky mantles of planets or other celestial bodies.

To this end, we need to go through the list of issues that have to be described
and that were outlined in Section~\ref{sec:cookbooks-overview}, and address them
one by one:
\begin{itemize}
  \item \textit{What internal forces act on the medium (the equation)?}
    This may in fact be the most difficult to answer part of it all. The real
    material in Earth's mantle is certainly no Newtonian fluid where the stress
    is a linear function of the strain with a proportionality constant (the
    viscosity) $\eta$ that only depends on the temperature. Rather, the
    real viscosity almost surely also depends on the pressure and the strain
    rate. Because the issue is complicated and the exact material model not
    entirely clear, for the next few subsections we will therefore ignore the
    issue and start with just using the ``simple'' material model where the
    viscosity is constant and most other coefficients depend at most on the
    temperature.

  \item \textit{What external forces do we have (the right hand side)}
    There are of course other issues: for example, should the model include terms
    that describe shear heating? Should it be compressible? Adiabatic heating due
    to compression? Most of the terms that pertain to these questions appear on
    the right hand sides of the equations, though some (such as the
    compressibility) also affect the differential operators on the left. Either
    way, for the moment, let us just go with the simplest models and come back to
    the more advanced questions in later examples.

    One right hand side that will certainly be there is that due to gravitational
    acceleration. To first order, within the mantle gravity points radially
    inward and has a roughly constant magnitude. In reality, of course, the
    strength and direction of gravity depends on the distribution and density of
    materials in Earth -- and, consequently, on the solution of the model at
    every time step. We will discuss some of the associated issues in the
    examples below.

  \item \textit{What is the domain (geometry)?}
    This question is easier to answer. To first order, the domains we want to
    simulate are spherical shells, and to second order ellipsoid shells that can
    be obtained by considering the isopotential surface of the gravity field of
    a homogenous, rotating fluid.
    A more accurate description is of course the geoid for which several
    parameterizations are available. A complication arises if we ask whether we
    want to include the mostly rigid crust in the domain and simply assume that
    it is part of the convecting mantle, albeit a rather viscous part due to its
    low temperature and the low pressure there, or whether we want to truncate
    the computation at the asthenosphere.

  \item \textit{What happens at the boundary for each variable involved
      (boundary conditions)?}
    The mantle has two boundaries: at the bottom where it contacts the outer core
    and at the top where it either touches the air or, depending on the outcome
    of the discussion of the previous question, where it contacts the
    lithospheric crust. At the bottom, a very good approximation of what is
    happening is certainly to assume that the velocity field is tangential
    (i.e., horizontal) and without friction forces due to the very low viscosity
    of the liquid metal in the outer core. Similarly, we can assume that the
    outer core is well mixed and at a constant temperature. At the top boundary,
    the situation is slightly more complex because in reality the boundary is not
    fixed but also allows vertical movement. If we ignore this, we can assume
    free tangential flow at the surface or, if we want, prescribe the tangential
    velocity as inferred from plate motion models. \aspect{} has a plugin that
    allows to query this kind of information from the \texttt{GPlates} program.

  \item \textit{How did it look at the beginning (initial conditions)?}
    This is of course a trick question. Convection in the mantle of earth-like
    planets did not start with a concrete initial temperature distribution when
    the mantle was already fully formed. Rather, convection already happened
    when primordial material was still separating into mantle and core. As a
    consequence, for models that only simulate convection using mantle-like
    geometries and materials, no physically reasonable initial conditions are
    possible that date back to the beginning of Earth. On the other hand, recall
    that we only need initial conditions for the temperature (and, if
    necessary, compositional fields). Thus, if we have a temperature profile at
    a given time, for example one inferred from seismic data at the current
    time, then we can use these as the starting point of a simulation.
\end{itemize}

This discussion shows that there are in fact many pieces with which one can play
and for which the answers are in fact not always clear. We will address some of
them in the cookbooks below. Recall in the descriptions we use in the input
files that \aspect{} uses physical units, rather than non-dimensionalizing
everything. The advantage, of course, is that we can immediately compare outputs
with actual measurements. The disadvantage is that we need to work a bit when
asked for, say, the Rayleigh number of a simulation.


\subsubsection{Simple convection in a quarter of a 2d annulus}
\label{sec:shell-simple-2d}

Let us start this sequence of cookbooks using a simpler situation: convection in
a quarter of a 2d shell. We choose this setup because 2d domains allow for much
faster computations (in turn allowing for more experimentation) and because
using a quarter of a shell avoids a pitfall with boundary conditions we will
discuss in the next section. Because it's simpler to explain what we want to
describe in pictures than in words, Fig.~\ref{fig:simple-shell-2d} shows the
domain and the temperature field at a few time steps. In addition, you can find
a movie of how the temperature evolves over this time period at
\url{http://www.youtube.com/watch?v=d4AS1FmdarU}.%
\footnote{In YouTube, click on the gear symbol at the bottom right of the
player window to select the highest resolution to see all the details of this
video.}

\begin{figure}[tb]
\includegraphics[width=0.32\textwidth]{cookbooks/shell_simple_2d/x-movie0000}
\hfill
\includegraphics[width=0.32\textwidth]{cookbooks/shell_simple_2d/x-movie0008}
\hfill
\includegraphics[width=0.32\textwidth]{cookbooks/shell_simple_2d/x-movie1000}
\caption{\it Simple convection in a quarter of an annulus: Snapshots of the
temperature field at times $t=0$, $t=1.2\cdot 10^7$ years (time step 2135), and
$t=10^9$ years (time step 25,662). The bottom right part of each figure shows an
overlay of the mesh used during that time step.}
\label{fig:simple-shell-2d}
\end{figure}

Let us just start by showing the input file (which you can find in
\url{cookbooks/shell_simple_2d.prm}):

\lstinputlisting[language=prmfile]{cookbooks/shell_simple_2d/shell.prm.out}

In the following, let us pick apart this input file:
\begin{enumerate}
  \item Lines 1--4 are just global parameters. Since we are interested in
  geophysically realistic simulations, we will use material parameters that
  lead to flows so slow that we need to measure time in years, and we will set
  the end time to 1.5 billion years -- enough to see a significant amount of
  motion.

  \item The next block (lines 7--14) describes the material that is convecting
  (for historical reasons, the remainder of the parameters that describe the
  equations is in a different section, see the fourth point below). We choose
  the simplest material model \aspect{} has to offer where the viscosity is
  constant (here, we set it to $\eta=10^{22} \text{Pa}\;\text{s}$) and so are
  all other parameters except for the density which we choose to be
  $\rho(T)=\rho_0(1-\beta (T-T_\text{ref}))$ with $\rho_0=3300
  \text{kg}\;\text{m}^{-3}$, $\beta=4\cdot 10^{-5} \text{K}^{-1}$ and
  $T_\text{ref}=293 \text{K}$. The remaining material parameters remain at their
  default values and you can find their values described in the documentation of
  the \texttt{simple} material model in
  Sections~\ref{parameters:Material_20model} and
  \ref{parameters:Material_20model/Simple_20model}.

  \item Lines 17--25 then describe the geometry. In this simple case, we will
  take a quarter of a 2d shell (recall that the dimension had previously been
  set as a global parameter) with inner and outer radii matching those of a
  spherical approximation of Earth.

  \item The second part of the model description and boundary values follows in
  lines 28--45. There, we describe that we want a model where equation
  \eqref{eq:temperature} contains the shear heating term $2\eta
  \varepsilon(\mathbf u):\varepsilon(\mathbf u)$ (noting that the default is to
  use an incompressible model for which the term $\frac{1}{3}(\nabla \cdot
  \mathbf u)\mathbf 1$ in the shear heating contribution is zero).

  The boundary conditions require us to look up how the geometry model
  we chose (the \texttt{spherical shell} model) assigns boundary indicators to
  the four sides of the domain. This is described in
  Section~\ref{parameters:Geometry_20model} where the model announces that
  boundary indicator zero is the inner boundary of the domain, boundary
  indicator one is the outer boundary, and the left and right boundaries for a
  2d model with opening angle of 90 degrees as chosen here get boundary
  indicators 2 and 3, respectively. In other words, the settings in the input
  file correspond to a zero velocity at the inner boundary and tangential flow
  at all other boundaries. We know that this is not realistic at the bottom, but
  for now there are of course many other parts of the model that are not
  realistic either and that we will have to address in subsequent cookbooks.
  Furthermore, the temperature is fixed at the inner and outer boundaries (with
  the left and right boundaries then chosen so that no heat flows across them,
  emulating symmetry boundary conditions) and, further down, set to values of
  700 and 4000 degrees Celsius -- roughly realistic for the bottom of the crust
  and the core-mantle boundary.

  \item The description of what we want to model is complete by specifying
  that the initial temperature is a perturbation with hexagonal symmetry from a
  linear interpolation between inner and outer temperatures (see
  Section~\ref{parameters:Initial_20conditions}), and what kind of gravity model
  we want to choose (one reminiscent of the one inside the Earth mantle, see
  Section~\ref{parameters:Gravity_20model}).

  \item The remainder of the input file consists of a description of how to
  choose the initial mesh and how to adapt it (lines 58--63) and what to do at
  the end of each time step with the solution that \aspect{} computes for us (lines 66--79).
  Here, we ask for a variety of statistical quantities and for graphical output
  in VTU format every million years.
\end{enumerate}

\note{Having described everything to \aspect{}, you may want to view the video linked
to above again and compare what you see with what you expect. In fact, this is
what one should always do having just run a model: compare it with expectations to make
sure that we have not overlooked anything when setting up the model or that
the code has produced something that doesn't match what we thought we should
get. Any such mismatch between expectation and observed result is typically a
learning opportunity: it either points to a bug in our input file, or it
provides us with insight about an aspect of reality that we had not foreseen.
Either way, accepting results uncritically is, more often than not, a way to
scientifically invalid results.}

The model we have chosen has a number of inadequacies that
make it not very realistic (some of those happened more as an accident while
playing with the input file and weren't a purposeful experiment, but we left
them in because they make for good examples to discuss below).
Let us discuss these issues in the following.

\paragraph{Dimension.} This is a cheap shot but it is nevertheless true that the
world is three-dimensional whereas the simulation here is 2d. We will address
this in the next section.

\paragraph{Incompressibility, adiabaticity and the initial conditions.} This one
requires a bit more discussion. In the model selected above, we have chosen a
model that is incompressible in the sense that the density does not depend on
the pressure and only very slightly depends on the temperature (the Boussinesq
approximation). In such models, material that rises up does not cool down due to
expansion resulting from the pressure dropping, and material that is transported
down does not adiabatically heat up. Consequently, the adiabatic temperature
profile would be constant with depth, and a well-mixed model with hot inner and
cold outer boundary would have a constant temperature with thin boundary layers
at the bottom and top of the mantle. In contrast to this, our initial
temperature field was a perturbation of a linear temperature profile.

There are multiple implications of this. First, the temperature difference
between outer and inner boundary of 3300 K we have chosen in the input file is
much too large. The temperature difference is what drives the convection, but
what really counts is of course the difference \textit{in addition} to the
temperature increase a volume of material would experience if it were to be
transported adiabatically from the surface to the core-mantle boundary. This
difference is much smaller than 3300 K in reality, and we can expect convection
to be significantly less vigorous than in the simulation here. Indeed, using
the values in the input file shown above, we can compute the Rayleigh number for
the current case to be%
\footnote{Note that the density in 2d has units $\text{kg}\,\text{m}^{-2}$}
\begin{equation*}
  \textrm{Ra}
  =
  \frac{g\; \beta \; \Delta T \; \rho \; L^3}{\alpha\eta}
  \approx
  \frac{10 \text{m}\,\text{s}^{-2} \; 4\cdot 10^{-5}\text{K}^{-1} \; 3300
  \text{K} \; 3300 \text{kg}\,\text{m}^{-2} \; (2.86 \cdot 10^6
  \text{m})^3}{10^{22}
  \text{kg}\,\text{m}^{-1}\,\text{s}^{-1}}.
\end{equation*}

Second, the initial temperature profile we chose is not realistic -- in fact, it
is a completely instable one: there is hot material underlying cold one, and
this is not just the result of boundary layers. Consequently, what happens in
the simulation is that we first overturn the entire temperature field with the
hot material in the lower half of the domain swapping places with the colder
material in the top, to achieve a stable layering except for the boundary
layers. After this, hot blobs rise from the bottom boundary layer into the cold
layer at the bottom of the mantle, and cold blobs sink from the top, but their
motion is impeded about half-way through the mantle once they reach material
that has roughly the same temperature as the plume material. This impedes
convection until we reach a state where these plumes have sufficiently mixed the
mantle to achieve a roughly constant temperature profile.

This effect is visible in the movie linked to above where convection does not
penetrate the entire depth of the mantle for the first 20 seconds
(corresponding to roughly the first 800 million years). We can also see this
effect by plotting the root mean square velocity, see the left panel of
Fig.~\ref{fig:simple-shell-2d-rms}. There, we can see how the average velocity
picks up once the stable layering of material that resulted from the initial
overturning has been mixed sufficiently to allow plumes to rise or sink through
the entire depth of the mantle.

\begin{figure}[tb]
\includegraphics[width=0.48\textwidth]{cookbooks/shell_simple_2d/rms}
\hfill
\includegraphics[width=0.48\textwidth]{cookbooks/shell_simple_2d/depth_average_temperature}
\caption{\it Simple convection in a quarter of an annulus. Left: Root mean
square values of the velocity field. The initial spike (off the scale) is due to
the overturning of the instable layering of the temperature. Convection is suppressed for the
first 800 million years due to the stable layering that results from it. The
maximal velocity encountered follows generally the same trend and is in the
range of 2--3 cm/year between 100 and 800 million years, and 4--8 cm/year
following that. Right: Average temperature at various depths for $t=0$,
$t=800,000$ years, $t=5\cdot 10^8$ years, and $t=10^9$ years.}
\label{fig:simple-shell-2d-rms}
\end{figure}

The right panel of Fig.~\ref{fig:simple-shell-2d-rms} shows a different way of
visualizing this, using the average temperature at various depths of the model
(this is what the \texttt{depth average} postprocessor computes). The figure
shows how the initially linear unstable layering almost immediately reverts
completely, and then slowly equilibrates towards a temperature profile that is
constant throughout the mantle (which in the incompressible model chosen here
equates to an adiabatic layering) except for the boundary layers at the inner
and outer boundaries. (The end points of these temperature profiles do not
exactly match the boundary values specified in the input file because we
average temperatures over shells of finite width.)

A conclusion of this discussion is that if we want to evaluate the statistical
properties of the flow field, e.g., the number of plumes, average velocities or
maximal velocities, then we need to restrict our efforts to times after
approximately 800 million years in this simulation to avoid the effects of our
inappropriately chosen initial conditions. Likewise, we may actually want to
choose initial conditions more like what we see in the model for later times,
i.e., constant in depth with the exception of thin boundary layers, if we want
to stick to incompressible models.

\paragraph{Material model.}
The model we use here involves viscosity, density, and thermal property
functions that do not depend on the pressure, and only the density varies
(slightly) with the temperature. We know that this is not the case in nature.

\paragraph{Shear heating.}
When we set up the input file, we started with a model that includes the
shear heating term $2\eta \varepsilon(\mathbf u):\varepsilon(\mathbf u)$ in
eq.~\eqref{eq:temperature}. In hindsight, this may have been the wrong decision,
but it provides an opportunity to investigate whether we think that the results of our
computations can possibly be correct.

We first realized the issue when looking at the heat flux that the
\texttt{heat flux statistics} postprocessor computes. This is shown in the left
panel of Fig.~\ref{fig:simple-shell-2d-heatflux}.%
\footnote{The \texttt{heat flux statistics} postprocessor computes heat fluxes
through parts of the boundary in \textit{outward} direction, i.e., from the
mantle to the air and to the core. However, we are typically interested in the
flux from the core into the mantle, so the figure plots the negative of the
computed quantity.}
There are two issues one should notice here.
The more obvious one is that the flux from the mantle to the air is consistently
higher than the heat flux from core to mantle. Since we have no radiogenic
heating model selected (see the \texttt{Model name}
\index[prmindex]{Model name}
\index[prmindexfull]{Heating model!Model name}
parameter in the \texttt{Heating model} section of the input file; see also
Section~\ref{parameters:Heating_20model}), in the long run the heat output
of the mantle must equal the input, unless is cools. Our misconception was that
after the 800 million year transition, we believed that we had reached a steady
state where the average temperature remains constant and convection simply
moves heat from the core-mantle boundary the surface. One could also be tempted
to believe this from the right panel in Fig.~\ref{fig:simple-shell-2d-rms} where
it looks like the average temperature does at least not change dramatically.
But, it is easy to convince oneself that that is not the case: the
\texttt{temperature statistics} postprocessor we had previously selected also
outputs data about the mean temperature in the model, and it looks like shown in
the left panel of Fig.~\ref{fig:simple-shell-2d-temperature}. Indeed, the
average temperature drops over the course of the 1.2 billion years shown here.
We could now convince ourselves that indeed the loss of thermal
energy in the mantle due to the drop in average temperature is exactly what
fuels the persistently imbalanced energy outflow. In essence, what this would
show is that if we kept the temperature at the boundaries constant, we would
have chosen a mantle that was initially too hot on average to be sustained by
the boundary values and that will cool until it will be in energetic balance and
on longer time scales, in- and outflow of thermal energy would balance each
other.


\begin{figure}[tb]
  \includegraphics[width=.48\textwidth]{cookbooks/shell_simple_2d/heat-flux}
  \hfill
  \includegraphics[width=0.48\textwidth]{cookbooks/shell_simple_2d/heat-flux-noshear}
  \caption{\it Simple convection in a quarter of an annulus. Left: Heat flux
  through the core-mantle and mantle-air boundaries of the domain for the
  model with shear heating. Right: Same for a model without shear heating.}
  \label{fig:simple-shell-2d-heatflux}
\end{figure}


\begin{figure}[tb]
  \includegraphics[width=.48\textwidth]{cookbooks/shell_simple_2d/avg-temperature}
  \hfill
  \includegraphics[width=0.48\textwidth]{cookbooks/shell_simple_2d/avg-temperature-noshear}
  \caption{\it Simple convection in a quarter of an annulus. Left: Average
  temperature throughout the model for the
  model with shear heating. Right: Same for a model without shear heating.}
  \label{fig:simple-shell-2d-temperature}
\end{figure}

However, there is a bigger problem. Fig.~\ref{fig:simple-shell-2d-heatflux}
shows that at the very beginning, there is a spike in energy flux through the
outer boundary. We can explain this away with the imbalanced initial temperature
field that leads to an overturning and, thus, a lot of hot material rising close
to the surface that will then lead to a high energy flux towards the cold upper
boundary. But, worse, there is initially a \textit{negative} heat flux into the
mantle from the core -- in other words, the mantle is \textit{losing} energy to
the core. How is this possible? After all, the hottest part of the mantle in our
initial temperature field is at the core-mantle boundary, no thermal energy
should be flowing from the colder overlying material towards the hotter material
at the boundary! A glimpse of the solution can be found in looking at the
average temperature in Fig.~\ref{fig:simple-shell-2d-temperature}: At the
beginning, the average temperature \textit{rises}, and apparently there are
parts of the mantle that become hotter than the 4273 K we have given the core,
leading to a downward heat flux. This heating can of course only come from the
shear heating term we have accidentally left in the model: at the beginning, the
instable layering leads to very large velocities, and large velocities lead to
large velocity gradients that in turn lead to a lot of shear heating! Once the
initial overturning has subsided, after say 100 million years (see the mean
velocity in Fig.~\ref{fig:simple-shell-2d-rms}), the shear heating becomes
largely irrelevant and the cooling of the mantle indeed begins.

Whether this is really the case is of course easily verified: The right panels
of Fig.s~\ref{fig:simple-shell-2d-heatflux}
and \ref{fig:simple-shell-2d-temperature} show heat fluxes and average
temperatures for a model where we have switched off the shear heating by setting

\lstinputlisting[language=prmfile]{cookbooks/shell_simple_2d/shearheat.part.prm.out}

Indeed, doing so leads to a model where the heat flux from core to mantle is
always positive, and where the average temperature strictly drops!


\paragraph{Summary.} As mentioned, we will address some of the issues we have
identified as unrealistic in the following sections.
However, despite all of this, some things are at least at the right order of
magnitude, confirming that what \aspect{} is computing is reasonable. For
example, the maximal velocities encounted in our model (after the 800 million
year boundary) are in the range of 6--7cm per year, with occasional excursions
up to 11cm. Clearly, something is going in the right direction.


\subsubsection{Simple convection in a spherical 3d shell}
\label{sec:shell-simple-3d}

The setup from the previous section can of course be extended to 3d shell
geometries as well -- though at significant computational cost. In fact, the
number of modifications necessary is relatively small, as we will discuss below.
To show an example up front, a picture of the temperature field one gets from
such a simulation is shown in Fig.~\ref{fig:simple-shell-3d}. The
corresponding movie can be found at \url{http://youtu.be/j63MkEc0RRw}.

\begin{figure}[tb]
\centering
\includegraphics[width=0.7\textwidth]{cookbooks/shell_simple_3d/x-movie0700}
\caption{\it Convection in a spherical shell: Snapshot of
isosurfaces of the temperature field at time $t\approx 1.06\cdot 10^9$ years
with a quarter of the geometry cut away. The surface shows
vectors indicating the flow velocity and direction.}
\label{fig:simple-shell-3d}
\end{figure}

\paragraph{The input file.}
Compared to the input file discussed in the previous section, the number of
changes is relatively small. However, when taking into account the various
discussions about which parts of the model were or were not realistic, they go
throughout the input file, so we reproduce it here in its entirety, interspersed
with comments (the full input file can also be found in
\url{cookbooks/shell_simple_3d.prm}). Let us start from the top where everything
looks the same except that we set the dimension to 3:

\lstinputlisting[language=prmfile]{cookbooks/shell_simple_3d/part1.part.prm.out}

The next section concerns the geometry. The geometry model remains unchanged at
``spherical shell'' but we omit the opening angle of 90 degrees as we would like
to get a complete spherical shell. Such a shell of course also only has two
boundaries (the inner one has indicator zero, the outer one indicator one) and
consequently these are the only ones we need to list in the ``Model settings''
section:

\lstinputlisting[language=prmfile]{cookbooks/shell_simple_3d/part2.part.prm.out}

Next, since we convinced ourselves that the temperature range from 973 to 4273
was too large given that we do not take into account adiabatic effects in this
model, we reduce the temperature at the inner edge of the mantle to 1973. One
can think of this as an approximation to the real temperature there minus the
amount of adiabatic heating material would experience as it is transported from
the surface to the core-mantle boundary. This is, in effect, the temperature
difference that drives the convection (because a completely adiabatic
temperature profile is stable despite the fact that it is much hotter at the
core mantle boundary than at the surface). What the real value for this
temperature difference is, is unclear from current research, but it is thought
to be around 1000 Kelvin, so let us choose these values.

\lstinputlisting[language=prmfile]{cookbooks/shell_simple_3d/part3.part.prm.out}

The second component to this is that we found that without adiabatic effects, an
initial temperature profile that decreases the temperature from the inner to the
outer boundary makes no sense. Rather, we expected a more or less constant
temperature with boundary layers at both ends. We could describe such an initial
temperature field, but since any initial temperature is mostly arbitrary anyway,
we opt to just assume a constant temperature in the middle between the inner and
outer temperature boundary values and let the simulation find the exact shape of
the boundary layers itself:

\lstinputlisting[language=prmfile]{cookbooks/shell_simple_3d/part4.part.prm.out}

As before, we need to determine how many mesh refinement steps we want. In 3d,
it is simply not possible to have as much mesh refinement as in 2d, so we choose
the following values that lead to meshes that have, after an initial transitory
phase, between 1.5 and 2.2 million cells and 50--75 million unknowns:

\lstinputlisting[language=prmfile]{cookbooks/shell_simple_3d/amr.part.prm.out}

Second to last, we specify what we want \aspect{} to do with the solutions it
computes. Here, we compute the same statistics as before, and we again generate
graphical output every million years. Computations of this size typically run
with ~1000 MPI processes, and it is not efficient to let every one of them write
their own file to disk every time we generate graphical output; rather, we group
all of these into a single file to keep file systems reasonably happy. Likewise,
to accomodate the large amount of data, we output depth averaged fields in VTU
format since it is easier to visualize:

\lstinputlisting[language=prmfile]{cookbooks/shell_simple_3d/postprocess.part.prm.out}

Finally, we realize that when we run very large parallel computations, nodes go
down or the scheduler aborts programs because they ran out of time. With
computations this big, we cannot afford to just lose the results, so we
checkpoint the computations every 50 time steps and can then resume it at the
last saved state if necessary (see Section~\ref{sec:checkpoint-restart}):

\lstinputlisting[language=prmfile]{cookbooks/shell_simple_3d/checkpoint.part.prm.out}




\paragraph{Evaluation.}
Just as in the 2d case above, there are still many things that are wrong from a
physical perspective in this setup, notably the no-slip boundary conditions at
the bottom and of course the simplistic material model with its fixed viscosity
and its neglect for adiabatic heating and compressibility.
But there are also a number of things that are already order of magnitude
correct here.

For example, if we look at the heat flux this model produces, we find that the
convection here produces approximately the correct number. Wikipedia's article
on \href{http://en.wikipedia.org/wiki/Earth's_internal_heat_budget}{Earth's
internal heat budget}%
\footnote{Not necessarily the most scientific source, but easily
accessible and typically about right in terms of numbers. The numbers stated
here are those listed on Wikipedia at the time this section was written in
March 2014.}
states that the overall heat flux through the Earth surface is about $47 \cdot
10^{12}$ W (i.e., 47 terawatts) of which an estimated 12--30 TW are primordial
heat released from cooling the Earth and 15--41 TW from radiogenic heating.%
\footnote{As a point of reference, for the mantle an often used number for the
release of heat due to radioactive decay is $7.4\cdot 10^{-12}$ W/kg. Taking a
density of $3300\; \text{kg}/\text{m}^3$ and a volume of $10^{12}\; \text{m}^3$
would yield roughly $2.4\cdot 10^{13}$ W of heat produced. This back of the
envelope calculation lies within the uncertain range stated above.}
Our model does not include radiogenic heating (though \aspect{} has a number of 
\texttt{Heating models} to switch this on, see 
Section~\ref{parameters:Heating_20model}) but we can compare what the model
gives us in terms of heat flux through the inner and outer boundaries of our
shell geometry. This is shown in the left panel of
Fig.~\ref{fig:shell-simple-3d-eval} where we plot the heat flux through
boundaries zero and one, corresponding to the core-mantle boundary and Earth's
surface. \aspect{} always computes heat fluxes in outward direction, so the flux
through boundary zero will be negative, indicating the we have a net flux
\textit{into} the mantle as expected. The figure indicates that after some
initial jitters, heat flux from the core to the mantle stabilizes at around 4.5
TW and that through the surface at around 10 TW, the difference of 5.5 TW
resulting from the overall cooling of the mantle. While we cannot expect our model to be
quantitatively correct, this can be compared with estimates heat fluxes of 5--15
TW for the core-mantle boundary, and an estimated heat loss due to cooling of
the mantle of 7--15 TW (values again taken from Wikipedia).

\begin{figure}
  \includegraphics[width=0.48\textwidth]{cookbooks/shell_simple_3d/heatflux}
  \hfill
  \includegraphics[width=0.48\textwidth]{cookbooks/shell_simple_3d/velocities}
  \caption{\it Evaluating the 3d spherical shell model. Left: Outward heat
  fluxes through the inner and outer boundaries of the shell. Right: Average
  and maximal velocities in the mantle.}
  \label{fig:shell-simple-3d-eval}
\end{figure}

A second measure of whether these results make sense is to compare velocities in
the mantle with what is known from observations. As shown in the right panel of
Fig.~\ref{fig:shell-simple-3d-eval}, the maximal velocities settle to values on
the order of 3 cm/year (each of the peaks in the line for the maximal velocity
corresponds to a particularly large plume rising or falling). This is, again, at
least not very far from what we know to be correct and we should expect that
with a more elaborate material model we should be able to get even closer to
reality.


\subsubsection{3D convection with an Earth-like initial condition}
\label{sec:cookbooks-S20RTS}
\textit{This section was contributed by Jacqueline Austermann}

For any model run with \aspect{} we have to choose an initial condition for the 
temperature field. If we want to model convection in the Earth's mantle we want 
to choose an initial temperature distribution that captures the Earth's buoyancy 
structure. In this cookbook we present how to use temperature perturbations 
based on the shear wave velocity model S20RTS \cite{S20RTS} to 
initialize a mantle convection calculation.

\paragraph{The input shear wave model.}

The current version of \aspect{} can read in the shear wave velocity models 
S20RTS \cite{S20RTS} and S40RTS \cite{S40RTS}, which are located 
in \texttt{data/initial-conditions/S40RTS/}. Those models provide 
spherical harmonic coefficients up do degree 20 and 40, respectively, for 21 
depth layers. The interpolation with depth is done through a cubic spline 
interpolation. The input files \texttt{S20RTS.sph} and \texttt{S40RTS.sph} were
downloaded from \url{http://www.earth.lsa.umich.edu/~jritsema/Research.html} 
and have the following format (this example is S20RTS): 

\lstinputlisting[language=prmfile]{cookbooks/initial-condition-S20RTS/S20RTS.input.sph}

The first number in the first line denotes the maximum degree. This is followed in 
the next line by the spherical harmonic coefficients from the surface down to the 
CMB. The coefficients are arranged in the following way:\\

\noindent $a_{00}$ \\
$a_{10}$ $a_{11}$ $b_{11}$ \\
$a_{20}$ $a_{21}$ $b_{21}$ $a_{22}$ $b_{22}$ \\
... \\

$a_{yz}$ is the cosine coefficient of degree $y$ and order $z$ and $b_{yz}$ is 
the sine coefficient of degree $y$ and order $z$. The depth layers are specified 
in the file \texttt{Spline\_knots.txt} by a normalized depth value ranging from the CMB (3480km, 
normalized to -1) to the moho (6346km, normalized to 1). This is the original
format provided on the homepage. 

Any other perturbation model in this same format can also be used, one only
has to specify the different filename in the parameter file (see next section).
For models with different depth layers one has to adjust the \texttt{Spline\_knots.txt} 
file as well as the number of depth layers, which is hard coded in the current 
code. A further note of caution when switching to a different input model 
concerns the normalization of the spherical harmonics, which might differ.  
After reading in the shear wave velocity perturbation one has several options
to scale this into temperature differences, which are then used to initialize 
the temperature field. 

\paragraph{Setting up the \aspect{} model.}

For this cookbook we will use the parameter file provided in 
\url{cookbooks/S20RTS.prm}, which uses a 3d spherical shell 
geometry similar to section \ref{sec:shell-simple-3d}. This plugin is only sensible 
for a 3D spherical shell with Earth-like dimensions. 

The relevant section in the input file is as follows:

\lstinputlisting[language=prmfile]{cookbooks/initial-condition-S20RTS/S20RTS.part.prm.out}

For this initial condition model we need to first specify the data directory in which 
the input files are located as well as the initial condition file (S20RTS.sph or 
S40RTS.sph) and the file that contains the normalized depth layers (Spline knots depth file name). 
We next have the option to remove the degree 0 perturbation from the shear 
wave model. This might be the case if we want to make sure that the depth 
average temperature follows the background (adiabatic or constant) temperature.  

The next input parameters describe the scaling from the shear wave velocity 
perturbation to the final temperature field. The shear wave velocity perturbation 
$\delta v_s / v_s$ (that is provided by S20RTS) is scaled into a density perturbation $\delta \rho / \rho$ with a 
constant that is specified in the initial condition section of the input parameter 
file as `Vs to density scaling'. Here we choose a constant scaling of 0.15. This 
perturbation is further translated into a temperature difference $\Delta T$ by 
multiplying it by the negative inverse of thermal expansion, which is also 
specified in this section of the parameter file as `Thermal expansion coefficient 
in initial temperature scaling'. This temperature difference is then added to the 
background temperature, which is the adiabatic temperature for a compressible 
model or the reference temperature (as specified in this section of the parameter file) for an 
incompressible model. Features in the upper mantle such as cratons might 
be chemically buoyant and therefore isostatically compensated, in which case 
their shear wave perturbation would not contribute buoyancy variations. We therefore included an 
additional option to zero out temperature perturbations within a certain depth, however, in this example we don't make use of this functionality. The chemical variation within the mantle 
might require a more sophisticated `Vs to density' scaling that varies for 
example with depth or as a function of the perturbation itself, which is not captured
in this model. The described procedure 
provides an absolute temperature for every point, which will only be adjusted 
at the boundaries if indicated in the Boundary temperature model. In this example
we chose a surface and core mantle boundary temperature that differ from the
reference mantle temperature in order to approximate thermal boundary layers.

\paragraph{Visualizing 3D models.}

In this cookbook we calculate the instantaneous solution to examine the flow 
field. Figure~\ref{fig:ic-1} shows some of the output for a resolution of 2 global refinement 
steps (c and e) as used in the cookbook, as well as 4 global refinement 
steps (other panels in this figure). The top panels show the density variation 
that has been obtained from scaling S20RTS in the way described above.
One can see the two large low shear wave velocity perturbations 
underneath Africa and the Pacific that lead to upwelling if they are assumed to 
be buoyant (as is done in this case). One can also see the subducting slabs 
underneath South America and the Philippine region that lead to local downwelling. 
This flow produces dynamic topography on the surface, which is shown 
in the middle row for 2 refinement steps (c) and 4 refinement steps (d). 
One can see that subduction zones are visible as depressed topography 
due to the downward flow, while regions such as Iceland, Hawaii, or mid ocean 
ridges are elevated due to (deep and) shallow upward flow. The bottom row
shows the heat flux density at the surface for 2 refinement steps (e, colorbar
ranges from 13 to 19 mW/$m^2$) and for 4 refinement steps (f, colorbar ranges
from 35 to 95 mW/$m^2$). Similar to the dynamic topography, a first order correlation
with upper mantle features such as high heat flow at mid ocean ridges and low
heat flow at cratons is correctly initialized by the tomography model. 
This model uses a highly simplified material model that is incompressible and 
isoviscous and does therefore not represent real mantle flow. More realistic 
material properties, density scaling as well as boundary conditions will affect the magnitude 
and pattern shown here. 

\begin{figure}
  \includegraphics[width=\textwidth]{cookbooks/initial-condition-S20RTS/full_mantle.png}
  \hfill
  \caption{\it Panels a and b show the density distribution as prescribed from the shear
  wave velocity model S20RTS and the resulting flow for a global refinement of 4. This
  model assumes a constant scaling between shear wave and density perturbations.
  Panel c shows the great circle (dashed blue line) along which the top slices
  are evaluated. Panels c and d show the resulting dynamic topography for a global refinement of 2 (c, cookbook)
  and 4 (d). Panels e and f show the resulting heat flux density for a global refinement of
  2 (e, cookbook) and 4 (f). The colorbar ranges from 13 to 19 mW/$m^2$ for panel e and
  from 35 to 95 mW/$m^2$ for panel f.}
  \label{fig:ic-1}
\end{figure}


\marginpar{Use cookbooks/periodic-box.prm}
\marginpar{Finish the GPlates section}

\subsubsection{Using reconstructed surface velocities by GPlates}
\label{sec:cookbooks-gplates}
\textit{This section was contributed by Ren{\'e} Ga{\ss}m{\"o}ller}

In a number of model setups one may want to include a surface velocity boundary
condition that prescribes the velocity according to a specific geologic
reconstruction. The purpose of this kind of models is often to test a proposed
geologic model and compare characteristic convection results to present-day
observables in order to gain information about the initially assumed geologic
input. In this cookbook we present \aspect{}'s interface to the  widely used
plate reconstruction software GPlates, and the steps to go from a geologic plate
reconstruction to a geodynamic model incorporating these velocities as boundary
condition.

\paragraph{Acquiring a plate reconstruction.}

The plate reconstruction that is used in this cookbook is included in
the \texttt{data/velocity-boundary-conditions/gplates/} directory of
your \aspect{} installation.
For a new model setup however, a user eventually needs to create her own data
files, and so we will briefly discuss the process of acquiring a usable plate
reconstruction and transferring it into a format usable by \aspect{}.
Both the necessary software and data are provided by the GPlates project.
GPlates is an open-source software for interactive visualization of plate
tectonics. It is developed by the EarthByte Project in the School of Geosciences
at the University of Sydney, the Division of Geological and Planetary Sciences
(GPS) at CalTech and the Center for Geodynamics at the Norwegian Geological
Survey (NGU). For extensive documentation and support we refer to the GPlates
website (\url{http://www.gplates.org}). Apart from the software one needs the
actual plate reconstruction that consists of closed polygons covering the
complete model domain. For our case we will use the data provided by
\cite{GTZDSMBSMB12} that is available from the GPlates website under ``Download
$\rightarrow$ Download GPlates-compatible data $\rightarrow$ Global
reconstructions with continuously closing plates from 140 Ma to the present''.
The data is provided under a Creative Commons Attribution 3.0 Unported License
(\url{http://creativecommons.org/licenses/by/3.0/}).

\paragraph{Converting GPlates data to \aspect{} input.}

After loading the data
files into GPlates (*.gpml for plate polygons, *.rot for plate rotations over
time) the user needs to convert the GPlates data to velocity
information usable in \aspect{}. The purpose of this step is to convert from the
description GPlates uses internally (namely a representation of plates as
polygons that rotate with a particular velocity around a pole) to one that can
be used by \aspect{} (which needs velocity vectors defined at individual points
at the surface).

With loaded plate polygon and rotation information the conversion from GPlates
data to \aspect{}-readable velocity files is rather straightforward. First the
user needs to generate (or import) so-called ``velocity domain points'', which
are discrete sets of points at which GPlates will evaluate velocity
information. This is done using the ``Features $\rightarrow$ Generate Velocity
Domain Points $\rightarrow$ Latitude Longitude'' menu option. Because \aspect{}
is using an adaptive mesh it is not possible for GPlates to generate velocity
information at the precise surface node positions like for CitcomS or Terra (the
other currently available interfaces). Instead GPlates will output the
information on a general Latitude/Longitude grid with nodes on all crossing
points. \aspect{} then internally interpolates this information to the current
node locations during the model run. This requires the user to
choose a sensible resolution of the GPlates output, which can be adjusted in
the ``Generate Latitude/Longitude Velocity Domain Points'' dialog of GPlates. In
general a resolution that resolves the important features is necessary, while a
resolution that is higher than the maximal mesh size for the \aspect{}
model is unnecessary and only increases the computational cost and memory consumption of
the model. 

\textbf{Important note:} The Mesh creation routine in GPlates has significantly 
changed from version 1.3 to 1.4. In GPlates 1.4 and later the user has to make 
sure that the number of longitude intervals is set as twice the number of 
latitude intervals, the ``Place node points at centre of latitude/longitude
cells'' box is \textbf{un}checked and the ``Latitude/Longitude extents'' are set
to ``Use Global Extents''. \aspect{} does check for most possible combinations that
can not be read and will cancel the calculation in these cases, however some
mistakes can not be checked against from the information provided in the GPlates file.

After creating the Velocity Domain Points the user should see the
created points and their velocities indicated as points and arrows in GPlates.
To export the calculated velocities one would use the ``Reconstruction
$\rightarrow$ Export'' menu. In this dialog the user may specify the time
instant or range at which the velocities shall be exported. The only necessary option is
to include the ``Velocities'' data type in the ``Add Export'' sub-dialog. The
velocities need to be exported in the native GPlates \texttt{*.gpml} format,
which is based on XML and can be read by \aspect{}. In case of a time-range the
user needs to add a pattern specifier to the name to create a series of files.
The \texttt{\%u} flag is especially suited for the interaction with \aspect{},
since it can easily be
replaced by a calculated file index (see also
\ref{sec:time-dependent-gplates-velocities}).

\paragraph{Setting up the \aspect{} model.}

For this cookbook we will use the parameter file provided in
\url{cookbooks/gplates-2d.prm} which uses the 2d shell geometry previously
discussed in Section~\ref{sec:shell-simple-2d}. \aspect{}'s GPlates plugin
allows for the use of two- and three-dimensional models incorporating the
GPlates velocities. Since the output by GPlates is threedimensional in any case,
\aspect{} internally handles the 2D model by rotating the model plane to the
orientation specified by the user and projecting the plate velocities into this plane. The
user specifies the orientation of the model plane by prescribing two points that
define a plane together with the coordinate origin (i.e. in the current
formulation only great-circle slices are allowed). The coordinates need to be in
spherical coordinates $\theta$ and $\phi$ with $\theta$ being the colatitude (0
at north pole) and $\phi$ being the longitude (0 at Greenwich meridian, positive
eastwards) both given in radians.
The approach of identifying two points on the surface of the Earth along with
its center allows to run computations on arbitrary two-dimensional slices
through the Earth with realistic boundary conditions.

The relevant section of the input file is then as follows:

\lstinputlisting[language=prmfile]{cookbooks/gplates/gplates.part.prm.out}

In the model settings subsection the user prescribes the boundary that is supposed to 
use the GPlates plugin. Although currently nothing forbids the user to use GPlates plugin for other
boundaries than the surface, its current usage and the provided sample data only make sense
for the surface of a spherical shell (boundary number 1 in the above provided parameter file). 
In case you are familiar with this kind of modeling and the plugin you could however also use it to prescribe mantle
movements \textit{below} a lithosphere model. All plugin specific options may be set in 
section~\ref{parameters:Boundary_20velocity_20model}. Possible options include the data directory
and file name of the velocity file/files, the time step (in model units, mostly seconds or years depending on the 
``\texttt{Use years in output instead of seconds}'' flag) and the points that define the 2D plane.
The parameter ``\texttt{Interpolation width}'' is used to smooth the provided velocity files by
a moving average filter. All velocity data points within this distance are averaged to determine the
actual boundary velocity at a certain mesh point. This parameter is usually set to 0 (no interpolation, use
nearest velocity point data) and is only needed in case the original setting is unstable or slowly converging.

\paragraph{Comparing and visualizing 2D and 3D models.}

The implementation of plate velocities in both two- and three-dimensional model
setups allows for an easy comparison and test for common sources of error
in the interpretation of model results. The left top figure in Fig.~\ref{fig:gv-1} 
shows a modification of the above presented parameter file by setting 
``\texttt{Dimension = 3}'' and ``\texttt{Initial global refinement = 3}''. 
The top right plot of Fig.~\ref{fig:gv-1} shows an example of three independent 
two-dimensional computations of the same reduced resolution. The models were prescribed 
to be orthogonal slices by setting:

\lstinputlisting[language=prmfile]{cookbooks/gplates/slice1.part.prm.out}
and
\lstinputlisting[language=prmfile]{cookbooks/gplates/slice2.part.prm.out}


The results of these models are plotted simultaneously in a single three-dimensional figure
in their respective model planes. The necessary information 
to rotate the 2D models to their respective planes (rotation axis and angle) is provided by the 
GPlates plugin in the beginning of the model output.  The bottom plot 
of Fig.~\ref{fig:gv-1} finally shows the results of the original \url{cookbooks/gplates-2d.prm} 
also in the three mentioned planes. 

Now that we have model output for otherwise identical 2D and 3D models with equal resolution and additional 2D output
for a higher resolution an interesting question to ask would be: What additional information can be created by
either using three-dimensional geometry or higher resolution in mantle convection models with prescribed boundary velocities.
As one can see in the comparison between the top right and bottom plot in Fig.~\ref{fig:gv-1} additional resolution clearly
improves the geometry of small scale features like the shape of up- and downwellings as well as the maximal temperature
deviation from the background mantle. However, the limitation to two dimensions leads to inconsistencies, 
that are especially apparent at the cutting lines of the individual 2D models. 
Note for example that the Nacza slab of the South American subduction zone is only
present in the equatorial model plane and is not captured in the polar model plane west 
of the South American coastline. The (coarse) three-dimensional model on the other hand
shows the same location of up- and downwellings but additionally provides a consistent solution
that is different from the two dimensional setups. Note that the Nazca slab is subducting eastward,
while all 2D models (even in high resolution) predict a westward subduction.

Finally we would like to emphasize that these models (especially the used material model)
are way too simplified to draw any scientific conclusion out of it. Rather it is thought
as a proof-of-concept what is possible with the dimension independent approach of
\aspect{} and its plugins.

\begin{figure}
  \includegraphics[width=\textwidth]{cookbooks/gplates/gplates-comparison.png}
  \hfill
  \caption{\it Using GPlates for velocity boundary conditions: The top left figure shows
  the results of a three-dimensional model using the present day plate velocities
  provided by GPlates as surface boundary condition. 
  The top right figure shows three independent computations 
  on two-dimensional slices through Earth. The boundary conditions for each of these slices (white
  arrows) are tangential to the slices and are projections of the
  three-dimensional velocity vectors into the two-dimensional plane
  occupied by the slice. While the two top models are created with the same mesh resolution
  the bottom figure shows three independent two-dimensional models using a higher resolution. 
  The view is centered on South America with Antarctica being near the bottom of
  the figure (coastlines provided by NGU and the GPlates project).}
  \label{fig:gv-1}
\end{figure}

\paragraph{Time-dependent boundary conditions.}
\label{sec:time-dependent-gplates-velocities}
The example presented above uses a constant velocity boundary field that
equals the present day plate movements. For a number of purposes one may want to
use a prescribed velocity boundary condition that changes over time, for example
to investigate the effect of trench migration on subduction. Therefore \aspect{}'s
GPlates plugin is able to read in multiple velocity files and linearly interpolate
between pairs of files to the current model time. To achieve this, one needs
to use the \texttt{\%d} wildcard in the velocity file name, which represents the current
velocity file index (e.g. \texttt{time\_dependent.\%d.gpml}). This index is
calculated by dividing the current model time by the user-defined time step
between velocity files (see parameter file above). As the model time progresses
the plugin will update the interpolation accordingly and if necessary read in
new velocity files. In case it can not read the next velocity file, it assumes
the last velocity file to be the constant boundary condition until the end of
the model run. One can test this behavior with the provided data files
\texttt{data/velocity\_boundary\_conditions/gplates/time\_dependent.\%d.gpml}
with the index \texttt{d} ranging from 0 to 3 and representing the plate movements of
the last 3 million years corresponding to the same plate reconstruction as used
above. Additionally, the parameter \texttt{Velocity file start time} allows for
a period of no-slip boundary conditions before starting the use of the GPlates plugin.
This is a comfort implementation, which could also be achieved by using the checkpointing
possibility described in section~\ref{sec:checkpoint-restart}. 

\subsubsection{Reproducing rheology of Morency and Doin, 2004}
\label{sec:cookbooks-morency-doin}
\textit{This section was contributed by Jonathan Perry-Houts}

Modeling interactions between the upper mantle and the lithosphere can be difficult because
of the dynamic range of temperatures and pressures involved. Many simple material models
will assign very high viscosities at low temperature thermal boundary layers. The
pseudo-brittle rheology described in \cite{MD04} was developed to limit the strength of
lithosphere at low temperature. The effective viscosity can be described as the harmonic mean
of two non-Newtonian rheologies:
\[v_\text{eff} = \left(\frac{1}{v_\text{eff}^v}+\frac{1}{v_\text{eff}^p}\right)^{-1}\]
where
\begin{align*}
  v_\text{eff}^v = B \left(\frac{\dot{\epsilon}}{\dot{\epsilon}_{ref}}\right)^{-1+1/n_v}
  \exp\left(\frac{E_a +V_a \rho_m g z}{n_v R T}\right),
  \\
  v_\text{eff}^p = (\tau_0 + \gamma \rho_m g z) \left( \frac{\dot{\epsilon}^{-1+1/n_p}}
  {\dot{\epsilon}_{ref}^{1/n_p}} \right),
\end{align*}
where $B$ is a scaling constant; $\dot{\epsilon}$ is defined as the quadratic sum of the
second invariant of the strain rate tensor and a minimum strain rate, $\dot{\epsilon}_0$;
$\dot{\epsilon}_{ref}$ is a reference strain rate; $n_v$, and $n_p$ are stress exponents;
$E_a$ is the activation energy; $V_a$ is the activation volume; $\rho_m$ is the mantle density;
$R$ is the gas constant; $T$ is temperature; $\tau_0$ is the cohesive strength of rocks at
the surface; $\gamma$ is a coefficient of yield stress increase with depth; and $z$ is depth.

By limiting the strength of the lithosphere at low temperature, this rheology allows one to
more realistically model processes like lithospheric delamination and foundering in the
presence of weak crustal layers. A similar model setup to the one described in \cite{MD04}
can be reproduced with the following parameters.

\note{\cite{MD04} defines the second invariant of the strain rate in a nonstandard way.
    The formulation in the paper is given as $\epsilon_{II} = \sqrt{\frac{1}{2}
    (\epsilon_{11}^2 + \epsilon_{12}^2)}$, where $\epsilon$ is the strain rate tensor.
    For consistency, that is also the formulation implemented in \aspect{}. Because of this
    irregularity it is inadvisable to use this material model for purposes beyond
    reproducing published results.}

\note{The viscosity profile in Figure 1 of \cite{MD04} appears to be wrong. The published
    parameters do not reproduce those viscosities; it is unclear why. The values used here
    get very close. See Figure~\ref{fig:md-1} for an approximate reproduction of the
    original figure.}

\lstinputlisting[language=prmfile]{cookbooks/morency_doin/morency_doin.prm.out}

\begin{figure}[h!]
  \includegraphics[width=\textwidth]{cookbooks/morency_doin/morency_doin_2004_fig1.pdf}
  \caption{\it Approximate reproduction of figure 1 from \cite{MD04} using the `morency doin'
  material model with almost all default parameters. Note the low-viscosity Moho, enabled by
  the low activation energy of the crustal component.}
  \label{fig:md-1}
\end{figure}

\subsubsection{Crustal deformation}
\label{sec:cookbooks-crustal-deformation}

\textit{This section was contributed by Cedric Thieulot, and makes use of the Drucker-Prager 
material model written by Anne Glerum and the free surface plugin by Ian Rose.}

This is a simple example of an upper-crust undergoing compression or extension.
It is characterized by a single layer of visco-plastic material subjected to basal 
kinematical boundary conditions. In compression, this setup is somewhat analogous 
to \cite{will99}, and in extension to \cite{alht11}.

Brittle failure is approximated by adapting the viscosity to limit the stress 
that is generated during deformation. 
This ``cap'' on the stress level is parameterized in this experiment by the pressure-dependent 
Drucker Prager yield criterion  and we therefore make use of the Drucker-Prager 
material model (see Section \ref{parameters:Material_20model}) in the 
{\tt cookbooks/crustal\_model\_2D.prm}.

The layer is assumed to have dimensions of 80km $\times$ 16km and to have a density  $\rho=2800$ kg/m$^3$. 
The plasticity parameters are specified as follows:

\lstinputlisting[language=prmfile]{cookbooks/crustal_deformation/crustal_model_2D_part1.prm}

The yield strength $\sigma_y$ is a function of pressure, cohesion and angle of friction 
(see Drucker-Prager material model in Section \ref{parameters:Material_20model}),
and the effective viscosity is then computed as follows:
\[
\mu_\text{eff} = \left( \frac{1}{ \frac{\sigma_y}{2 \dot{\epsilon}}+
\mu_\text{min}} + \frac{1}{\mu_\text{max}}  \right)^{-1}
\]
where $\dot{\epsilon}$ is the square root of the second invariant of the deviatoric strain rate.
The viscosity cutoffs insure that the viscosity remains within computationally acceptable values.

During the first iteration of the first timestep, the strain rate is zero, so
we avoid dividing by zero by setting the strain rate to {\tt Reference strain rate}.


The top boundary is a free surface while the left, right and bottom boundaries are subjected 
to the following boundary conditions:

\lstinputlisting[language=prmfile]{cookbooks/crustal_deformation/crustal_model_2D_part2.prm}

Note that compressive boundary conditions are simply achieved by reversing  
the sign of the imposed velocity.

The free surface will be advected up and down according to the solution of the Stokes solve.
We have a choice whether to advect the free surface in the direction of the surface normal
or in the direction of the local vertical (i.e., in the direction of gravity).
For small deformations, these directions are almost the same, but in this example the deformations 
are quite large. We have found that when the deformation is large, advecting the surface vertically 
results in a better behaved mesh, so we set the following in the free surface subsection:

\lstinputlisting[language=prmfile]{cookbooks/crustal_deformation/crustal_model_2D_part3.prm}

We also make use of the strain rate-based mesh refinement plugin:

\lstinputlisting[language=prmfile]{cookbooks/crustal_deformation/crustal_model_2D_part4.prm}

Setting 
{\tt   set Initial adaptive refinement        = 4}
yields a series of meshes as shown in Fig. (\ref{fig:meshes}), all produced during the 
first timestep. As expected, we see that the location of the highest mesh refinement
corresponds to the location of a set of conjugated shear bands.

If we now set this parameter to 1 and allow the simulation to evolve
for 500kyr, a central graben or plateau (depending on the nature
of the boundary conditions) develops and deepens/thickens over time, nicely showcasing 
the unique capabilities of the code to handle free surface large deformation, localised
strain rates through visco-plasticity and adaptive mesh refinement as
shown in Fig. (\ref{fig:extcompr}).

\begin{figure}
   \centering
   \includegraphics[width=0.7\textwidth]{cookbooks/crustal_deformation/grids}
   \caption{\it Mesh evolution during the first timestep (refinement is based on strain rate).}
   \label{fig:meshes}
\end{figure}

 

Deformation localizes at the basal velocity discontinuity and plastic shear bands
form at an angle of approximately $53^\circ$ to the bottom in extension and 
$35^\circ$ in compression, both of which correspond to the reported Arthur angle \cite{kaus10,buit12}. 

\begin{figure}
  \centering
  \includegraphics[width=\textwidth]{cookbooks/crustal_deformation/both}
  \caption{\it Finite element mesh, velocity, viscosity and strain rate fields
  in the case of extensional boundary conditions (top) and compressive boundary conditions (bottom) at t=500kyr.}
  \label{fig:extcompr}
\end{figure}



\paragraph{Extension to 3D.} We can easily modify the previous 
input file to produce {\tt crustal\_model\_3D.prm}
which implements a similar setup, with the additional constraint that the position 
of the velocity discontinuity varies with the $y$-coordinate, 
as shown in Fig. (\ref{fig:bottombc}). 
The domain is now 
$128\times96\times16$km and the boundary conditions are implemented as
follows:

\lstinputlisting[language=prmfile]{cookbooks/crustal_deformation/crustal_model_3D_part1.prm}

The presence of 
an offset between the two velocity discontinuity zones leads to a transform 
fault which connects them. 

\begin{figure}
  \centering
  \includegraphics[width=\textwidth]{cookbooks/crustal_deformation/bottombc2}
  \caption{\it Basal velocity boundary conditions and corresponding 
  strain rate field for the 3D model.} 
  \label{fig:bottombc}
\end{figure}

The Finite Element mesh, the velocity, viscosity and strain rate fields are shown 
in Fig. (\ref{fig:ext3D}) at the end of the first time steps. The reader is encouraged
to run this setup in time to look at how the two grabens interact as a function 
of their initial offset \cite{alht11,alht12,alhf13}.

\begin{figure}
\centering
\includegraphics[width=\textwidth]{cookbooks/crustal_deformation/all3D.png}
\caption{\it Finite element mesh, velocity, viscosity and strain rate fields at
the end of the first time step after one level of strain rate-based adaptive mesh refinement.} 
\label{fig:ext3D}
\end{figure}


\subsubsection{Inner core convection}
\label{sec:cookbooks-inner-core-convection}

\textit{This section was contributed by Juliane Dannberg, and the model setup was inspired 
by discussions with John Rudge.}

This is an example of convection in the inner core of the Earth. It uses a spherical geometry 
and is characterized by a single material with temperature dependent density that makes it
unstably stratified as temperature increases towards the center of the core. 
The boundary conditions combine normal stress and normal velocity, and take into account the
rate of phase change (melting/freezing) at the inner-outer core boundary. Gravity decreases 
linearly from the boundary to zero at the center of the inner core. 
The setup is analogous to the models described in \cite{Deguen2013}, and all material properties 
are chosen in a way so that the equations are non-dimensional.

The required heating model and changes to the material model are implemented in a shared library
(\url{cookbooks/inner_core_convection/inner_core_convection.cc}). 

In their non-dimensional form, the equations are

\begin{align}
  \label{eq:inner-core-1}
  \nabla \cdot \sigma &=
  -Ra T \mathbf g,
  \\
  \label{eq:inner-core-2}
  \nabla \cdot \mathbf u &= 0,
  \\
  \label{eq:inner-core-3}
  \left(\frac{\partial T}{\partial t} + \mathbf u\cdot\nabla T\right)
  - \nabla^2 T
  &=
  H,
\end{align}

where $Ra$ is the Rayleigh number. 

The mechanical boundary conditions for the inner core are
tangential stress-free and continuity of the normal stress at the
inner-outer core boundary. For the non-dimensional equations, that
means that we can define a ``phase change number'' $\mathcal{P}$ so that the
normal stress at the boundary is $-\mathcal{P} u_r$ with the radial velocity
$u_r$. This number characterizes the resistance to phase change at
the boundary, with $\mathcal{P}\rightarrow\infty$ corresponding to infinitely slow
melting/freezing (or a free slip boundary), and $\mathcal{P}\rightarrow0$ corresponding to
instantaneous melting/freezing (or a zero normal stress, corresponding to an open boundary).

In the weak form, this results in boundary conditions of the form
of a surface integral:
\begin{equation*}
\int_S \mathcal{P} (\mathbf u \cdot \mathbf n) (\mathbf v \cdot \mathbf n) dS,
\end{equation*}
with the normal vector $\mathbf n$.

This phase change term is added to the matrix in the 
\url{cookbooks/inner_core_convection/inner_core_assembly.cc} plugin by using a signal 
(as described in Section~\ref{sec:extending-signals}). The signal 
connects the function \verb!set_assemblers_phase_boundary!, which is only called once at the beginning of 
the model run. It creates the new assembler \texttt{PhaseBoundaryAssembler} for the boundary faces of the Stokes 
system and adds it to the list of assemblers executed in every time step. 
The assembler contains the function \verb!phase_change_boundary_conditions! that loops over all faces at the model 
boundary, queries the value of $\mathcal{P}$ from the material model, and adds the surface integral given above
to the matrix:
\lstinputlisting[language=C++,,basicstyle=\footnotesize\ttfamily,]{../../cookbooks/inner_core_convection/inner_core_assembly.cc}
Instructions for how to compile and run models with a shared library are given in Section~\ref{sec:benchmark-run}.

Analyzing Equations~\eqref{eq:inner-core-1}--\eqref{eq:inner-core-3} reveals that two parameters determine 
the dynamics of convection in the inner core: the Rayleigh number $Ra$ and the phase change number $\mathcal{P}$. 
For low Rayleigh numbers below a critical value $Ra_c$ that is approximately $1000$ for high $\mathcal{P}$ ($\gg 29$), and 
that depends linearly on $\mathcal{P}$ for smaller $\mathcal{P}$ (see \cite{Deguen2013}), there is no convection 
and thermal diffusion dominates the heat transport. 
For higher values of $Ra > Ra_c$, the style of convection in the inner core depends on $\mathcal{P}$: 
For high $\mathcal{P}$ ($\gg 29$, see \cite{Deguen2013}), convection is dominated by thermal plumes with the number 
of plumes scaling with the Rayleigh number; whereas for low values of $\mathcal{P}$, the inner core is in a 
translation regime, where material freezes at one side and melts at the other side, so that the velocity field 
is uniform, pointing from the freezing to the melting side. 

\begin{figure}
    \centering
    \begin{tabular}{lcl}
    \includegraphics[height=0.25\textwidth]{cookbooks/inner_core_convection/convection.png}
    & \hspace{1cm} &
    \includegraphics[height=0.25\textwidth]{cookbooks/inner_core_convection/convection_contours.png}
    \\
    \includegraphics[height=0.25\textwidth]{cookbooks/inner_core_convection/translation.png}
    & \hspace{1cm} &
    \includegraphics[height=0.25\textwidth]{cookbooks/inner_core_convection/translation_contours.png}
    \\
    \includegraphics[height=0.25\textwidth]{cookbooks/inner_core_convection/no_convection.png}
    & \hspace{1cm} &
    \includegraphics[height=0.25\textwidth]{cookbooks/inner_core_convection/no_convection_contours.png}
    \end{tabular}
    \caption{Convection regimes in the inner core for different values of $Ra$ and $\mathcal{P}$.
             From top to bottom: plume convection, translation, no convection; full 3d models
             on the left and 2d slices on the right. Temperature is shown as contours, 
             velocity is shown as arrows.}
    \label{fig:inner-core-regimes}
\end{figure}

Changing the values of $Ra$ and $\mathcal{P}$ in the input file allows switching between the different regimes.
The Rayleigh number can be changed by adjusting the magnitude of the gravity:
\lstinputlisting[language=prmfile]{cookbooks/inner_core_convection/inner_core_traction.part.2.prm}
The phase change number is implemented as part of the material model, and as a function that can depend on the 
spatial coordinates and/or on time: 
\lstinputlisting[language=prmfile]{cookbooks/inner_core_convection/inner_core_traction.part.1.prm}

Figure~\ref{fig:inner-core-regimes} shows examples of the three regimes with $Ra=3000, \mathcal{P}=1000$ (plume convection), 
$Ra=10^5, \mathcal{P}=0.01$ (translation), $Ra=10, \mathcal{P}=30$ (no convection).


In addition, \cite{Deguen2013} give scaling laws for the velocities in each regime derived from linear stability 
analysis, and show how numerical results compare to them. In the regimes of low $\mathcal{P}$, translation will start 
at a critical ratio of Rayleigh number and phase change number $\frac{Ra}{\mathcal{P}}=\frac{175}{2}$ with steady-state 
translation velocities being zero below this threshold and tending to $v_0=\frac{175}{2}\sqrt{\frac{6}{5}\frac{Ra}{\mathcal{P}}}$ 
going towards higher values of $\frac{Ra}{\mathcal{P}}$.
In the same way, translation velocities will decrease from $v_0$ with increasing $\mathcal{P}$, with translation transitioning
to plume convection at $\mathcal{P}\sim29$. 
Both trends are shown in Figure~\ref{fig:inner-core-trends} and can be compared to Figure~8 and 9 in \cite{Deguen2013}.
 
\begin{figure}
    \centering
    \includegraphics[width=0.49\textwidth]{cookbooks/inner_core_convection/translation_over_Ra_P.pdf}
    \hfill
    \includegraphics[width=0.49\textwidth]{cookbooks/inner_core_convection/translation_over_P.pdf}
    \caption{Translation rate (approximated by the average of the velocity component in the direction of translation), 
    normalized to the low $\mathcal{P}$ 
    limit estimate given in \cite{Deguen2013}, as a function of $\frac{Ra}{\mathcal{P}}$ for $\mathcal{P}=10^{-2}$ 
    (left) and as a function of $\mathcal{P}$ for $Ra=10^5$ (right).
    The dashed gray line gives the translation velocity predicted in the limit of low $\mathcal{P}$. Disagreement 
    for larger values of $\mathcal{P}$ indicates that higher order terms (not included in the low $\mathcal{P}$
    approximation) become important. Additionally, differences between the analytical and numerical model might 
    be the result of limited resolution (only 12 elements in radial direction).}
    \label{fig:inner-core-trends}
\end{figure}

\subsection{Benchmarks}
\label{sec:cookbooks-benchmarks}

Benchmarks are used to verify that a solver solves the problem correctly,
i.e., to \textit{verify} correctness of a code.%
\footnote{Verification is the first half of the \textit{verification and
    validation} (V\&V) procedure: \textit{verification} intends to ensure that the
  mathematical model is solved correctly, while \textit{validation} intends to
  ensure that the mathematical model is correct. Obviously, much of the aim of
  computational geodynamics is to validate the models that we have.}
Over the past decades, the geodynamics community has come up with a large
number of benchmarks. Depending on the goals of their original inventors, they
describe stationary problems in which only the solution of the flow problem is
of interest (but the flow may be compressible or incompressible, with constant
or variable viscosity, etc), or they may actually model time-dependent
processes. Some of them have solutions that are analytically known and can be
compared with, while for others, there are only sets of numbers that are
approximately known. We have implemented a number of them in \aspect{} to
convince ourselves (and our users) that \aspect{} indeed works as intended and
advertised. Some of these benchmarks are discussed below. Numerical results
for several of these benchmarks are also presented in \cite{KHB12} in much more
detail than shown here.

\subsubsection{Running benchmarks that require code}
\label{sec:benchmark-run}

Some of the benchmarks require plugins like custom material models, boundary
conditions, or postprocessors. To not pollute \aspect{} with all these
purpose-built plugins, they are kept separate from the more generic plugins in
the normal source tree. Instead, the benchmarks have all the necessary code in
\texttt{.cc} files in the benchmark directories. Those are then compiled into a shared
library that will be used by \aspect{} if it is referenced in a \texttt{.prm}
file. Let's take the SolCx benchmark as an example (see Section \ref{sec:benchmark-solcx}).
The directory contains:
\begin{itemize}
 \item \texttt{solcx.cc} -- the code file containing a material model
   ``SolCxMaterial'' and a postprocessor ``SolCxPostprocessor'',
 \item \texttt{solcx.prm} -- the parameter file referencing these plugins,
 \item \texttt{CMakeLists.txt} -- a cmake configuration that allows you to
   compile \texttt{solcx.cc}.
\end{itemize}
To run this benchmark you need to follow the general outline of
steps discussed in Section~\ref{sec:write-plugin}. For the current case, this
amounts to the following:
\begin{enumerate}
 \item Move into the directory of that particular benchmark:
\begin{verbatim}
 $ cd benchmark/solcx
\end{verbatim} 
 \item Set up the project:
\begin{verbatim}
 $ cmake .  
\end{verbatim}
 By default, \texttt{cmake} will look for the \aspect{} binary and other
 information in a number of directories relative to the current one.
 If it is unable to pick up where \aspect{} was built and installed, you can
 specify this directory explicitly this using \texttt{-D
   Aspect\_DIR=$<$...$>$} as an additional flag to \texttt{cmake}, where
 \texttt{$<$...$>$} is the path to the build directory.
 \item Build the library:
\begin{verbatim}
 $ make
\end{verbatim}
 This will generate the file \texttt{libsolcx.so}.
\end{enumerate}
Finally, you can run \aspect{} with \texttt{solcx.prm}:
\begin{verbatim}
 $ ../../aspect solcx.prm
\end{verbatim}
where again you may have to use the appropriate path to get to the \aspect{}
executable. You will need to run \aspect{} from the current directory because
\texttt{solcx.prm} refers to the plugin as \texttt{./libsolcx.so}, i.e., in
the current directory.



\subsubsection{The van Keken thermochemical composition benchmark}
\label{sec:benchmark-van-keken}

\textit{This section is a co-production of Cedric Thieulot, Juliane Dannberg,
Timo Heister and Wolfgang Bangerth with an extension to this benchmark provided by the Virginia Tech Department of Geosciences class ``Geodynamics and ASPECT'' co-taught by Scott King and D.~Sarah Stamps.}

One of the most widely used benchmarks for mantle convection codes is the
isoviscous Rayleigh-Taylor case (``case 1a'') published by van Keken \textit{et
al.} in \cite{KKSCND97}.
The benchmark considers a 2d situation where a lighter fluid underlies a heavier
one with a non-horizontal interface between the two of them. This unstable
layering causes the lighter fluid to start rising at the point where the
interface is highest. Fig.~\ref{fig:vk-1} shows a time series of images to
illustrate this.

\begin{figure}
  \includegraphics[width=0.23\textwidth]{cookbooks/benchmarks/van-keken/movie0000.png}
  \hfill
  \includegraphics[width=0.23\textwidth]{cookbooks/benchmarks/van-keken/movie0003.png}
  \hfill
  \includegraphics[width=0.23\textwidth]{cookbooks/benchmarks/van-keken/movie0009.png}
  \hfill
  \includegraphics[width=0.23\textwidth]{cookbooks/benchmarks/van-keken/movie0018.png}
  \caption{\it Van Keken benchmark (using a smoothed out interface, see the main
  text):
  Compositional field at times $t=0, 300, 900, 1800$.}
  \label{fig:vk-1}
\end{figure}


Although van Keken's paper title suggests that the paper is really about
thermochemical convection, the part we look here can equally be considered as
thermal or chemical convection: all that is necessary is that we describe the
fluid's density somehow. We can do that by using an inhomogenous initial
temperature field, or an inhomogenous initial composition field. We will use the
input file in \url{cookbooks/van-keken-discontinuous.prm} as input, the central
piece of which is as follows (go to the actual input file
to see the remainder of the input parameters):

\lstinputlisting[language=prmfile]{cookbooks/benchmarks/van-keken/main.part.prm.out}

The first part of this selects the \texttt{simple} material model and sets the
thermal expansion to zero (resulting in a density that does not depend on the
temperature, making the temperature a passively advected field) and instead
makes the density depend on the first compositional field. The second section
prescribes that the first compositional field's
initial conditions are 0 above a line describes by a cosine and 1 below it.
Because the dependence of the density on the compositional field is negative,
this means that a lighter fluid underlies a heavier one.

The dynamics of the resulting flow have already been shown in
Fig.~\ref{fig:vk-1}. The measure commonly considered in papers comparing
different methods is the root mean square of the velocity, which we can get
using the following block in the input file (the actual input file also enables
other postprocessors):

\lstinputlisting[language=prmfile]{cookbooks/benchmarks/van-keken/postprocess.part.prm.out}

Using this, we can plot the evolution of the fluid's average velocity over time,
as shown in the left panel of Fig.~\ref{fig:vk-2}. Looking at this graph, we
find that both the timing and the height of the first peak is already
well converged on a simple
$32\times 32$ mesh (5 global refinements) and is very consistent (to better
than 1\% accuracy) with the results in the van Keken paper.

\begin{figure}
  \includegraphics[width=0.48\textwidth]{cookbooks/benchmarks/van-keken/velocity-discontinuous.png}
  \hfill
  \includegraphics[width=0.48\textwidth]{cookbooks/benchmarks/van-keken/velocity-smooth.png}
  \caption{\it Van Keken benchmark with discontinuous (left) and smoothed,
  continuous (right) initial conditions for the compositional field:
  Evolution of the root mean square velocity $\left(\frac 1{|\Omega|}\int_\Omega |\mathbf u(\mathbf x,t)|^2 \;
  dx\right)^{1/2}$ as a function of time for different numbers of global mesh
  refinements. 5 global refinements correspond to a $32\times 32$ mesh, 9
  refinements to a $512\times 512$ mesh.}
  \label{fig:vk-2}
\end{figure}

That said, it is startling that the second peak does not appear to converge
despite the fact that the various codes compared in \cite{KKSCND97} show good
agreement in this comparison. Tracking down the cause for this proved to be a
lesson in benchmark design; in hindsight, it may also explain why van Keken
\textit{et al.} stated presciently in their abstract that ``\textit{\ldots good
agreement is found for the initial rise of the unstable lower layer; however, the timing
  and location of the later smaller-scale instabilities may differ between
  methods.}''
To understand what is happening here, note that the first peak in these plots
corresponds to the plume that rises along the left edge of the domain and whose
evolution is primarily determined by the large-scale shape of the initial
interface (i.e., the cosine used to describe the initial conditions in the
input file). This is a first order deterministic effect, and is obviously
resolved already on the coarsest mesh shown used. The second peak corresponds to
the plume that rises along the right edge, and its origin along the interface is
much harder to trace -- its position and the timing when it starts to rise is
certainly not obvious from the initial location of the interface. Now recall
that we are using a finite element field using continuous shape functions for
the composition that determines the density differences that drive the flow. But
this interface is neither aligned with the mesh, nor can a discontinuous
function be represented by continuous shape functions to begin with. In other
words, we may \textit{input} the initial conditions as a discontinuous functions
of zero and one in the parameter file, but the initial conditions used in the
program are in fact different: they are the \textit{interpolated} values of this
discontinuous function on a finite element mesh. This is shown in
Fig.~\ref{fig:vk-3}. It is obvious that these initial conditions agree on the
large scale (the determinant of the first plume), but not in the steps that may
(and do, in fact) determine when and where the second plume will rise. The
evolution of the resulting compositional field is shown in Fig.~\ref{fig:vk-4}
and it is obvious that the second, smaller plume starts to rise from a
completely different location -- no wonder the second peak in the root mean
square velocity plot is in a different location and with different height!

\begin{figure}
  \centering
  \includegraphics[width=0.7\textwidth]{cookbooks/benchmarks/van-keken/mesh-comparison-initial-conditions.png}
  \caption{\it Van Keken benchmark with discontinuous initial conditions for the
  compositional field:
  Initial compositional field interpolated onto a $32\times 32$ (left) and
  $64\times 64$ finite element mesh (right).}
  \label{fig:vk-3}
\end{figure}

\begin{figure}
  \centering
  \includegraphics[height=0.8\textheight]{cookbooks/benchmarks/van-keken/mesh-comparison.png}
  \caption{\it Van Keken benchmark with discontinuous initial conditions for the
  compositional field:
  Evolution of the compositional field over time on a $32\times 32$ (first and
  third column; left to right and top to bottom) and $64\times 64$ finite
  element mesh (second and fourth column).}
  \label{fig:vk-4}
\end{figure}

The conclusion one can draw from this is that if the outcome of a computational
experiment depends so critically on very small details like the steps of an
initial condition, then it's probably not a particularly good measure to look at
in a benchmark. That said, the benchmark is what it is, and so we should try to
come up with ways to look at the benchmark in a way that allows us to reproduce
what van Keken \textit{et al.} had agreed upon. To this end, note that the codes
compared in that paper use all sorts of different methods, and one can certainly
agree on the fact that these methods are not identical on small length scales.
One approach to make the setup more mesh-independent is to replace the original
discontinuous initial condition with a smoothed out version; of course, we can
still not represent it exactly on any given mesh, but we can at least get closer
to it than for discontinuous variables. Consequently, let us use the following
initial conditions instead (see also the file
\url{cookbooks/van-keken-smooth.prm}):
\lstinputlisting[language=prmfile]{cookbooks/benchmarks/van-keken/smooth.part.prm.out}

This replaces the discontinuous initial conditions with a smoothed out version
with a half width of around 0.01. Using this, the root mean square plot now
looks as shown in the right panel of Fig.~\ref{fig:vk-2}. Here, the second peak
also converges quickly, as hoped for.

The exact location and height of the two peaks is in good agreement with those
given in the paper by van Keken \textit{et al.}, but not exactly where desired
(the error is within a couple of per cent for the first peak, and probably
better for the second, for both the timing and height of the peaks).
This has to do with the fact that they depend on the exact size of the smoothing
parameter (the division by 0.02 in the formula for the smoothed initial
condition). However, for more exact results, one can choose
this half width parameter proportional to the mesh size and thereby get more
accurate results. The point of the section was to demonstrate the reason
for the lack of convergence.

In this section we extend the van Keken cookbook following up the work previously completed by Cedric Thieulot, Juliane Dannberg,
Timo Heister and Wolfgang Bangerth.  \textit{This section contributed by Grant Euen, Tahiry Rajaonarison, and Shangxin Liu as part of the Geodynamics and ASPECT class at Virginia Tech.}  

As already mentioned above, using a half width parameter proportional to the mesh size allows for more accurate results.  We test the effect of the half width size of the smoothed discontinuity by changing the division by 0.02, the smoothing parameter, in the formula for the smoothed initial conditions into values proportional to the mesh size.  We use 7 global refinements because the root mean square velocity converges at greater resolution while keeping average runtime around 5 to 25 minutes.  These runtimes were produced by the BlueRidge cluster of the Advanced Research Computing (ARC) program at Virginia Tech.  BlueRidge is a 408-node Cray CS-300 cluster; each node outfitted with two octa-core Intel Sandy Bridge CPUs and 64 GB of memory.  A chart of average runtimes for 5 through 10 global refinements on one node can be seen in Table~\ref{tab:runtime-table}.  For 7 global refinements (128$\times$128 mesh size), the size of the mesh is 0.0078 corresponding to a half width parameter of 0.0039.  The smooth model allows for much better convergence of the secondary plumes, although they are still more scattered than the primary plumes.

\begin{table}[htb]
        \center
        \begin{tabular}{|c|ccccc|}
                \hline
                Global & \multicolumn{4}{|c|}{Number of Processors} \\
                Refinements & 4 & 8 & 12 & 16
                \\ \hline
                5 & 28.1 seconds & 19.8 seconds & 19.6 seconds & 17.1 seconds \\
                6 & 3.07 minutes & 1.95 minutes & 1.49 minutes & 1.21 minutes \\
                7 & 23.33 minutes & 13.92 minutes & 9.87 minutes & 7.33 minutes \\
                8 & 3.08 hours & 1.83 hours & 1.30 hours & 56.33 minutes \\
                9 & 1.03 days & 15.39 hours & 10.44 hours & 7.53 hours \\
                10 & More than 6 days & More than 6 days & 3.39 days & 2.56 days \\ \hline
        \end{tabular}
        \caption{\it Average runtimes for the van Keken Benchmark with smoothed initial conditions.  These times are for the entire computation, a final time step number of 2000.  All of these tests were run using ASPECT version 1.3 in release mode, and used different numbers of processors on one node on the BlueRidge cluster of ARC at Virginia Tech.}
        \label{tab:runtime-table}
\end{table}

This convergence is due to changing the smoothing parameter, which controls how much of the problem is smoothed over.  As the parameter is increased, the smoothed boundary grows and vice versa.  As the smoothed boundary shrinks it becomes sharper until the original discontinuous behavior is revealed.  As it grows, the two layers eventually become one large, transitioning layer rather than two distinct layers separated by a boundary.  These effects can be seen in Fig.~\ref{fig:vk-5}.  The overall effect is that the secondary rise is at different times based on these conditions.  In general, as the smoothing parameter is decreased, the smoothed boundary shrinks and the plumes rise more quickly.  As it is increased, the boundary grows and the plumes rise more slowly.  This trend can be used to force a more accurate convergence from the secondary plumes.

\begin{figure}
        \centering
        \includegraphics[width=0.6\textwidth]{cookbooks/benchmarks/van-keken/smoothing-parameter.png}
        \caption{\it Van Keken Benchmark using smoothed out interface at 7 global refinements: compositional field at time $t=0$ using smoothing parameter size: a) 0.0039, b) 0.0078, c) 0.0156, d) 0.0234, e) 0.0312, f) 0.0390, g) 0.0468, h) 0.0546, i) 0.0624.}
        \label{fig:vk-5}
\end{figure}

The evolution in time of the resulting compositional fields (Fig.~\ref{fig:vk-6}) shows that the first peak converges as the smoothed interface decreases. There is a good agreement for the first peak for all smoothing parameters.  As the width of the discontinuity increases, the second peak rises both later and more slowly.

\begin{figure}
        \centering
        \includegraphics[width=0.4\textwidth]{cookbooks/benchmarks/van-keken/smoothing-parameter-velocity.png}
        \caption{\it Van Keken benchmark with smoothed initial conditions for the compositional field using 7 global refinements for different smoothing parameters.  Number of the time step is shown on the $x$-axis, while root mean square velocity is shown on the $y$-axis.}
        \label{fig:vk-6}
\end{figure}

Now let us further add a two-layer viscosity model to the domain. This is done to recreate the two nonisoviscous Rayleigh-Taylor instability cases (``cases 1b and 1c'') published in van Keken \textit{et al.} in \cite{KKSCND97}.  Let's assume the viscosity value of the upper heavier layer is $\eta_{t}$ and the viscosity value of the lower lighter layer is $\eta_{b}$. Based on the initial constant viscosity value 1$\times10^{2}$ Pa~s, we set the viscosity proportion $\frac{\eta_{t}}{\eta_{b}}=0.1, 0.01$, meaning the viscosity of the upper, heavier layer is still 1$\times10^{2}$ Pa~s, but the viscosity of the lower, lighter layer is now either 10 or 1 Pa~s, respectively. The viscosity profiles of the discontinuous and smooth models are shown in Fig.~\ref{fig:vk-7}.

\begin{figure}
        \centering
        \includegraphics[width=0.7\textwidth]{cookbooks/benchmarks/van-keken/contrast_viscosity.png}
        \caption{\it Van Keken benchmark using layers of different viscosities. The left image is the discontinuous case, while right is the smooth.  Both are shown at t=0.}
        \label{fig:vk-7}
\end{figure}

For both benchmark cases, discontinuous and smooth, and both viscosity proportions, 0.1 and 0.01, the results are shown at the end time step number, 2000, in Fig.~\ref{fig:vk-8}.  This was generated using the original input parameter file, running the cases with 8 global refinement steps, and also adding the two-layer viscosity model.

\begin{figure}
        \centering
        \includegraphics[width=0.5\textwidth]{cookbooks/benchmarks/van-keken/2viscosities-final.png}
        \caption{\it Van Keken benchmark two-layer viscosity model at final time step number, 2000. These images show layers of different compositions and viscosities. Discontinuous cases are the left images, smooth cases are the right. The upper images are $\frac{\eta_{t}}{\eta_{b}}=0.1$, and the lower are $\frac{\eta_{t}}{\eta_{b}}=0.01$.}
        \label{fig:vk-8}
\end{figure}

Compared to the results of the constant viscosity throughout the domain, the plumes rise faster when adding the two-layer viscosity model. Also, the larger the viscosity difference is, the earlier the plumes appear and the faster their ascent. To further reveal the effect of the two-layer viscosity model, we also plot the evolution of the fluids' average velocity over time, as shown in Fig.~\ref{fig:vk-9}.

\begin{figure}
        \centering
        \includegraphics[width=0.4\textwidth]{cookbooks/benchmarks/van-keken/2viscosities-velocity.png}
        \caption{\it Van Keken benchmark: Evolution of the root mean square velocity as a function of time for different viscosity contrast proportions (0.1/0.01) for both discontinuous and smooth models.}
        \label{fig:vk-9}
\end{figure}

We can observe that when the two-layer viscosity model is added, there is only one apparent peak for each case. The first peaks of the 0.01 viscosity contrast tests appear earlier and are larger in magnitude than those of 0.1 viscosity contrast tests.  There are no secondary plumes and the whole system tends to reach stability after around 500 time steps.

\subsubsection{The SolCx Stokes benchmark}
\label{sec:benchmark-solcx}

The SolCx benchmark is intended to test the accuracy of the solution to a
problem that has a large jump in the viscosity along a line through the
domain. Such situations are common in geophysics: for example, the viscosity
in a cold, subducting slab is much larger than in the surrounding, relatively
hot mantle material.

The SolCx benchmark computes the Stokes flow field of a fluid driven by
spatial density variations, subject to a spatially variable
viscosity. Specifically, the domain is $\Omega=[0,1]^2$, gravity is $\mathbf
g=(0,-1)^T$ and the density is given
by $\rho(\mathbf x)=\sin(\pi x_1)\cos(\pi x_2)$; this can be considered a
density perturbation to a constant background density. The viscosity is
\begin{align*}
  \eta(\mathbf x) = \left\{
    \begin{matrix}
      1 & \text{for}\ x_1 \le 0.5, \\
      10^6 & \text{for}\ x_1  > 0.5.
    \end{matrix}
  \right.
\end{align*}
This strongly discontinuous viscosity field yields an almost stagnant flow in
the right half of the domain and consequently a singularity in the pressure
along the interface.
Boundary conditions are free slip on all of $\partial\Omega$. The temperature
plays no role in this benchmark. The prescribed density field and the
resulting velocity field are shown in Fig.~\ref{fig:solcx}.

The SolCx benchmark was previously used in \cite[Section 4.1.1]{DMGT11}
(references to earlier uses of the benchmark are available there) and its analytic
solution is given in \cite{Zho96}. \aspect{} contains an implementation of
this analytic solution taken from the Underworld package (see \cite{MQLMAM07}
and \url{http://www.underworldproject.org/}, and correcting for the mismatch
in sign between the implementation and the description in \cite{DMGT11}).

\begin{figure}
  \begin{center}
    \includegraphics[width=0.45\textwidth]{cookbooks/benchmarks/solcx-solution}
    \hfill
    \includegraphics[width=0.45\textwidth]{cookbooks/benchmarks/solcx-solution-pressure}
    \caption{\it SolCx Stokes benchmark. Left: The density perturbation field
    and overlaid to it some velocity vectors. The viscosity is very large in the
      right hand, leading to a stagnant flow in this region. Right: The
      pressure on a relatively coarse mesh, showing the internal layer along
      the line where the viscosity jumps.}
    \label{fig:solcx}
  \end{center}
\end{figure}

To run this benchmark, the following input file will do (see the files in \url{benchmark/solcx/} to rerun the benchmark):
\lstinputlisting[language=prmfile]{cookbooks/benchmarks/solcx.prm.out}

Since this is the first cookbook in the benchmarking section, let us go
through the different parts of this file in more detail:
\begin{itemize}
\item The material model and the postprocessor 
\item The first part consists of parameter setting for overall
  parameters. Specifically, we set the dimension in which this benchmark runs
  to two and choose an output directory. Since we are not interested in a time
  dependent solution, we set the end time equal to the start time, which
  results in only a single time step being computed.

  The last parameter of this section, \texttt{Pressure normalization},
\index[prmindex]{Pressure normalization}
\index[prmindexfull]{Pressure normalization}
  is set in such a way that the pressure is chosen so that its \textit{domain}
  average is zero, rather than the pressure along the surface, see
  Section~\ref{sec:pressure}.

\item The next part of the input file describes the setup of the
  benchmark. Specifically, we have to say how the geometry should look like (a
  box of size $1\times 1$) and what the velocity boundary conditions shall be
  (tangential flow all around -- the box geometry defines four boundary
\index[prmindex]{Model name}
\index[prmindexfull]{Geometry model!Model name}
  indicators for the left, right, bottom and top boundaries, see also
  Section~\ref{parameters:Geometry_20model}). This is followed by subsections
  choosing the material model (where we choose a particular model implemented
  in \aspect{} that describes the spatially variable density and viscosity
  fields, along with the size of the viscosity jump) and finally the chosen
  gravity model (a gravity field that is the constant vector $(0,-1)^T$, see
\index[prmindex]{Model name}
\index[prmindexfull]{Gravity model!Model name}
  Section~\ref{parameters:Gravity_20model}).

\item The part that follows this describes the boundary and initial values for
  the temperature. While we are not interested in the evolution of the
  temperature field in this benchmark, we nevertheless need to set
  something. The values given here are the minimal set of inputs.

\item The second-to-last part sets discretization parameters. Specifically, it
  determines what kind of Stokes element to choose (see
\index[prmindex]{Stokes velocity polynomial degree}
\index[prmindexfull]{Discretization!Stokes velocity polynomial degree}
  Section~\ref{parameters:Discretization} and the extensive discussion in
  \cite{KHB12}). We do not adaptively refine the mesh but only do four global
  refinement steps at the very beginning. This is obviously a parameter worth
\index[prmindex]{Initial global refinement}
\index[prmindexfull]{Mesh refinement!Initial global refinement}
  playing with.

\item The final section on postprocessors determines what to do with the
  solution once computed. Here, we do two things: we ask \aspect{} to compute
  the error in the solution using the setup described in the Duretz et
  al.~paper \cite{DMGT11}, and we request that output files for later
  visualization are generated and placed in the output directory. The
  functions that compute the error automatically query which kind of material
  model had been chosen, i.e., they can know whether we are solving the SolCx
  benchmark or one of the other benchmarks discussed in the following
  subsections.
\end{itemize}

Upon running \aspect{} with this input file, you will get output of the
following kind (obviously with different timings, and details of the output
may also change as development of the code continues):
\begin{lstlisting}[frame=single,language=ksh]
aspect/cookbooks> ../aspect solcx.prm
Number of active cells: 256 (on 5 levels)
Number of degrees of freedom: 3,556 (2,178+289+1,089)

*** Timestep 0:  t=0 years
   Solving temperature system... 0 iterations.
   Rebuilding Stokes preconditioner...
   Solving Stokes system... 30+3 iterations.

   Postprocessing:
     Errors u_L1, p_L1, u_L2, p_L2: 1.125997e-06, 2.994143e-03, 1.670009e-06, 9.778441e-03
     Writing graphical output:      output/solution-00000



+---------------------------------------------+------------+------------+
| Total wallclock time elapsed since start    |      1.51s |            |
|                                             |            |            |
| Section                         | no. calls |  wall time | % of total |
+---------------------------------+-----------+------------+------------+
| Assemble Stokes system          |         1 |     0.114s |       7.6% |
| Assemble temperature system     |         1 |     0.284s |        19% |
| Build Stokes preconditioner     |         1 |    0.0935s |       6.2% |
| Build temperature preconditioner|         1 |    0.0043s |      0.29% |
| Solve Stokes system             |         1 |    0.0717s |       4.8% |
| Solve temperature system        |         1 |  0.000753s |      0.05% |
| Postprocessing                  |         1 |     0.627s |        42% |
| Setup dof systems               |         1 |      0.19s |        13% |
+---------------------------------+-----------+------------+------------+
\end{lstlisting}

One can then visualize the solution in a number of different ways (see
Section~\ref{sec:viz}), yielding pictures like those shown in
Fig.~\ref{fig:solcx}. One can also analyze the error as shown in various
different ways, for example as a function of the mesh refinement level, the
element chosen, etc.; we have done so extensively in \cite{KHB12}.


\subsubsection{The SolKz Stokes benchmark}
\label{sec:benchmark-solkz}

The SolKz benchmark is another variation on the same theme as the SolCx
benchmark above: it solves a Stokes problem with a spatially variable
viscosity but this time the viscosity is not a discontinuous function but
grows exponentially with the vertical coordinate so that it's overall
variation is again $10^6$. The forcing is again chosen by imposing a spatially
variable density variation. For details, refer again to \cite{DMGT11}.

The following input file, only a small variation of the one in the previous
section, solves the benchmark (see \url{benchmark/solkz/}):

\lstinputlisting[language=prmfile]{cookbooks/benchmarks/solkz.prm.out}

The output when running \aspect{} on this parameter file looks similar to the
one shown for the SolCx case. The solution when computed with one more level
of global refinement is visualized in Fig.~\ref{fig:solkz}.

\begin{figure}
  \begin{center}
    \includegraphics[width=0.45\textwidth]{cookbooks/benchmarks/solkz-solution}
    \hfill
    \includegraphics[width=0.45\textwidth]{cookbooks/benchmarks/solkz-solution-pressure}
    \caption{\it SolKz Stokes benchmark. Left: The density perturbation field
    and overlaid to it some velocity vectors. The viscosity grows exponentially
      in the vertical direction, leading to small velocities at the top
      despite the large density variations. Right: The
      pressure.}
    \label{fig:solkz}
  \end{center}
\end{figure}


\subsubsection{The ``inclusion'' Stokes benchmark}
\label{sec:benchmark-inclusion}

The ``inclusion'' benchmark again solves a problem with a discontinuous
viscosity, but this time the viscosity is chosen in such a way that the
discontinuity is along a circle. This ensures that, unlike in the SolCx
benchmark discussed above, the discontinuity in the viscosity never aligns to
cell boundaries, leading to much larger difficulties in obtaining an accurate
representation of the pressure. Specifically, the almost discontinuous
pressure along this interface leads to oscillations in the numerical
solution. This can be seen in the visualizations shown in
Fig.~\ref{fig:inclusion}. As before, for details we refer to
\cite{DMGT11}. The analytic solution against which we compare is given in
\cite{SP03}. An extensive discussion of convergence properties is given in
\cite{KHB12}.

\begin{figure}
  \begin{center}
    \includegraphics[width=0.45\textwidth]{cookbooks/benchmarks/inclusion-solution}
    \hfill
    \includegraphics[width=0.45\textwidth]{cookbooks/benchmarks/inclusion-solution-pressure}
    \caption{\it Inclusion Stokes benchmark. Left: The viscosity field
      when interpolated onto the mesh (internally, the ``exact'' viscosity
      field -- large inside a circle, small outside -- is used),
      and overlaid to it some velocity vectors. Right: The
      pressure with its oscillations along the interface. The oscillations
      become more localized as the mesh is refined.}
    \label{fig:inclusion}
  \end{center}
\end{figure}

The benchmark can be run using the parameter files in \url{benchmark/inclusion/}. The material model, boundary condition, and postprocessor are defined in \url{benchmark/inclusion/inclusion.cc}. Consequently, this code needs to be compiled into a shared lib before you can run the tests.

\marginpar{Link to a general section on how you can compile libs for the benchmarks.}

\marginpar{Revisit this once we have the machinery in place to choose nonzero
  boundary conditions in a more elegant way.}

\marginpar{The following prm file isn't annotated yet. How to annotate if we have a .lib?}

\lstinputlisting[language=prmfile]{cookbooks/benchmarks/inclusion.prm.out}


\subsubsection{The Burstedde variable viscosity benchmark}
\label{sec:benchmark-burstedde}

\textit{This section was contributed by Iris van Zelst.}

This benchmark is intended to test solvers for variable viscosity Stokes
problems. It begins with postulating a smooth exact polynomial solution to the Stokes equation for a unit cube, first proposed by \cite{dobo04} and also described by \cite{busa13}:
\begin{align}
  {\mathbf u} &= \left( \begin{array}{c}
      x+x^2+xy+x^3y \\
      y + xy + y^2 + x^2 y^2\\
      -2z - 3xz - 3yz - 5x^2 yz
    \end{array}
  \right)
  \label{eq:burstedde-velocity}
  \\
  p &= xyz + x^3 y^3z - \frac{5}{32}.
  \label{eq:burstedde-pressure}
\end{align}

It is then trivial to verify that the velocity field is divergence-free. The
constant $-\frac{5}{32}$ has been added to the expression of $p$ to ensure
that the volume pressure normalization of \aspect{} can be used in this
benchmark (in other words, to ensure that the exact pressure has mean value
zero and, consequently, can easily be compared with the numerically computed
pressure). Following \cite{busa13}, the viscosity $\mu$ is given by the smoothly varying function 
\begin{equation}
  \mu = \exp\left\{1 - \beta\left[x (1-x) + y(1-y) + z(1-z)\right]\right\}.
  \label{eq:burstedde-mu}
\end{equation}
The maximum of this function is $\mu = e$, for example at $(x,y,z)=(0,0,0)$, and the minimum of this function is $\mu = \exp \Big( 1-\frac{3\beta}{4}\Big)$ at $(x,y,z) = (0.5,0.5,0.5)$. The viscosity ratio $\mu^*$ is then given by 
\begin{equation}
  \mu^* = \frac{\exp\Big(1-\frac{3\beta}{4}\Big)}{\exp(1)} = \exp\Big(\frac{-3\beta}{4}\Big).
\end{equation}
Hence, by varying $\beta$ between 1 and 20, a difference of up to 7 orders of
magnitude viscosity is obtained. $\beta$ will be one of the parameters that
can be selected in the input file that accompanies this benchmark.

The corresponding body force of the Stokes equation can then be computed by inserting this solution into the momentum equation,
\begin{equation}
  {\nabla} p - \nabla \cdot (2  \mu {\epsilon(\mathbf u)}) = \rho \mathbf g.
  \label{eq:burstedde-momentum}
\end{equation}
Using equations \eqref{eq:burstedde-velocity}, \eqref{eq:burstedde-pressure}
and \eqref{eq:burstedde-mu} in the
momentum equation \eqref{eq:burstedde-momentum}, the following expression for the body force
$\rho\mathbf g$ can be found:
\begin{multline}
  {\rho\mathbf g} 
  =
  \left(
    \begin{array}{c}
      yz+3x^2y^3z\\
      xz +3x^3y^2z \\
      xy+x^3y^3
    \end{array}
  \right)
  -\mu
  \left(
    \begin{array}{c}
      2+6xy  \\
      2 + 2x^2 +  2y^2 \\
      -10yz 
    \end{array}
  \right) \\
  +
  (1-2x)\beta \mu 
  \left(
    \begin{array}{c}
      2+4x+2y+6x^2y \\
      x+y+2xy^2+x^3 \\
      -3z -10xyz 
    \end{array}
  \right)
  +(1-2y)\beta \mu 
  \left(
    \begin{array}{c}
      x+y+2xy^2+x^3 \\
      2+2x+4y+4x^2y \\
      -3z-5x^2z \\
    \end{array}
  \right)
  \\
  +(1-2z)\beta \mu
  \left(
    \begin{array}{c}
      -3z -10xyz \\
      -3z-5x^2z \\
      -4-6x-6y-10x^2y
    \end{array}
  \right)
\end{multline}
Assuming $\rho = 1$, the above expression translates into an expression for the
gravity vector $\mathbf g$. This expression for the gravity (even though it is
completely unphysical), has consequently been incorporated into the
\texttt{BursteddeGravity} gravity model that is described in the
\texttt{benchmarks/burstedde/burstedde.cc} file that accompanies this benchmark.

We will use the input file \texttt{benchmark/burstedde/burstedde.prm} as
input, which is very similar to the input file
\texttt{benchmark/inclusion/adaptive.prm} discussed above in
Section~\ref{sec:benchmark-inclusion}. The major changes for the 3D polynomial
Stokes benchmark are listed below: 

\lstinputlisting[language=prmfile]{cookbooks/benchmarks/burstedde/burstedde.prm.out}

The boundary conditions that are used are simply the velocities from equation
\eqref{eq:burstedde-velocity} prescribed on each boundary. The viscosity parameter in the input
file is $\beta$. Furthermore, in order to compute the velocity and pressure
$L_1$ and $L_2$ norm, the postprocessor \texttt{BursteddePostprocessor} is
used. Please note that the linear solver tolerance is set to a very small
value (deviating from the default value), in order to ensure that the solver
can solve the system accurately enough to make sure that the iteration
error is smaller than the discretization error.

Expected analytical solutions at two locations are summarised in Table~\ref{tab:burstedde-table} and can be deduced from equations \eqref{eq:burstedde-velocity} and
\eqref{eq:burstedde-pressure}.
Figure~\ref{fig:burstedde-benchmark} shows that the analytical solution is indeed retrieved by the model.

\begin{table}[h!]
\caption{Analytical solutions \label{tab:burstedde-table}}
\centering
\begin{tabular}{l|c|c}
Quantity & $\mathbf{r} = (0,0,0)$ & $\mathbf{r} = (1,1,1)$ \\ \hline
$p$ & $-0.15625$ & $1.84375$ \\
$\mathbf{u}$ & $(0,0,0)$  & $(4,4,-13)$ \\ 
$|\mathbf{u}|$ & $0$ &  $14.177$ \\
\end{tabular}
\end{table}

\begin{figure}[t!]
  \centering
  \subfigure[]{
    \includegraphics[width=0.48\textwidth]{cookbooks/benchmarks/burstedde/viscosity.png}}%
  ~ 
  \subfigure[]{
    \includegraphics[width=0.48\textwidth]{cookbooks/benchmarks/burstedde/pressure.png}}%
  \\
  \subfigure[]{
    \includegraphics[width=0.48\textwidth]{cookbooks/benchmarks/burstedde/velocity_x.png}}
  ~
  \subfigure[]{
    \includegraphics[width=0.48\textwidth]{cookbooks/benchmarks/burstedde/velocity_z.png}}
  \caption{Burstedde benchmark: Results for the 3D polynomial Stokes benchmark, obtained with a resolution of $16\times 16$ elements, with $\beta = 10$.}\label{fig:burstedde-benchmark}
\end{figure}

The convergence of the numerical error of this benchmark has been analysed by
playing with the mesh refinement level in the input file, and
results can be found in Figure~\ref{errors}. The velocity shows cubic error
convergence, while the pressure shows quadratic convergence in the $L_1$ and
$L_2$ norms, as one would hope for using $Q_2$ elements for the velocity and
$Q_1$ elements for the pressure.

\begin{figure}[tbp]
  \centering
  \includegraphics[width=\textwidth]{cookbooks/benchmarks/burstedde/errors.pdf}
  \caption{Burstedde benchmark: Error convergence for the 3D polynomial Stokes
    benchmark.
    \label{errors}}
\end{figure}



\subsubsection{The ``Stokes' law'' benchmark}
\label{sec:benchmark-stokes_law}

\textit{This section was contributed by Juliane Dannberg.}

Stokes' law was derived by George Gabriel Stokes in 1851 and describes the frictional force
a sphere with a density different than the surrounding fluid experiences in a
laminar flowing viscous medium.
A setup for testing this law is a sphere with the radius $r$ falling in a highly
viscous fluid with lower density. Due to its higher density the sphere is
accelerated by the gravitational force. While
the frictional force increases with the velocity of the falling particle,
the buoyancy force remains constant. Thus, after some time the forces will
be balanced and the settling velocity of the sphere $v_s$ will remain constant:

\begin{align}
  \label{eq:stokes-law}
  \underbrace{6 \pi \, \eta \, r \, v_s}_{\text{frictional force}} =
  \underbrace{4/3 \pi \, r^3 \, \Delta\rho \, g,}_{\text{buoyancy force}}
\end{align}
where $\eta$ is the dynamic viscosity of the fluid, $\Delta\rho$ is the
density difference between sphere and fluid and $g$ the gravitational
acceleration. The resulting settling velocity is then given by
\begin{align}
  \label{eq:stokes-velo}
  v_s = \frac{2}{9} \frac{\Delta\rho \, r^2 \, g}{\eta}.
\end{align}
Because we do not take into account inertia in our numerical computation,
the falling particle will reach the constant settling velocity right after
the first timestep.

For the setup of this benchmark, we chose the following parameters:
\begin{align*}
  \label{eq:stokes-parameters}
  r &= 200 \, \text{km}\\
  \Delta\rho &= 100 \, \text{kg}/\text{m}^3\\
  \eta &= 10^{22} \, \text{Pa s}\\
  g &= 9.81 \, \text{m}/\text{s}^2.
\end{align*}
With these values, the exact value of sinking velocity is $v_s =
8.72 \cdot 10^{-10} \, \text{m}/\text{s}$.

To run this benchmark, we need to set up an input file that describes the
situation. In principle, what we need to do is to describe a spherical object
with a density that is larger than the surrounding material. There are multiple
ways of doing this. For example, we could simply set the initial temperature of
the material in the sphere to a lower value, yielding a higher density with any
of the common material models. Or, we could use \aspect{}'s facilities to advect
along what are called ``compositional fields'' and make the density dependent on
these fields.

We will go with the second approach and tell \aspect{} to advect a single
compositional field. The initial conditions for this field will be zero outside
the sphere and one inside. We then need to also tell the material model to
increase the density by $\Delta\rho=100 kg\, m^{-3}$ times the concentration of
the compositional field. This can be done, like everything else, from the input
file.

All of this setup is then described by the following input file.
(You can find the input file to run this cookbook example in
\url{cookbooks/stokes.prm}. For your first runs you will probably want to
reduce the number of mesh refinement steps to make things run more quickly.)

\lstinputlisting[language=prmfile]{cookbooks/benchmarks/stokes/stokeslaw.prm.out}

Using this input file, let us try to evaluate the results of the current
computations for the settling velocity of the sphere. You can visualize the output in different
ways, one of it being ParaView and shown in
Fig.~\ref{fig:stokes-falling-sphere-2d} (an alternative is to use Visit as
described in Section~\ref{sec:viz}; 3d images of this simulation using Visit
are shown in Fig.~\ref{fig:stokes-falling-sphere-3d}).
Here, Paraview has the advantage that you can calculate the average velocity
of the sphere using the following filters:
\begin{enumerate}
 \item Threshold (Scalars: C\_1, Lower Threshold 0.5, Upper Threshold 1),
 \item Integrate Variables,
 \item Cell Data to Point Data,
 \item Calculator (use the formula sqrt(velocity\_x\textasciicircum2+
       velocity\_y\textasciicircum2+velocity\_z\textasciicircum2)/Volume).
\end{enumerate}
If you then look at
the Calculator object in the Spreadsheet View, you can see the average sinking
velocity of the sphere in the column ``Result'' and compare it to the theoretical
value $v_s = 8.72 \cdot 10^{-10} \, \text{m}/\text{s}$.
In this case, the numerical result is 8.865 $\cdot 10^{-10} \,
\text{m}/\text{s}$ when you add a few more refinement steps to actually resolve
the 3d flow field adequately. The ``velocity statistics'' postprocessor we have
selected above also provides us with a maximal velocity that is on the same
order of magnitude. The difference between the analytical and the numerical
values can be explained by different at least the following three points:
(i) In our case the sphere is viscous and not rigid as assumed in Stokes' initial model, leading to
a velocity field that varies inside the sphere rather than being constant.
(ii) Stokes' law is derived using an infinite domain but we have a finite box
instead. (iii) The mesh may not yet fine enough to provide a fully converges
solution. Nevertheless, the fact that we get a result that is accurate to less
than 2\% is a good indication that \aspect{} implements the equations correctly.

\begin{figure}
  \begin{center}
    \includegraphics[width=0.55\textwidth]{cookbooks/benchmarks/stokes/stokes-velocity}
    \hfill
    \includegraphics[width=0.44\textwidth]{cookbooks/benchmarks/stokes/stokes-density}
  \end{center}
  \caption{\it Stokes benchmark. Both figures show only a 2D slice of the
      three-dimensional model.
      Left: The compositional field and overlaid to it some velocity vectors.
      The composition is 1 inside a sphere with the radius of 200 km and 0
      outside of this sphere. As the velocity vectors show, the sphere sinks
      in the viscous medium.
      Right: The density distribution of the model. The compositional density
      contrast of 100 kg$/\text{m}^3$ leads to a higher density inside of the
      sphere.}
  \label{fig:stokes-falling-sphere-2d}
\end{figure}

\begin{figure}
  \begin{center}
    \includegraphics[width=0.3\textwidth]{cookbooks/benchmarks/stokes/composition}
    \hfill
    \includegraphics[width=0.3\textwidth]{cookbooks/benchmarks/stokes/mesh}
    \hfill
    \includegraphics[width=0.3\textwidth]{cookbooks/benchmarks/stokes/velocity}
  \end{center}
  \caption{\it Stokes benchmark. Three-dimensional views of the compositional
  field (left), the adaptively refined mesh (center) and the resulting velocity field
  (right).}
  \label{fig:stokes-falling-sphere-3d}
\end{figure}


\subsubsection{Latent heat benchmark}
\label{sec:benchmark-latent_heat}

\textit{This section was contributed by Juliane Dannberg.}

The setup of this benchmark is taken from Schubert, Turcotte and Olson \cite{STO01} (part 1, p. 194) and is illustrated in Fig.~\ref{fig:latent-heat-benchmark}.
\begin{figure}
  \begin{center}
    \includegraphics[width=0.52\textwidth]{cookbooks/benchmarks/latent-heat/latent-heat-setup}
    \hfill
    \includegraphics[width=0.47\textwidth]{cookbooks/benchmarks/latent-heat/latent-heat-temperature}
  \end{center}
  \caption{\it Latent heat benchmark. Both figures show the 2D box model domain.
      Left: Setup of the benchmark together with a sketch of the expected
      temperature profile across the phase transition. The dashed line marks
      the phase transition. Material flows in with a prescribed temperature and
      velocity at the top, crosses the phase transition in the center and flows
      out at the bottom. The predicted bottom temperature is $T_2 = 1109.08 \, \text{K}$.
      Right: Temperature distribution of the model together with the associated
      temperature profile across the phase transition. The modelled bottom
      temperature is $T_2 = 1107.39 \, \text{K}$.}
  \label{fig:latent-heat-benchmark}
\end{figure}
It tests whether the latent heat production when material crosses a phase
transition is calculated correctly according to the laws of thermodynamics. The material
model defines two phases in the model domain with the phase transition
approximately in the center. The material flows in from the top due to a
prescribed downward velocity, and crosses the phase transition before it leaves
the model domain at the bottom. As initial condition, the model uses a uniform
temperature field, however, upon the phase change, latent heat is released. This
leads to a characteristic temperature profile across the phase transition with a
higher temperature in the bottom half of the domain. To compute it, we have to solve 
equation \eqref{eq:temperature} or its reformulation
\eqref{eq:temperature-reformulated}. For
steady-state one-dimensional downward flow with vertical velocity $v_y$, it
simplifies to the following:
\begin{gather*}
\rho C_p
v_y
\frac{\partial T}{\partial y} = 
\rho T \Delta S v_y \frac{\partial X}{\partial y} 
+ \rho C_p \kappa
\frac{\partial^2 T}{\partial y^2}.
\end{gather*}
Here, $\rho C_p \kappa = k$ with $k$ the thermal conductivity and $\kappa$ the
thermal diffusivity.
The first term on the right-hand side of the equation describes the latent heat
produced at the phase transition: It is proportional to the temperature T, the
entropy change $\Delta S$ across the phase transition divided by the specific
heat capacity and the derivative of the phase function X. If the velocity is
smaller than a critical value, and under the assumption of a discontinuous phase
transition (i.e. with a step function as phase function), this latent heating
term will be zero everywhere except for the one point $y_{tr}$ where the phase
transition takes place. This means, we have a region above the phase transition
with only phase 1, and below a certain depth a jump to a region with only phase
2. Inside of these one-phase regions, we can solve the equation above (using the
boundary conditions $T=T_1$ for $y \rightarrow \infty $ and $T=T_2$ for $y
\rightarrow -\infty $) and get
\begin{align*}
T(y) =\begin{cases}
T_1 + (T_2-T_1) e^\frac{v_y (y-y_{tr})}{\kappa}, & y>y_{tr}\\
T_2, & y<y_{tr}
\end{cases}
\end{align*}
While it is not entirely obvious while this equation for $T(y)$ should be
correct (in particular why it should be asymmetric), it is not difficult to
verify that it indeed satisfies the equation stated above for both $y<y_{tr}$
and $y>y_{tr}$. Furthermore, it indeed satisfies the jump condition we get by
evaluating the equation at $y=y_{tr}$.
Indeed, the jump condition can be reinterpreted as a balance of heat conduction:
We know the amount of heat that is produced at the phase boundary, and as
we consider only steady-state, the same amount of heat is conducted upwards from
the transition:

\begin{gather*}
\underbrace{\rho v_y T \Delta S}_{\text{latent heat release}} = \underbrace{\frac{\kappa}{\rho_0 C_p} \frac{\partial T}{\partial y} \vert_{y=y_{tr^-}} = \frac{v_y}{\rho_0 C_p} (T_2-T_1)}_{\text{heat conduction}}
\end{gather*}

In contrast to \cite{STO01}, we also consider the density change $\Delta\rho$ across the phase transition: While the heat conduction takes place above the transition and the density can be assumed as $\rho=\rho_0=$ const., the latent heat is released directly at the phase transition. Thus, we assume an average density $\rho=\rho_0 + 0.5\Delta\rho$ for the left side of the equation. Rearranging this equation gives

\begin{gather*}
T_2 = \frac{T_1}{1 - (1+\frac{\Delta \rho}{2 \rho_0}) \frac{\Delta S}{C_p}}
\end{gather*}

In addition, we have tested the approach exactly as it is described in \cite{STO01} by setting the entropy change to a specific value and in spite of that using a constant density. However, this is physically inconsistent, as the entropy change is proportional to the density change across the phase transition. With this method, we could reproduce the analytic results from \cite{STO01}.

The exact values of the parameters used for this benchmark can be found in
Fig.~\ref{fig:latent-heat-benchmark}. They result in a predicted value of $T_2 =
1109.08 \, \text{K}$ for the temperature in the bottom half of the model, and
we will demonstrate below that we can match this value in our numerical
computations. However, it is not as simple as suggested above. In actual
numerical computations, we can not exactly reproduce the behavior of Dirac delta
functions as would result from taking the derivative $\frac{\partial
X}{\partial y}$ of a discontinuous function $X(y)$. Rather, we have to model
$X(y)$ as a function that has a smooth transition from one value to another,
over a depth region of a certain width. In the material model plugin we will use
below, this depth is an input parameter and we will play with it in the
numerical results shown after the input file.

To run this benchmark, we need to set up an input file that describes the
situation. In principle, what we need to do is to describe the position and
entropy change of the phase transition in addition to the previously outlined
boundary and initial conditions. For this purpose, we use the ``latent heat''
material model that allows us to set the density change $\Delta\rho$ and
Clapeyron slope $\gamma$ (which together determine the entropy change via
$\Delta S = \gamma \frac{\Delta\rho}{\rho^2}$) as well as the depth of the phase
transition as input parameters.

All of this setup is then described by the input file
\url{cookbooks/latent-heat.prm} that models flow in a box of $10^6$ meters of
height and width, and a fixed downward velocity. The following section shows the
central part of this file:

\lstinputlisting[language=prmfile]{cookbooks/benchmarks/latent-heat/material.part.prm.out}

The complete input file referenced above also sets the number of mesh refinement
steps. For your first runs you will probably want to reduce the number of mesh
refinement steps to make things run more quickly. Later on, you might also want
to change the phase transition width to look how this influences the result.

\begin{figure}
  \begin{center}
    \includegraphics[width=0.49\textwidth]{cookbooks/benchmarks/latent-heat/latent-heat-results-1}
    \hfill
    \includegraphics[width=0.49\textwidth]{cookbooks/benchmarks/latent-heat/latent-heat-results-2}
  \end{center}
  \caption{\it Results of the latent heat benchmark. Both figures show the modelled temperature $T_2$ at the bottom of the model domain.
      Left: $T_2$ in dependence of resolution using a constant phase transition width of 20\,km. With an increasing number of global refinements of the mesh, the bottom temperature converges against a value of $T_2 = 1105.27 \, \text{K}$.
      Right: $T_2$ in dependence of phase transition width. The model resolution is chosen proportional to the phase transition width, starting with 5 global refinements for a width of 20\,km. With decreasing phase transition width, $T_2$ approaches the theoretical value of $1109.08 \, \text{K}$}
  \label{fig:latent-heat-benchmark-results}
\end{figure}

Using this input file, let us try to evaluate the results of the current
computations. We note that it takes some time for the model to reach a steady
state and only then does the bottom temperature reach the theoretical value.
Therefore, we use the last output step to compare predicted and computed values.
You can visualize the output in different ways, one of it being ParaView and shown in
Fig.~\ref{fig:latent-heat-benchmark} on the right side (an alternative is to use Visit as
described in Section~\ref{sec:viz}). In ParaView, you can plot the temperature profile
using the filter ``Plot Over Line'' (Point1: 500000,0,0; Point2:
500000,1000000,0, then go to the ``Display'' tab and select ``T'' as only
variable in the ``Line series'' section) or ``Calculator'' (as seen in
Fig.~\ref{fig:latent-heat-benchmark}). In
Fig.~\ref{fig:latent-heat-benchmark-results} (left) we can see that with
increasing resolution, the value for the bottom temperature converges to a value
of $T_2 = 1105.27 \, \text{K}$. 

However, this is not what the analytic solution
predicted. The reason for this difference is the width of the phase transition
with which we smooth out the Dirac delta function that results from
differentiating the $X(y)$ we would have liked to use in an ideal world.
(In reality, however, for the Earth's mantle we also expect phase transitions
that are distributed over a certain depth range and so the smoothed out
approach may not be a bad approximation.)
Of course, the results shown above result from an the analytical approach that
is only correct if the phase transition is discontinuous and constrained to one
specific depth $y=y_{tr}$. Instead, we chose a hyperbolic
tangent as our phase function. Moreover,
Fig.~\ref{fig:latent-heat-benchmark-results} (right) illustrates what happens to
the temperature at the bottom when we vary the width of the phase transition:
The smaller the width, the closer the temperature gets to the predicted value of
$T_2 = 1109.08 \, \text{K}$, demonstrating that we converge to the correct
solution.


\subsubsection{The 2D cylindrical shell benchmarks by Davies et al.}
\label{sec:benchmark-2D_cylindrical_shell}

\textit{This section was contributed by William Durkin and Wolfgang Bangerth.}

All of the benchmarks presented so far take place in a Cartesian domain. 
Davies et al.~describe a benchmark (in a paper that is currently still being
written) for a 2D spherical Earth that is  
nondimensionalized such that 
\begin{table*}[h]
 \centering
 \begin{tabular}{ l l }
    $r_{\min}$ = 1.22 &  $\left. T \right|_{r_{min}}$ = 1 \\
    $r_{\max}$ = 2.22 &  $\left. T \right|_{r_{max}}$ = 0
 \end{tabular}
\end{table*}

The benchmark is run for a series of approximations (Boussinesq, Extended Boussinesq,
Truncated Anelastic Liquid, and Anelastic Liquid), and temperature, velocity, and heat flux 
calculations are compared with the results of other mantle modeling programs. \aspect{}
will output all of these values directly except for the Nusselt number, which
we must calculate ourselves from the heat fluxes that \aspect{} can compute.
The Nusselt number of the top and bottom surfaces, ${Nu}_T$ and ${Nu}_B$,
respectively, are defined by the authors of the benchmarks as
\begin{equation}
\label{eq:davies-NuTop}
{Nu}_{T} = \frac{\ln(f)}{2{\pi}r_{\max}(1-f)}\int \limits_{0}^{2\pi} \frac{\partial T}{\partial r}\, d\theta  \\
\end{equation}
and
\begin{equation*}
\label{eq:davies-NuBottom}
{Nu}_{B} = \frac{f \ln(f)}{2{\pi}r_{\min}(1-f)}\int \limits_{0}^{2\pi} \frac{\partial T}{\partial r}\, d\theta \\
\end{equation*}
where $f$ is the ratio $\frac{r_{\min}}{r_{\max}}$.

We can put this in terms of heat flux
\begin{equation*}
  q_r = -k\frac{\partial T}{\partial r}
\end{equation*}
through the inner and outer surfaces,
where $q_r$ is heat flux in the radial direction. Let $Q$ be the total heat that flows throug a surface,
\begin{equation*}
  Q = \int \limits_{0}^{2\pi} q_r\, d\theta,
\end{equation*}
then \eqref{eq:davies-NuTop} becomes
\begin{equation*}
  {Nu}_{T} = \frac{-Q_{T}\ln(f)}{2\pi{r_{\max}}(1-f)k}
\end{equation*}
and similarly
\begin{equation*}
  {Nu}_{B} = \frac{-Q_{B}f\ln(f)}{2\pi{r_{\min}}(1-f)k}.
\end{equation*} 
$Q_T$ and $Q_B$ are heat fluxes that \aspect{} can readily compute through the
\texttt{heat flux statistics} postprocessor (see
Section~\ref{parameters:Postprocess/List of postprocessors}).
For further details on the nondimensionalization and equations used for each
approximation, refer to Davies et al.

The series of benchmarks is then defined by a number of cases relating to the
exact equations chosen to model the fluid. We will discuss these in the
following.


\paragraph{Case 1.1: BA\_Ra104\_Iso\_ZS.}
\label{sec:davies-case11_BA}

This case is run with the following settings:
\begin{itemize}
\item Boussinesq Approximation
\item Boundary Condition: Zero-Slip
\item Rayleigh Number = $10^4$ 
\item Initial Conditions: $D = 0, O = 4$
\item $\eta(T) = 1$
\end{itemize}
where $D$ and $O$ refer to the degree and order of a spherical harmonic that describes the 
initial temperature. While the initial conditions matter, what is important
here though is that the system evolve to four convective cells since we are
only interested in the long term, steady state behavior.

The model is relatively straightforward to set up, basing the input file on
that discussed in Section~\ref{sec:shell-simple-2d}. The full input file can
be found at \url{benchmark/davies_et_al/case-1.1.prm}, with the interesting
parts excerpted as follows:

\lstinputlisting[language=prmfile]{cookbooks/benchmarks/davies_et_al/case-1.1.prm.out}

We use the same trick here as in Section~\ref{sec:cookbooks-simple-box} to
produce a model in which the density $\rho(T)$ in the temperature equation
\eqref{eq:temperature} is almost constant (namely, by choosing a very small
thermal expansion coefficient) as required by the benchmark, and instead
prescribe the desired Rayleigh number by choosing a correspondingly large
gravity.

Results for this and the other cases are shown below.


\paragraph{Case 2.1: BA\_Ra104\_Iso\_FS.}
\label{sec:davies-case21_BA}

Case 2.1 uses the following setup, differing only in the boundary conditions:
\begin{itemize}
\item Boussinesq Approximation
\item Boundary Condition: Free-Slip
\item Rayleigh Number = $10^4$ 
\item Initial Conditions: $D = 0, O = 4$
\item $\eta(T) = 1$
\end{itemize}

As a consequence of the free slip boundary conditions, any solid body rotation
of the entire system satisfies the Stokes equations with their boundary
conditions. In other words, the solution of the problem is not unique: given a
solution, adding a solid body rotation yields another solution. We select
arbitrarily the one that has no net rotation (see
Section~\ref{parameters:Model_20settings}). The section in the input file 
that is relevant is then as follows (the full input file resides at
\url{benchmark/davies_et_al/case-2.1.prm}):
\index[prmindex]{Remove nullspace}
\index[prmindexfull]{Model settings!Remove nullspace}

\lstinputlisting[language=prmfile]{cookbooks/benchmarks/davies_et_al/case-2.1.prm.out}

Again, results are shown below.


\paragraph{Case 2.2: BA\_Ra105\_Iso\_FS.}
\label{sec:davies-case22_BA}

Case 2.2 is described as follows:
\begin{itemize}
\item Boussinesq Approximation
\item Boundary Condition: Free-Slip
\item Rayleigh Number = $10^5$ 
\item Initial Conditions: Final conditions of case 2.1 (BA\_Ra104\_Iso\_FS)
\item $\eta(T) = 1$
\end{itemize}
In other words, we have an increased Rayleigh number and begin with the final
steady state of case 2.1. To start the model where case 2.1 left off, the
input file of case 2.1, \url{benchmark/davies_et_al/case-2.1.prm}, instructs
\aspect{} to checkpoint itself every few time steps (see
Section~\ref{sec:checkpoint-restart}). If case 2.2 uses the same
output directory, we can then resume the computations from this checkpoint with
an input file that prescribes a different Rayleigh number and a later input time:

\lstinputlisting[language=prmfile]{cookbooks/benchmarks/davies_et_al/case-2.2.prm.out}

We increase the Rayleigh number to $10^5$ by increasing the magnitude of
gravity in the input file.  The full script for case 2.2 is located in
\url{benchmark/davies_et_al/case-2.2.prm}


\paragraph{Case 2.3: BA\_Ra103\_vv\_FS.}
\label{sec:davies-case23_BA}

Case 2.3 is a variation on the previous one:
\begin{itemize}
\item Boussinesq Approximation
\item Boundary Condition: Free-Slip
\item Rayleigh Number = $10^3$ 
\item Initial Conditions: Final conditions of case 2.1 (BA\_Ra104\_Iso\_FS)
\item $\eta(T) = 1000^{-T}$
\end{itemize}
The Rayleigh number is smaller here (and is selected using the gravity
parameter in the input file, as before), but the more important change is that the
viscosity is now a function of temperature. At the time of writing, there is
no material model that would implement such a viscosity, so we
create a plugin that does so for us (see Sections~\ref{sec:extending} and
\ref{sec:write-plugin} in
general, and Section~\ref{sec:material-models} for material models in
particular). The code for it is located in 
\url{benchmarks/davies_et_al/case-2.3-plugin/VoT.cc} (where ``VoT'' is short
for ``viscosity as a function of temperature'') and is essentially a copy of
the \texttt{simpler} material model. The primary change compared to the 
\texttt{simpler} material model is the line about the viscosity in the
following function:
\begin{lstlisting}[frame=single,language=C++]
template <int dim>
void
VoT<dim>::
evaluate(const typename Interface<dim>::MaterialModelInputs &in,
         typename Interface<dim>::MaterialModelOutputs &out) const
{
  for (unsigned int i=0; i<in.position.size(); ++i)
    {
      out.viscosities[i] = eta*std::pow(1000,(-in.temperature[i]));
      out.densities[i] = reference_rho * (1.0 - thermal_alpha * (in.temperature[i] - reference_T));
      out.thermal_expansion_coefficients[i] = thermal_alpha;
      out.specific_heat[i] = reference_specific_heat;
      out.thermal_conductivities[i] = k_value;
      out.compressibilities[i] = 0.0;
    }
}
\end{lstlisting}
Using the method described in Sections~\ref{sec:benchmark-run} and
\ref{sec:write-plugin}, and the files in 
the \texttt{benchmarks/davies\_et\_al/case-2.3-plugin}, we can compile our new
material model into a shared library that we can then reference from the input file.
The complete input file for case 2.3 is located in
\url{benchmark/davies_et_al/case-2.3.prm} and contains among others the
following parts:

\lstinputlisting[language=prmfile]{cookbooks/benchmarks/davies_et_al/case-2.3.prm.out}


\paragraph{Results.}

In the following, let us discuss some of the results of the benchmark setups
discussed above. First, the final steady state temperature fields are shown in
Fig.~\ref{fig:davies-2DcylinderFSS}. It is immediately obvious how the
different Rayleigh numbers affect the width of the plumes. If one imagines a
setup with constant gravity, constant inner and outer temperatures and
constant thermal expension coefficient (this is not how we describe it in the
input files, but we could have done so and it is closer to how we intuit about
fluids than adjusting the gravity), then the Rayleigh number is inversely
proportional to the viscosity -- and it is immediately clear that larger
Rayleigh numbers (corresponding to lower viscosities) then lead to thinner
plumes. This is nicely reflected in the visualizations.

\begin{figure}[h]
  \subfigure[Case 1.1]{\includegraphics[width=0.23\textwidth]{cookbooks/benchmarks/davies_et_al/case11_final}}
  \hfill
  \subfigure[Case 2.1]{\includegraphics[width=0.23\textwidth]{cookbooks/benchmarks/davies_et_al/case21_final}}
  \hfill
  \subfigure[Case 2.2]{\includegraphics[width=0.23\textwidth]{cookbooks/benchmarks/davies_et_al/case22_final}}
  \hfill
  \subfigure[Case 2.3]{\includegraphics[width=0.23\textwidth]{cookbooks/benchmarks/davies_et_al/case23_final}}
  \hfill
  \caption{Davies et al.~benchmarks: Final steady state temperature fields for
    the 2D cylindrical benchmark cases.}
  \label{fig:davies-2DcylinderFSS}
\end{figure}

Secondly, Fig.~\ref{fig:davies-2DcylinderVrms} shows the root mean square
velocity as a function of time for the various cases. It is obvious that they
all converge to steady state solutions. However, there is an initial transient
stage and, in cases 2.2 and 2.3, a sudden jolt to the system at the time where
we switch from the model used to compute up to time $t=2$ to the
different models used after that.

\begin{figure}[h]
  \subfigure[Case 1.1]{\includegraphics[width=0.48\textwidth]{cookbooks/benchmarks/davies_et_al/Case11Vrms}}
  \hfill
  \subfigure[Case 2.1]{\includegraphics[width=0.48\textwidth]{cookbooks/benchmarks/davies_et_al/Case21Vrms}}
  \\
  \subfigure[Case 2.2]{\includegraphics[width=0.48\textwidth]{cookbooks/benchmarks/davies_et_al/Case22Vrms}}
  \hfill
  \subfigure[Case 2.3]{\includegraphics[width=0.48\textwidth]{cookbooks/benchmarks/davies_et_al/Case23Vrms}}
  \hfill
  \caption{Davies et al.~benchmarks: $V_\text{rms}$ for 2D Cylindrical Cases. Large jumps occur when transitioning from case 2.1 to cases 2.2 and 2.3 due to the instantaneous change of parameter settings.}
  \label{fig:davies-2DcylinderVrms}
\end{figure}

These runs also produce quantitative data that will be published along with
the concise descriptions of the benchmarks and a comparison with other
codes. In particular, some of the criteria listed above to judge the accuracy
of results are listed in Table~\ref{tab:davies-et-al-results}.%
\footnote{The input files available in the \texttt{benchmark/davies\_et\_al}
  directory use 5 global refinements in order to provide cases that can be run
  without excessive trouble on a normal computer. However, this is not enough
  to achieve reasonable accuracy and both the data shown below and the data
  submitted to the benchmarking effort uses 7
  global refinement steps, corresponding to a mesh with 1536 cells in
  tangential and 128 cells in radial direction. Computing on such meshes is
  not cheap, as it leads to a problem size of more than 2.5 million
  unknowns. It is best done using a parallel computation.}

\begin{table}[tbp]
  \centering
  \begin{tabular}{|l|c|c|c|c|}
    \hline
    Case & $\left<T\right>$ & $Nu_T$ & $Nu_B$ & $V_\text{rms}$
    \\ \hline
    1.1 & 0.403 & 2.464 & 2.468 & 19.053 
    \\ 
    2.1 & 0.382 & 4.7000 & 4.706 & 46.244
    \\ 
    2.2 & 0.382 & 9.548 & 9.584 & 193.371
    \\ 
    2.3 & 0.582 & 5.102 & 5.121 & 79.632
    \\ \hline
  \end{tabular}
  \caption{\it Davies et al. benchmarks: Numerical results for some of the output quantities required by the benchmarks and the various cases considered.}
  \label{tab:davies-et-al-results}
\end{table}


\subsubsection{The Crameri et al.~benchmarks}
\label{sec:benchmark-crameri}

\textit{This section was contributed by Ian Rose.}

This section follows the two free surface benchmarks described by Crameri et al. \cite{CSG12}.  

\paragraph{Case 1: Relaxation of topography.}
\label{sec:benchmark-crameri-case-1}

The first benchmark involves a high viscosity lid sitting on top of a lower viscosity 
mantle. There is an initial sinusoidal topography which is then allowed to relax.
This benchmark has a semi-analytical solution (which is exact for infinitesimally small
topography). Details for the benchmark setup are in Figure~\ref{fig:crameri-benchmark-initial-topography}.


\begin{figure}
  \begin{center}
    \includegraphics[width=0.95\textwidth]{cookbooks/benchmarks/crameri/initial_topography}
  \end{center}
  \caption{\it Setup for the topography relaxation benchmark. The box is $2800$ km wide and $700$ km high, with 
    a $100$ km lid on top. The lid has a viscosity of $10^{23} \, {Pa\,s}$, while the mantle has a viscosity of $10^{21} \, {Pa\,s}$.  The sides are 
    free slip, the bottom is no slip, and the top is a free surface.  Both the lid and the mantle have 
    a density of $3300 \,{kg/m^3}$, and gravity is $10 \, {m/s^2}$. There is a $7 \, {km}$ 
    sinusoidal initial topography on the free surface.}
  \label{fig:crameri-benchmark-initial-topography}
\end{figure}

The complete parameter file for this benchmark can be found in 
\url{benchmarks/crameri_et_al/case_1/crameri_benchmark_1.prm}, 
the  most relevant parts of which are excerpted here: 
\lstinputlisting[language=prmfile]{cookbooks/benchmarks/crameri/crameri_benchmark_1.prm}
In particular, this benchmark uses a custom geometry model to set the initial geometry. 
This geometry model, called ``\texttt{ReboundBox}'', is based on the \texttt{Box} geometry model. 
It generates a domain in using the same parameters as \texttt{Box}, but then displaces all 
the nodes vertically with a sinusoidal perturbation, where the magnitude and order of that 
perturbation are specified in the \texttt{ReboundBox} subsection.


The characteristic timescales of topography relaxation are significantly smaller than those of 
mantle convection. Taking timesteps larger than this relaxation timescale tends to cause sloshing
instabilities, which are described further in Section~\ref{sec:freesurface}. Some sort of stabilization 
is required to take large timesteps. In this benchmark, however, we are interested in the relaxation 
timescale, so we are free to take very small timesteps (in this case, 0.01 times the CFL 
number).  As can be seen in Figure~\ref{fig:crameri-benchmark-relaxation-topography}, the results of all the 
codes which are included in this comparison are basically indistinguishable.

\begin{figure}
  \begin{center}
    \includegraphics[width=0.95\textwidth]{cookbooks/benchmarks/crameri/crameri_1_comparison}
  \end{center}
  \caption{\it Results for the topography relaxation benchmark, showing maximum topography 
   versus time. Over about $100$ ka the topography completely disappears. The results of four 
   free surface codes, as well as the semi-analytic solution, are nearly identical.}
  \label{fig:crameri-benchmark-relaxation-topography}
\end{figure}

\paragraph{Case 2: Dynamic topography.}
\label{sec:benchmark-crameri-case-2}

Case two is more complicated. Unlike the case one, it occurs over mantle convection 
timescales.  In this benchmark there is the same high viscosity lid over a lower 
viscosity mantle. However, now there is a blob of buoyant material rising in the 
center of the domain, causing dynamic topography at the surface. The details for the setup
are in the caption of Figure~\ref{fig:crameri-benchmark-rising-blob}.

\begin{figure}
  \begin{center}
    \includegraphics[width=0.95\textwidth]{cookbooks/benchmarks/crameri/rising_blob}
  \end{center}
  \caption{\it Setup for the dynamic topography benchmark. Again, the domain is $2800$ km 
  wide and $700$ km high.  A $100$ km thick lid with viscosity $10^{23}$ overlies a mantle
  with viscosity $10^{21}$.  Both the lid and the mantle have a density of $3300\,kg/m^3$.
  A blob with diameter $100$ km lies $300$ km from the bottom of the domain.  The blob has 
  a density of $3200 kg/m^3$ and a viscosity of $10^{20}$ Pa s.}
  \label{fig:crameri-benchmark-rising-blob}
\end{figure}

Case two requires higher resolution and longer time integrations than case one. The benchmark
is over 20 million years and builds dynamic topography of $\sim 800$ meters.  

\begin{figure}
  \begin{center}
    \includegraphics[width=0.95\textwidth]{cookbooks/benchmarks/crameri/crameri_2_comparison}
  \end{center}
  \caption{\it Evolution of topography for the dynamic topography benchmark. The maximum topography
   is shown as a function of time, for \aspect{} as well as for several other codes participating in
   the benchmark. This benchmark shows considerably more scatter between the codes.}
  \label{fig:crameri-2-comparison}
\end{figure}

Again, we excerpt the most relevant parts of the parameter file for this benchmark, with the 
full thing available in \url{benchmarks/crameri_et_al/case_2/crameri_benchmark_2.prm}.
Here we use the ``Multicomponent'' material model, which allows us to easily set up a number 
of compositional fields with different material properties. The first compositional field 
corresponds to background mantle, the second corresponds to the rising blob, and the third 
corresponds to the viscous lid.


Furthermore, the results of this benchmark are sensitive to the mesh refenement and timestepping 
parameters. Here we have nine refinement levels, and refine according to density and the 
compositional fields.

\lstinputlisting[language=prmfile]{cookbooks/benchmarks/crameri/crameri_benchmark_2.prm}

Unlike the first benchmark, for case two there is no (semi) analytical solution to compare against.
Furthermore, the time integration for this benchmark is much longer, allowing for errors to 
accumulate. As such, there is considerably more scatter between the participating codes.  \aspect{}
does, however, fall within the range of the other results, and the curve is somewhat less wiggly.
The results for maximum topography versus time are shown in~\ref{fig:crameri-2-comparison}

The precise values for topography at a given time are quite dependent on the resolution and
timestepping parameters. Following \cite{CSG12} we investigate the convergence of the maximum 
topography at 3 Ma as a function of CFL number and mesh resolution.  The results are shown in 
figure~\ref{fig:crameri-benchmark-convergence}. 

\begin{figure}
  \begin{center}
    \includegraphics[width=1.0\textwidth]{cookbooks/benchmarks/crameri/crameri_2_convergence}
  \end{center}
  \caption{\it Convergence for case two.  Left: Logarithm of the error with decreasing CFL number. 
As the CFL number decreases, the error gets smaller. However, once it reaches a value of $\sim0.1$, there
stops being much improvement in accuracy. Right: Logarithm of the error with increasing maximum mesh 
resolution. As the resolution increases, so does the accuracy.}
  \label{fig:crameri-benchmark-convergence}
\end{figure}

We find that at 3 Ma \aspect{}  converges to a maximum topography of $\sim$396 meters. 
This is slightly different from what MILAMIN\_VEP reported as its convergent value in \cite{CSG12},
but still well within the range of variation of the codes. Additionally, we note that \aspect{}
is able to achieve good results with relatively less mesh resolution due to the ability 
to adaptively refine in the regions of interest (namely, the blob and the high viscosity lid).

Accuracy improves roughly linearly with decreasing CFL number, though stops improving at CFL $\sim 0.1$.
Accuracy also improves with increasing mesh resolution, though its convergence order does not seem 
to be excellent.  It is possible that other mesh refinement parameters than we tried in this benchmark 
could improve the convergence. The primary challenge in accuracy is limiting numerical diffusion 
of the rising blob. If the blob becomes too diffuse, its ability to lift topography is diminished.
It would be instructive to compare the results of this benchmark using tracer particles with the 
results using compositional fields.

\subsubsection{The solitary wave benchmark}
\label{sec:benchmark-solitary_wave}

\textit{This section was contributed by Juliane Dannberg and is based on a section in \cite{DH2015} by Juliane Dannberg and Timo Heister.}

One of the most widely used benchmarks for codes that model the migration of melt through a compacting and dilating matrix is the propagation of solitary waves (e.g. \cite{SS11, KMK2013, Schm00}). 
The benchmark is intended to test the accuracy of the solution of the two-phase flow equations as described in Section \ref{sec:melt_transport} and makes use of the fact that there is an analytical solution for the shape of solitary waves that travel through a partially molten rock with a constant background porosity without changing their shape and with a constant wave speed. 
Here, we follow the setup of the benchmark as it is described in \cite{BR86}, which considers one-dimensional solitary waves.

The model features a perturbation of higher porosity with the amplitude $A \phi_0$ in a uniform low-porosity ($\phi=\phi_0$) background.  Due to its lower density, melt migrates upwards, dilating the solid matrix at its front and compacting it at its end. 

Assuming constant shear and compaction viscosities and using a permeability law of the form
%
\begin{align*}
k_\phi &= k_0 \phi^3, && \text{ implying a Darcy coefficient }
K_D(\phi) = \frac{k_0}{\eta_f} \phi^3 , \\
\intertext{and the non-dimensionalization }
x &= \delta x' 
  && \text{ with the compaction length } \delta = \sqrt{K_D(\phi_0)(\xi + \frac{4}{3}\eta)} , \\
\phi &= \phi_0 \phi ' 
  && \text{ with the background porosity } \phi_0 , \\
(\mathbf u_s, \mathbf u_f) &= u_0 (\mathbf u_s, \mathbf u_f)' 
  && \text{ with the separation flux } \phi_0 u_0 = K_D(\phi_0) \Delta\rho g ,
\end{align*}
%
the analytical solution for the shape of the solitary wave can be written in implicit form as:
\begin{align*}
x(\phi) &= \pm (A + 0.5) 
\left[ -2 \sqrt{A-\phi} + \frac{1}{\sqrt{A-1}} 
\ln \frac{\sqrt{A-1} - \sqrt{A-\phi}}{\sqrt{A-1} + \sqrt{A-\phi}} \right]
\end{align*}
and the phase speed $c$, scaled back to physical units, is $c = u_0 (2A+1)$. 
This is only valid in the limit of small porosity $\phi_0 \ll 1$. Figure~\ref{fig:setup-solitary-wave} illustrates the model setup. 

\begin{figure}
  \begin{center}
    \includegraphics[width=0.65\textwidth]{cookbooks/benchmarks/solitary_wave/setup.pdf}
  \end{center}
  \caption{\it Setup of the solitary wave benchmark. The domain is $400$ m high and includes a low porosity
  ($\phi = 0.001$) background with an initial perturbation ($\phi = 0.1$). The solid density is $3300\,kg/m^3$
  and the melt density is $2500\,kg/m^3$. We apply the negative phase speed 
  of the solitary wave $\mathbf u_s = -c \, \mathbf e_z$ as velocity boundary condition, so that the wave will 
  stay at its original position while the background is moving.}
  \label{fig:setup-solitary-wave}
\end{figure}

The parameter file and material model for this setup can be found in \url{benchmark/solitary_wave/solitary_wave.prm} and \url{benchmark/solitary_wave/solitary_wave.cc}. The most relevant sections are shown in the following paragraph. 

\lstinputlisting[language=prmfile]{cookbooks/benchmarks/solitary_wave/solitary_wave.prm}

The benchmark uses a custom model to generate the initial condition for the porosity field as specified by the analytical solution, and its own material model, which includes the additional material properties needed by models with melt migration, such as the permeability, melt density and compaction viscosity. The solitary wave postprocessor compares the porosity and pressure in the model to the analytical solution, and computes the errors for the shape of the porosity, shape of the compaction pressure and the phase speed. 
We apply the negative phase speed of the solitary wave as a boundary condition for the solid velocity. This changes the reference frame, so that the solitary wave stays in the center of the domain, while the solid moves downwards. The temperature evolution does not play a role in this benchmark, so all temperature and heating-related parameters are disabled or set to zero. 

And extensive discussion of the results and convergence behavior can be found in \cite{DH2015}.


\section{Extending \aspect}
\label{sec:extending}

\aspect{} is designed to be an extensible code. In particular, it
uses both a plugin architecture and a set of signals through which it is
trivial to replace or extend certain components of the program. Examples of
things that are simple to extend are:
\begin{itemize}
\item the material description,
\item the geometry,
\item the gravity description,
\item the initial conditions,
\item the boundary conditions,
\item the functions that postprocess the solution, i.e., that can compute
  derived quantities such as heat fluxes over part of the boundary, mean
  velocities, etc.,
\item the functions that generate derived quantities that can be put into
  graphical output files for visualization such as fields that depict the
  strength of the friction heating term, spatially dependent actual
  viscosities, and so on,
\item the computation of refinement indicators,
\item the determination of how long a computation should run.
\end{itemize}
This list may also have grown since this section was written.
We will discuss the way this is achieved in Sections~\ref{sec:plugins} and
\ref{sec:plugins-concrete}. Changing the core functionality, i.e., the basic equations
\eqref{eq:stokes-1}--\eqref{eq:temperature}, and how they are solved is
arguably more involved. We will discuss this in Section
\ref{sec:extending-solver}.

\note{The purpose of coming up with ways to make extensibility simple is that if
you want to extend \aspect{} for your own purposes, you can do this in a
separate set of files that describe your situation, rather than by modifying
the \aspect{} source files themselves. This is important, because (i) it makes
it possible for you to update \aspect{} itself to a newer version without
losing the functionality you added (because you did not make any changes to
the \aspect{} files themselves), (ii) because it makes it possible to keep
unrelated changes separate in your own set of files, in a place where they are
simple to find, and (iii) because it makes it much easier for you to share
your modifications and additions with others.}

Since \aspect{} is written in C++ using the \dealii{} library, you
will have to be proficient in C++. You will also likely have
to familiarize yourself with this library for which there is an extensive
amount of documentation:
\begin{itemize}
\item The manual at
  \url{https://www.dealii.org/developer/doxygen/deal.II/index.html} that
  describes in detail what every class, function and variable in \dealii{}
  does.
\item A collection of modules at
  \url{https://www.dealii.org/developer/doxygen/deal.II/modules.html} that give
  an overview of whole groups of classes and functions and how they work
  together to achieve their goal.
\item The \dealii{} tutorial at
  \url{https://www.dealii.org/developer/doxygen/tutorial/index.html} that
  provides a step-by-step introduction to the library using a sequence of
  several dozen programs that introduce gradually more complex topics. In
  particular, you will learn \dealii's way of \textit{dimension independent
  programming} that allows you to write the program once, test it in 2d, and
  run the exact same code in 3d without having to debug it a second time.
\item The step-31 and step-32 tutorial programs at
  \url{https://www.dealii.org/developer/doxygen/deal.II/step_31.html} and
  \url{https://www.dealii.org/developer/doxygen/deal.II/step_32.html} from
  which \aspect{} directly descends.
\item An overview of many general approaches to numerical methods, but also
  a discussion of \dealii{} and tools we use in programming, debugging and
  visualizing data are given in Wolfgang Bangerth's video lectures. These
  are linked from the \dealii{} website at \url{https://www.dealii.org/}
  and directly available at
  \url{http://www.math.tamu.edu/~bangerth/videos.html}.
\item The \dealii{} Frequently Asked Questions at
  \url{https://github.com/dealii/dealii/wiki/Frequently-Asked-Questions}
  that also have extensive sections on developing code with \dealii{} as well
  as on debugging. It also answers a number of questions we frequently get
  about the use of C++ in \dealii{}.
\item Several other parts of the \dealii{} website at
  \url{https://www.dealii.org/} also have information that may be relevant if
  you dive deeper into developing code. If you have questions, the mailing
  lists at \url{https://www.dealii.org/mail.html} are also of general help.
\item A general overview of \dealii{} is also provided in the paper
  \cite{BHK07}.
\end{itemize}

As a general note, by default \aspect{} utilizes a \dealii{} feature called \textit{debug
  mode}, see also the introduction to this topic in
Section~\ref{sec:debug-mode}. If you develop code, you will definitely want
this feature to be on, as it will capture the vast majority of bugs you
will invariably introduce in your code.

When you write new functionality and run
the code for the first time, you will almost invariably first have to deal
with a number of these assertions that point out problems in your code. While
this may be annoying at first, remember that these are actual bugs in your
code that have to be fixed anyway and that are much easier to find if the
program aborts than if you have to go by their more indirect results such as
wrong answers. The Frequently Asked Questions at
\url{https://github.com/dealii/dealii/wiki/Frequently-Asked-Questions}
contain a section on how to debug \dealii{} programs.

The downside of debug mode, as mentioned before, is that it makes the program
much slower. Consequently, once you are
confident that your program actually does what it is intended to do --
\textbf{but no earlier!} --, you may want to switch to optimized mode that
links \aspect{} with a version of the \dealii{} libraries that uses compiler
optimizations and that does not contain the \texttt{assert} statements
discussed above. This switch can be facilitated by editing the top of the
\aspect{} \url{Makefile} and recompiling the program.

In addition to these general comments, \aspect{} is itself extensively
documented. You can find documentation on all classes, functions and
namespaces starting from the \url{doc/doxygen/index.html} page.


\subsection{The idea of plugins and the \texttt{SimulatorAccess} and \texttt{Introspection} classes}
\label{sec:plugins}

The most common modification you will probably want to do to \aspect{} are to
switch to a different material model (i.e., have different values of
functional dependencies for the coefficients $\eta,\rho,C_p, \ldots$ discussed
in Section~\ref{sec:coefficients}); change the geometry; change the direction
and magnitude of the gravity vector $\mathbf g$; or change the initial and
boundary conditions.

To make this as simple as possible, all of these parts of the program (and some more) have
been separated into modules that can be replaced quickly and where it is
simple to add a new implementation and make it available to the rest of the
program and the input parameter file. The way this is achieved is through the
following two steps:
\begin{itemize}
\item The core of \aspect{} really only communicates with material models,
  geometry descriptions, etc., through a simple and very basic
  interface. These interfaces are declared in the
  \url{include/aspect/material_model/interface.h},
  \url{include/aspect/geometry_model/interface.h}, etc., header files. These
  classes are always called \texttt{Interface}, are located in namespaces that
  identify their purpose, and their documentation can be found from the
  general class overview in \url{doc/doxygen/classes.html}.

  To show an example of a rather minimal case, here is the declaration of the
\href{doc/doxygen/classaspect_1_1GravityModel_1_1Interface.html}{aspect::GravityModel::Interface} class (documentation comments have
  been removed):
  \begin{lstlisting}[frame=single,language=C++]
    class Interface
    {
      public:
        virtual ~Interface();

        virtual
        Tensor<1,dim>
        gravity_vector (const Point<dim> &position) const = 0;

        static void declare_parameters (ParameterHandler &prm);

        virtual void parse_parameters (ParameterHandler &prm);
    };
  \end{lstlisting}

  If you want to implement a new model for gravity, you just need to write a
  class that derives from this base class and implements the
  \texttt{gravity\_vector} function. If your model wants to read parameters
  from the input file, you also need to have functions called
  \texttt{declare\_parameters} and \texttt{parse\_parameters} in your class
  with the same signatures as the ones above. On the other hand, if the new
  model does not need any run-time parameters, you do not need to overload
  these functions.%
  \footnote{At first glance one may think that only the
    \texttt{parse\_parameters} function can be overloaded since
    \texttt{declare\_parameters} is not virtual. However, while the latter is
    called by the class that manages plugins through pointers to the interface
    class, the former function is called essentially at the time of
    registering a plugin, from code that knows the actual type and name of the
    class you are implementing. Thus, it can call the function -- if it exists
    in your class, or the default implementation in the base class if it doesn't
    -- even without it being declared as virtual.}

  Each of the categories above that allow plugins have several implementations
  of their respective interfaces that you can use to get an idea of how to
  implement a new model.

\item At the end of the file where you implement your new model, you need to
  have a call to the macro \texttt{ASPECT\_REGISTER\_GRAVITY\_MODEL} (or the
  equivalent for the other kinds of plugins). For
  example, let us say that you had implemented a gravity model that takes
  actual gravimetric readings from the GRACE satellites into account, and had
  put everything that is necessary into a class
  \texttt{aspect::GravityModel::GRACE}. Then you need a statement like this at
  the bottom of the file:
  \begin{lstlisting}[frame=single,language=C++]
    ASPECT_REGISTER_GRAVITY_MODEL
    (GRACE,
     "grace",
     "A gravity model derived from GRACE "
     "data. Run-time parameters are read from the parameter "
     "file in subsection 'Radial constant'.");
  \end{lstlisting}
  Here, the first argument to the macro is the name of the class. The second
  is the name by which this model can be selected in the parameter file. And
  the third one is a documentation string that describes the purpose of the
  class (see, for example, Section~\ref{parameters:Gravity_20model} for an
  example of how existing models describe themselves).

  This little piece of code ensures several things: (i) That the parameters
  this class declares are known when reading the parameter file. (ii) That you
  can select this model (by the name ``grace'') via the run-time parameter
  \texttt{Gravity model/Model name}. (iii) That \aspect{} can create an object
  of this kind when selected in the parameter file.

  Note that you need not announce the existence of this class in any other
  part of the code: Everything should just work automatically.%
  \footnote{The existing implementations of models of the gravity and other interfaces
  declare the class in a header file and define the member functions in a
  \texttt{.cc} file. This is done so that these classes show up in our
  doxygen-generated documentation, but it is not necessary: you can put your
  entire class declaration and implementation into a single file as long as
  you call the macro discussed above on it. This single file is all you need
  to touch to add a new model.}
  This has the advantage that things are neatly separated: You do not need to
  understand the core of \aspect{} to be able to add a new gravity model that
  can then be selected in an input file. In fact, this is true for
  all of the plugins we have: by and large, they just receive some data
  from the simulator and do something with it (e.g., postprocessors), or they
  just provide information (e.g., initial meshes, gravity models), but their
  writing does not require that you have a fundamental understanding
  of what the core of the program does.
\end{itemize}

The procedure for the other areas where plugins are supported works
essentially the same, with the obvious change in namespace for the interface
class and macro name.

In the following, we will discuss the requirements for individual plugins. Before
doing so, however, let us discuss ways in which plugins can query other
information, in particular about the current state of the simulation.
To this end, let us not consider those plugins that by and large just
provide information without any context of the simulation, such as gravity models,
prescribed boundary velocities, or initial temperatures. Rather, let us
consider things like postprocessors that can compute things like boundary heat
fluxes. Taking this as an example (see Section~\ref{sec:postprocessors}), you are
required to write a function with the following interface
\begin{lstlisting}[frame=single,language=C++]
    template <int dim>
    class MyPostprocessor : public aspect::Postprocess::Interface
    {
      public:
        virtual
        std::pair<std::string,std::string>
        execute (TableHandler &statistics);

      // ... more things ...
\end{lstlisting}
The idea is that in the implementation of the \texttt{execute} function
you would compute whatever you are interested in (e.g., heat fluxes)
and return this information in the statistics object that then gets written
to a file (see Sections~\ref{sec:running-overview} and \ref{sec:viz-stat}).
A postprocessor may also generate other files if it so likes -- e.g., graphical
output, a file that stores the locations of tracers, etc.
To do so, obviously you need access to the current solution. This is
stored in a vector somewhere in the core of \aspect{}. However, this
vector is, by itself, not sufficient: you also need to know the finite
element space it is associated with, and for that the triangulation it
is defined on. Furthermore, you may need to know what the current
simulation time is. A variety of other pieces of information enters
computations in these kinds of plugins.

All of this information is of course part of the core of \aspect{},
as part of the
\href{doc/doxygen/classaspect_1_1Simulator.html}{aspect::Simulator
class}. However, this is a rather heavy class: it's got dozens of
member variables and functions, and it is the one that does all
of the numerical heavy lifting. Furthermore, to access data in
this class would require that you need to learn about the internals,
the data structures, and the design of this class.
It would be poor design if plugins had to access information from this
core class directly. Rather, the way this works is that those plugin
classes that wish to access information about the state of the simulation
inherit from the
\href{doc/doxygen/classaspect_1_1SimulatorAccess.html}{aspect::SimulatorAccess
class}. This class has an interface that looks like this:
\begin{lstlisting}[frame=single,language=C++]
    template <int dim>
    class SimulatorAccess
    {
    protected:
      double       get_time () const;

      std::string  get_output_directory () const;

      const LinearAlgebra::BlockVector &
      get_solution () const;

      const DoFHandler<dim> &
      get_dof_handler () const;

      // ... many more things ...
\end{lstlisting}
This way, \href{doc/doxygen/classaspect_1_1SimulatorAccess.html}{SimulatorAccess} makes information available to plugins
without the need for them to understand details of the core of \aspect{}.
Rather, if the core changes, the \href{doc/doxygen/classaspect_1_1SimulatorAccess.html}{SimulatorAccess} class can still
provide exactly the same interface. Thus, it insulates plugins from having
to know the core. Equally importantly, since \href{doc/doxygen/classaspect_1_1SimulatorAccess.html}{SimulatorAccess} only
offers its information in a read-only way it insulates the core from
plugins since they can not interfere in the workings of the core except
through the interface they themselves provide to the core.

Using this class, if a plugin class \texttt{MyPostprocess} is then not only
derived from the corresponding \texttt{Interface} class but \textit{also}
from the \href{doc/doxygen/classaspect_1_1SimulatorAccess.html}{SimulatorAccess} class, then you can write a member
function of the following kind (a nonsensical but instructive example; see
Section~\ref{sec:postprocessors} for more details on what postprocessors do
and how they are implemented):%
\footnote{For complicated, technical reasons, in the code below we need to
  access elements of the \href{doc/doxygen/classaspect_1_1SimulatorAccess.html}{SimulatorAccess} class using the notation
  \texttt{this->get\_solution()}, etc. This is due to the fact that both the
  current class and the base class are templates. A long description of
  why it is necessary to use \texttt{this->} can be found in the \dealii{}
  Frequently Asked Questions.}
\begin{lstlisting}[frame=single,language=C++]
    template <int dim>
    std::pair<std::string,std::string>
    MyPostprocessor<dim>::execute (TableHandler &statistics)
    {
      // compute the mean value of vector component 'dim' of the solution
      // (which here is the pressure block) using a deal.II function:
      const double
        average_pressure = VectorTools::compute_mean_value (this->get_mapping(),
                                                            this->get_dof_handler(),
                                                            QGauss<dim>(2),
                                                            this->get_solution(),
                                                            dim);
      statistics.add_value ("Average pressure", average_pressure);

      // return that there is nothing to print to screen (a useful
      // plugin would produce something more elaborate here):
      return std::pair<std::string,std::string>();
    }
\end{lstlisting}

The second piece of information that plugins can use is called ``introspection''.
In the code snippet above, we had to use that the pressure variable is at
position \texttt{dim}. This kind of \textit{implicit knowledge} is usually
bad style: it is error prone because one can easily forget where each
component is located; and it is an obstacle to the extensibility of a code
if this kind of knowledge is scattered all across the code base.

Introspection is a way out of this dilemma. Using the \texttt{SimulatorAccess::introspection()}
function returns a reference to an object (of type
\href{doc/doxygen/structaspect_1_1Introspection.html}{aspect::Introspection})
that plugins can use to learn about these sort of conventions. For example,
\texttt{this->introspection().component\_mask.pressure} returns a
component mask (a deal.II concept that describes a list of booleans for each
component in a finite element that
are true if a component is part of a variable we would like to select and
false otherwise) that describes which component of the finite element
corresponds to the pressure. The variable, \texttt{dim}, we need above
to indicate that we want the pressure component can be accessed
as \texttt{this->introspection().component\_indices.pressure}. While this
is certainly not shorter than just writing \texttt{dim}, it may in
fact be easier to remember. It is most definitely less prone to
errors and makes it simpler to extend the code in the future because
we don't litter the sources with ``magic constants'' like the one
above.

This \href{doc/doxygen/structaspect_1_1Introspection.html}{aspect::Introspection} class
has a significant number of variables that can be used in this way, i.e.,
they provide symbolic names for things one frequently has to do and
that would otherwise require implicit knowledge of things such as the
order of variables, etc.


\subsection{How to write a plugin}
\label{sec:write-plugin}

Before discussing what each kind of plugin actually has to implement (see the
next subsection), let us briefly go over what you actually have to do when
implementing a new plugin. Essentially, the following steps are all you need to
do:
\begin{itemize}
  \item Create a file, say \texttt{my\_plugin.cc} that contains the declaration
  of the class you want to implement. This class must be derived from one of the
  \texttt{Interface} classes we will discuss below. The file also needs to
  contain the implementation of all member functions of your class.

  As discussed above, it is possible -- but not necessary -- to split this file
  into two: a header file, say \texttt{my\_plugin.h}, and the
  \texttt{my\_plugin.cc} file (or, if you prefer, into multiple source files).
  We do this for all the existing plugins in \aspect{} so that the documentation
  of these plugins shows up in the
  doxygen-generated documentation. However, for your own plugins, there is
  typically no need for this split. The only occasion where this would be useful
  is if some plugin actually makes use of a different plugin (e.g., the
  implementation of a gravity model of your own may want to query some
  specifics of a geometry model you also implemented); in that case the
  \textit{using} plugin needs to be able to see the declaration of the class of
  the \textit{used} plugin, and for this you will need to put the declaration of
  the latter into a header file.

  \item At the bottom of the \texttt{my\_plugin.cc} file, put a statement that
  instantiates the plugin, documents it, and makes it available to the parameter
  file handlers by registering it. This is always done using one of the
  \texttt{ASPECT\_REGISTER\_*} macros that will be discussed in the next
  subsections; take a look at how they are used in the existing plugins in the
  \aspect{} source files.

  \item You need to compile the file. There are two ways by which this can be
  achieved:
  \begin{itemize}
    \item Put the \texttt{my\_plugin.cc} into one of the \aspect{} source
    directories and call \texttt{cmake .} followed by \texttt{make} to ensure
    that it actually gets compiled. This approach has the advantage that you do
    not need to worry much about how the file actually gets compiled. On the
    other hand, every time you modify the file, calling \texttt{make} requires
    not only compiling this one file, but also link \aspect{}. Furthermore, when
    you upgrade from one version of \aspect{} to another, you need to remember
    to copy the \texttt{my\_plugin.cc} file.

    \item Put the  \texttt{my\_plugin.cc} file into a directory of your choice
    and compile it into a shared library yourself. This may be as easy as
    calling
    \begin{verbatim}
 g++ -I/path/to/aspect/headers -I/path/to/deal.II/headers \backslash
     -fPIC -shared my_plugin.cc -o my_plugin.so
    \end{verbatim}
    on Linux, but the command may be different on other systems. Now you only
    need to tell \aspect{} to load this shared library at startup so that the
    plugin becomes available at run time and can be selected from the input
    parameter file. This is done using the \texttt{Additional shared libraries}
    \index[prmindex]{Additional shared libraries}
    \index[prmindexfull]{Additional shared libraries}
    parameter in the input file, see Section~\ref{parameters:global}. This
    approach has the upside that you can keep all files that define new plugins
    in your own directories where you also run the simulations, also making it
    easier to keep around your plugins as you upgrade your \aspect{}
    installation. On the other hand, compiling the file into a shared library is
    a bit more that you need to do yourself. Nevertheless, this is the preferred
    approach.

    In practice, the compiler line above can become tedious because it includes
    paths to the \aspect{} and \dealii{} header files, but possibly also other
    things such as Trilinos headers, etc. Having to remember all of these pieces
    is a hassle, and a much easier way is in fact to set up a mini-CMake project
    for this. To this end, simply copy the file \url{doc/plugin-CMakeLists.txt}
    to the directory where you have your plugin source files and rename it to
    \texttt{CMakeLists.txt}.
  \end{itemize}
  You can then just run the commands
    \begin{verbatim}
 cmake -DAspect_DIR=/path/to/aspect .
 make
    \end{verbatim}
    and it should compile your plugin files into a shared library
    \texttt{my\_plugin.so}. A concrete example of this process is discussed in
    Section~\ref{sec:benchmark-run}. Of course, you may want to choose different names
    for the source files \texttt{source\_1.cc}, \texttt{source\_2.cc} or the name of
    the plugin \texttt{my\_plugin}.

    In essence, what these few lines do is that they find an \aspect{}
    installation (i.e., the directory where you configured and compiled it,
    which may be the same directory as where you keep your sources, or a
    different one, as discussed in Section~\ref{sec:installation}) in either the
    directory explicitly specified in the \texttt{Aspect\_DIR} variable passed
    to \texttt{cmake}, the shell environment variable \texttt{ASPECT\_DIR}, or just one directory up. It then
    sets up compiler paths and similar, and the following lines simply define
    the name of a plugin, list the source files for it, and define everything
    that's necessary to compile them into a shared library. Calling
    \texttt{make} on the command line then simply compiles everything.
\end{itemize}

\note{Complex projects built on \aspect{} often require plugins of more than
just one kind. For example, they may have plugins for the geometry, the
material model, and for postprocessing. In such cases, you can either define
multiple shared libraries by repeating the calls to \texttt{PROJECT},
\texttt{ADD\_LIBRARY} and \texttt{ASPECT\_SETUP\_PLUGIN} for each shared
library in your
\texttt{CMakeLists.txt} file above, or you can just compile all of your source
files into a single shared library. In the latter case, you only need to list a
single library in your input file, but each plugin will still be selectable in
the various sections of your input file as long as each of your classes has a
corresponding \texttt{ASPECT\_REGISTER\_*} statement somewhere in the file
where you have its definition. An even simpler approach is to just put
everything into a single file -- there is no requirement that different
plugins are in separate files, though this is often convenient from a code
organization point of view.}

\note{If you choose to compile your plugins into a shared library yourself, you
  will need to recompile them every time you upgrade your \aspect{} installation
  since we do not guarantee that the \aspect{} application binary interface
  (ABI) will remain stable, even if it may not be necessary to actually change
  anything in the \textit{implementation} of your plugin.}


\subsection{Materials, geometries, gravitation and other plugin types}
\label{sec:plugins-concrete}

\subsubsection{Material models}
\label{sec:material-models}

\index[prmindex]{Model name}
\index[prmindexfull]{Material model!Model name}
The material model is responsible for describing the various coefficients in
the equations that \aspect{} solves. To implement a new material model, you
need to overload the \href{doc/doxygen/classaspect_1_1MaterialModel_1_1Interface.html}{aspect::MaterialModel::Interface} class and use
the \texttt{ASPECT\_REGISTER\_MATERIAL\_MODEL} macro to register your new
class. The implementation of the new class should be in namespace
\texttt{aspect::MaterialModel}. An example of a material model implemented
this way is given in Section~\ref{sec:davies-case23_BA}.

Specifically, your new class needs to implement the following interface:
\begin{lstlisting}[frame=single,language=C++]
    template <int dim>
    class aspect::MaterialModel::Interface
    {
      public:
        // Physical parameters used in the basic equations
        virtual void evaluate(const MaterialModelInputs &in, MaterialModelOutputs &out) const=0;

        virtual bool is_compressible () const = 0;


        // Reference quantities
        virtual double reference_viscosity () const = 0;

        virtual double reference_density () const = 0;

        virtual double reference_thermal_expansion_coefficient () const = 0;


        // Auxiliary material properties used for postprocessing
        virtual double
        seismic_Vp (const double      temperature,
                    const double      pressure,
                    const std::vector<double> &compositional_fields,
                    const Point<dim> &position) const;

        virtual double
        seismic_Vs (const double      temperature,
                    const double      pressure,
                    const std::vector<double> &compositional_fields,
                    const Point<dim> &position) const;

        virtual unsigned int
        thermodynamic_phase (const double      temperature,
                             const double      pressure,
                             const std::vector<double> &compositional_fields) const;


        // Functions used in dealing with run-time parameters
        static void
        declare_parameters (ParameterHandler &prm);

        virtual void
        parse_parameters (ParameterHandler &prm);


        // Optional:
        virtual void initialize ();

        virtual void update ();
}
\end{lstlisting}
The main properties of the material are computed in the function
evaluate() that takes a struct of type MaterialModelInputs and is
supposed to fill a MaterialModelOutputs structure. For performance
reasons this function is handling lookups at an arbitrary number
of positions, so for each variable (for example viscosity), a
std::vector is returned. The following members of MaterialModelOutputs
need to be filled:
\begin{lstlisting}[frame=single,language=C++]
struct MaterialModelOutputs
{
          std::vector<double> viscosities;
          std::vector<double> densities;
          std::vector<double> thermal_expansion_coefficients;
          std::vector<double> specific_heat;
          std::vector<double> thermal_conductivities;
          std::vector<double> compressibilities;
}
\end{lstlisting}
The variables refer to the coefficients $\eta,C_p,k,\rho$ in
equations \eqref{eq:stokes-1}--\eqref{eq:temperature}, each as a function of
temperature, pressure, position, compositional fields and, in the case of the viscosity, the strain
rate (all handed in by MaterialModelInputs). Implementations of evaluate() may of course choose to ignore
dependencies on any of these arguments.

The remaining functions are used in postprocessing as well as
handling run-time parameters. The exact meaning of these member functions is
documented in the
\href{doc/doxygen/classaspect_1_1MaterialModel_1_1Interface.html}{aspect::MaterialModel::Interface
class documentation}. Note that some of the functions listed above have a
default implementation, as discussed on the documentation page just
mentioned.

The function \texttt{is\_compressible} returns whether we should consider the
material as compressible or not, see Section~\ref{sec:boussinesq} on the
Boussinesq model. As discussed there, incompressibility as described by this function
does not necessarily imply that the density is constant; rather, it
may still depend on temperature or pressure. In the current
context, compressibility simply means whether we should solve the continuity
equation as $\nabla \cdot (\rho \mathbf u)=0$ (compressible Stokes)
or as $\nabla \cdot \mathbf{u}=0$ (incompressible Stokes).

The purpose of the parameter handling functions has been discussed in the general
overview of plugins above.

The functions initialize() and update() can be implemented if desired (the default implementation does nothing) and are useful if the material model has internal state. The function
initialize() is called once during the initialization of \aspect{} and
can be used to allocate memory, initialize state, or read information from
an external file. The function update() is called at the beginning of
every time step.

Additionally, every material model has a member variable ``model\textunderscore dependence'',
declared in the Interface class, which can be accessed from the plugin as
``this$\rightarrow$model\textunderscore dependence''. This structure describes the
nonlinear dependence of the various coefficients on pressure, temperature, composition
or strain rate. This information will be used in future versions of \aspect{} to
implement a fully nonlinear solution scheme based on, for example, a Newton
iteration. The initialization of this variable is optional, but only plugins
that declare correct dependencies can benefit from these solver types. All
packaged material models declare their dependencies in the 
parse\textunderscore parameters() function and can be used as a
starting point for implementations of new material models.

Older versions of \aspect{} used to have individual functions like viscosity()
instead of the evaluate() function discussed above. They are now a deprecated
way of implementing a material model. You can get your old model working
by deriving from InterfaceCompatibility instead of Interface.

\subsubsection{Heating models}
\label{sec:heating-models}


\marginpar{To be written}


\subsubsection{Geometry models}
\label{sec:geometry-models}

\index[prmindex]{Model name}
\index[prmindexfull]{Geometry model!Model name}
The geometry model is responsible for describing the domain in which we want
to solve the equations. A domain is described in \dealii{} by a coarse mesh
and, if necessary, an object that characterizes the boundary. Together, these
two suffice to reconstruct any domain by adaptively refining the coarse mesh
and placing new nodes generated by refining cells onto the surface described
by the boundary object. The geometry model is also responsible for marking
different parts of the boundary with different \textit{boundary indicators}
for which one can then, in the input file, select whether these boundaries
should be Dirichlet-type
(fixed temperature) or Neumann-type (no heat flux) boundaries for the
temperature, and what kind of velocity conditions should hold there. In
\dealii{}, a boundary indicator is a number of type
\texttt{types::boundary\_id}, but since boundaries are hard to remember and
get right in input files, geometry models also have a function that provide a
map from symbolic names that can be used to describe pieces of the boundary to
the corresponding boundary indicators. For example, the simple \texttt{box}
geometry model in 2d provides the map
\texttt{\{"left"$\rightarrow$0, "right"$\rightarrow$1,
"bottom"$\rightarrow$2,"top"$\rightarrow$3\}}, and we have consistently used
these symbolic names in the input files used in this manual.

To implement a new geometry model, you need to overload the
\href{doc/doxygen/classaspect_1_1GeometryModel_1_1Interface.html}{aspect::GeometryModel::Interface}
class and use
the \texttt{ASPECT\_REGISTER\_GEOMETRY\_MODEL} macro to register your new
class. The implementation of the new class should be in namespace
\texttt{aspect::GeometryModel}.

Specifically, your new class needs to implement the following basic interface:
\begin{lstlisting}[frame=single,language=C++]
    template <int dim>
    class aspect::GeometryModel::Interface
    {
      public:
        virtual
        void
        create_coarse_mesh (parallel::distributed::Triangulation<dim> &coarse_grid) const = 0;

        virtual
        double
        length_scale () const = 0;

        virtual
        double depth(const Point<dim> &position) const = 0;

        virtual
        Point<dim> representative_point(const double depth) const = 0;

        virtual
        double maximal_depth() const = 0;

        virtual
        std::set<types::boundary_id_t>
        get_used_boundary_indicators () const = 0;

        virtual
        std::map<std::string,types::boundary_id>
        get_symbolic_boundary_names_map () const;

        static
        void
        declare_parameters (ParameterHandler &prm);

        virtual
        void
        parse_parameters (ParameterHandler &prm);
    };
\end{lstlisting}
The kind of information these functions need to provide is extensively
discussed in the documentation of this interface class at
\href{doc/doxygen/classaspect_1_1GeometryModel_1_1Interface.html}{aspect::GeometryModel::Interface}.
The purpose of the last two functions has been discussed in the general
overview of plugins above.


The \texttt{create\_coarse\_mesh} function does not only create the actual
mesh (i.e., the locations of the vertices of the coarse mesh and how they
connect to cells) but it must also set the boundary indicators for all parts
of the boundary of the mesh. The \dealii{} glossary describes the purpose of
boundary indicators as follows:
\begin{quote}
  In a \texttt{Triangulation} object, every part of the boundary is associated with
  a unique number (of type \texttt{types::boundary\_id}) that is used to identify which
  boundary geometry object is responsible to generate new points when the mesh
  is refined. By convention, this boundary indicator is also often used to
  determine what kinds of boundary conditions are to be applied to a particular
  part of a boundary. The boundary is composed of the faces of the cells and, in 3d,
  the edges of these faces.

  By default, all boundary indicators of a mesh are zero, unless you are
  reading from a mesh file that specifically sets them to something different,
  or unless you use one of the mesh generation functions in namespace \texttt{GridGenerator}
  that have a 'colorize' option. A typical piece of code that sets the boundary
  indicator on part of the boundary to something else would look like
  this, here setting the boundary indicator to 42 for all faces located at
  $x=-1$:
  \begin{lstlisting}[frame=single,language=C++]
  for (typename Triangulation<dim>::active_cell_iterator
         cell = triangulation.begin_active();
       cell != triangulation.end();
       ++cell)
    for (unsigned int f=0; f<GeometryInfo<dim>::faces_per_cell; ++f)
      if (cell->face(f)->at_boundary())
        if (cell->face(f)->center()[0] == -1)
          cell->face(f)->set_boundary_indicator (42);
  \end{lstlisting}
  This calls functions \texttt{TriaAccessor::set\_boundary\_indicator}. In 3d, it may
  also be appropriate to call \texttt{TriaAccessor::set\_all\_boundary\_indicators} instead
  on each of the selected faces. To query the boundary indicator of a particular
  face or edge, use \texttt{TriaAccessor::boundary\_indicator}.

  The code above only sets the boundary indicators of a particular part
  of the boundary, but it does not by itself change the way the Triangulation
  class treats this boundary for the purposes of mesh refinement. For this,
  you need to call \texttt{Triangulation::set\_boundary} to associate a boundary
  object with a particular boundary indicator. This allows the Triangulation
  object to use a different method of finding new points on faces and edges
  to be refined; the default is to use a \texttt{StraightBoundary} object for all
  faces and edges. The results section of step-49 has a worked example that
  shows all of this in action.

  The second use of boundary indicators is to describe not only which geometry
  object to use on a particular boundary but to select a part of the boundary
  for particular boundary conditions. \textit{[...]}

  \textbf{Note:} Boundary indicators are inherited from mother faces and edges to
  their children upon mesh refinement. Some more information about boundary
  indicators is also presented in a section of the documentation of the
  Triangulation class.
\end{quote}

Two comments are in order here. First, if a coarse triangulation's faces
already accurately represent where you want to pose which boundary condition
(for example to set temperature values or determine which are no-flow and
which are tangential flow boundary conditions), then it is sufficient to set
these boundary indicators only once at the beginning of the program since they
will be inherited upon mesh refinement to the child faces. Here, \textit{at the
beginning of the program} is equivalent to inside the
\texttt{create\_coarse\_mesh())} function of the geometry module shown above
that generates the coarse mesh.

Secondly, however, if you can only accurately determine which boundary
indicator should hold where on a refined mesh -- for example because the
coarse mesh is the cube $[0,L]^3$ and you want to have a fixed velocity
boundary describing an extending slab only for those faces for which
$z>L-L_\text{slab}$ -- then you need a way to set the boundary indicator
for all boundary faces either to the value representing the slab or the fluid
underneath \textit{after every mesh refinement step}. By doing so, child faces
can obtain boundary indicators different from that of their parents. \dealii{}
triangulations support this kind of operations using a so-called
\textit{post-refinement signal}. In essence, what this means is that you can
provide a function that will be called by the triangulation immediately after
every mesh refinement step.

The way to do this is by writing a function that sets boundary
indicators and that will be called by the \texttt{Triangulation} class. The
triangulation does not provide a pointer to itself to the function being
called, nor any other information, so the trick is to get this information
into the function. C++ provides a nice mechanism for this that is best
explained using an example:
\begin{lstlisting}[frame=single,language=C++]
    #include <deal.II/base/std_cxx1x/bind.h>

    template <int dim>
    void set_boundary_indicators (parallel::distributed::Triangulation<dim> &triangulation)
    {
      ... set boundary indicators on the triangulation object ...
    }

    template <int dim>
    void
    MyGeometry<dim>::
    create_coarse_mesh (parallel::distributed::Triangulation<dim> &coarse_grid) const
    {
      ... create the coarse mesh ...

      coarse_grid.signals.post_refinement.connect
        (std_cxx1x::bind (&set_boundary_indicators<dim>,
                          std_cxx1x::ref(coarse_grid)));

    }
\end{lstlisting}

What the call to \texttt{std\_cxx1x::bind} does is to produce an object that
can be called like a function with no arguments. It does so by taking the
address of a function that does, in fact, take an argument but permanently fix
this one argument to a reference to the coarse grid triangulation. After each
refinement step, the triangulation will then call the object so created which
will in turn call \texttt{set\_boundary\_indicators<dim>} with the reference
to the coarse grid as argument.

This approach can be generalized. In the example above, we have used a global
function that will be called. However, sometimes it is necessary that this
function is in fact a member function of the class that generates the mesh,
for example because it needs to access run-time parameters. This can be
achieved as follows: assuming the \texttt{set\_boundary\_indicators()}
function has been declared as a (non-static, but possibly private) member
function of the \texttt{MyGeometry} class, then the following will work:
\begin{lstlisting}[frame=single,language=C++]
    #include <deal.II/base/std_cxx1x/bind.h>

    template <int dim>
    void
    MyGeometry<dim>::
    set_boundary_indicators (parallel::distributed::Triangulation<dim> &triangulation) const
    {
      ... set boundary indicators on the triangulation object ...
    }

    template <int dim>
    void
    MyGeometry<dim>::
    create_coarse_mesh (parallel::distributed::Triangulation<dim> &coarse_grid) const
    {
      ... create the coarse mesh ...

      coarse_grid.signals.post_refinement.connect
        (std_cxx1x::bind (&MyGeometry<dim>::set_boundary_indicators,
                          std_cxx1x::cref(*this),
                          std_cxx1x::ref(coarse_grid)));
    }
\end{lstlisting}
Here, like any other member function, \texttt{set\_boundary\_indicators}
implicitly takes a pointer or reference to the object it belongs to as first
argument. \texttt{std::bind} again creates an object that can be called like a
global function with no arguments, and this object in turn calls
\texttt{set\_boundary\_indicators} with a pointer to the current object and a
reference to the triangulation to work on. Note that because the
\texttt{create\_coarse\_mesh} function is declared as \texttt{const}, it is
necessary that the \texttt{set\_boundary\_indicators} function is also
declared \texttt{const}.

\note{For reasons that have to do with the way the
  \texttt{parallel::distributed::Triangulation} is implemented, functions that
  have been attached to the post-refinement signal of the triangulation are
  called more than once, sometimes several times, every time the triangulation
  is actually refined.}


\subsubsection{Gravity models}
\label{sec:gravity-models}

\index[prmindex]{Model name}
\index[prmindexfull]{Gravity model!Model name}
The gravity model is responsible for describing the magnitude and direction of
the gravity vector at each point inside the domain. To implement a new gravity model, you
need to overload the
\href{doc/doxygen/classaspect_1_1GravityModel_1_1Interface.html}{aspect::GravityModel::Interface}
class and use
the \texttt{ASPECT\_REGISTER\_GRAVITY\_MODEL} macro to register your new
class. The implementation of the new class should be in namespace
\texttt{aspect::GravityModel}.

Specifically, your new class needs to implement the following basic interface:
\begin{lstlisting}[frame=single,language=C++]
    template <int dim>
    class aspect::GravityModel::Interface
    {
      public:
        virtual
        Tensor<1,dim>
        gravity_vector (const Point<dim> &position) const = 0;

        virtual
        void
        update ();

        static
        void
        declare_parameters (ParameterHandler &prm);

        virtual
        void
        parse_parameters (ParameterHandler &prm);
    };
\end{lstlisting}
The kind of information these functions need to provide is discussed in the
documentation of this interface class at
\href{doc/doxygen/classaspect_1_1GravityModel_1_1Interface.html}{aspect::GravityModel::Interface}. The first needs to return a gravity
vector at a given position, whereas the second is called at the beginning of
each time step, for example to allow a model to update itself based on the
current time or the solution of the previous time step.
The purpose of the last two functions has been
discussed in the general overview of plugins above.


\subsubsection{Initial conditions}
\label{sec:initial-conditions}

\index[prmindex]{Model name}
\index[prmindexfull]{Initial conditions!Model name}
The initial conditions model is responsible for describing the initial
temperature distribution throughout the domain. It essentially has to provide
a function that for each point can return the initial temperature. Note that
the model \eqref{eq:stokes-1}--\eqref{eq:temperature} does not require initial
values for the pressure or velocity. However, if coefficients are nonlinear,
one can significantly reduce the number of initial nonlinear iterations if a
good guess for them is available; consequently, \aspect{} initializes the
pressure with the adiabatically computed hydrostatic pressure, and a zero
velocity. Neither of these two has to be provided by the objects considered in
this section.

To implement a new initial conditions model, you
need to overload the
\href{doc/doxygen/classaspect_1_1InitialConditions_1_1Interface.html}{aspect::InitialConditions::Interface}
class and use
the \texttt{ASPECT\_REGISTER\_INITIAL\_CONDITIONS} macro to register your new
class. The implementation of the new class should be in namespace
\texttt{aspect::InitialConditions}.

Specifically, your new class needs to implement the following basic interface:
\begin{lstlisting}[frame=single,language=C++]
    template <int dim>
    class aspect::InitialConditions::Interface
    {
      public:
        void
        initialize (const GeometryModel::Interface<dim>       &geometry_model,
                    const BoundaryTemperature::Interface<dim> &boundary_temperature,
                    const AdiabaticConditions<dim>            &adiabatic_conditions);

        virtual
        double
        initial_temperature (const Point<dim> &position) const = 0;

        static
        void
        declare_parameters (ParameterHandler &prm);

        virtual
        void
        parse_parameters (ParameterHandler &prm);
    };
\end{lstlisting}
The meaning of the first class should be clear. The purpose
of the last two functions has been discussed in the general overview of
plugins above.


\subsubsection{Prescribed velocity boundary conditions}
\label{sec:prescribed-velocity-boundary-conditions}

\index[prmindex]{Prescribed velocity boundary indicators}
\index[prmindexfull]{Model settings!Prescribed velocity boundary indicators}

Most of the time, one chooses relatively simple boundary values for the
velocity: either a zero boundary velocity, a tangential flow model in which
the tangential velocity is unspecified but the normal velocity is zero at the
boundary, or one in which all components of the velocity are unspecified (i.e.,
for example, an outflow or inflow condition where the total stress in the fluid
is assumed to be zero). However, sometimes we want to choose a velocity model in
which the velocity on the boundary equals some prescribed value. A typical
example is one in which plate velocities are known, for example their current
values or historical reconstructions. In that case, one needs a model in which
one needs to be able to evaluate the velocity at individual points at the
boundary. This can be implemented via plugins.

To implement a new boundary velocity model, you
need to overload the
\href{doc/doxygen/classaspect_1_1VelocityBoundaryConditions_1_1Interface.html}{aspect::VelocityBoundaryConditions::Interface}
class and use
the \texttt{ASPECT\_REGISTER\_VELOCITY\_BOUNDARY\_CONDITIONS} macro to
register your new class. The implementation of the new class should be in namespace
\texttt{aspect::VelocityBoundaryConditions}.

Specifically, your new class needs to implement the following basic interface:
\begin{lstlisting}[frame=single,language=C++]
    template <int dim>
    class aspect::VelocityBoundaryConditions::Interface
    {
      public:
        virtual
        Tensor<1,dim>
        boundary_velocity (const Point<dim> &position) const = 0;

        virtual
        void
        initialize (const GeometryModel::Interface<dim> &geometry_model);

        virtual
        void
        update ();

        static
        void
        declare_parameters (ParameterHandler &prm);

        virtual
        void
        parse_parameters (ParameterHandler &prm);
    };
\end{lstlisting}
The first of these functions needs to provide the velocity at the
given point. The next two are other member functions that can
(but need not) be overloaded if a model wants to do initialization steps at the
beginning of the program or at the beginning of each time step. Examples are
models that need to call an external program to obtain plate velocities for the
current time, or from historical records, in which case it is far cheaper to do
so only once at the beginning of the time step than for every boundary point
separately. See, for example, the 
\href{doc/doxygen/classaspect_1_1VelocityBoundaryConditions_1_1GPlates.html}{aspect::VelocityBoundaryConditions::GPlates}
class.

The remaining functions are obvious, and are also
discussed in the documentation of this interface class at
\href{doc/doxygen/classaspect_1_1VelocityBoundaryConditions_1_1Interface.html}{aspect::VelocityBoundaryConditions::Interface}.
The purpose
of the last two functions has been discussed in the general overview of
plugins above.


\subsubsection{Temperature boundary conditions}
\label{sec:temperature-boundary-conditions}

\index[prmindex]{Fixed temperature boundary indicators}
\index[prmindexfull]{Model settings!Fixed temperature boundary indicators}
The boundary conditions are responsible for describing the temperature values
at those parts of the boundary at which the temperature is fixed (see
Section~\ref{sec:geometry-models} for how it is determined which parts of the
boundary this applies to).

To implement a new boundary conditions model, you
need to overload the
\href{doc/doxygen/classaspect_1_1BoundaryTemperature_1_1Interface.html}{aspect::BoundaryTemperature::Interface}
class and use
the \texttt{ASPECT\_REGISTER\_BOUNDARY\_TEMPERATURE\_MODEL} macro to register your new
class. The implementation of the new class should be in namespace
\texttt{aspect::BoundaryTemperature}.

Specifically, your new class needs to implement the following basic interface:
\begin{lstlisting}[frame=single,language=C++]
    template <int dim>
    class aspect::BoundaryTemperature::Interface
    {
      public:
        virtual
        double
        temperature (const GeometryModel::Interface<dim> &geometry_model,
                     const unsigned int                   boundary_indicator,
                     const Point<dim>                    &location) const = 0;

        virtual
        double minimal_temperature () const = 0;

        virtual
        double maximal_temperature () const = 0;

        static
        void
        declare_parameters (ParameterHandler &prm);

        virtual
        void
        parse_parameters (ParameterHandler &prm);
    };
\end{lstlisting}
The first of these functions needs to provide the fixed temperature at the
given point. The geometry model and the boundary indicator of the particular
piece of boundary on which the point is located is also given as a hint in
determining where this point may be located; this may, for example, be used to
determine if a point is on the inner or outer boundary of a spherical
shell. The remaining functions are obvious, and are also
discussed in the documentation of this interface class at
\href{doc/doxygen/classaspect_1_1BoundaryTemperature_1_1Interface.html}{aspect::BoundaryTemperature::Interface}. The
purpose
of the last two functions has been discussed in the general overview of
plugins above.


\subsubsection{Postprocessors: Evaluating the solution after each time step}
\label{sec:postprocessors}

\index[prmindex]{List of postprocessors}
\index[prmindexfull]{Postprocess!List of postprocessors}
Postprocessors are arguably the most complex and powerful of the plugins
available in \aspect{} since they do not only passively provide any
information but can actually compute quantities derived from the
solution. They are executed once at the end of each time step and,
unlike all the other plugins discussed above, there can be an arbitrary number
of active postprocessors in the same program (for the plugins discussed in
previous sections it was clear that there is always exactly one material
model, geometry model, etc.).

\paragraph{Motivation.}
The original motivation for postprocessors is that the goal of a simulation is
of course not the simulation itself, but that we want to do something with the
solution. Examples for already existing postprocessors are:
\begin{itemize}
\item Generating output in file formats that are understood by visualization
  programs. This is facilitated by the
  \href{doc/doxygen/classaspect_1_1Postprocess_1_1Visualization.html}{aspect::Postprocess::Visualization}
  class and a separate class of visualization postprocessors, see
  Section~\ref{sec:viz-postpostprocessors}.
\item Computing statistics about the velocity field (e.g., computing minimal,
  maximal, and average velocities), temperature field (minimal, maximal, and
  average temperatures), or about the heat fluxes across boundaries of the
  domain. This is provided by the
  \href{doc/doxygen/classaspect_1_1Postprocess_1_1VelocityStatistics.html}{aspect::Postprocess::VelocityStatistics},
  \href{doc/doxygen/classaspect_1_1Postprocess_1_1TemperatureStatistics.html}{aspect::Postprocess::TemperatureStatistics},
  \href{doc/doxygen/classaspect_1_1Postprocess_1_1HeatFluxStatistics.html}{aspect::Postprocess::HeatFluxStatistics}
  classes, respectively.
\end{itemize}
Since writing this text, there may have been other additions as well.

However, postprocessors can be more powerful than this. For example, while the
ones listed above are by and large stateless, i.e., they do not carry
information from one invocation at one timestep to the next invocation,%
\footnote{This is not entirely true. The visualization plugin keeps track of
  how many output files it has already generated, so that they can be numbered
  consecutively.}
there is nothing that prohibits postprocessors from doing so. For example, the
following ideas would fit nicely into the postprocessor framework:
\begin{itemize}
\item \textit{Passive tracers:} If one would like to follow the trajectory of
  material as it is advected along with the flow field, one technique is to
  use tracer particles. To implement this, one would start with an initial
  population of particles distributed in a certain way, for example close to
  the core-mantle boundary. At the end of each time step, one would then need
  to move them forward with the flow field by one time increment. As long as
  these particles do not affect the flow field (i.e., they do not carry any
  information that feeds into material properties; in other words, they are
  \textit{passive}), their location could well
  be stored in a postprocessor object and then be output in periodic intervals
  for visualization. In fact, such a passive tracer postprocessor is already
  available.

\item \textit{Surface or crustal processes:} Another possibility would be to keep track
  of surface or crustal processes induced by mantle flow. An example would be
  to keep track of the thermal history of a piece of crust by updating it
  every time step with the heat flux from the mantle below. One could also
  imagine integrating changes in the surface topography by considering the
  surface divergence of the surface velocity computed in the previous time
  step: if the surface divergence is positive, the topography is lowered,
  eventually forming a trench; if the divergence is negative, a mountain belt
  eventually forms.
\end{itemize}
In all of these cases, the essential limitation is that postprocessors are
\textit{passive}, i.e., that they do not affect the simulation but only
observe it.

\paragraph{The statistics file.}
Postprocessors fall into two categories: ones that produce lots of output
every time they run (e.g., the visualization postprocessor), and ones that
only produce one, two, or in any case a small and fixed number of often
numerical results (e.g., the postprocessors computing velocity, temperature,
or heat flux statistics). While the former are on their own in implementing
how they want to store their data to disk, there is a mechanism in place that
allows the latter class of postprocessors to store their data into a central
file that is updated at the end of each time step, after all postprocessors
are run.

To this end, the function that executes each of the postprocessors is given a
reference to a \texttt{dealii::TableHandler} object that allows to store data
in named columns, with one row for each time step. This table is then stored
in the \texttt{statistics} file in the directory designated for output in the
input parameter file. It allows for easy visualization of trends over all time
steps. To see how to put data into this statistics object, take a look at the
existing postprocessor objects.

Note that the data deposited into the statistics object need not be numeric in
type, though it often is. An example of text-based entries in this table is
the visualization class that stores the name of the graphical output file
written in a particular time step.

\paragraph{Implementing a postprocessor.}
Ultimately, implementing a new postprocessor is no different than any of the
other plugins. Specifically, you'll have to write a class that
overloads the
\href{doc/doxygen/classaspect_1_1Postprocess_1_1Interface.html}{aspect::Postprocess::Interface}
base class and use
the \texttt{ASPECT\_REGISTER\_POSTPROCESSOR} macro to register your new
class. The implementation of the new class should be in namespace
\texttt{aspect::Postprocess}.

In reality, however, implementing new postprocessors is often more
difficult. Primarily, this difficulty results from two facts:
\begin{itemize}
\item Postprocessors are not self-contained (only providing information) but
  in fact need to access the solution of the model at each time step. That is,
  of course, the purpose of postprocessors, but it requires that the writer of
  a plugin has a certain amount of knowledge of how the solution is computed
  by the main \texttt{Simulator} class, and how it is represented in data
  structures. To alleviate this somewhat, and to insulate the two worlds from
  each other, postprocessors do not directly access the data structures of the
  simulator class. Rather, in addition to deriving from the
  \href{doc/doxygen/classaspect_1_1Postprocess_1_1Interface.html}{aspect::Postprocess::Interface}
  base class, postprocessors also
  derive from the \href{doc/doxygen/classaspect_1_1SimulatorAccess.html}{SimulatorAccess} class that
  has a number of member functions postprocessors can call to obtain read-only
  access to some of the information stored in the main class of \aspect{}. See
  \href{doc/doxygen/classaspect_1_1SimulatorAccess.html}{the
    documentation of this class} to see what kind of information is available to
  postprocessors. See also Section~\ref{sec:plugins} for more information
  about the \texttt{SimulatorAccess} class.

\item Writing a new postprocessor typically
  requires a fair amount of knowledge how to leverage the \dealii{} library to
  extract information from the solution. The existing postprocessors are
  certainly good examples to start from in trying to understand how to do this.
\end{itemize}

Given these comments, the interface a postprocessor class has to implement is
rather basic:
\begin{lstlisting}[frame=single,language=C++]
    template <int dim>
    class aspect::Postprocess::Interface
    {
      public:
        virtual
        std::pair<std::string,std::string>
        execute (TableHandler &statistics) = 0;

        virtual
        void
        save (std::map<std::string, std::string> &status_strings) const;

        virtual
        void
        load (const std::map<std::string, std::string> &status_strings);

        static
        void
        declare_parameters (ParameterHandler &prm);

        virtual
        void
        parse_parameters (ParameterHandler &prm);
    };
\end{lstlisting}
The purpose of these functions is described in detail in the documentation of
the
\href{doc/doxygen/classaspect_1_1Postprocess_1_1Interface.html}{aspect::Postprocess::Interface}
class. While the first one is responsible for evaluating the solution at the
end of a time step, the \texttt{save/load} functions are used in checkpointing
the program and restarting it at a previously saved point during the
simulation. The first of these functions therefore needs to store the status
of the object as a string under a unique key in the database described by the
argument, while the latter function restores the same state as before by
looking up the status string under the same key. The default implementation of
these functions is to do nothing; postprocessors that do have non-static
member variables that contain a state need to overload these functions.

There are numerous postprocessors already implemented. If you want to
implement a new one, it would be helpful to look at the existing ones to see
how they implement their functionality.

\paragraph{Postprocessors and checkpoint/restart.} Postprocessors have
\texttt{save()} and \texttt{load()} functions that are used to write the data
a postprocessor has into a checkpoint file, and to load it again upon
restart. This is important since many postprocessors store some state -- say,
a temporal average over all the time steps seen so far, or the number of the
last graphical output file generated so that we know how the next one needs
to be numbered.

The typical case is that this state is the same across all processors of a
parallel computation. Consequently, what \aspect{} writes into the checkpoint
file is only the state obtained from the postprocessors on processor 0 of a
parallel computation. On restart, all processors read from the same file and
the postprocessors on \textit{all} processors will be initialized by what the
same postprocessor on processor 0 wrote.

There are situations where postprocessors do in fact store complementary
information on different processors. At the time of writing this, one example
is the postprocessor that supports advecting passive particles along the
velocity field: on every processor, it handles only those particles that lie
inside the part of the domain that is owned by this MPI rank. The
serialization approach outlined above can not work in this case, for obvious
reasons. In cases like this, one needs to implement the \texttt{save()} and
\texttt{load()} differently than usual: one needs to put all variables that
are common across processors into the maps of string as usual, but one then
also needs to save all state that is different across processors, from all
processors. There are two ways: If the amount of data is small, you can use
MPI communications to send the state of all processors to processor zero, and
have processor zero store it in the result so that it gets written into the
checkpoint file; in the \texttt{load()} function, you will then have to
identify which part of the text written by processor 0 is relevant to the
current processor. Or, if your postprocessor stores a large amount of data, you
may want to open a restart file specifically for this postprocessor, use MPI
I/O or other ways to write into it, and do the reverse operation in
\texttt{load()}.

Note that this approach requires that \aspect{} actually calls the
\texttt{save()} function on all processors. This in fact happens -- though
\aspect{} also discards the result on all but processor zero.


\subsubsection{Visualization postprocessors}
\label{sec:viz-postpostprocessors}

\index[prmindex]{List of output variables}
\index[prmindexfull]{Postprocess!Visualization!List of output variables}
As mentioned in the previous section, one of the postprocessors that are
already implemented in \aspect{} is the \href{doc/doxygen/classaspect_1_1Postprocess_1_1Visualization.html}{aspect::Postprocess::Visualization}
class that takes the solution and outputs it as a collection of files that can
then be visualized graphically, see Section~\ref{sec:viz}. The question is
which variables to output: the solution of the basic equations we solve here
is characterized by the velocity, pressure and temperature; on the other hand,
we are frequently interested in derived, spatially and temporally variable
quantities such as the viscosity for the actual pressure, temperature and
strain rate at a given location, or seismic wave speeds.

\aspect{} already implements a good number of such derived quantities that one
may want to visualize. On the other hand, always outputting \textit{all} of
them would yield very large output files, and would furthermore not scale very
well as the list continues to grow. Consequently, as with the postprocessors
described in the previous section, what \textit{can} be computed is
implemented in a number of plugins and what \textit{is} computed is selected
in the input parameter file (see
Section~\ref{parameters:Postprocess/Visualization}).

Defining visualization postprocessors works in much the same way as for the
other plugins discussed in this section. Specifically, an implementation of
such a plugin needs to be a class that derives from interface classes,
should by convention be in namespace
\texttt{aspect::Postprocess::VisualizationPostprocessors},
and is registered using a macro, here called
\texttt{ASPECT\_REGISTER\_VISUALIZATION\_POSTPROCESSOR}. Like the
postprocessor plugins, visualization postprocessors can derive from class
\href{doc/doxygen/classaspect_1_1Postprocess_1_1SimulatorAccess.html}{aspect::Postprocess::SimulatorAccess} if they need to know specifics
of the simulation such as access to the material models and to get
access to the introspection facility outlined in Section~\ref{sec:plugins}. A typical example is
the plugin that produces the viscosity as a spatially variable field by
evaluating the viscosity function of the material model using the pressure,
temperature and location of each visualization point (implemented in the
\texttt{aspect::Postprocess::VisualizationPostprocessors::Viscosity}
class). On the other hand, a hypothetical plugin
that
simply outputs the norm of the strain rate $\sqrt{\varepsilon(\mathbf
  u):\varepsilon(\mathbf u)}$ would not need access to anything but the
solution vector (which the plugin's main function is given as an argument)
and consequently is not derived from the
\href{doc/doxygen/classaspect_1_1Postprocess_1_1SimulatorAccess.html}{aspect::Postprocess::SimulatorAccess}
class.%
\footnote{The actual plugin
  \texttt{aspect::Postprocess::VisualizationPostprocessors::StrainRate}
  only computes $\sqrt{\varepsilon(\mathbf
    u):\varepsilon(\mathbf u)}$ in the incompressible case. In the compressible
  case, it computes
  $\sqrt{[\varepsilon(\mathbf u)-\tfrac 13(\textrm{tr}\;\varepsilon(\mathbf
    u))\mathbf I]:[\varepsilon(\mathbf u)-\tfrac
    13(\textrm{tr}\;\varepsilon(\mathbf u))\mathbf I]}$ instead. To test whether
  the model is compressible or not, the plugin needs access to the material
  model object, which the class gains by deriving from
  \href{doc/doxygen/classaspect_1_1Postprocess_1_1SimulatorAccess.html}{aspect::Postprocess::SimulatorAccess}
  and then calling \texttt{this->get\_material\_model().is\_compressible()}.}

Visualization plugins can come in two flavors:
\begin{itemize}
  \item \textit{Plugins that compute things from the solution in a pointwise way:}
   The classes in this group are derived not only from the respective interface class (and possibly
   the \href{doc/doxygen/classaspect_1_1SimulatorAccess.html}{SimulatorAccess} class) but also from the deal.II class
   \texttt{DataPostprocessor} or any of
   the classes like \texttt{DataPostprocessorScalar} or \texttt{DataPostprocessorVector}.
   These classes can be thought of as filters: DataOut will call a function in
   them for every cell and this function will transform the values or gradients
   of the solution and other information such as the location of quadrature
   points into the desired quantity to output. A typical case would be
   if the quantity $g(x)$ you want to output can be written as a function
   $g(x) = G(u(x),\nabla u(x), x, ...)$ in a pointwise sense where $u(x)$
   is the value of the solution vector (i.e., the velocities, pressure,
   temperature, etc) at an evaluation point. In the context
   of this program an example would be to output the density of the medium as
   a spatially variable function since this is a quantity that for realistic
   media depends pointwise on the values of the solution.

To sum this, slightly confusing multiple inheritance up, visualization
postprocessors do the following:
\begin{itemize}
\item If necessary, they derive from
  \href{doc/doxygen/classaspect_1_1Postprocess_1_1SimulatorAccess.html}{aspect::Postprocess::SimulatorAccess}.
\item They derive from
  \href{doc/doxygen/classaspect_1_1Postprocess_1_1VisualizationPostprocessors_1_1Interface.html}{aspect::Postprocess::VisualizationPostprocessors::Interface}. The
  functions of this interface class are all already implemented as doing
  nothing in the base class but can be overridden in a plugin. Specifically,
  the following functions exist:
  \begin{lstlisting}[frame=single,language=C++]
    class Interface
    {
      public:
        static
        void
        declare_parameters (ParameterHandler &prm);

        virtual
        void
        parse_parameters (ParameterHandler &prm);

        virtual
        void save (std::map<std::string, std::string> &status_strings) const;

        virtual
        void load (const std::map<std::string, std::string> &status_strings);
    };
  \end{lstlisting}

\item They derive from either the \texttt{dealii::DataPostprocessor} class,
  or the simpler to use \texttt{dealii::DataPostprocessorScalar}
  or \texttt{dealii::DataPostprocessorVector} classes. For example, to derive
  from the second of these classes, the following interface functions has to be
  implemented:
  \begin{lstlisting}[frame=single,language=C++]
    class dealii::DataPostprocessorScalar
    {
      public:
        virtual
        void
        compute_derived_quantities_vector
          (const std::vector<Vector<double> >              &uh,
           const std::vector<std::vector<Tensor<1,dim> > > &duh,
           const std::vector<std::vector<Tensor<2,dim> > > &dduh,
           const std::vector<Point<dim> >                  &normals,
           const std::vector<Point<dim> >                  &evaluation_points,
           std::vector<Vector<double> >                    &computed_quantities) const;
    };
  \end{lstlisting}
  What this function does is described in detail in the deal.II
  documentation. In addition, one has to write a suitable constructor to call
  \texttt{dealii::DataPostprocessorScalar::DataPostprocessorScalar}.
\end{itemize}

  \item \textit{Plugins that compute things from the solution in a cellwise way:}
   The second possibility is for a class to not derive from
   \texttt{dealii::DataPostprocessor} but instead from the
   \href{doc/doxygen/classaspect_1_1Postprocess_1_1VisualizationPostprocessors_1_1CellDataVectorCreator.html}{aspect::Postprocess::VisualizationPostprocessors::CellDataVectorCreator}
   class. In this case, a visualization postprocessor would generate
   and return a vector that consists of one element per cell. The
   intent of this option is to output quantities that are not pointwise
   functions of the solution but instead can only be computed as
   integrals or other functionals on a per-cell basis. A typical
   case would be error estimators that do depend on the solution but
   not in a pointwise sense; rather, they yield one value per cell of
   the mesh. See the documentation of the
   \texttt{CellDataVectorCreator} class
   for more information.
\end{itemize}


If all of this sounds confusing, we recommend consulting the implementation of
the various visualization plugins that already exist in the \aspect{} sources,
and using them as a template.


\subsubsection{Mesh refinement criteria}
\label{sec:mesh-refinement-criteria}

\index[prmindex]{Mesh refinement}
\index[prmindexfull]{Mesh refinement}

Despite research since the mid-1980s, it isn't completely clear how to refine
meshes for complex situations like the ones modeled by \aspect{}. The basic
problem is that mesh refinement criteria either can refine based on some
variable such as the temperature, the pressure, the velocity, or a compositional
field, but that oftentimes this by itself is not quite what one wants. For
example, we know that Earth has discontinuities, e.g., at 440km and 610km depth.
In these places, densities and other material properties suddenly change. Their
resolution in computation models is important as we know that they affect
convection patterns. At the same time, there is only a small effect on the
primary variables in a computation -- maybe a jump in the second or third
derivative, for example, but not a discontinuity that would be clear to see. As
a consequence, automatic refinement criteria do not always refine these
interfaces as well as necessary.

To alleviate this, \aspect{} has plugins for mesh refinement. Through the
parameters in Section~\ref{parameters:Mesh_20refinement}, one can select when to
refine but also which refinement criteria should be used and how they should be
combined if multiple refinement criteria are selected. Furthermore, through the
usual plugin mechanism, one can extend the list of available mesh refinement
criteria (see the parameter ``Strategy'' in
Section~\ref{parameters:Mesh_20refinement}).
\index[prmindex]{Strategy}
\index[prmindexfull]{Mesh refinement!Strategy}
Each such plugin is responsible for producing a vector of values (one per
active cell on the current processor, though only those values for cells that
the current processor owns are used) with an indicator of how badly this cell
needs to be refined: large values mean that the cell should be refined, small
values that the cell may be coarsened away.

To implement a new mesh refinement criterion, you
need to overload the
\href{doc/doxygen/classaspect_1_1MeshRefinement_1_1Interface.html}{aspect::MeshRefinement::Interface}
class and use
the \texttt{ASPECT\_REGISTER\_MESH\_REFINEMENT\_CRITERION} macro to register
your new class. The implementation of the new class should be in namespace
\texttt{aspect::MeshRefinement}.

Specifically, your new class needs to implement the following basic interface:
\begin{lstlisting}[frame=single,language=C++]
    template <int dim>
    class aspect::MeshRefinement::Interface
    {
      public:
        virtual
        void
        execute (Vector<float> &error_indicators) const = 0;

        static
        void
        declare_parameters (ParameterHandler &prm);

        virtual
        void
        parse_parameters (ParameterHandler &prm);
    };
\end{lstlisting}
The first of these functions computes the set of refinement criteria (one per
cell) and returns it in the given argument. Typical examples can be found in the
existing implementations in the \texttt{source/mesh\_refinement} directory. As usual, your termination
criterion implementation will likely need to be derived from the
\texttt{SimulatorAccess} to get access to the current state of the simulation.

The
remaining functions are obvious, and are also discussed in the documentation of this interface class at \href{doc/doxygen/classaspect_1_1MeshRefinement_1_1Interface.html}{aspect::MeshRefinement::Interface}.
The purpose
of the last two functions has been discussed in the general overview of
plugins above.



\subsubsection{Criteria for terminating a simulation}
\label{sec:terminators}

\index[prmindex]{Termination criteria}
\index[prmindexfull]{Termination criteria}

\aspect{} allows for different ways of terminating a simulation. For example,
the simulation may have reached a final time specified in the input file.
However, it also allows for ways to terminate a simulation when it has reached a
steady state (or, rather, some criterion determines that it is close enough to
steady state), or by an external action such as placing a specially named file
in the output directory. The criteria
determining termination of a simulation are all implemented in plugins. The
parameters describing these criteria are listed in
Section~\ref{parameters:Termination_20criteria}.

To implement a termination criterion, you
need to overload the
\href{doc/doxygen/classaspect_1_1TerminationCriteria_1_1Interface.html}{aspect::TerminationCriteria::Interface}
class and use
the \texttt{ASPECT\_REGISTER\_TERMINATION\_CRITERION} macro to register
your new class. The implementation of the new class should be in namespace
\texttt{aspect::TerminationCriteria}.

Specifically, your new class needs to implement the following basic interface:
\begin{lstlisting}[frame=single,language=C++]
    template <int dim>
    class aspect::TerminationCriteria::Interface
    {
      public:
        virtual
        bool
        execute () const = 0;

        static
        void
        declare_parameters (ParameterHandler &prm);

        virtual
        void
        parse_parameters (ParameterHandler &prm);
    };
\end{lstlisting}
The first of these functions returns a value that indicates whether the
simulation should be terminated.
Typical examples can be found in the existing implementations in the
\texttt{source/termination\_criteria} directory. As usual, your termination
criterion implementation will likely need to be derived from the
\texttt{SimulatorAccess} to get access to the current state of the simulation.

The remaining functions are
obvious, and are also discussed in the documentation of this interface class at
\href{doc/doxygen/classaspect_1_1TerminationCriteria_1_1Interface.html}{aspect::TerminationCriteria::Interface}.
The purpose
of the last two functions has been discussed in the general overview of
plugins above.



\subsection{Extending \aspect{} through the signals mechanism}
\label{sec:extending-signals}

Not all things you may want to do fit neatly into the list of plugins of the
previous sections. Rather, there are cases where you may want to change things
that are more of the one-off kind and that require code that is at a lower level
and requires more knowledge about \aspect{}'s internal workings. For such
changes, we still want to stick with the general principle outlined at the
beginning of Section~\ref{sec:extending}: You should be able to make all of your
changes and extensions in your own files, without having to modify \aspect{}'s
own sources.

To support this, \aspect{} uses a ``signals'' mechanism. Signals are, in
essence, objects that represent \textit{events}, for example the fact that the
solver has finished a time step. The core of \aspect{} defines a number of such
signals, and \textit{triggers} them at the appropriate points. The idea of
signals is now that you can \textit{connect} to them: you can tell the signal
that it should call a particular function every time the signal is triggered.
The functions that are connected to a signal are called ``slots'' in common
diction. One, several, or no slots may be connected to each signal.

There are two kinds of signals that \aspect{} provides:
\begin{itemize}
  \item Signals that are triggered at startup of the program: These are, in
  essence, signals that live in some kind of global scope. Examples are signals
  that declare additional parameters for use in input files, or that read the
  values of these parameters from a \texttt{ParameterHandler} object. These
  signals are static member variables of the structure that contains them and
  consequently exist only once for the entire program.
  \item Signals that reference specific events that happen inside a simulator
  object. These are regular member variables of the structure that contains
  them, and because each simulator object has such a structure, the signals
  exist once per simulator object. (Which in practice is only once per program,
  of course.)
\end{itemize}
For both of these kinds, a user-written plugin file can (but does not need) to
register functions that connect functions in this file (i.e., slots) to their
respective signals.

In the first case, code that registers slots with global signals would look like
this:
\begin{lstlisting}[frame=single,language=C++]
// A function that will be called at the time when parameters are declared.
// It receives the dimension in which ASPECT will be run as the first argument,
// and the ParameterHandler object that holds the runtime parameter
// declarations as second argument.
void declare_parameters(const unsigned int dim,
                        ParameterHandler &prm)
{
  prm.declare_entry("My parameter", ...);
}


// The same for parsing parameters. 'my_parameter' is a parameter
// that stores something we want to read from the input file
// and use in other functions in this file (which we don't show here).
// For simplicity, we assume that it is an integer.
//
// The function also receives a first argument that contains all
// of the other (already parsed) arguments of the simulation, in
// case what you want to do here wants to refer to other parameters.
int my_parameter;

template <int dim>
void parse_parameters(const Parameters<dim> &parameters,
                      ParameterHandler &prm)
{
  my_parameter = prm.get_integer ("My parameter");
}


// Now have a function that connects slots (i.e., the two functions
// above) to the static signals. Do this for both the 2d and 3d
// case for generality.
void parameter_connector ()
{
  SimulatorSignals<2>::declare_additional_parameters.connect (&declare_parameters);
  SimulatorSignals<3>::declare_additional_parameters.connect (&declare_parameters);

  SimulatorSignals<2>::parse_additional_parameters.connect (&parse_parameters<2>);
  SimulatorSignals<3>::parse_additional_parameters.connect (&parse_parameters<3>);
}


// Finally register the connector function above to make sure it gets run
// whenever we load a user plugin that is mentioned among the additional
// shared libraries in the input file:
ASPECT_REGISTER_SIGNALS_PARAMETER_CONNECTOR(parameter_connector)
 \end{lstlisting}

The second kind of signal can be connected to once a simulator object has been
created. As above, one needs to define the slots, define a connector function,
and register the connector function. The following gives an example:
\begin{lstlisting}[frame=single,language=C++]
// A function that is called at the end of creating the current constraints
// on degrees of freedom (i.e., the constraints that describe, for example,
// hanging nodes, boundary conditions, etc).
template <int dim>
void post_constraints_creation (const SimulatorAccess<dim> &simulator_access,
                                ConstraintMatrix &current_constraints)
{
  ...; // do whatever you want to do here
}


// A function that is called from the simulator object and that can connect
// a slot (such as the function above) to any of the signals declared in the
// structure passed as argument:
template <int dim>
void signal_connector (SimulatorSignals<dim> &signals)
{
  signals.post_constraints_creation.connect (&post_constraints_creation<dim>);
}


// Finally register the connector function so that it is called whenever
// a simulator object has been set up. For technical reasons, we need to
// register both 2d and 3d versions of this function:
ASPECT_REGISTER_SIGNALS_CONNECTOR(signal_connector<2>,
                                  signal_connector<3>)
\end{lstlisting}

As mentioned above, each signal may be connected to zero, one, or many slots.
Consequently, you could have multiple plugins each of which connect to the same
slot, or the connector function above may just connect multiple slots (i.e.,
functions in your program) to the same signal.

So what could one do in a place like this? One option would be to just monitor
what is going on, e.g., in code like this that simply outputs into the
statistics file (see Section~\ref{sec:viz-stat}):
\begin{lstlisting}[frame=single,language=C++]
template <int dim>
void post_constraints_creation (const SimulatorAccess<dim> &simulator_access,
                                ConstraintMatrix &current_constraints)
{
  simulator_access.get_statistics_object()
    .add_value ("number of constraints",
                current_constraints.n_constraints());
}
\end{lstlisting}
This will produce, for every time step (because this is how often the signal is
called) an entry in a new column in the statistics file that records the number
of constraints. On the other hand, it is equally possible to also modify the
constraints object at this point. An application would be if you wanted to run a
simulation where you prescribe the velocity in a part of the domain, e.g., for a
subducting slab (see Section \ref{sec:prescribed-velocities}).

Signals exist for various waypoints in a simulation and you can consequently
monitor and change what is happening inside a simulation by connecting your own
functions to these signals. It would be pointless to list here what signals
actually exist -- simply refer to the documentation of the
\href{doc/doxygen/structaspect_1_1SimulatorSignals.html}{SimulatorSignals
class} for a complete list of signals you can connect to.

As a final note, it is generally true that writing functions that can connect to
signals require significantly more internal knowledge of the workings of
\aspect{} than writing plugins through the mechanisms outlined above. It also
allows to affect the course of a simulation by working on the internal data
structures of \aspect{} in ways that are not available to the largely
passive and reactive plugins discussed in previous sections. With this
obviously also comes the potential for trouble. On the other hand, it also
allows to do things with \aspect{} that were not initially intended by the
authors, and that would be hard or impossible to implement through plugins. An
example would be to couple different codes by exchanging details of the internal
data structures, or even update the solution vectors using information received
from another code.


\note{Chances are that if you think about using the signal mechanism, there is
not yet a signal that is triggered at exactly the point where you need it. Consequently,
you will be tempted to just put your code into the place where it fits inside
\aspect{} where it fits best. This is poor practice: it prevents you from
upgrading to a newer version of \aspect{} at a later time because this would
overwrite the code you inserted.

Rather, a more productive approach would be to either define a new signal that
is triggered where you need it, and connect a function (slot) in your own plugin
file to this signal using the mechanisms outlined above. Then send the code that
defines and triggers the signal to the developers of \aspect{} to make sure that
it is also included in the next release. Alternatively, you can also simply ask
on the mailing lists for someone to add such a signal in the place where you
want it. Either way, adding signals is something that is easy to do, and we would 
much rather add signals than have people who modify the \aspect{} source files
for their own needs and are then stuck on a particular version.}



\subsection{Extending the basic solver}
\label{sec:extending-solver}

The core functionality of the code, i.e., that part of the code that
implements the time stepping, assembles matrices, solves linear and nonlinear
systems, etc., is in the \texttt{aspect::Simulator} class (see the
\href{doc/doxygen/classaspect_1_1Simulator.html}{doxygen documentation of this
  class}). Since the implementation of this class has more than 3,000 lines of
code, it is split into several files that are all located in the
\texttt{source/simulator} directory. Specifically, functionality is split into
the following files:
\begin{itemize}
\item \texttt{source/simulator/core.cc}: This file contains the functions that
  drive the overall algorithm (in particular \texttt{Simulator::run}) through
  the main time stepping loop and the functions immediately called by
  \texttt{Simulator::run}.
\item \texttt{source/simulator/assembly.cc}: This is where all the functions
  are located that are related to assembling linear systems.
\item \texttt{source/simulator/solver.cc}: This file provides everything that
  has to do with solving and preconditioning the linear systems.
\item \texttt{source/simulator/initial\_conditions.cc}: The functions in this
  file deal with setting initial conditions for all variables.
\item \texttt{source/simulator/checkpoint\_restart.cc}: The location of
  functionality related to saving the current state of the program to a set of
  files and restoring it from these files again.
\item \texttt{source/simulator/helper\_functions.cc}: This file contains a set
  of functions that do the odd thing in support of the rest of the simulator
  class.
\item \texttt{source/simulator/parameters.cc}: This is where we define and
  read run-time parameters that pertain to the top-level functionality of the
  program.
\end{itemize}

Obviously, if you want to extend this core functionality, it is useful to
first understand the numerical methods this class implements. To this end,
take a look at the paper that describes these methods, see
\cite{KHB12}. Further, there are two predecessor programs whose extensive
documentation is at a much higher level than the one typically found inside
\aspect{} itself, since they are meant to teach the basic components of
convection simulators as part of the \dealii{} tutorial:
\begin{itemize}
\item The step-31 program at
  \url{https://www.dealii.org/developer/doxygen/deal.II/step_31.html}: This
  program is the first version of a convection solver. It does not run in
  parallel, but it introduces many of the concepts relating to the time
  discretization, the linear solvers, etc.
\item The step-32 program at
  \url{https://www.dealii.org/developer/doxygen/deal.II/step_32.html}: This is
  a parallel version of the step-31 program that already solves on a spherical
  shell geometry. The focus of the documentation in this program is on the
  techniques necessary to make the program run in parallel, as well as some of
  the consequences of making things run with realistic geometries, material
  models, etc.
\end{itemize}
Neither of these two programs is nearly as modular as \aspect{}, but that was
also not the goal in creating them. They will, however, serve as good
introductions to the general approach for solving thermal convection problems.

\note{Neither this manual, nor the documentation in \aspect{} makes much of an
attempt at teaching how to use the \dealii{} library upon which \aspect{} is
built. Nevertheless, you will likely have to know at least the basics of
\dealii{} to successfully work on the \aspect{} code. We refer to the
resources listed at the beginning of this section as well as references
\cite{BHK07,BK99m}.}


\section{Future plans for \aspect}
\label{sec:future}

We have a number of near-term plans for \aspect{} that we hope to implement
soon:
\begin{itemize}
\item \textit{Iterating out the nonlinearity:} In the current version of
  \aspect{}, we use the velocity, pressure and temperature of the previous
  time step to evaluate the coefficients that appear in the flow equations
  \eqref{eq:stokes-1}--\eqref{eq:stokes-2}; and the velocity and pressure of
  the current time step as well as the previous time step's temperature to
  evaluate the coefficients in the temperature equation
  \eqref{eq:temperature}. This is an appropriate strategy if the model is not
  too nonlinear; however, it introduces inaccuracies and limits the size of
  the time step if coefficients strongly depend on the solution variables.

  To avoid this, one can iterate out the equations using either a fixed point
  or Newton scheme. Both approaches ensure that at the end of a time step, the
  values of coefficients and solution variables are consistent. On the other
  hand, one may have to solve the linear systems that describe a time step
  more than once, increasing the computational effort.

  We have started implementing such methods using a testbase code, based on
  earlier experiments by Jennifer Worthen \cite{Wor12}. We hope to implement
  this feature in \aspect{} early in 2012.

\item \textit{Faster 3d computations:} Whichever way you look at it, 3d
  computations are expensive. In parallel computations, the Stokes solve
  currently takes upward of 90\% of the overall wallclock time, suggesting an
  obvious target for improvements based on better algorithms as well as from
  profiling the code to find hot spots. In particular, playing with better
  solver and/or preconditioner options would seem to be a useful goal.

\item \textit{Particle-based methods:} It is often useful to employ particle
  tracers to visualize where material is being transported. While conceptually
  simple, their implementation is made difficult in parallel computations
  if particles cross the boundary between parts of the regions owned by
  individual processors, as well as during re-partitioning the mesh between
  processors following mesh refinement. Eric Heien is
  working on an implementation of such passive tracers.

\item \textit{More realistic material models:} The number of
  material models available in \aspect{} is currently relatively
  small. Obviously, how realistic a simulation is depends on how realistic a
  material model is. We hope to obtain descriptions of more realistic material
  descriptions over time, either given analytically or based on table-lookup
  of material properties.

\item \textit{Melting:} An important part of mantle behavior is
  melting. Melting not only affects the properties of the material such as
  density or viscosity, but it also leads to chemical segregation and, in
  fact, to the flow of two different fluids (the melt and the rock matrix)
  relative to each other. Modeling this additional process would yield
  significant insight.

\item \textit{Converting output into seismic velocities:} The predictions of
  mantle convection codes are often difficult to verify experimentally. On the
  other hand, simulations can be used to predict a seismic signature of the
  earth mantle -- for example the location of transition zones that can be
  observed using seismic imaging. To facilitate such comparisons, it is of
  interest to output not only the primary solution variables but also convert
  them into the primary quantity visible in seismic imaging: compressive and
  shear wave velocities. Implementing this should be relatively
  straightforward if given a formula or table that expresses velocities in terms of the
  variables computed by \aspect.
\end{itemize}

To end this section, let us repeat something already stated in the
introduction:

\note{\aspect{} is a community project. As such, we encourage contributions
  from the community to improve this code over time. Obvious candidates for
  such contributions are implementations of new plugins as discussed in
  Section~\ref{sec:plugins-concrete} since they are typically self-contained and do not
  require much knowledge of the details of the remaining code. Obviously,
  however, we also encourage contributions to the core functionality in any
  form!}


\section{Finding answers to more questions}

If you have questions that go beyond this manual, there are a number of
resources:
\begin{itemize}
\item For questions on the source code of \aspect{}, portability, installation,
  etc., use the \aspect{} development mailing list at
  \url{http://lists.geodynamics.org/cgi-bin/mailman/listinfo/aspect-devel}. This
  mailing list is where the \aspect{} developers all hang out.

\item \aspect{} is primarily based on the deal.II library (the dependency
  on Trilinos and p4est is primarily through deal.II, and not directly
  visible in the \aspect{} source code). If you have particular questions
  about deal.II, contact
  the mailing lists described at \url{https://www.dealii.org/mail.html}.

\item In case of more general questions about mantle convection, you can
  contact the CIG mantle
  convection mailing lists at 
  \url{http://lists.geodynamics.org/cgi-bin/mailman/listinfo/cig-MC}.

\item If you have specific questions about \aspect{} that are not suitable
  for public and archived mailing lists, you can contact the
  primary developers:
  \begin{itemize}
  \item Wolfgang
    Bangerth: \url{bangerth@math.tamu.edu}.
  \item Timo
    Heister: \url{heister@clemson.edu}.
  \end{itemize}
\end{itemize}


\appendix

\section{Run-time input parameters}
\label{sec:parameters}

% now include a file that describes all currently available run-time parameters
\subsection{Global parameters}
\label{parameters:global}


\begin{itemize}
\item {\it Parameter name:} {\tt Additional shared libraries}
\phantomsection\label{parameters:Additional shared libraries}


\index[prmindex]{Additional shared libraries}
\index[prmindexfull]{Additional shared libraries}
{\it Value:} 


{\it Default:} 


{\it Description:} A list of names of additional shared libraries that should be loaded upon starting up the program. The names of these files can contain absolute or relative paths (relative to the directory in which you call ASPECT). In fact, file names that are do not contain any directory information (i.e., only the name of a file such as $<$myplugin.so$>$ will not be found if they are not located in one of the directories listed in the \texttt{LD\_LIBRARY\_PATH} environment variable. In order to load a library in the current directory, use $<$./myplugin.so$>$ instead.

The typical use of this parameter is so that you can implement additional plugins in your own directories, rather than in the ASPECT source directories. You can then simply compile these plugins into a shared library without having to re-compile all of ASPECT. See the section of the manual discussing writing extensions for more information on how to compile additional files into a shared library.


{\it Possible values:} [List list of [FileName (Type: input)] of length 0...4294967295 (inclusive)]
\item {\it Parameter name:} {\tt Adiabatic surface temperature}
\phantomsection\label{parameters:Adiabatic surface temperature}


\index[prmindex]{Adiabatic surface temperature}
\index[prmindexfull]{Adiabatic surface temperature}
{\it Value:} 0


{\it Default:} 0


{\it Description:} In order to make the problem in the first time step easier to solve, we need a reasonable guess for the temperature and pressure. To obtain it, we use an adiabatic pressure and temperature field. This parameter describes what the `adiabatic' temperature would be at the surface of the domain (i.e. at depth zero). Note that this value need not coincide with the boundary condition posed at this point. Rather, the boundary condition may differ significantly from the adiabatic value, and then typically induce a thermal boundary layer.

For more information, see the section in the manual that discusses the general mathematical model.


{\it Possible values:} [Double -1.79769e+308...1.79769e+308 (inclusive)]
\item {\it Parameter name:} {\tt CFL number}
\phantomsection\label{parameters:CFL number}


\index[prmindex]{CFL number}
\index[prmindexfull]{CFL number}
{\it Value:} 1.0


{\it Default:} 1.0


{\it Description:} In computations, the time step $k$ is chosen according to $k = c \min_K \frac {h_K} {\|u\|_{\infty,K} p_T}$ where $h_K$ is the diameter of cell $K$, and the denominator is the maximal magnitude of the velocity on cell $K$ times the polynomial degree $p_T$ of the temperature discretization. The dimensionless constant $c$ is called the CFL number in this program. For time discretizations that have explicit components, $c$ must be less than a constant that depends on the details of the time discretization and that is no larger than one. On the other hand, for implicit discretizations such as the one chosen here, one can choose the time step as large as one wants (in particular, one can choose $c>1$) though a CFL number significantly larger than one will yield rather diffusive solutions. Units: None.


{\it Possible values:} [Double 0...1.79769e+308 (inclusive)]
\item {\it Parameter name:} {\tt Composition solver tolerance}
\phantomsection\label{parameters:Composition solver tolerance}


\index[prmindex]{Composition solver tolerance}
\index[prmindexfull]{Composition solver tolerance}
{\it Value:} 1e-12


{\it Default:} 1e-12


{\it Description:} The relative tolerance up to which the linear system for the composition system gets solved. See 'linear solver tolerance' for more details.


{\it Possible values:} [Double 0...1 (inclusive)]
\item {\it Parameter name:} {\tt Dimension}
\phantomsection\label{parameters:Dimension}


\index[prmindex]{Dimension}
\index[prmindexfull]{Dimension}
{\it Value:} 2


{\it Default:} 2


{\it Description:} The number of space dimensions you want to run this program in. ASPECT can run in 2 and 3 space dimensions.


{\it Possible values:} [Integer range 2...4 (inclusive)]
\item {\it Parameter name:} {\tt End time}
\phantomsection\label{parameters:End time}


\index[prmindex]{End time}
\index[prmindexfull]{End time}
{\it Value:} 5.69e+300


{\it Default:} 5.69e+300


{\it Description:} The end time of the simulation. The default value is a number so that when converted from years to seconds it is approximately equal to the largest number representable in floating point arithmetic. For all practical purposes, this equals infinity. Units: Years if the 'Use years in output instead of seconds' parameter is set; seconds otherwise.


{\it Possible values:} [Double -1.79769e+308...1.79769e+308 (inclusive)]
\item {\it Parameter name:} {\tt Linear solver A block tolerance}
\phantomsection\label{parameters:Linear solver A block tolerance}


\index[prmindex]{Linear solver A block tolerance}
\index[prmindexfull]{Linear solver A block tolerance}
{\it Value:} 1e-2


{\it Default:} 1e-2


{\it Description:} A relative tolerance up to which the approximate inverse of the A block of the Stokes system is computed. This approximate A is used in the preconditioning used in the GMRES solver.


{\it Possible values:} [Double 0...1 (inclusive)]
\item {\it Parameter name:} {\tt Linear solver S block tolerance}
\phantomsection\label{parameters:Linear solver S block tolerance}


\index[prmindex]{Linear solver S block tolerance}
\index[prmindexfull]{Linear solver S block tolerance}
{\it Value:} 1e-6


{\it Default:} 1e-6


{\it Description:} A relative tolerance up to which the approximate inverse of the S block (Schur complement matrix, $S = BA^{-1}B^{T}$) of the Stokes system is computed. This approximate inverse of the S block is used in the preconditioning used in the GMRES solver.


{\it Possible values:} [Double 0...1 (inclusive)]
\item {\it Parameter name:} {\tt Linear solver tolerance}
\phantomsection\label{parameters:Linear solver tolerance}


\index[prmindex]{Linear solver tolerance}
\index[prmindexfull]{Linear solver tolerance}
{\it Value:} 1e-7


{\it Default:} 1e-7


{\it Description:} A relative tolerance up to which the linear Stokes systems in each time or nonlinear step should be solved. The absolute tolerance will then be $\| M x_0 - F \| \cdot \text{tol}$, where $x_0 = (0,p_0)$ is the initial guess of the pressure, $M$ is the system matrix, F is the right-hand side, and tol is the parameter specified here. We include the initial guess of the pressure to remove the dependency of the tolerance on the static pressure. A given tolerance value of 1 would mean that a zero solution vector is an acceptable solution since in that case the norm of the residual of the linear system equals the norm of the right hand side. A given tolerance of 0 would mean that the linear system has to be solved exactly, since this is the only way to obtain a zero residual.

In practice, you should choose the value of this parameter to be so that if you make it smaller the results of your simulation do not change any more (qualitatively) whereas if you make it larger, they do. For most cases, the default value should be sufficient. In fact, a tolerance of 1e-4 might be accurate enough.


{\it Possible values:} [Double 0...1 (inclusive)]
\item {\it Parameter name:} {\tt Max nonlinear iterations}
\phantomsection\label{parameters:Max nonlinear iterations}


\index[prmindex]{Max nonlinear iterations}
\index[prmindexfull]{Max nonlinear iterations}
{\it Value:} 10


{\it Default:} 10


{\it Description:} The maximal number of nonlinear iterations to be performed.


{\it Possible values:} [Integer range 0...2147483647 (inclusive)]
\item {\it Parameter name:} {\tt Max nonlinear iterations in pre-refinement}
\phantomsection\label{parameters:Max nonlinear iterations in pre-refinement}


\index[prmindex]{Max nonlinear iterations in pre-refinement}
\index[prmindexfull]{Max nonlinear iterations in pre-refinement}
{\it Value:} 2147483647


{\it Default:} 2147483647


{\it Description:} The maximal number of nonlinear iterations to be performed in the pre-refinement steps. This does not include the last refinement step before moving to timestep 1. When this parameter has a larger value than max nonlinear iterations, the latter is used.


{\it Possible values:} [Integer range 0...2147483647 (inclusive)]
\item {\it Parameter name:} {\tt Maximum time step}
\phantomsection\label{parameters:Maximum time step}


\index[prmindex]{Maximum time step}
\index[prmindexfull]{Maximum time step}
{\it Value:} 5.69e+300


{\it Default:} 5.69e+300


{\it Description:} Set a maximum time step size for the solver to use. Generally the time step based on the CFL number should be sufficient, but for complicated models or benchmarking it may be useful to limit the time step to some value. The default value is a value so that when converted from years into seconds it equals the largest number representable by a floating point number, implying an unlimited time step.Units: Years or seconds, depending on the ``Use years in output instead of seconds'' parameter.


{\it Possible values:} [Double 0...1.79769e+308 (inclusive)]
\item {\it Parameter name:} {\tt Nonlinear solver scheme}
\phantomsection\label{parameters:Nonlinear solver scheme}


\index[prmindex]{Nonlinear solver scheme}
\index[prmindexfull]{Nonlinear solver scheme}
{\it Value:} IMPES


{\it Default:} IMPES


{\it Description:} The kind of scheme used to resolve the nonlinearity in the system. 'IMPES' is the classical IMplicit Pressure Explicit Saturation scheme in which ones solves the temperatures and Stokes equations exactly once per time step, one after the other. The 'iterated IMPES' scheme iterates this decoupled approach by alternating the solution of the temperature and Stokes systems. The 'iterated Stokes' scheme solves the temperature equation once at the beginning of each time step and then iterates out the solution of the Stokes equation. The 'Stokes only' scheme only solves the Stokes system and ignores compositions and the temperature equation (careful, the material model must not depend on the temperature; mostly useful for Stokes benchmarks). The 'Advection only'scheme only solves the temperature and other advection systems and instead of solving for the Stokes system, a prescribed velocity and pressure is used


{\it Possible values:} [Selection IMPES|iterated IMPES|iterated Stokes|Stokes only|Advection only ]
\item {\it Parameter name:} {\tt Nonlinear solver tolerance}
\phantomsection\label{parameters:Nonlinear solver tolerance}


\index[prmindex]{Nonlinear solver tolerance}
\index[prmindexfull]{Nonlinear solver tolerance}
{\it Value:} 1e-5


{\it Default:} 1e-5


{\it Description:} A relative tolerance up to which the nonlinear solver will iterate. This parameter is only relevant if Nonlinear solver scheme is set to 'iterated Stokes' or 'iterated IMPES'.


{\it Possible values:} [Double 0...1 (inclusive)]
\item {\it Parameter name:} {\tt Number of cheap Stokes solver steps}
\phantomsection\label{parameters:Number of cheap Stokes solver steps}


\index[prmindex]{Number of cheap Stokes solver steps}
\index[prmindexfull]{Number of cheap Stokes solver steps}
{\it Value:} 30


{\it Default:} 30


{\it Description:} As explained in the ASPECT paper (Kronbichler, Heister, and Bangerth, GJI 2012) we first try to solve the Stokes system in every time step using a GMRES iteration with a poor but cheap preconditioner. By default, we try whether we can converge the GMRES solver in 30 such iterations before deciding that we need a better preconditioner. This is sufficient for simple problems with constant viscosity and we never need the second phase with the more expensive preconditioner. On the other hand, for more complex problems, and in particular for problems with strongly varying viscosity, the 30 cheap iterations don't actually do very much good and one might skip this part right away. In that case, this parameter can be set to zero, i.e., we immediately start with the better but more expensive preconditioner.


{\it Possible values:} [Integer range 0...2147483647 (inclusive)]
\item {\it Parameter name:} {\tt Output directory}
\phantomsection\label{parameters:Output directory}


\index[prmindex]{Output directory}
\index[prmindexfull]{Output directory}
{\it Value:} output


{\it Default:} output


{\it Description:} The name of the directory into which all output files should be placed. This may be an absolute or a relative path.


{\it Possible values:} [DirectoryName]
\item {\it Parameter name:} {\tt Pressure normalization}
\phantomsection\label{parameters:Pressure normalization}


\index[prmindex]{Pressure normalization}
\index[prmindexfull]{Pressure normalization}
{\it Value:} surface


{\it Default:} surface


{\it Description:} If and how to normalize the pressure after the solution step. This is necessary because depending on boundary conditions, in many cases the pressure is only determined by the model up to a constant. On the other hand, we often would like to have a well-determined pressure, for example for table lookups of material properties in models or for comparing solutions. If the given value is `surface', then normalization at the end of each time steps adds a constant value to the pressure in such a way that the average pressure at the surface of the domain is zero; the surface of the domain is determined by asking the geometry model whether a particular face of the geometry has a zero or small `depth'. If the value of this parameter is `volume' then the pressure is normalized so that the domain average is zero. If `no' is given, the no pressure normalization is performed.


{\it Possible values:} [Selection surface|volume|no ]
\item {\it Parameter name:} {\tt Resume computation}
\phantomsection\label{parameters:Resume computation}


\index[prmindex]{Resume computation}
\index[prmindexfull]{Resume computation}
{\it Value:} false


{\it Default:} false


{\it Description:} A flag indicating whether the computation should be resumed from a previously saved state (if true) or start from scratch (if false). If auto is selected, models will be resumed if there is an existing checkpoint file, otherwise started from scratch.


{\it Possible values:} [Selection true|false|auto ]
\item {\it Parameter name:} {\tt Start time}
\phantomsection\label{parameters:Start time}


\index[prmindex]{Start time}
\index[prmindexfull]{Start time}
{\it Value:} 0


{\it Default:} 0


{\it Description:} The start time of the simulation. Units: Years if the 'Use years in output instead of seconds' parameter is set; seconds otherwise.


{\it Possible values:} [Double -1.79769e+308...1.79769e+308 (inclusive)]
\item {\it Parameter name:} {\tt Surface pressure}
\phantomsection\label{parameters:Surface pressure}


\index[prmindex]{Surface pressure}
\index[prmindexfull]{Surface pressure}
{\it Value:} 0


{\it Default:} 0


{\it Description:} The mathematical equations that describe thermal convection only determine the pressure up to an arbitrary constant. On the other hand, for comparison and for looking up material parameters it is important that the pressure be normalized somehow. We do this by enforcing a particular average pressure value at the surface of the domain, where the geometry model determines where the surface is. This parameter describes what this average surface pressure value is supposed to be. By default, it is set to zero, but one may want to choose a different value for example for simulating only the volume of the mantle below the lithosphere, in which case the surface pressure should be the lithostatic pressure at the bottom of the lithosphere.

For more information, see the section in the manual that discusses the general mathematical model.


{\it Possible values:} [Double -1.79769e+308...1.79769e+308 (inclusive)]
\item {\it Parameter name:} {\tt Temperature solver tolerance}
\phantomsection\label{parameters:Temperature solver tolerance}


\index[prmindex]{Temperature solver tolerance}
\index[prmindexfull]{Temperature solver tolerance}
{\it Value:} 1e-12


{\it Default:} 1e-12


{\it Description:} The relative tolerance up to which the linear system for the temperature system gets solved. See 'linear solver tolerance' for more details.


{\it Possible values:} [Double 0...1 (inclusive)]
\item {\it Parameter name:} {\tt Timing output frequency}
\phantomsection\label{parameters:Timing output frequency}


\index[prmindex]{Timing output frequency}
\index[prmindexfull]{Timing output frequency}
{\it Value:} 100


{\it Default:} 100


{\it Description:} How frequently in timesteps to output timing information. This is generally adjusted only for debugging and timing purposes. If the value is set to zero it will also output timing information at the initiation timesteps.


{\it Possible values:} [Integer range 0...2147483647 (inclusive)]
\item {\it Parameter name:} {\tt Use conduction timestep}
\phantomsection\label{parameters:Use conduction timestep}


\index[prmindex]{Use conduction timestep}
\index[prmindexfull]{Use conduction timestep}
{\it Value:} false


{\it Default:} false


{\it Description:} Mantle convection simulations are often focused on convection dominated systems. However, these codes can also be used to investigate systems where heat conduction plays a dominant role. This parameter indicates whether the simulator should also use heat conduction in determining the length of each time step.


{\it Possible values:} [Bool]
\item {\it Parameter name:} {\tt Use direct solver for Stokes system}
\phantomsection\label{parameters:Use direct solver for Stokes system}


\index[prmindex]{Use direct solver for Stokes system}
\index[prmindexfull]{Use direct solver for Stokes system}
{\it Value:} false


{\it Default:} false


{\it Description:} If set to true the linear system for the Stokes equation will be solved using Trilinos klu, otherwise an iterative Schur complement solver is used. The direct solver is only efficient for small problems.


{\it Possible values:} [Bool]
\item {\it Parameter name:} {\tt Use years in output instead of seconds}
\phantomsection\label{parameters:Use years in output instead of seconds}


\index[prmindex]{Use years in output instead of seconds}
\index[prmindexfull]{Use years in output instead of seconds}
{\it Value:} true


{\it Default:} true


{\it Description:} When computing results for mantle convection simulations, it is often difficult to judge the order of magnitude of results when they are stated in MKS units involving seconds. Rather, some kinds of results such as velocities are often stated in terms of meters per year (or, sometimes, centimeters per year). On the other hand, for non-dimensional computations, one wants results in their natural unit system as used inside the code. If this flag is set to 'true' conversion to years happens; if it is 'false', no such conversion happens. Note that when 'true', some input such as prescribed velocities should also use years instead of seconds.


{\it Possible values:} [Bool]
\end{itemize}



\subsection{Parameters in section \tt Adiabatic conditions model}
\label{parameters:Adiabatic_20conditions_20model}

\begin{itemize}
\item {\it Parameter name:} {\tt Model name}
\phantomsection\label{parameters:Adiabatic conditions model/Model name}


\index[prmindex]{Model name}
\index[prmindexfull]{Adiabatic conditions model!Model name}
{\it Value:} initial profile


{\it Default:} initial profile


{\it Description:} Select one of the following models:

`initial profile': A model in which the adiabatic profile is calculated once at the start of the model run. The gravity is assumed to be in depth direction and the composition is evaluated at reference points, no lateral averaging is performed. All material parameters are used from the material model plugin.


{\it Possible values:} [Selection initial profile ]
\end{itemize}

\subsection{Parameters in section \tt Boundary composition model}
\label{parameters:Boundary_20composition_20model}

\begin{itemize}
\item {\it Parameter name:} {\tt Model name}
\phantomsection\label{parameters:Boundary composition model/Model name}


\index[prmindex]{Model name}
\index[prmindexfull]{Boundary composition model!Model name}
{\it Value:} unspecified


{\it Default:} unspecified


{\it Description:} Select one of the following models:

`ascii data': Implementation of a model in which the boundary composition is derived from files containing data in ascii format. Note the required format of the input data: The first lines may contain any number of commentsif they begin with '\#', but one of these lines needs tocontain the number of grid points in each dimension asfor example '\# POINTS: 3 3'.The order of the data columns has to be 'x', 'composition1', 'composition2', etc. in a 2d model and 'x', 'y', 'composition1', 'composition2', etc., in a 3d model, according to the number of compositional fields, which means that there has to be a single column for every composition in the model.Note that the data in the input files need to be sorted in a specific order: the first coordinate needs to ascend first, followed by the second in order to assign the correct data to the prescribed coordinates.If you use a spherical model, then the data will still be handled as cartesian,however the assumed grid changes. 'x' will be replaced by the radial distance of the point to the bottom of the model, 'y' by the azimuth angle and 'z' by the polar angle measured positive from the north pole. The grid will be assumed to be a latitude-longitude grid. Note that the order of spherical coordinates is 'r', 'phi', 'theta' and not 'r', 'theta', 'phi', since this allows for dimension independent expressions. 

`box': A model in which the composition is chosen constant on all the sides of a box.

`box with lithosphere boundary indicators': A model in which the composition is chosen constant on all the sides of a box. Additional boundary indicators are added to the lithospheric parts of the vertical boundaries. This model is to be used with the 'Two Merged Boxes' Geometry Model.

`initial composition': A model in which the composition at the boundary is chosen to be the same as given in the initial conditions.

Because this class simply takes what the initial composition had described, this class can not know certain pieces of information such as the minimal and maximal composition on the boundary. For operations that require this, for example in postprocessing, this boundary composition model must therefore be told what the minimal and maximal values on the boundary are. This is done using parameters set in section ``Boundary composition model/Initial composition''.

`spherical constant': A model in which the composition is chosen constant on the inner and outer boundaries of a spherical shell or chunk. Parameters are read from subsection 'Spherical constant'.


{\it Possible values:} [Selection ascii data|box|box with lithosphere boundary indicators|initial composition|spherical constant|unspecified ]
\end{itemize}



\subsection{Parameters in section \tt Boundary composition model/Ascii data model}
\label{parameters:Boundary_20composition_20model/Ascii_20data_20model}

\begin{itemize}
\item {\it Parameter name:} {\tt Data directory}
\phantomsection\label{parameters:Boundary composition model/Ascii data model/Data directory}


\index[prmindex]{Data directory}
\index[prmindexfull]{Boundary composition model!Ascii data model!Data directory}
{\it Value:} \$ASPECT\_SOURCE\_DIR/data/boundary-composition/ascii-data/test/


{\it Default:} \$ASPECT\_SOURCE\_DIR/data/boundary-composition/ascii-data/test/


{\it Description:} The name of a directory that contains the model data. This path may either be absolute (if starting with a '/') or relative to the current directory. The path may also include the special text '\$ASPECT\_SOURCE\_DIR' which will be interpreted as the path in which the ASPECT source files were located when ASPECT was compiled. This interpretation allows, for example, to reference files located in the 'data/' subdirectory of ASPECT. 


{\it Possible values:} [DirectoryName]
\item {\it Parameter name:} {\tt Data file name}
\phantomsection\label{parameters:Boundary composition model/Ascii data model/Data file name}


\index[prmindex]{Data file name}
\index[prmindexfull]{Boundary composition model!Ascii data model!Data file name}
{\it Value:} box\_2d\_\%s.\%d.txt


{\it Default:} box\_2d\_\%s.\%d.txt


{\it Description:} The file name of the material data. Provide file in format: (Velocity file name).\%s\%d where \%s is a string specifying the boundary of the model according to the names of the boundary indicators (of a box or a spherical shell).\%d is any sprintf integer qualifier, specifying the format of the current file number. 


{\it Possible values:} [Anything]
\item {\it Parameter name:} {\tt Data file time step}
\phantomsection\label{parameters:Boundary composition model/Ascii data model/Data file time step}


\index[prmindex]{Data file time step}
\index[prmindexfull]{Boundary composition model!Ascii data model!Data file time step}
{\it Value:} 1e6


{\it Default:} 1e6


{\it Description:} Time step between following velocity files. Depending on the setting of the global 'Use years in output instead of seconds' flag in the input file, this number is either interpreted as seconds or as years. The default is one million, i.e., either one million seconds or one million years.


{\it Possible values:} [Double 0...1.79769e+308 (inclusive)]
\item {\it Parameter name:} {\tt Decreasing file order}
\phantomsection\label{parameters:Boundary composition model/Ascii data model/Decreasing file order}


\index[prmindex]{Decreasing file order}
\index[prmindexfull]{Boundary composition model!Ascii data model!Decreasing file order}
{\it Value:} false


{\it Default:} false


{\it Description:} In some cases the boundary files are not numbered in increasing but in decreasing order (e.g. 'Ma BP'). If this flag is set to 'True' the plugin will first load the file with the number 'First velocity file number' and decrease the file number during the model run.


{\it Possible values:} [Bool]
\item {\it Parameter name:} {\tt First data file model time}
\phantomsection\label{parameters:Boundary composition model/Ascii data model/First data file model time}


\index[prmindex]{First data file model time}
\index[prmindexfull]{Boundary composition model!Ascii data model!First data file model time}
{\it Value:} 0


{\it Default:} 0


{\it Description:} Time from which on the velocity file with number 'First velocity file number' is used as boundary condition. Previous to this time, a no-slip boundary condition is assumed. Depending on the setting of the global 'Use years in output instead of seconds' flag in the input file, this number is either interpreted as seconds or as years.


{\it Possible values:} [Double 0...1.79769e+308 (inclusive)]
\item {\it Parameter name:} {\tt First data file number}
\phantomsection\label{parameters:Boundary composition model/Ascii data model/First data file number}


\index[prmindex]{First data file number}
\index[prmindexfull]{Boundary composition model!Ascii data model!First data file number}
{\it Value:} 0


{\it Default:} 0


{\it Description:} Number of the first velocity file to be loaded when the model time is larger than 'First velocity file model time'.


{\it Possible values:} [Integer range -2147483648...2147483647 (inclusive)]
\item {\it Parameter name:} {\tt Scale factor}
\phantomsection\label{parameters:Boundary composition model/Ascii data model/Scale factor}


\index[prmindex]{Scale factor}
\index[prmindexfull]{Boundary composition model!Ascii data model!Scale factor}
{\it Value:} 1


{\it Default:} 1


{\it Description:} Scalar factor, which is applied to the boundary velocity. You might want to use this to scale the velocities to a reference model (e.g. with free-slip boundary) or another plate reconstruction. Another way to use this factor is to convert units of the input files. The unit is assumed to bem/s or m/yr depending on the 'Use years in output instead of seconds' flag. If you provide velocities in cm/yr set this factor to 0.01.


{\it Possible values:} [Double 0...1.79769e+308 (inclusive)]
\end{itemize}

\subsection{Parameters in section \tt Boundary composition model/Box}
\label{parameters:Boundary_20composition_20model/Box}

\begin{itemize}
\item {\it Parameter name:} {\tt Bottom composition}
\phantomsection\label{parameters:Boundary composition model/Box/Bottom composition}


\index[prmindex]{Bottom composition}
\index[prmindexfull]{Boundary composition model!Box!Bottom composition}
{\it Value:} 


{\it Default:} 


{\it Description:} A comma separated list of composition boundary values at the bottom boundary (at minimal y-value in 2d, or minimal z-value in 3d). This list must have as many entries as there are compositional fields. Units: none.


{\it Possible values:} [List list of [Double -1.79769e+308...1.79769e+308 (inclusive)] of length 0...4294967295 (inclusive)]
\item {\it Parameter name:} {\tt Left composition}
\phantomsection\label{parameters:Boundary composition model/Box/Left composition}


\index[prmindex]{Left composition}
\index[prmindexfull]{Boundary composition model!Box!Left composition}
{\it Value:} 


{\it Default:} 


{\it Description:} A comma separated list of composition boundary values at the left boundary (at minimal x-value). This list must have as many entries as there are compositional fields. Units: none.


{\it Possible values:} [List list of [Double -1.79769e+308...1.79769e+308 (inclusive)] of length 0...4294967295 (inclusive)]
\item {\it Parameter name:} {\tt Right composition}
\phantomsection\label{parameters:Boundary composition model/Box/Right composition}


\index[prmindex]{Right composition}
\index[prmindexfull]{Boundary composition model!Box!Right composition}
{\it Value:} 


{\it Default:} 


{\it Description:} A comma separated list of composition boundary values at the right boundary (at maximal x-value). This list must have as many entries as there are compositional fields. Units: none.


{\it Possible values:} [List list of [Double -1.79769e+308...1.79769e+308 (inclusive)] of length 0...4294967295 (inclusive)]
\item {\it Parameter name:} {\tt Top composition}
\phantomsection\label{parameters:Boundary composition model/Box/Top composition}


\index[prmindex]{Top composition}
\index[prmindexfull]{Boundary composition model!Box!Top composition}
{\it Value:} 


{\it Default:} 


{\it Description:} A comma separated list of composition boundary values at the top boundary (at maximal y-value in 2d, or maximal z-value in 3d). This list must have as many entries as there are compositional fields. Units: none.


{\it Possible values:} [List list of [Double -1.79769e+308...1.79769e+308 (inclusive)] of length 0...4294967295 (inclusive)]
\end{itemize}

\subsection{Parameters in section \tt Boundary composition model/Box with lithosphere boundary indicators}
\label{parameters:Boundary_20composition_20model/Box_20with_20lithosphere_20boundary_20indicators}

\begin{itemize}
\item {\it Parameter name:} {\tt Bottom composition}
\phantomsection\label{parameters:Boundary composition model/Box with lithosphere boundary indicators/Bottom composition}


\index[prmindex]{Bottom composition}
\index[prmindexfull]{Boundary composition model!Box with lithosphere boundary indicators!Bottom composition}
{\it Value:} 


{\it Default:} 


{\it Description:} A comma separated list of composition boundary values at the bottom boundary (at minimal y-value in 2d, or minimal z-value in 3d). This list must have as many entries as there are compositional fields. Units: none.


{\it Possible values:} [List list of [Double -1.79769e+308...1.79769e+308 (inclusive)] of length 0...4294967295 (inclusive)]
\item {\it Parameter name:} {\tt Left composition}
\phantomsection\label{parameters:Boundary composition model/Box with lithosphere boundary indicators/Left composition}


\index[prmindex]{Left composition}
\index[prmindexfull]{Boundary composition model!Box with lithosphere boundary indicators!Left composition}
{\it Value:} 


{\it Default:} 


{\it Description:} A comma separated list of composition boundary values at the left boundary (at minimal x-value). This list must have as many entries as there are compositional fields. Units: none.


{\it Possible values:} [List list of [Double -1.79769e+308...1.79769e+308 (inclusive)] of length 0...4294967295 (inclusive)]
\item {\it Parameter name:} {\tt Left composition lithosphere}
\phantomsection\label{parameters:Boundary composition model/Box with lithosphere boundary indicators/Left composition lithosphere}


\index[prmindex]{Left composition lithosphere}
\index[prmindexfull]{Boundary composition model!Box with lithosphere boundary indicators!Left composition lithosphere}
{\it Value:} 


{\it Default:} 


{\it Description:} A comma separated list of composition boundary values at the left boundary (at minimal x-value). This list must have as many entries as there are compositional fields. Units: none.


{\it Possible values:} [List list of [Double -1.79769e+308...1.79769e+308 (inclusive)] of length 0...4294967295 (inclusive)]
\item {\it Parameter name:} {\tt Right composition}
\phantomsection\label{parameters:Boundary composition model/Box with lithosphere boundary indicators/Right composition}


\index[prmindex]{Right composition}
\index[prmindexfull]{Boundary composition model!Box with lithosphere boundary indicators!Right composition}
{\it Value:} 


{\it Default:} 


{\it Description:} A comma separated list of composition boundary values at the right boundary (at maximal x-value). This list must have as many entries as there are compositional fields. Units: none.


{\it Possible values:} [List list of [Double -1.79769e+308...1.79769e+308 (inclusive)] of length 0...4294967295 (inclusive)]
\item {\it Parameter name:} {\tt Right composition lithosphere}
\phantomsection\label{parameters:Boundary composition model/Box with lithosphere boundary indicators/Right composition lithosphere}


\index[prmindex]{Right composition lithosphere}
\index[prmindexfull]{Boundary composition model!Box with lithosphere boundary indicators!Right composition lithosphere}
{\it Value:} 


{\it Default:} 


{\it Description:} A comma separated list of composition boundary values at the right boundary (at maximal x-value). This list must have as many entries as there are compositional fields. Units: none.


{\it Possible values:} [List list of [Double -1.79769e+308...1.79769e+308 (inclusive)] of length 0...4294967295 (inclusive)]
\item {\it Parameter name:} {\tt Top composition}
\phantomsection\label{parameters:Boundary composition model/Box with lithosphere boundary indicators/Top composition}


\index[prmindex]{Top composition}
\index[prmindexfull]{Boundary composition model!Box with lithosphere boundary indicators!Top composition}
{\it Value:} 


{\it Default:} 


{\it Description:} A comma separated list of composition boundary values at the top boundary (at maximal y-value in 2d, or maximal z-value in 3d). This list must have as many entries as there are compositional fields. Units: none.


{\it Possible values:} [List list of [Double -1.79769e+308...1.79769e+308 (inclusive)] of length 0...4294967295 (inclusive)]
\end{itemize}

\subsection{Parameters in section \tt Boundary composition model/Initial composition}
\label{parameters:Boundary_20composition_20model/Initial_20composition}

\begin{itemize}
\item {\it Parameter name:} {\tt Maximal composition}
\phantomsection\label{parameters:Boundary composition model/Initial composition/Maximal composition}


\index[prmindex]{Maximal composition}
\index[prmindexfull]{Boundary composition model!Initial composition!Maximal composition}
{\it Value:} 1


{\it Default:} 1


{\it Description:} Maximal composition. Units: none.


{\it Possible values:} [Double -1.79769e+308...1.79769e+308 (inclusive)]
\item {\it Parameter name:} {\tt Minimal composition}
\phantomsection\label{parameters:Boundary composition model/Initial composition/Minimal composition}


\index[prmindex]{Minimal composition}
\index[prmindexfull]{Boundary composition model!Initial composition!Minimal composition}
{\it Value:} 0


{\it Default:} 0


{\it Description:} Minimal composition. Units: none.


{\it Possible values:} [Double -1.79769e+308...1.79769e+308 (inclusive)]
\end{itemize}

\subsection{Parameters in section \tt Boundary composition model/Spherical constant}
\label{parameters:Boundary_20composition_20model/Spherical_20constant}

\begin{itemize}
\item {\it Parameter name:} {\tt Inner composition}
\phantomsection\label{parameters:Boundary composition model/Spherical constant/Inner composition}


\index[prmindex]{Inner composition}
\index[prmindexfull]{Boundary composition model!Spherical constant!Inner composition}
{\it Value:} 1


{\it Default:} 1


{\it Description:} Composition at the inner boundary (core mantle boundary). Units: none.


{\it Possible values:} [Double -1.79769e+308...1.79769e+308 (inclusive)]
\item {\it Parameter name:} {\tt Outer composition}
\phantomsection\label{parameters:Boundary composition model/Spherical constant/Outer composition}


\index[prmindex]{Outer composition}
\index[prmindexfull]{Boundary composition model!Spherical constant!Outer composition}
{\it Value:} 0


{\it Default:} 0


{\it Description:} Composition at the outer boundary (lithosphere water/air). Units: none.


{\it Possible values:} [Double -1.79769e+308...1.79769e+308 (inclusive)]
\end{itemize}

\subsection{Parameters in section \tt Boundary temperature model}
\label{parameters:Boundary_20temperature_20model}

\begin{itemize}
\item {\it Parameter name:} {\tt Model name}
\phantomsection\label{parameters:Boundary temperature model/Model name}


\index[prmindex]{Model name}
\index[prmindexfull]{Boundary temperature model!Model name}
{\it Value:} box


{\it Default:} unspecified


{\it Description:} Select one of the following models:

`ascii data': Implementation of a model in which the boundary data is derived from files containing data in ascii format. Note the required format of the input data: The first lines may contain any number of commentsif they begin with '\#', but one of these lines needs tocontain the number of grid points in each dimension asfor example '\# POINTS: 3 3'.The order of the data columns has to be 'x', 'Temperature [K]' in a 2d model and  'x', 'y', 'Temperature [K]' in a 3d model, which means that there has to be a single column containing the temperature. Note that the data in the input files need to be sorted in a specific order: the first coordinate needs to ascend first, followed by the second in order to assign the correct data to the prescribed coordinates.If you use a spherical model, then the data will still be handled as cartesian,however the assumed grid changes. 'x' will be replaced by the radial distance of the point to the bottom of the model, 'y' by the azimuth angle and 'z' by the polar angle measured positive from the north pole. The grid will be assumed to be a latitude-longitude grid. Note that the order of spherical coordinates is 'r', 'phi', 'theta' and not 'r', 'theta', 'phi', since this allows for dimension independent expressions. 

`box': A model in which the temperature is chosen constant on all the sides of a box.

`box with lithosphere boundary indicators': A model in which the temperature is chosen constant on all the sides of a box. Additional boundary indicators are added to the lithospheric parts of the vertical boundaries. This model is to be used with the 'Two Merged Boxes' Geometry Model.

`constant': A model in which the temperature is chosen constant on a given boundary indicator.  Parameters are read from the subsection 'Constant'.

`function': Implementation of a model in which the boundary temperature is given in terms of an explicit formula that is elaborated in the parameters in section ``Boundary temperature model|Function''. 

Since the symbol $t$ indicating time may appear in the formulas for the prescribed temperatures, it is interpreted as having units seconds unless the global input parameter ``Use years in output instead of seconds'' is set, in which case we interpret the formula expressions as having units year.

Because this class simply takes what the function calculates, this class can not know certain pieces of information such as the minimal and maximal temperature on the boundary. For operations that require this, for example in postprocessing, this boundary temperature model must therefore be told what the minimal and maximal values on the boundary are. This is done using parameters set in section ``Boundary temperature model/Initial temperature''.

The format of these functions follows the syntax understood by the muparser library, see Section~\ref{sec:muparser-format}.

`initial temperature': A model in which the temperature at the boundary is chosen to be the same as given in the initial conditions.

Because this class simply takes what the initial temperature had described, this class can not know certain pieces of information such as the minimal and maximal temperature on the boundary. For operations that require this, for example in postprocessing, this boundary temperature model must therefore be told what the minimal and maximal values on the boundary are. This is done using parameters set in section ``Boundary temperature model/Initial temperature''.

`spherical constant': A model in which the temperature is chosen constant on the inner and outer boundaries of a spherical shell, ellipsoidal chunk or chunk. Parameters are read from subsection 'Spherical constant'.


{\it Possible values:} [Selection ascii data|box|box with lithosphere boundary indicators|constant|function|initial temperature|spherical constant|unspecified ]
\end{itemize}



\subsection{Parameters in section \tt Boundary temperature model/Ascii data model}
\label{parameters:Boundary_20temperature_20model/Ascii_20data_20model}

\begin{itemize}
\item {\it Parameter name:} {\tt Data directory}
\phantomsection\label{parameters:Boundary temperature model/Ascii data model/Data directory}


\index[prmindex]{Data directory}
\index[prmindexfull]{Boundary temperature model!Ascii data model!Data directory}
{\it Value:} \$ASPECT\_SOURCE\_DIR/data/boundary-temperature/ascii-data/test/


{\it Default:} \$ASPECT\_SOURCE\_DIR/data/boundary-temperature/ascii-data/test/


{\it Description:} The name of a directory that contains the model data. This path may either be absolute (if starting with a '/') or relative to the current directory. The path may also include the special text '\$ASPECT\_SOURCE\_DIR' which will be interpreted as the path in which the ASPECT source files were located when ASPECT was compiled. This interpretation allows, for example, to reference files located in the 'data/' subdirectory of ASPECT. 


{\it Possible values:} [DirectoryName]
\item {\it Parameter name:} {\tt Data file name}
\phantomsection\label{parameters:Boundary temperature model/Ascii data model/Data file name}


\index[prmindex]{Data file name}
\index[prmindexfull]{Boundary temperature model!Ascii data model!Data file name}
{\it Value:} box\_2d\_\%s.\%d.txt


{\it Default:} box\_2d\_\%s.\%d.txt


{\it Description:} The file name of the material data. Provide file in format: (Velocity file name).\%s\%d where \%s is a string specifying the boundary of the model according to the names of the boundary indicators (of a box or a spherical shell).\%d is any sprintf integer qualifier, specifying the format of the current file number. 


{\it Possible values:} [Anything]
\item {\it Parameter name:} {\tt Data file time step}
\phantomsection\label{parameters:Boundary temperature model/Ascii data model/Data file time step}


\index[prmindex]{Data file time step}
\index[prmindexfull]{Boundary temperature model!Ascii data model!Data file time step}
{\it Value:} 1e6


{\it Default:} 1e6


{\it Description:} Time step between following velocity files. Depending on the setting of the global 'Use years in output instead of seconds' flag in the input file, this number is either interpreted as seconds or as years. The default is one million, i.e., either one million seconds or one million years.


{\it Possible values:} [Double 0...1.79769e+308 (inclusive)]
\item {\it Parameter name:} {\tt Decreasing file order}
\phantomsection\label{parameters:Boundary temperature model/Ascii data model/Decreasing file order}


\index[prmindex]{Decreasing file order}
\index[prmindexfull]{Boundary temperature model!Ascii data model!Decreasing file order}
{\it Value:} false


{\it Default:} false


{\it Description:} In some cases the boundary files are not numbered in increasing but in decreasing order (e.g. 'Ma BP'). If this flag is set to 'True' the plugin will first load the file with the number 'First velocity file number' and decrease the file number during the model run.


{\it Possible values:} [Bool]
\item {\it Parameter name:} {\tt First data file model time}
\phantomsection\label{parameters:Boundary temperature model/Ascii data model/First data file model time}


\index[prmindex]{First data file model time}
\index[prmindexfull]{Boundary temperature model!Ascii data model!First data file model time}
{\it Value:} 0


{\it Default:} 0


{\it Description:} Time from which on the velocity file with number 'First velocity file number' is used as boundary condition. Previous to this time, a no-slip boundary condition is assumed. Depending on the setting of the global 'Use years in output instead of seconds' flag in the input file, this number is either interpreted as seconds or as years.


{\it Possible values:} [Double 0...1.79769e+308 (inclusive)]
\item {\it Parameter name:} {\tt First data file number}
\phantomsection\label{parameters:Boundary temperature model/Ascii data model/First data file number}


\index[prmindex]{First data file number}
\index[prmindexfull]{Boundary temperature model!Ascii data model!First data file number}
{\it Value:} 0


{\it Default:} 0


{\it Description:} Number of the first velocity file to be loaded when the model time is larger than 'First velocity file model time'.


{\it Possible values:} [Integer range -2147483648...2147483647 (inclusive)]
\item {\it Parameter name:} {\tt Scale factor}
\phantomsection\label{parameters:Boundary temperature model/Ascii data model/Scale factor}


\index[prmindex]{Scale factor}
\index[prmindexfull]{Boundary temperature model!Ascii data model!Scale factor}
{\it Value:} 1


{\it Default:} 1


{\it Description:} Scalar factor, which is applied to the boundary velocity. You might want to use this to scale the velocities to a reference model (e.g. with free-slip boundary) or another plate reconstruction. Another way to use this factor is to convert units of the input files. The unit is assumed to bem/s or m/yr depending on the 'Use years in output instead of seconds' flag. If you provide velocities in cm/yr set this factor to 0.01.


{\it Possible values:} [Double 0...1.79769e+308 (inclusive)]
\end{itemize}

\subsection{Parameters in section \tt Boundary temperature model/Box}
\label{parameters:Boundary_20temperature_20model/Box}

\begin{itemize}
\item {\it Parameter name:} {\tt Bottom temperature}
\phantomsection\label{parameters:Boundary temperature model/Box/Bottom temperature}


\index[prmindex]{Bottom temperature}
\index[prmindexfull]{Boundary temperature model!Box!Bottom temperature}
{\it Value:} 0


{\it Default:} 0


{\it Description:} Temperature at the bottom boundary (at minimal z-value). Units: K.


{\it Possible values:} [Double -1.79769e+308...1.79769e+308 (inclusive)]
\item {\it Parameter name:} {\tt Left temperature}
\phantomsection\label{parameters:Boundary temperature model/Box/Left temperature}


\index[prmindex]{Left temperature}
\index[prmindexfull]{Boundary temperature model!Box!Left temperature}
{\it Value:} 1


{\it Default:} 1


{\it Description:} Temperature at the left boundary (at minimal x-value). Units: K.


{\it Possible values:} [Double -1.79769e+308...1.79769e+308 (inclusive)]
\item {\it Parameter name:} {\tt Right temperature}
\phantomsection\label{parameters:Boundary temperature model/Box/Right temperature}


\index[prmindex]{Right temperature}
\index[prmindexfull]{Boundary temperature model!Box!Right temperature}
{\it Value:} 0


{\it Default:} 0


{\it Description:} Temperature at the right boundary (at maximal x-value). Units: K.


{\it Possible values:} [Double -1.79769e+308...1.79769e+308 (inclusive)]
\item {\it Parameter name:} {\tt Top temperature}
\phantomsection\label{parameters:Boundary temperature model/Box/Top temperature}


\index[prmindex]{Top temperature}
\index[prmindexfull]{Boundary temperature model!Box!Top temperature}
{\it Value:} 0


{\it Default:} 0


{\it Description:} Temperature at the top boundary (at maximal x-value). Units: K.


{\it Possible values:} [Double -1.79769e+308...1.79769e+308 (inclusive)]
\end{itemize}

\subsection{Parameters in section \tt Boundary temperature model/Box with lithosphere boundary indicators}
\label{parameters:Boundary_20temperature_20model/Box_20with_20lithosphere_20boundary_20indicators}

\begin{itemize}
\item {\it Parameter name:} {\tt Bottom temperature}
\phantomsection\label{parameters:Boundary temperature model/Box with lithosphere boundary indicators/Bottom temperature}


\index[prmindex]{Bottom temperature}
\index[prmindexfull]{Boundary temperature model!Box with lithosphere boundary indicators!Bottom temperature}
{\it Value:} 0


{\it Default:} 0


{\it Description:} Temperature at the bottom boundary (at minimal z-value). Units: K.


{\it Possible values:} [Double -1.79769e+308...1.79769e+308 (inclusive)]
\item {\it Parameter name:} {\tt Left temperature}
\phantomsection\label{parameters:Boundary temperature model/Box with lithosphere boundary indicators/Left temperature}


\index[prmindex]{Left temperature}
\index[prmindexfull]{Boundary temperature model!Box with lithosphere boundary indicators!Left temperature}
{\it Value:} 1


{\it Default:} 1


{\it Description:} Temperature at the left boundary (at minimal x-value). Units: K.


{\it Possible values:} [Double -1.79769e+308...1.79769e+308 (inclusive)]
\item {\it Parameter name:} {\tt Left temperature lithosphere}
\phantomsection\label{parameters:Boundary temperature model/Box with lithosphere boundary indicators/Left temperature lithosphere}


\index[prmindex]{Left temperature lithosphere}
\index[prmindexfull]{Boundary temperature model!Box with lithosphere boundary indicators!Left temperature lithosphere}
{\it Value:} 0


{\it Default:} 0


{\it Description:} Temperature at the additional left lithosphere boundary (specified by user in Geometry Model). Units: K.


{\it Possible values:} [Double -1.79769e+308...1.79769e+308 (inclusive)]
\item {\it Parameter name:} {\tt Right temperature}
\phantomsection\label{parameters:Boundary temperature model/Box with lithosphere boundary indicators/Right temperature}


\index[prmindex]{Right temperature}
\index[prmindexfull]{Boundary temperature model!Box with lithosphere boundary indicators!Right temperature}
{\it Value:} 0


{\it Default:} 0


{\it Description:} Temperature at the right boundary (at maximal x-value). Units: K.


{\it Possible values:} [Double -1.79769e+308...1.79769e+308 (inclusive)]
\item {\it Parameter name:} {\tt Right temperature lithosphere}
\phantomsection\label{parameters:Boundary temperature model/Box with lithosphere boundary indicators/Right temperature lithosphere}


\index[prmindex]{Right temperature lithosphere}
\index[prmindexfull]{Boundary temperature model!Box with lithosphere boundary indicators!Right temperature lithosphere}
{\it Value:} 0


{\it Default:} 0


{\it Description:} Temperature at the additional right lithosphere boundary (specified by user in Geometry Model). Units: K.


{\it Possible values:} [Double -1.79769e+308...1.79769e+308 (inclusive)]
\item {\it Parameter name:} {\tt Top temperature}
\phantomsection\label{parameters:Boundary temperature model/Box with lithosphere boundary indicators/Top temperature}


\index[prmindex]{Top temperature}
\index[prmindexfull]{Boundary temperature model!Box with lithosphere boundary indicators!Top temperature}
{\it Value:} 0


{\it Default:} 0


{\it Description:} Temperature at the top boundary (at maximal x-value). Units: K.


{\it Possible values:} [Double -1.79769e+308...1.79769e+308 (inclusive)]
\end{itemize}

\subsection{Parameters in section \tt Boundary temperature model/Constant}
\label{parameters:Boundary_20temperature_20model/Constant}

\begin{itemize}
\item {\it Parameter name:} {\tt Boundary indicator to temperature mappings}
\phantomsection\label{parameters:Boundary temperature model/Constant/Boundary indicator to temperature mappings}


\index[prmindex]{Boundary indicator to temperature mappings}
\index[prmindexfull]{Boundary temperature model!Constant!Boundary indicator to temperature mappings}
{\it Value:} 


{\it Default:} 


{\it Description:} A comma separated list of mappings between boundary indicators and the temperature associated with the boundary indicators. The format for this list is ``indicator1 : value1, indicator2 : value2, ...'', where each indicator is a valid boundary indicator (either a number or the symbolic name of a boundary as provided by the geometry model) and each value is the temperature of that boundary.


{\it Possible values:} [Map map of [Anything]:[Double -1.79769e+308...1.79769e+308 (inclusive)] of length 0...4294967295 (inclusive)]
\end{itemize}

\subsection{Parameters in section \tt Boundary temperature model/Function}
\label{parameters:Boundary_20temperature_20model/Function}

\begin{itemize}
\item {\it Parameter name:} {\tt Function constants}
\phantomsection\label{parameters:Boundary temperature model/Function/Function constants}


\index[prmindex]{Function constants}
\index[prmindexfull]{Boundary temperature model!Function!Function constants}
{\it Value:} 


{\it Default:} 


{\it Description:} Sometimes it is convenient to use symbolic constants in the expression that describes the function, rather than having to use its numeric value everywhere the constant appears. These values can be defined using this parameter, in the form `var1=value1, var2=value2, ...'.

A typical example would be to set this runtime parameter to `pi=3.1415926536' and then use `pi' in the expression of the actual formula. (That said, for convenience this class actually defines both `pi' and `Pi' by default, but you get the idea.)


{\it Possible values:} [Anything]
\item {\it Parameter name:} {\tt Function expression}
\phantomsection\label{parameters:Boundary temperature model/Function/Function expression}


\index[prmindex]{Function expression}
\index[prmindexfull]{Boundary temperature model!Function!Function expression}
{\it Value:} 0


{\it Default:} 0


{\it Description:} The formula that denotes the function you want to evaluate for particular values of the independent variables. This expression may contain any of the usual operations such as addition or multiplication, as well as all of the common functions such as `sin' or `cos'. In addition, it may contain expressions like `if(x>0, 1, -1)' where the expression evaluates to the second argument if the first argument is true, and to the third argument otherwise. For a full overview of possible expressions accepted see the documentation of the muparser library at http://muparser.beltoforion.de/.

If the function you are describing represents a vector-valued function with multiple components, then separate the expressions for individual components by a semicolon.


{\it Possible values:} [Anything]
\item {\it Parameter name:} {\tt Maximal temperature}
\phantomsection\label{parameters:Boundary temperature model/Function/Maximal temperature}


\index[prmindex]{Maximal temperature}
\index[prmindexfull]{Boundary temperature model!Function!Maximal temperature}
{\it Value:} 3773


{\it Default:} 3773


{\it Description:} Maximal temperature. Units: K.


{\it Possible values:} [Double -1.79769e+308...1.79769e+308 (inclusive)]
\item {\it Parameter name:} {\tt Minimal temperature}
\phantomsection\label{parameters:Boundary temperature model/Function/Minimal temperature}


\index[prmindex]{Minimal temperature}
\index[prmindexfull]{Boundary temperature model!Function!Minimal temperature}
{\it Value:} 273


{\it Default:} 273


{\it Description:} Minimal temperature. Units: K.


{\it Possible values:} [Double -1.79769e+308...1.79769e+308 (inclusive)]
\item {\it Parameter name:} {\tt Variable names}
\phantomsection\label{parameters:Boundary temperature model/Function/Variable names}


\index[prmindex]{Variable names}
\index[prmindexfull]{Boundary temperature model!Function!Variable names}
{\it Value:} x,y,t


{\it Default:} x,y,t


{\it Description:} The name of the variables as they will be used in the function, separated by commas. By default, the names of variables at which the function will be evaluated is `x' (in 1d), `x,y' (in 2d) or `x,y,z' (in 3d) for spatial coordinates and `t' for time. You can then use these variable names in your function expression and they will be replaced by the values of these variables at which the function is currently evaluated. However, you can also choose a different set of names for the independent variables at which to evaluate your function expression. For example, if you work in spherical coordinates, you may wish to set this input parameter to `r,phi,theta,t' and then use these variable names in your function expression.


{\it Possible values:} [Anything]
\end{itemize}

\subsection{Parameters in section \tt Boundary temperature model/Initial temperature}
\label{parameters:Boundary_20temperature_20model/Initial_20temperature}

\begin{itemize}
\item {\it Parameter name:} {\tt Maximal temperature}
\phantomsection\label{parameters:Boundary temperature model/Initial temperature/Maximal temperature}


\index[prmindex]{Maximal temperature}
\index[prmindexfull]{Boundary temperature model!Initial temperature!Maximal temperature}
{\it Value:} 3773


{\it Default:} 3773


{\it Description:} Maximal temperature. Units: K.


{\it Possible values:} [Double -1.79769e+308...1.79769e+308 (inclusive)]
\item {\it Parameter name:} {\tt Minimal temperature}
\phantomsection\label{parameters:Boundary temperature model/Initial temperature/Minimal temperature}


\index[prmindex]{Minimal temperature}
\index[prmindexfull]{Boundary temperature model!Initial temperature!Minimal temperature}
{\it Value:} 0


{\it Default:} 0


{\it Description:} Minimal temperature. Units: K.


{\it Possible values:} [Double -1.79769e+308...1.79769e+308 (inclusive)]
\end{itemize}

\subsection{Parameters in section \tt Boundary temperature model/Spherical constant}
\label{parameters:Boundary_20temperature_20model/Spherical_20constant}

\begin{itemize}
\item {\it Parameter name:} {\tt Inner temperature}
\phantomsection\label{parameters:Boundary temperature model/Spherical constant/Inner temperature}


\index[prmindex]{Inner temperature}
\index[prmindexfull]{Boundary temperature model!Spherical constant!Inner temperature}
{\it Value:} 6000


{\it Default:} 6000


{\it Description:} Temperature at the inner boundary (core mantle boundary). Units: K.


{\it Possible values:} [Double -1.79769e+308...1.79769e+308 (inclusive)]
\item {\it Parameter name:} {\tt Outer temperature}
\phantomsection\label{parameters:Boundary temperature model/Spherical constant/Outer temperature}


\index[prmindex]{Outer temperature}
\index[prmindexfull]{Boundary temperature model!Spherical constant!Outer temperature}
{\it Value:} 0


{\it Default:} 0


{\it Description:} Temperature at the outer boundary (lithosphere water/air). Units: K.


{\it Possible values:} [Double -1.79769e+308...1.79769e+308 (inclusive)]
\end{itemize}

\subsection{Parameters in section \tt Boundary traction model}
\label{parameters:Boundary_20traction_20model}


\subsection{Parameters in section \tt Boundary traction model/Function}
\label{parameters:Boundary_20traction_20model/Function}

\begin{itemize}
\item {\it Parameter name:} {\tt Function constants}
\phantomsection\label{parameters:Boundary traction model/Function/Function constants}


\index[prmindex]{Function constants}
\index[prmindexfull]{Boundary traction model!Function!Function constants}
{\it Value:} 


{\it Default:} 


{\it Description:} Sometimes it is convenient to use symbolic constants in the expression that describes the function, rather than having to use its numeric value everywhere the constant appears. These values can be defined using this parameter, in the form `var1=value1, var2=value2, ...'.

A typical example would be to set this runtime parameter to `pi=3.1415926536' and then use `pi' in the expression of the actual formula. (That said, for convenience this class actually defines both `pi' and `Pi' by default, but you get the idea.)


{\it Possible values:} [Anything]
\item {\it Parameter name:} {\tt Function expression}
\phantomsection\label{parameters:Boundary traction model/Function/Function expression}


\index[prmindex]{Function expression}
\index[prmindexfull]{Boundary traction model!Function!Function expression}
{\it Value:} 0; 0


{\it Default:} 0; 0


{\it Description:} The formula that denotes the function you want to evaluate for particular values of the independent variables. This expression may contain any of the usual operations such as addition or multiplication, as well as all of the common functions such as `sin' or `cos'. In addition, it may contain expressions like `if(x>0, 1, -1)' where the expression evaluates to the second argument if the first argument is true, and to the third argument otherwise. For a full overview of possible expressions accepted see the documentation of the muparser library at http://muparser.beltoforion.de/.

If the function you are describing represents a vector-valued function with multiple components, then separate the expressions for individual components by a semicolon.


{\it Possible values:} [Anything]
\item {\it Parameter name:} {\tt Variable names}
\phantomsection\label{parameters:Boundary traction model/Function/Variable names}


\index[prmindex]{Variable names}
\index[prmindexfull]{Boundary traction model!Function!Variable names}
{\it Value:} x,y,t


{\it Default:} x,y,t


{\it Description:} The name of the variables as they will be used in the function, separated by commas. By default, the names of variables at which the function will be evaluated is `x' (in 1d), `x,y' (in 2d) or `x,y,z' (in 3d) for spatial coordinates and `t' for time. You can then use these variable names in your function expression and they will be replaced by the values of these variables at which the function is currently evaluated. However, you can also choose a different set of names for the independent variables at which to evaluate your function expression. For example, if you work in spherical coordinates, you may wish to set this input parameter to `r,phi,theta,t' and then use these variable names in your function expression.


{\it Possible values:} [Anything]
\end{itemize}

\subsection{Parameters in section \tt Boundary velocity model}
\label{parameters:Boundary_20velocity_20model}


\subsection{Parameters in section \tt Boundary velocity model/Ascii data model}
\label{parameters:Boundary_20velocity_20model/Ascii_20data_20model}

\begin{itemize}
\item {\it Parameter name:} {\tt Data directory}
\phantomsection\label{parameters:Boundary velocity model/Ascii data model/Data directory}


\index[prmindex]{Data directory}
\index[prmindexfull]{Boundary velocity model!Ascii data model!Data directory}
{\it Value:} \$ASPECT\_SOURCE\_DIR/data/velocity-boundary-conditions/ascii-data/test/


{\it Default:} \$ASPECT\_SOURCE\_DIR/data/velocity-boundary-conditions/ascii-data/test/


{\it Description:} The name of a directory that contains the model data. This path may either be absolute (if starting with a '/') or relative to the current directory. The path may also include the special text '\$ASPECT\_SOURCE\_DIR' which will be interpreted as the path in which the ASPECT source files were located when ASPECT was compiled. This interpretation allows, for example, to reference files located in the 'data/' subdirectory of ASPECT. 


{\it Possible values:} [DirectoryName]
\item {\it Parameter name:} {\tt Data file name}
\phantomsection\label{parameters:Boundary velocity model/Ascii data model/Data file name}


\index[prmindex]{Data file name}
\index[prmindexfull]{Boundary velocity model!Ascii data model!Data file name}
{\it Value:} box\_2d\_\%s.\%d.txt


{\it Default:} box\_2d\_\%s.\%d.txt


{\it Description:} The file name of the material data. Provide file in format: (Velocity file name).\%s\%d where \%s is a string specifying the boundary of the model according to the names of the boundary indicators (of a box or a spherical shell).\%d is any sprintf integer qualifier, specifying the format of the current file number. 


{\it Possible values:} [Anything]
\item {\it Parameter name:} {\tt Data file time step}
\phantomsection\label{parameters:Boundary velocity model/Ascii data model/Data file time step}


\index[prmindex]{Data file time step}
\index[prmindexfull]{Boundary velocity model!Ascii data model!Data file time step}
{\it Value:} 1e6


{\it Default:} 1e6


{\it Description:} Time step between following velocity files. Depending on the setting of the global 'Use years in output instead of seconds' flag in the input file, this number is either interpreted as seconds or as years. The default is one million, i.e., either one million seconds or one million years.


{\it Possible values:} [Double 0...1.79769e+308 (inclusive)]
\item {\it Parameter name:} {\tt Decreasing file order}
\phantomsection\label{parameters:Boundary velocity model/Ascii data model/Decreasing file order}


\index[prmindex]{Decreasing file order}
\index[prmindexfull]{Boundary velocity model!Ascii data model!Decreasing file order}
{\it Value:} false


{\it Default:} false


{\it Description:} In some cases the boundary files are not numbered in increasing but in decreasing order (e.g. 'Ma BP'). If this flag is set to 'True' the plugin will first load the file with the number 'First velocity file number' and decrease the file number during the model run.


{\it Possible values:} [Bool]
\item {\it Parameter name:} {\tt First data file model time}
\phantomsection\label{parameters:Boundary velocity model/Ascii data model/First data file model time}


\index[prmindex]{First data file model time}
\index[prmindexfull]{Boundary velocity model!Ascii data model!First data file model time}
{\it Value:} 0


{\it Default:} 0


{\it Description:} Time from which on the velocity file with number 'First velocity file number' is used as boundary condition. Previous to this time, a no-slip boundary condition is assumed. Depending on the setting of the global 'Use years in output instead of seconds' flag in the input file, this number is either interpreted as seconds or as years.


{\it Possible values:} [Double 0...1.79769e+308 (inclusive)]
\item {\it Parameter name:} {\tt First data file number}
\phantomsection\label{parameters:Boundary velocity model/Ascii data model/First data file number}


\index[prmindex]{First data file number}
\index[prmindexfull]{Boundary velocity model!Ascii data model!First data file number}
{\it Value:} 0


{\it Default:} 0


{\it Description:} Number of the first velocity file to be loaded when the model time is larger than 'First velocity file model time'.


{\it Possible values:} [Integer range -2147483648...2147483647 (inclusive)]
\item {\it Parameter name:} {\tt Scale factor}
\phantomsection\label{parameters:Boundary velocity model/Ascii data model/Scale factor}


\index[prmindex]{Scale factor}
\index[prmindexfull]{Boundary velocity model!Ascii data model!Scale factor}
{\it Value:} 1


{\it Default:} 1


{\it Description:} Scalar factor, which is applied to the boundary velocity. You might want to use this to scale the velocities to a reference model (e.g. with free-slip boundary) or another plate reconstruction. Another way to use this factor is to convert units of the input files. The unit is assumed to bem/s or m/yr depending on the 'Use years in output instead of seconds' flag. If you provide velocities in cm/yr set this factor to 0.01.


{\it Possible values:} [Double 0...1.79769e+308 (inclusive)]
\end{itemize}

\subsection{Parameters in section \tt Boundary velocity model/Function}
\label{parameters:Boundary_20velocity_20model/Function}

\begin{itemize}
\item {\it Parameter name:} {\tt Function constants}
\phantomsection\label{parameters:Boundary velocity model/Function/Function constants}


\index[prmindex]{Function constants}
\index[prmindexfull]{Boundary velocity model!Function!Function constants}
{\it Value:} 


{\it Default:} 


{\it Description:} Sometimes it is convenient to use symbolic constants in the expression that describes the function, rather than having to use its numeric value everywhere the constant appears. These values can be defined using this parameter, in the form `var1=value1, var2=value2, ...'.

A typical example would be to set this runtime parameter to `pi=3.1415926536' and then use `pi' in the expression of the actual formula. (That said, for convenience this class actually defines both `pi' and `Pi' by default, but you get the idea.)


{\it Possible values:} [Anything]
\item {\it Parameter name:} {\tt Function expression}
\phantomsection\label{parameters:Boundary velocity model/Function/Function expression}


\index[prmindex]{Function expression}
\index[prmindexfull]{Boundary velocity model!Function!Function expression}
{\it Value:} 0; 0


{\it Default:} 0; 0


{\it Description:} The formula that denotes the function you want to evaluate for particular values of the independent variables. This expression may contain any of the usual operations such as addition or multiplication, as well as all of the common functions such as `sin' or `cos'. In addition, it may contain expressions like `if(x>0, 1, -1)' where the expression evaluates to the second argument if the first argument is true, and to the third argument otherwise. For a full overview of possible expressions accepted see the documentation of the muparser library at http://muparser.beltoforion.de/.

If the function you are describing represents a vector-valued function with multiple components, then separate the expressions for individual components by a semicolon.


{\it Possible values:} [Anything]
\item {\it Parameter name:} {\tt Variable names}
\phantomsection\label{parameters:Boundary velocity model/Function/Variable names}


\index[prmindex]{Variable names}
\index[prmindexfull]{Boundary velocity model!Function!Variable names}
{\it Value:} x,y,t


{\it Default:} x,y,t


{\it Description:} The name of the variables as they will be used in the function, separated by commas. By default, the names of variables at which the function will be evaluated is `x' (in 1d), `x,y' (in 2d) or `x,y,z' (in 3d) for spatial coordinates and `t' for time. You can then use these variable names in your function expression and they will be replaced by the values of these variables at which the function is currently evaluated. However, you can also choose a different set of names for the independent variables at which to evaluate your function expression. For example, if you work in spherical coordinates, you may wish to set this input parameter to `r,phi,theta,t' and then use these variable names in your function expression.


{\it Possible values:} [Anything]
\end{itemize}

\subsection{Parameters in section \tt Boundary velocity model/GPlates model}
\label{parameters:Boundary_20velocity_20model/GPlates_20model}

\begin{itemize}
\item {\it Parameter name:} {\tt Data directory}
\phantomsection\label{parameters:Boundary velocity model/GPlates model/Data directory}


\index[prmindex]{Data directory}
\index[prmindexfull]{Boundary velocity model!GPlates model!Data directory}
{\it Value:} \$ASPECT\_SOURCE\_DIR/data/velocity-boundary-conditions/gplates/


{\it Default:} \$ASPECT\_SOURCE\_DIR/data/velocity-boundary-conditions/gplates/


{\it Description:} The name of a directory that contains the model data. This path may either be absolute (if starting with a '/') or relative to the current directory. The path may also include the special text '\$ASPECT\_SOURCE\_DIR' which will be interpreted as the path in which the ASPECT source files were located when ASPECT was compiled. This interpretation allows, for example, to reference files located in the 'data/' subdirectory of ASPECT. 


{\it Possible values:} [DirectoryName]
\item {\it Parameter name:} {\tt Interpolation width}
\phantomsection\label{parameters:Boundary velocity model/GPlates model/Interpolation width}


\index[prmindex]{Interpolation width}
\index[prmindexfull]{Boundary velocity model!GPlates model!Interpolation width}
{\it Value:} 0


{\it Default:} 0


{\it Description:} Determines the width of the velocity interpolation zone around the current point. Currently equals the arc distance between evaluation point and velocity data point that is still included in the interpolation. The weighting of the points currently only accounts for the surface area a single data point is covering ('moving window' interpolation without distance weighting).


{\it Possible values:} [Double 0...1.79769e+308 (inclusive)]
\item {\it Parameter name:} {\tt Point one}
\phantomsection\label{parameters:Boundary velocity model/GPlates model/Point one}


\index[prmindex]{Point one}
\index[prmindexfull]{Boundary velocity model!GPlates model!Point one}
{\it Value:} 1.570796,0.0


{\it Default:} 1.570796,0.0


{\it Description:} Point that determines the plane in which a 2D model lies in. Has to be in the format 'a,b' where a and b are theta (polar angle)  and phi in radians.


{\it Possible values:} [Anything]
\item {\it Parameter name:} {\tt Point two}
\phantomsection\label{parameters:Boundary velocity model/GPlates model/Point two}


\index[prmindex]{Point two}
\index[prmindexfull]{Boundary velocity model!GPlates model!Point two}
{\it Value:} 1.570796,1.570796


{\it Default:} 1.570796,1.570796


{\it Description:} Point that determines the plane in which a 2D model lies in. Has to be in the format 'a,b' where a and b are theta (polar angle)  and phi in radians.


{\it Possible values:} [Anything]
\item {\it Parameter name:} {\tt Scale factor}
\phantomsection\label{parameters:Boundary velocity model/GPlates model/Scale factor}


\index[prmindex]{Scale factor}
\index[prmindexfull]{Boundary velocity model!GPlates model!Scale factor}
{\it Value:} 1


{\it Default:} 1


{\it Description:} Scalar factor, which is applied to the boundary velocity. You might want to use this to scale the velocities to a reference model (e.g. with free-slip boundary) or another plate reconstruction.


{\it Possible values:} [Double 0...1.79769e+308 (inclusive)]
\item {\it Parameter name:} {\tt Time step}
\phantomsection\label{parameters:Boundary velocity model/GPlates model/Time step}


\index[prmindex]{Time step}
\index[prmindexfull]{Boundary velocity model!GPlates model!Time step}
{\it Value:} 1e6


{\it Default:} 1e6


{\it Description:} Time step between following velocity files. Depending on the setting of the global 'Use years in output instead of seconds' flag in the input file, this number is either interpreted as seconds or as years. The default is one million, i.e., either one million seconds or one million years.


{\it Possible values:} [Double 0...1.79769e+308 (inclusive)]
\item {\it Parameter name:} {\tt Velocity file name}
\phantomsection\label{parameters:Boundary velocity model/GPlates model/Velocity file name}


\index[prmindex]{Velocity file name}
\index[prmindexfull]{Boundary velocity model!GPlates model!Velocity file name}
{\it Value:} phi.\%d


{\it Default:} phi.\%d


{\it Description:} The file name of the material data. Provide file in format: (Velocity file name).\%d.gpml where \%d is any sprintf integer qualifier, specifying the format of the current file number.


{\it Possible values:} [Anything]
\item {\it Parameter name:} {\tt Velocity file start time}
\phantomsection\label{parameters:Boundary velocity model/GPlates model/Velocity file start time}


\index[prmindex]{Velocity file start time}
\index[prmindexfull]{Boundary velocity model!GPlates model!Velocity file start time}
{\it Value:} 0.0


{\it Default:} 0.0


{\it Description:} Time at which the velocity file with number 0 shall be loaded. Previous to this time, a no-slip boundary condition is assumed. Depending on the setting of the global 'Use years in output instead of seconds' flag in the input file, this number is either interpreted as seconds or as years.


{\it Possible values:} [Double 0...1.79769e+308 (inclusive)]
\end{itemize}

\subsection{Parameters in section \tt Checkpointing}
\label{parameters:Checkpointing}

\begin{itemize}
\item {\it Parameter name:} {\tt Steps between checkpoint}
\phantomsection\label{parameters:Checkpointing/Steps between checkpoint}


\index[prmindex]{Steps between checkpoint}
\index[prmindexfull]{Checkpointing!Steps between checkpoint}
{\it Value:} 0


{\it Default:} 0


{\it Description:} The number of timesteps between performing checkpoints. If 0 and time between checkpoint is not specified, checkpointing will not be performed. Units: None.


{\it Possible values:} [Integer range 0...2147483647 (inclusive)]
\item {\it Parameter name:} {\tt Time between checkpoint}
\phantomsection\label{parameters:Checkpointing/Time between checkpoint}


\index[prmindex]{Time between checkpoint}
\index[prmindexfull]{Checkpointing!Time between checkpoint}
{\it Value:} 0


{\it Default:} 0


{\it Description:} The wall time between performing checkpoints. If 0, will use the checkpoint step frequency instead. Units: Seconds.


{\it Possible values:} [Integer range 0...2147483647 (inclusive)]
\end{itemize}

\subsection{Parameters in section \tt Compositional fields}
\label{parameters:Compositional_20fields}

\begin{itemize}
\item {\it Parameter name:} {\tt List of normalized fields}
\phantomsection\label{parameters:Compositional fields/List of normalized fields}


\index[prmindex]{List of normalized fields}
\index[prmindexfull]{Compositional fields!List of normalized fields}
{\it Value:} 


{\it Default:} 


{\it Description:} A list of integers smaller than or equal to the number of compositional fields. All compositional fields in this list will be normalized before the first timestep. The normalization is implemented in the following way: First, the sum of the fields to be normalized is calculated at every point and the global maximum is determined. Second, the compositional fields to be normalized are divided by this maximum.


{\it Possible values:} [List list of [Integer range 0...2147483647 (inclusive)] of length 0...4294967295 (inclusive)]
\item {\it Parameter name:} {\tt Names of fields}
\phantomsection\label{parameters:Compositional fields/Names of fields}


\index[prmindex]{Names of fields}
\index[prmindexfull]{Compositional fields!Names of fields}
{\it Value:} 


{\it Default:} 


{\it Description:} A user-defined name for each of the compositional fields requested.


{\it Possible values:} [List list of [Anything] of length 0...4294967295 (inclusive)]
\item {\it Parameter name:} {\tt Number of fields}
\phantomsection\label{parameters:Compositional fields/Number of fields}


\index[prmindex]{Number of fields}
\index[prmindexfull]{Compositional fields!Number of fields}
{\it Value:} 0


{\it Default:} 0


{\it Description:} The number of fields that will be advected along with the flow field, excluding velocity, pressure and temperature.


{\it Possible values:} [Integer range 0...2147483647 (inclusive)]
\end{itemize}

\subsection{Parameters in section \tt Compositional initial conditions}
\label{parameters:Compositional_20initial_20conditions}

\begin{itemize}
\item {\it Parameter name:} {\tt Model name}
\phantomsection\label{parameters:Compositional initial conditions/Model name}


\index[prmindex]{Model name}
\index[prmindexfull]{Compositional initial conditions!Model name}
{\it Value:} function


{\it Default:} function


{\it Description:} Select one of the following models:

`ascii data': Implementation of a model in which the initial composition is derived from files containing data in ascii format. Note the required format of the input data: The first lines may contain any number of commentsif they begin with '\#', but one of these lines needs tocontain the number of grid points in each dimension asfor example '\# POINTS: 3 3'.The order of the data columns has to be 'x', 'y', 'composition1', 'composition2', etc. in a 2d model and 'x', 'y', 'z', 'composition1', 'composition2', etc. in a 3d model, according to the number of compositional fields, which means that there has to be a single column for every composition in the model.Note that the data in the input files need to be sorted in a specific order: the first coordinate needs to ascend first, followed by the second and the third at last in order to assign the correct data to the prescribed coordinates.If you use a spherical model, then the data will still be handled as cartesian,however the assumed grid changes. 'x' will be replaced by the radial distance of the point to the bottom of the model, 'y' by the azimuth angle and 'z' by the polar angle measured positive from the north pole. The grid will be assumed to be a latitude-longitude grid. Note that the order of spherical coordinates is 'r', 'phi', 'theta' and not 'r', 'theta', 'phi', since this allows for dimension independent expressions. 

`function': Specify the composition in terms of an explicit formula. The format of these functions follows the syntax understood by the muparser library, see Section~\ref{sec:muparser-format}.


{\it Possible values:} [Selection ascii data|function ]
\end{itemize}



\subsection{Parameters in section \tt Compositional initial conditions/Ascii data model}
\label{parameters:Compositional_20initial_20conditions/Ascii_20data_20model}

\begin{itemize}
\item {\it Parameter name:} {\tt Data directory}
\phantomsection\label{parameters:Compositional initial conditions/Ascii data model/Data directory}


\index[prmindex]{Data directory}
\index[prmindexfull]{Compositional initial conditions!Ascii data model!Data directory}
{\it Value:} \$ASPECT\_SOURCE\_DIR/data/compositional-initial-conditions/ascii-data/test/


{\it Default:} \$ASPECT\_SOURCE\_DIR/data/compositional-initial-conditions/ascii-data/test/


{\it Description:} The name of a directory that contains the model data. This path may either be absolute (if starting with a '/') or relative to the current directory. The path may also include the special text '\$ASPECT\_SOURCE\_DIR' which will be interpreted as the path in which the ASPECT source files were located when ASPECT was compiled. This interpretation allows, for example, to reference files located in the 'data/' subdirectory of ASPECT. 


{\it Possible values:} [DirectoryName]
\item {\it Parameter name:} {\tt Data file name}
\phantomsection\label{parameters:Compositional initial conditions/Ascii data model/Data file name}


\index[prmindex]{Data file name}
\index[prmindexfull]{Compositional initial conditions!Ascii data model!Data file name}
{\it Value:} box\_2d.txt


{\it Default:} box\_2d.txt


{\it Description:} The file name of the material data. Provide file in format: (Velocity file name).\%s\%d where \%s is a string specifying the boundary of the model according to the names of the boundary indicators (of a box or a spherical shell).\%d is any sprintf integer qualifier, specifying the format of the current file number. 


{\it Possible values:} [Anything]
\item {\it Parameter name:} {\tt Scale factor}
\phantomsection\label{parameters:Compositional initial conditions/Ascii data model/Scale factor}


\index[prmindex]{Scale factor}
\index[prmindexfull]{Compositional initial conditions!Ascii data model!Scale factor}
{\it Value:} 1


{\it Default:} 1


{\it Description:} Scalar factor, which is applied to the boundary velocity. You might want to use this to scale the velocities to a reference model (e.g. with free-slip boundary) or another plate reconstruction. Another way to use this factor is to convert units of the input files. The unit is assumed to bem/s or m/yr depending on the 'Use years in output instead of seconds' flag. If you provide velocities in cm/yr set this factor to 0.01.


{\it Possible values:} [Double 0...1.79769e+308 (inclusive)]
\end{itemize}

\subsection{Parameters in section \tt Compositional initial conditions/Function}
\label{parameters:Compositional_20initial_20conditions/Function}

\begin{itemize}
\item {\it Parameter name:} {\tt Function constants}
\phantomsection\label{parameters:Compositional initial conditions/Function/Function constants}


\index[prmindex]{Function constants}
\index[prmindexfull]{Compositional initial conditions!Function!Function constants}
{\it Value:} 


{\it Default:} 


{\it Description:} Sometimes it is convenient to use symbolic constants in the expression that describes the function, rather than having to use its numeric value everywhere the constant appears. These values can be defined using this parameter, in the form `var1=value1, var2=value2, ...'.

A typical example would be to set this runtime parameter to `pi=3.1415926536' and then use `pi' in the expression of the actual formula. (That said, for convenience this class actually defines both `pi' and `Pi' by default, but you get the idea.)


{\it Possible values:} [Anything]
\item {\it Parameter name:} {\tt Function expression}
\phantomsection\label{parameters:Compositional initial conditions/Function/Function expression}


\index[prmindex]{Function expression}
\index[prmindexfull]{Compositional initial conditions!Function!Function expression}
{\it Value:} 0


{\it Default:} 0


{\it Description:} The formula that denotes the function you want to evaluate for particular values of the independent variables. This expression may contain any of the usual operations such as addition or multiplication, as well as all of the common functions such as `sin' or `cos'. In addition, it may contain expressions like `if(x>0, 1, -1)' where the expression evaluates to the second argument if the first argument is true, and to the third argument otherwise. For a full overview of possible expressions accepted see the documentation of the muparser library at http://muparser.beltoforion.de/.

If the function you are describing represents a vector-valued function with multiple components, then separate the expressions for individual components by a semicolon.


{\it Possible values:} [Anything]
\item {\it Parameter name:} {\tt Variable names}
\phantomsection\label{parameters:Compositional initial conditions/Function/Variable names}


\index[prmindex]{Variable names}
\index[prmindexfull]{Compositional initial conditions!Function!Variable names}
{\it Value:} x,y,t


{\it Default:} x,y,t


{\it Description:} The name of the variables as they will be used in the function, separated by commas. By default, the names of variables at which the function will be evaluated is `x' (in 1d), `x,y' (in 2d) or `x,y,z' (in 3d) for spatial coordinates and `t' for time. You can then use these variable names in your function expression and they will be replaced by the values of these variables at which the function is currently evaluated. However, you can also choose a different set of names for the independent variables at which to evaluate your function expression. For example, if you work in spherical coordinates, you may wish to set this input parameter to `r,phi,theta,t' and then use these variable names in your function expression.


{\it Possible values:} [Anything]
\end{itemize}

\subsection{Parameters in section \tt Discretization}
\label{parameters:Discretization}

\begin{itemize}
\item {\it Parameter name:} {\tt Composition polynomial degree}
\phantomsection\label{parameters:Discretization/Composition polynomial degree}


\index[prmindex]{Composition polynomial degree}
\index[prmindexfull]{Discretization!Composition polynomial degree}
{\it Value:} 2


{\it Default:} 2


{\it Description:} The polynomial degree to use for the composition variable(s). Units: None.


{\it Possible values:} [Integer range 1...2147483647 (inclusive)]
\item {\it Parameter name:} {\tt Stokes velocity polynomial degree}
\phantomsection\label{parameters:Discretization/Stokes velocity polynomial degree}


\index[prmindex]{Stokes velocity polynomial degree}
\index[prmindexfull]{Discretization!Stokes velocity polynomial degree}
{\it Value:} 2


{\it Default:} 2


{\it Description:} The polynomial degree to use for the velocity variables in the Stokes system. The polynomial degree for the pressure variable will then be one less in order to make the velocity/pressure pair conform with the usual LBB (Babuska-Brezzi) condition. In other words, we are using a Taylor-Hood element for the Stoeks equations and this parameter indicates the polynomial degree of it. Units: None.


{\it Possible values:} [Integer range 1...2147483647 (inclusive)]
\item {\it Parameter name:} {\tt Temperature polynomial degree}
\phantomsection\label{parameters:Discretization/Temperature polynomial degree}


\index[prmindex]{Temperature polynomial degree}
\index[prmindexfull]{Discretization!Temperature polynomial degree}
{\it Value:} 2


{\it Default:} 2


{\it Description:} The polynomial degree to use for the temperature variable. Units: None.


{\it Possible values:} [Integer range 1...2147483647 (inclusive)]
\item {\it Parameter name:} {\tt Use locally conservative discretization}
\phantomsection\label{parameters:Discretization/Use locally conservative discretization}


\index[prmindex]{Use locally conservative discretization}
\index[prmindexfull]{Discretization!Use locally conservative discretization}
{\it Value:} false


{\it Default:} false


{\it Description:} Whether to use a Stokes discretization that is locally conservative at the expense of a larger number of degrees of freedom (true), or to go with a cheaper discretization that does not locally conserve mass, although it is globally conservative (false).

When using a locally conservative discretization, the finite element space for the pressure is discontinuous between cells and is the polynomial space $P_ {-q}$ of polynomials of degree $q$ in each variable separately. Here, $q$ is one less than the value given in the parameter ``Stokes velocity polynomial degree''. As a consequence of choosing this element, it can be shown if the medium is considered incompressible that the computed discrete velocity field $\mathbf u_h$ satisfies the property $\int_ {\partial K} \mathbf u_h \cdot \mathbf n = 0$ for every cell $K$, i.e., for each cell inflow and outflow exactly balance each other as one would expect for an incompressible medium. In other words, the velocity field is locally conservative.

On the other hand, if this parameter is set to ``false'', then the finite element space is chosen as $Q_q$. This choice does not yield the local conservation property but has the advantage of requiring fewer degrees of freedom. Furthermore, the error is generally smaller with this choice.

For an in-depth discussion of these issues and a quantitative evaluation of the different choices, see \cite {KHB12} .


{\it Possible values:} [Bool]
\end{itemize}



\subsection{Parameters in section \tt Discretization/Stabilization parameters}
\label{parameters:Discretization/Stabilization_20parameters}

\begin{itemize}
\item {\it Parameter name:} {\tt Use artificial viscosity smoothing}
\phantomsection\label{parameters:Discretization/Stabilization parameters/Use artificial viscosity smoothing}


\index[prmindex]{Use artificial viscosity smoothing}
\index[prmindexfull]{Discretization!Stabilization parameters!Use artificial viscosity smoothing}
{\it Value:} false


{\it Default:} false


{\it Description:} If set to false, the artificial viscosity of a cell is computed andis computed on every cell separately as discussed in \cite{KHB12}. If set to true, the maximum of the artificial viscosity in the cell as well as the neighbors of the cell is computed and used instead.


{\it Possible values:} [Bool]
\item {\it Parameter name:} {\tt alpha}
\phantomsection\label{parameters:Discretization/Stabilization parameters/alpha}


\index[prmindex]{alpha}
\index[prmindexfull]{Discretization!Stabilization parameters!alpha}
{\it Value:} 2


{\it Default:} 2


{\it Description:} The exponent $\alpha$ in the entropy viscosity stabilization. Valid options are 1 or 2. The recommended setting is 2. (This parameter does not correspond to any variable in the 2012 GJI paper by Kronbichler, Heister and Bangerth that describes ASPECT. Rather, the paper always uses 2 as the exponent in the definition of the entropy, following eq. (15).).Units: None.


{\it Possible values:} [Integer range 1...2 (inclusive)]
\item {\it Parameter name:} {\tt beta}
\phantomsection\label{parameters:Discretization/Stabilization parameters/beta}


\index[prmindex]{beta}
\index[prmindexfull]{Discretization!Stabilization parameters!beta}
{\it Value:} 0.078


{\it Default:} 0.078


{\it Description:} The $\beta$ factor in the artificial viscosity stabilization. An appropriate value for 2d is 0.078 and 0.117 for 3d. (For historical reasons, the name used here is different from the one used in the 2012 GJI paper by Kronbichler, Heister and Bangerth that describes ASPECT. This parameter corresponds to the factor $\alpha_\text {max}$ in the formulas following equation (15) of the paper. After further experiments, we have also chosen to use a different value than described there: It can be chosen as stated there for uniformly refined meshes, but it needs to be chosen larger if the mesh has cells that are not squares or cubes.) Units: None.


{\it Possible values:} [Double 0...1.79769e+308 (inclusive)]
\item {\it Parameter name:} {\tt cR}
\phantomsection\label{parameters:Discretization/Stabilization parameters/cR}


\index[prmindex]{cR}
\index[prmindexfull]{Discretization!Stabilization parameters!cR}
{\it Value:} 0.33


{\it Default:} 0.33


{\it Description:} The $c_R$ factor in the entropy viscosity stabilization. (For historical reasons, the name used here is different from the one used in the 2012 GJI paper by Kronbichler, Heister and Bangerth that describes ASPECT. This parameter corresponds to the factor $\alpha_E$ in the formulas following equation (15) of the paper. After further experiments, we have also chosen to use a different value than described there.) Units: None.


{\it Possible values:} [Double 0...1.79769e+308 (inclusive)]
\end{itemize}

\subsection{Parameters in section \tt Free surface}
\label{parameters:Free_20surface}

\begin{itemize}
\item {\it Parameter name:} {\tt Additional tangential mesh velocity boundary indicators}
\phantomsection\label{parameters:Free surface/Additional tangential mesh velocity boundary indicators}


\index[prmindex]{Additional tangential mesh velocity boundary indicators}
\index[prmindexfull]{Free surface!Additional tangential mesh velocity boundary indicators}
{\it Value:} 


{\it Default:} 


{\it Description:} A comma separated list of names denoting those boundaries where there the mesh is allowed to move tangential to the boundary. All tangential mesh movements along those boundaries that have tangential material velocity boundary conditions are allowed by default, this parameters allows to generate mesh movements along other boundaries that are open, or have prescribed material velocities or tractions.

The names of the boundaries listed here can either be numbers (in which case they correspond to the numerical boundary indicators assigned by the geometry object), or they can correspond to any of the symbolic names the geometry object may have provided for each part of the boundary. You may want to compare this with the documentation of the geometry model you use in your model.


{\it Possible values:} [List list of [Anything] of length 0...4294967295 (inclusive)]
\item {\it Parameter name:} {\tt Free surface stabilization theta}
\phantomsection\label{parameters:Free surface/Free surface stabilization theta}


\index[prmindex]{Free surface stabilization theta}
\index[prmindexfull]{Free surface!Free surface stabilization theta}
{\it Value:} 0.5


{\it Default:} 0.5


{\it Description:} Theta parameter described in Kaus et. al. 2010. An unstabilized free surface can overshoot its equilibrium position quite easily and generate unphysical results.  One solution is to use a quasi-implicit correction term to the forces near the free surface.  This parameter describes how much the free surface is stabilized with this term, where zero is no stabilization, and one is fully implicit.


{\it Possible values:} [Double 0...1 (inclusive)]
\item {\it Parameter name:} {\tt Surface velocity projection}
\phantomsection\label{parameters:Free surface/Surface velocity projection}


\index[prmindex]{Surface velocity projection}
\index[prmindexfull]{Free surface!Surface velocity projection}
{\it Value:} normal


{\it Default:} normal


{\it Description:} After each time step the free surface must be advected in the direction of the velocity field. Mass conservation requires that the mesh velocity is in the normal direction of the surface. However, for steep topography or large curvature, advection in the normal direction can become ill-conditioned, and instabilities in the mesh can form. Projection of the mesh velocity onto the local vertical direction can preserve the mesh quality better, but at the cost of slightly poorer mass conservation of the domain.


{\it Possible values:} [Selection normal|vertical ]
\end{itemize}

\subsection{Parameters in section \tt Geometry model}
\label{parameters:Geometry_20model}

\begin{itemize}
\item {\it Parameter name:} {\tt Model name}
\phantomsection\label{parameters:Geometry model/Model name}


\index[prmindex]{Model name}
\index[prmindexfull]{Geometry model!Model name}
{\it Value:} box


{\it Default:} unspecified


{\it Description:} Select one of the following models:

`box': A box geometry parallel to the coordinate directions. The extent of the box in each coordinate direction is set in the parameter file. The box geometry labels its 2*dim sides as follows: in 2d, boundary indicators 0 through 3 denote the left, right, bottom and top boundaries; in 3d, boundary indicators 0 through 5 indicate left, right, front, back, bottom and top boundaries (see also the documentation of the deal.II class ``GeometryInfo''). You can also use symbolic names ``left'', ``right'', etc., to refer to these boundaries in input files.

`box with lithosphere boundary indicators': A box geometry parallel to the coordinate directions. The extent of the box in each coordinate direction is set in the parameter file. This geometry model labels its sides with 2*dim+2*(dim-1) boundary indicators: in 2d, boundary indicators 0 through 3 denote the left, right, bottom and top boundaries, while indicators4 and 5 denote the upper part of the left and right vertical boundary, respectively. In 3d, boundary indicators 0 through 5 indicate left, right, front, back, bottom and top boundaries (see also the documentation of the deal.II class ``GeometryInfo''), while indicators 6, 7, 8 and 9 denote the left, rigth, front and back upper parts of the vertical boundaries, respectively. You can also use symbolic names ``left'', ``right'', ``left lithosphere'', etc., to refer to these boundaries in input files.

Note that for a given ``Global refinement level'' and no user-specified ``Repetitions'', the lithosphere part of the mesh will be more refined. 

The additional boundary indicators for the lithosphere allow for selecting boundary conditions for the lithosphere different from those for the underlying mantle. An example application of this geometry is to prescribe a velocity on the lithospheric plates, but use open boundary conditions underneath. 

`chunk': A geometry which can be described as a chunk of a spherical shell, bounded by lines of longitude, latitude and radius. The minimum and maximum longitude, (latitude) and depth of the chunk is set in the parameter file. The chunk geometry labels its 2*dim sides as follows: ``west'' and ``east'': minimum and maximum longitude, ``south'' and ``north'': minimum and maximum latitude, ``inner'' and ``outer'': minimum and maximum radii. Names in the parameter files are as follows: Chunk (minimum || maximum) (longitude || latitude): edges of geographical quadrangle (in degrees)Chunk (inner || outer) radius: Radii at bottom and top of chunk(Longitude || Latitude || Radius) repetitions: number of cells in each coordinate direction.

`ellipsoidal chunk': A 3D chunk geometry that accounts for Earth's ellipticity (default assuming the WGS84 ellipsoid definition) which can be defined in non-coordinate directions. In the description of the ellipsoidal chunk, two of the ellipsoidal axes have the same length so that there is only a semi-major axis and a semi-minor axis. The user has two options for creating an ellipsoidal chunk geometry: 1) by defining two opposing points (SW and NE or NW and SE) a coordinate parallel ellipsoidal chunk geometry will be created. 2) by defining three points a non-coordinate parallel ellipsoidal chunk will be created. The points are defined in the input file by longitude:latitude. It is also possible to define additional subdivisions of the mesh in each direction. Faces of the model are defined as 0, west; 1,east; 2, south; 3, north; 4, inner; 5, outer. 

`sphere': Geometry model for sphere with a user specified radius. This geometry has only a single boundary, so the only valid boundary indicator to specify in the input file is ``0''. It can also be referenced by the symbolic name ``surface'' in input files.

`spherical shell': A geometry representing a spherical shell or a piece of it. Inner and outer radii are read from the parameter file in subsection 'Spherical shell'.

The model assigns boundary indicators as follows: In 2d, inner and outer boundaries get boundary indicators zero and one, and if the opening angle set in the input file is less than 360, then left and right boundaries are assigned indicators two and three. These boundaries can also be referenced using the symbolic names 'inner', 'outer' and (if applicable) 'left', 'right'.

In 3d, inner and outer indicators are treated as in 2d. If the opening angle is chosen as 90 degrees, i.e., the domain is the intersection of a spherical shell and the first octant, then indicator 2 is at the face $x=0$, 3 at $y=0$, and 4 at $z=0$. These last three boundaries can then also be referred to as 'east', 'west' and 'south' symbolically in input files.


{\it Possible values:} [Selection box|box with lithosphere boundary indicators|chunk|ellipsoidal chunk|sphere|spherical shell|unspecified ]
\end{itemize}



\subsection{Parameters in section \tt Geometry model/Box}
\label{parameters:Geometry_20model/Box}

\begin{itemize}
\item {\it Parameter name:} {\tt Box origin X coordinate}
\phantomsection\label{parameters:Geometry model/Box/Box origin X coordinate}


\index[prmindex]{Box origin X coordinate}
\index[prmindexfull]{Geometry model!Box!Box origin X coordinate}
{\it Value:} 0


{\it Default:} 0


{\it Description:} X coordinate of box origin. Units: m.


{\it Possible values:} [Double -1.79769e+308...1.79769e+308 (inclusive)]
\item {\it Parameter name:} {\tt Box origin Y coordinate}
\phantomsection\label{parameters:Geometry model/Box/Box origin Y coordinate}


\index[prmindex]{Box origin Y coordinate}
\index[prmindexfull]{Geometry model!Box!Box origin Y coordinate}
{\it Value:} 0


{\it Default:} 0


{\it Description:} Y coordinate of box origin. Units: m.


{\it Possible values:} [Double -1.79769e+308...1.79769e+308 (inclusive)]
\item {\it Parameter name:} {\tt Box origin Z coordinate}
\phantomsection\label{parameters:Geometry model/Box/Box origin Z coordinate}


\index[prmindex]{Box origin Z coordinate}
\index[prmindexfull]{Geometry model!Box!Box origin Z coordinate}
{\it Value:} 0


{\it Default:} 0


{\it Description:} Z coordinate of box origin. This value is ignored if the simulation is in 2d. Units: m.


{\it Possible values:} [Double -1.79769e+308...1.79769e+308 (inclusive)]
\item {\it Parameter name:} {\tt X extent}
\phantomsection\label{parameters:Geometry model/Box/X extent}


\index[prmindex]{X extent}
\index[prmindexfull]{Geometry model!Box!X extent}
{\it Value:} 1


{\it Default:} 1


{\it Description:} Extent of the box in x-direction. Units: m.


{\it Possible values:} [Double 0...1.79769e+308 (inclusive)]
\item {\it Parameter name:} {\tt X periodic}
\phantomsection\label{parameters:Geometry model/Box/X periodic}


\index[prmindex]{X periodic}
\index[prmindexfull]{Geometry model!Box!X periodic}
{\it Value:} false


{\it Default:} false


{\it Description:} Whether the box should be periodic in X direction


{\it Possible values:} [Bool]
\item {\it Parameter name:} {\tt X repetitions}
\phantomsection\label{parameters:Geometry model/Box/X repetitions}


\index[prmindex]{X repetitions}
\index[prmindexfull]{Geometry model!Box!X repetitions}
{\it Value:} 1


{\it Default:} 1


{\it Description:} Number of cells in X direction.


{\it Possible values:} [Integer range 1...2147483647 (inclusive)]
\item {\it Parameter name:} {\tt Y extent}
\phantomsection\label{parameters:Geometry model/Box/Y extent}


\index[prmindex]{Y extent}
\index[prmindexfull]{Geometry model!Box!Y extent}
{\it Value:} 1


{\it Default:} 1


{\it Description:} Extent of the box in y-direction. Units: m.


{\it Possible values:} [Double 0...1.79769e+308 (inclusive)]
\item {\it Parameter name:} {\tt Y periodic}
\phantomsection\label{parameters:Geometry model/Box/Y periodic}


\index[prmindex]{Y periodic}
\index[prmindexfull]{Geometry model!Box!Y periodic}
{\it Value:} false


{\it Default:} false


{\it Description:} Whether the box should be periodic in Y direction


{\it Possible values:} [Bool]
\item {\it Parameter name:} {\tt Y repetitions}
\phantomsection\label{parameters:Geometry model/Box/Y repetitions}


\index[prmindex]{Y repetitions}
\index[prmindexfull]{Geometry model!Box!Y repetitions}
{\it Value:} 1


{\it Default:} 1


{\it Description:} Number of cells in Y direction.


{\it Possible values:} [Integer range 1...2147483647 (inclusive)]
\item {\it Parameter name:} {\tt Z extent}
\phantomsection\label{parameters:Geometry model/Box/Z extent}


\index[prmindex]{Z extent}
\index[prmindexfull]{Geometry model!Box!Z extent}
{\it Value:} 1


{\it Default:} 1


{\it Description:} Extent of the box in z-direction. This value is ignored if the simulation is in 2d. Units: m.


{\it Possible values:} [Double 0...1.79769e+308 (inclusive)]
\item {\it Parameter name:} {\tt Z periodic}
\phantomsection\label{parameters:Geometry model/Box/Z periodic}


\index[prmindex]{Z periodic}
\index[prmindexfull]{Geometry model!Box!Z periodic}
{\it Value:} false


{\it Default:} false


{\it Description:} Whether the box should be periodic in Z direction


{\it Possible values:} [Bool]
\item {\it Parameter name:} {\tt Z repetitions}
\phantomsection\label{parameters:Geometry model/Box/Z repetitions}


\index[prmindex]{Z repetitions}
\index[prmindexfull]{Geometry model!Box!Z repetitions}
{\it Value:} 1


{\it Default:} 1


{\it Description:} Number of cells in Z direction.


{\it Possible values:} [Integer range 1...2147483647 (inclusive)]
\end{itemize}

\subsection{Parameters in section \tt Geometry model/Box with lithosphere boundary indicators}
\label{parameters:Geometry_20model/Box_20with_20lithosphere_20boundary_20indicators}

\begin{itemize}
\item {\it Parameter name:} {\tt Box origin X coordinate}
\phantomsection\label{parameters:Geometry model/Box with lithosphere boundary indicators/Box origin X coordinate}


\index[prmindex]{Box origin X coordinate}
\index[prmindexfull]{Geometry model!Box with lithosphere boundary indicators!Box origin X coordinate}
{\it Value:} 0


{\it Default:} 0


{\it Description:} X coordinate of box origin. Units: m.


{\it Possible values:} [Double -1.79769e+308...1.79769e+308 (inclusive)]
\item {\it Parameter name:} {\tt Box origin Y coordinate}
\phantomsection\label{parameters:Geometry model/Box with lithosphere boundary indicators/Box origin Y coordinate}


\index[prmindex]{Box origin Y coordinate}
\index[prmindexfull]{Geometry model!Box with lithosphere boundary indicators!Box origin Y coordinate}
{\it Value:} 0


{\it Default:} 0


{\it Description:} Y coordinate of box origin. Units: m.


{\it Possible values:} [Double -1.79769e+308...1.79769e+308 (inclusive)]
\item {\it Parameter name:} {\tt Box origin Z coordinate}
\phantomsection\label{parameters:Geometry model/Box with lithosphere boundary indicators/Box origin Z coordinate}


\index[prmindex]{Box origin Z coordinate}
\index[prmindexfull]{Geometry model!Box with lithosphere boundary indicators!Box origin Z coordinate}
{\it Value:} 0


{\it Default:} 0


{\it Description:} Z coordinate of box origin. This value is ignored if the simulation is in 2d. Units: m.


{\it Possible values:} [Double -1.79769e+308...1.79769e+308 (inclusive)]
\item {\it Parameter name:} {\tt Lithospheric thickness}
\phantomsection\label{parameters:Geometry model/Box with lithosphere boundary indicators/Lithospheric thickness}


\index[prmindex]{Lithospheric thickness}
\index[prmindexfull]{Geometry model!Box with lithosphere boundary indicators!Lithospheric thickness}
{\it Value:} 0.2


{\it Default:} 0.2


{\it Description:} The thickness of the lithosphere used to create additional boundary indicators to set specific boundary conditions for the lithosphere. 


{\it Possible values:} [Double 0...1.79769e+308 (inclusive)]
\item {\it Parameter name:} {\tt X extent}
\phantomsection\label{parameters:Geometry model/Box with lithosphere boundary indicators/X extent}


\index[prmindex]{X extent}
\index[prmindexfull]{Geometry model!Box with lithosphere boundary indicators!X extent}
{\it Value:} 1


{\it Default:} 1


{\it Description:} Extent of the box in x-direction. Units: m.


{\it Possible values:} [Double 0...1.79769e+308 (inclusive)]
\item {\it Parameter name:} {\tt X periodic}
\phantomsection\label{parameters:Geometry model/Box with lithosphere boundary indicators/X periodic}


\index[prmindex]{X periodic}
\index[prmindexfull]{Geometry model!Box with lithosphere boundary indicators!X periodic}
{\it Value:} false


{\it Default:} false


{\it Description:} Whether the box should be periodic in X direction.


{\it Possible values:} [Bool]
\item {\it Parameter name:} {\tt X periodic lithosphere}
\phantomsection\label{parameters:Geometry model/Box with lithosphere boundary indicators/X periodic lithosphere}


\index[prmindex]{X periodic lithosphere}
\index[prmindexfull]{Geometry model!Box with lithosphere boundary indicators!X periodic lithosphere}
{\it Value:} false


{\it Default:} false


{\it Description:} Whether the box should be periodic in X direction in the lithosphere.


{\it Possible values:} [Bool]
\item {\it Parameter name:} {\tt X repetitions}
\phantomsection\label{parameters:Geometry model/Box with lithosphere boundary indicators/X repetitions}


\index[prmindex]{X repetitions}
\index[prmindexfull]{Geometry model!Box with lithosphere boundary indicators!X repetitions}
{\it Value:} 1


{\it Default:} 1


{\it Description:} Number of cells in X direction of the lower box. The same number of repetitions will be used in the upper box.


{\it Possible values:} [Integer range 1...2147483647 (inclusive)]
\item {\it Parameter name:} {\tt Y extent}
\phantomsection\label{parameters:Geometry model/Box with lithosphere boundary indicators/Y extent}


\index[prmindex]{Y extent}
\index[prmindexfull]{Geometry model!Box with lithosphere boundary indicators!Y extent}
{\it Value:} 1


{\it Default:} 1


{\it Description:} Extent of the box in y-direction. Units: m.


{\it Possible values:} [Double 0...1.79769e+308 (inclusive)]
\item {\it Parameter name:} {\tt Y periodic}
\phantomsection\label{parameters:Geometry model/Box with lithosphere boundary indicators/Y periodic}


\index[prmindex]{Y periodic}
\index[prmindexfull]{Geometry model!Box with lithosphere boundary indicators!Y periodic}
{\it Value:} false


{\it Default:} false


{\it Description:} Whether the box should be periodic in Y direction.


{\it Possible values:} [Bool]
\item {\it Parameter name:} {\tt Y periodic lithosphere}
\phantomsection\label{parameters:Geometry model/Box with lithosphere boundary indicators/Y periodic lithosphere}


\index[prmindex]{Y periodic lithosphere}
\index[prmindexfull]{Geometry model!Box with lithosphere boundary indicators!Y periodic lithosphere}
{\it Value:} false


{\it Default:} false


{\it Description:} Whether the box should be periodic in Y direction in the lithosphere. This value is ignored if the simulation is in 2d. 


{\it Possible values:} [Bool]
\item {\it Parameter name:} {\tt Y repetitions}
\phantomsection\label{parameters:Geometry model/Box with lithosphere boundary indicators/Y repetitions}


\index[prmindex]{Y repetitions}
\index[prmindexfull]{Geometry model!Box with lithosphere boundary indicators!Y repetitions}
{\it Value:} 1


{\it Default:} 1


{\it Description:} Number of cells in Y direction of the lower box. If the simulation is in 3d, the same number of repetitions will be used in the upper box.


{\it Possible values:} [Integer range 1...2147483647 (inclusive)]
\item {\it Parameter name:} {\tt Y repetitions lithosphere}
\phantomsection\label{parameters:Geometry model/Box with lithosphere boundary indicators/Y repetitions lithosphere}


\index[prmindex]{Y repetitions lithosphere}
\index[prmindexfull]{Geometry model!Box with lithosphere boundary indicators!Y repetitions lithosphere}
{\it Value:} 1


{\it Default:} 1


{\it Description:} Number of cells in Y direction in the lithosphere. This value is ignored if the simulation is in 3d.


{\it Possible values:} [Integer range 1...2147483647 (inclusive)]
\item {\it Parameter name:} {\tt Z extent}
\phantomsection\label{parameters:Geometry model/Box with lithosphere boundary indicators/Z extent}


\index[prmindex]{Z extent}
\index[prmindexfull]{Geometry model!Box with lithosphere boundary indicators!Z extent}
{\it Value:} 1


{\it Default:} 1


{\it Description:} Extent of the box in z-direction. This value is ignored if the simulation is in 2d. Units: m.


{\it Possible values:} [Double 0...1.79769e+308 (inclusive)]
\item {\it Parameter name:} {\tt Z periodic}
\phantomsection\label{parameters:Geometry model/Box with lithosphere boundary indicators/Z periodic}


\index[prmindex]{Z periodic}
\index[prmindexfull]{Geometry model!Box with lithosphere boundary indicators!Z periodic}
{\it Value:} false


{\it Default:} false


{\it Description:} Whether the box should be periodic in Z direction. This value is ignored if the simulation is in 2d.


{\it Possible values:} [Bool]
\item {\it Parameter name:} {\tt Z repetitions}
\phantomsection\label{parameters:Geometry model/Box with lithosphere boundary indicators/Z repetitions}


\index[prmindex]{Z repetitions}
\index[prmindexfull]{Geometry model!Box with lithosphere boundary indicators!Z repetitions}
{\it Value:} 1


{\it Default:} 1


{\it Description:} Number of cells in Z direction of the lower box. This value is ignored if the simulation is in 2d.


{\it Possible values:} [Integer range 1...2147483647 (inclusive)]
\item {\it Parameter name:} {\tt Z repetitions lithosphere}
\phantomsection\label{parameters:Geometry model/Box with lithosphere boundary indicators/Z repetitions lithosphere}


\index[prmindex]{Z repetitions lithosphere}
\index[prmindexfull]{Geometry model!Box with lithosphere boundary indicators!Z repetitions lithosphere}
{\it Value:} 1


{\it Default:} 1


{\it Description:} Number of cells in Z direction in the lithosphere. This value is ignored if the simulation is in 2d.


{\it Possible values:} [Integer range 1...2147483647 (inclusive)]
\end{itemize}

\subsection{Parameters in section \tt Geometry model/Chunk}
\label{parameters:Geometry_20model/Chunk}

\begin{itemize}
\item {\it Parameter name:} {\tt Chunk inner radius}
\phantomsection\label{parameters:Geometry model/Chunk/Chunk inner radius}


\index[prmindex]{Chunk inner radius}
\index[prmindexfull]{Geometry model!Chunk!Chunk inner radius}
{\it Value:} 0


{\it Default:} 0


{\it Description:} Radius at the bottom surface of the chunk. Units: m.


{\it Possible values:} [Double 0...1.79769e+308 (inclusive)]
\item {\it Parameter name:} {\tt Chunk maximum latitude}
\phantomsection\label{parameters:Geometry model/Chunk/Chunk maximum latitude}


\index[prmindex]{Chunk maximum latitude}
\index[prmindexfull]{Geometry model!Chunk!Chunk maximum latitude}
{\it Value:} 1


{\it Default:} 1


{\it Description:} Maximum latitude of the chunk. This value is ignored if the simulation is in 2d. Units: degrees.


{\it Possible values:} [Double -90...90 (inclusive)]
\item {\it Parameter name:} {\tt Chunk maximum longitude}
\phantomsection\label{parameters:Geometry model/Chunk/Chunk maximum longitude}


\index[prmindex]{Chunk maximum longitude}
\index[prmindexfull]{Geometry model!Chunk!Chunk maximum longitude}
{\it Value:} 1


{\it Default:} 1


{\it Description:} Maximum longitude of the chunk. Units: degrees.


{\it Possible values:} [Double -180...360 (inclusive)]
\item {\it Parameter name:} {\tt Chunk minimum latitude}
\phantomsection\label{parameters:Geometry model/Chunk/Chunk minimum latitude}


\index[prmindex]{Chunk minimum latitude}
\index[prmindexfull]{Geometry model!Chunk!Chunk minimum latitude}
{\it Value:} 0


{\it Default:} 0


{\it Description:} Minimum latitude of the chunk. This value is ignored if the simulation is in 2d. Units: degrees.


{\it Possible values:} [Double -90...90 (inclusive)]
\item {\it Parameter name:} {\tt Chunk minimum longitude}
\phantomsection\label{parameters:Geometry model/Chunk/Chunk minimum longitude}


\index[prmindex]{Chunk minimum longitude}
\index[prmindexfull]{Geometry model!Chunk!Chunk minimum longitude}
{\it Value:} 0


{\it Default:} 0


{\it Description:} Minimum longitude of the chunk. Units: degrees.


{\it Possible values:} [Double -180...360 (inclusive)]
\item {\it Parameter name:} {\tt Chunk outer radius}
\phantomsection\label{parameters:Geometry model/Chunk/Chunk outer radius}


\index[prmindex]{Chunk outer radius}
\index[prmindexfull]{Geometry model!Chunk!Chunk outer radius}
{\it Value:} 1


{\it Default:} 1


{\it Description:} Radius at the top surface of the chunk. Units: m.


{\it Possible values:} [Double 0...1.79769e+308 (inclusive)]
\item {\it Parameter name:} {\tt Latitude repetitions}
\phantomsection\label{parameters:Geometry model/Chunk/Latitude repetitions}


\index[prmindex]{Latitude repetitions}
\index[prmindexfull]{Geometry model!Chunk!Latitude repetitions}
{\it Value:} 1


{\it Default:} 1


{\it Description:} Number of cells in latitude. This value is ignored if the simulation is in 2d


{\it Possible values:} [Integer range 1...2147483647 (inclusive)]
\item {\it Parameter name:} {\tt Longitude repetitions}
\phantomsection\label{parameters:Geometry model/Chunk/Longitude repetitions}


\index[prmindex]{Longitude repetitions}
\index[prmindexfull]{Geometry model!Chunk!Longitude repetitions}
{\it Value:} 1


{\it Default:} 1


{\it Description:} Number of cells in longitude.


{\it Possible values:} [Integer range 1...2147483647 (inclusive)]
\item {\it Parameter name:} {\tt Radius repetitions}
\phantomsection\label{parameters:Geometry model/Chunk/Radius repetitions}


\index[prmindex]{Radius repetitions}
\index[prmindexfull]{Geometry model!Chunk!Radius repetitions}
{\it Value:} 1


{\it Default:} 1


{\it Description:} Number of cells in radius.


{\it Possible values:} [Integer range 1...2147483647 (inclusive)]
\end{itemize}

\subsection{Parameters in section \tt Geometry model/Ellipsoidal chunk}
\label{parameters:Geometry_20model/Ellipsoidal_20chunk}

\begin{itemize}
\item {\it Parameter name:} {\tt Depth}
\phantomsection\label{parameters:Geometry model/Ellipsoidal chunk/Depth}


\index[prmindex]{Depth}
\index[prmindexfull]{Geometry model!Ellipsoidal chunk!Depth}
{\it Value:} 500000.0


{\it Default:} 500000.0


{\it Description:} Bottom depth of model region.


{\it Possible values:} [Double 0...1.79769e+308 (inclusive)]
\item {\it Parameter name:} {\tt Depth subdivisions}
\phantomsection\label{parameters:Geometry model/Ellipsoidal chunk/Depth subdivisions}


\index[prmindex]{Depth subdivisions}
\index[prmindexfull]{Geometry model!Ellipsoidal chunk!Depth subdivisions}
{\it Value:} 1


{\it Default:} 1


{\it Description:} The number of subdivisions of the coarse (initial) mesh in depth.


{\it Possible values:} [Integer range 0...2147483647 (inclusive)]
\item {\it Parameter name:} {\tt East-West subdivisions}
\phantomsection\label{parameters:Geometry model/Ellipsoidal chunk/East-West subdivisions}


\index[prmindex]{East-West subdivisions}
\index[prmindexfull]{Geometry model!Ellipsoidal chunk!East-West subdivisions}
{\it Value:} 1


{\it Default:} 1


{\it Description:} The number of subdivisions of the coarse (initial) mesh in the East-West direction.


{\it Possible values:} [Integer range 0...2147483647 (inclusive)]
\item {\it Parameter name:} {\tt Eccentricity}
\phantomsection\label{parameters:Geometry model/Ellipsoidal chunk/Eccentricity}


\index[prmindex]{Eccentricity}
\index[prmindexfull]{Geometry model!Ellipsoidal chunk!Eccentricity}
{\it Value:} 8.1819190842622e-2


{\it Default:} 8.1819190842622e-2


{\it Description:} Eccentricity of the ellipsoid. Zero is a perfect sphere, default (8.1819190842622e-2) is WGS84.


{\it Possible values:} [Double 0...1.79769e+308 (inclusive)]
\item {\it Parameter name:} {\tt NE corner}
\phantomsection\label{parameters:Geometry model/Ellipsoidal chunk/NE corner}


\index[prmindex]{NE corner}
\index[prmindexfull]{Geometry model!Ellipsoidal chunk!NE corner}
{\it Value:} 


{\it Default:} 


{\it Description:} Longitude:latitude in degrees of the North-East corner point of model region.The North-East direction is positive. If one of the three corners is not providedthe missing corner value will be calculated so all faces are parallel.


{\it Possible values:} [Anything]
\item {\it Parameter name:} {\tt NW corner}
\phantomsection\label{parameters:Geometry model/Ellipsoidal chunk/NW corner}


\index[prmindex]{NW corner}
\index[prmindexfull]{Geometry model!Ellipsoidal chunk!NW corner}
{\it Value:} 


{\it Default:} 


{\it Description:} Longitude:latitude in degrees of the North-West corner point of model region. The North-East direction is positive. If one of the three corners is not providedthe missing corner value will be calculated so all faces are parallel.


{\it Possible values:} [Anything]
\item {\it Parameter name:} {\tt North-South subdivisions}
\phantomsection\label{parameters:Geometry model/Ellipsoidal chunk/North-South subdivisions}


\index[prmindex]{North-South subdivisions}
\index[prmindexfull]{Geometry model!Ellipsoidal chunk!North-South subdivisions}
{\it Value:} 1


{\it Default:} 1


{\it Description:} The number of subdivisions of the coarse (initial) mesh in the North-South direction.


{\it Possible values:} [Integer range 0...2147483647 (inclusive)]
\item {\it Parameter name:} {\tt SE corner}
\phantomsection\label{parameters:Geometry model/Ellipsoidal chunk/SE corner}


\index[prmindex]{SE corner}
\index[prmindexfull]{Geometry model!Ellipsoidal chunk!SE corner}
{\it Value:} 


{\it Default:} 


{\it Description:} Longitude:latitude in degrees of the South-East corner point of model region. The North-East direction is positive. If one of the three corners is not providedthe missing corner value will be calculated so all faces are parallel.


{\it Possible values:} [Anything]
\item {\it Parameter name:} {\tt SW corner}
\phantomsection\label{parameters:Geometry model/Ellipsoidal chunk/SW corner}


\index[prmindex]{SW corner}
\index[prmindexfull]{Geometry model!Ellipsoidal chunk!SW corner}
{\it Value:} 


{\it Default:} 


{\it Description:} Longitude:latitude in degrees of the South-West corner point of model region. The North-East direction is positive. If one of the three corners is not providedthe missing corner value will be calculated so all faces are parallel.


{\it Possible values:} [Anything]
\item {\it Parameter name:} {\tt Semi-major axis}
\phantomsection\label{parameters:Geometry model/Ellipsoidal chunk/Semi-major axis}


\index[prmindex]{Semi-major axis}
\index[prmindexfull]{Geometry model!Ellipsoidal chunk!Semi-major axis}
{\it Value:} 6378137.0


{\it Default:} 6378137.0


{\it Description:} The semi-major axis (a) of an ellipsoid. This is the radius for a sphere (eccentricity=0). Default WGS84 semi-major axis.


{\it Possible values:} [Double 0...1.79769e+308 (inclusive)]
\end{itemize}

\subsection{Parameters in section \tt Geometry model/Sphere}
\label{parameters:Geometry_20model/Sphere}

\begin{itemize}
\item {\it Parameter name:} {\tt Radius}
\phantomsection\label{parameters:Geometry model/Sphere/Radius}


\index[prmindex]{Radius}
\index[prmindexfull]{Geometry model!Sphere!Radius}
{\it Value:} 6371000


{\it Default:} 6371000


{\it Description:} Radius of the sphere. Units: m.


{\it Possible values:} [Double 0...1.79769e+308 (inclusive)]
\end{itemize}

\subsection{Parameters in section \tt Geometry model/Spherical shell}
\label{parameters:Geometry_20model/Spherical_20shell}

\begin{itemize}
\item {\it Parameter name:} {\tt Cells along circumference}
\phantomsection\label{parameters:Geometry model/Spherical shell/Cells along circumference}


\index[prmindex]{Cells along circumference}
\index[prmindexfull]{Geometry model!Spherical shell!Cells along circumference}
{\it Value:} 0


{\it Default:} 0


{\it Description:} The number of cells in circumferential direction that are created in the coarse mesh in 2d. If zero, this number is chosen automatically in a way that produces meshes in which cells have a reasonable aspect ratio for models in which the depth of the mantle is roughly that of the Earth. For planets with much shallower mantles and larger cores, you may want to chose a larger number to avoid cells that are elongated in tangential and compressed in radial direction.

In 3d, the number of cells is computed differently and does not have an easy interpretation. Valid values for this parameter in 3d are 0 (let this class choose), 6, 12 and 96. Other possible values may be discussed in the documentation of the deal.II function GridGenerator::hyper\_shell. The parameter is best left at its default in 3d.

In either case, this parameter is ignored unless the opening angle of the domain is 360 degrees.


{\it Possible values:} [Integer range 0...2147483647 (inclusive)]
\item {\it Parameter name:} {\tt Inner radius}
\phantomsection\label{parameters:Geometry model/Spherical shell/Inner radius}


\index[prmindex]{Inner radius}
\index[prmindexfull]{Geometry model!Spherical shell!Inner radius}
{\it Value:} 3481000


{\it Default:} 3481000


{\it Description:} Inner radius of the spherical shell. Units: m.


{\it Possible values:} [Double 0...1.79769e+308 (inclusive)]
\item {\it Parameter name:} {\tt Opening angle}
\phantomsection\label{parameters:Geometry model/Spherical shell/Opening angle}


\index[prmindex]{Opening angle}
\index[prmindexfull]{Geometry model!Spherical shell!Opening angle}
{\it Value:} 360


{\it Default:} 360


{\it Description:} Opening angle in degrees of the section of the shell that we want to build. Units: degrees.


{\it Possible values:} [Double 0...360 (inclusive)]
\item {\it Parameter name:} {\tt Outer radius}
\phantomsection\label{parameters:Geometry model/Spherical shell/Outer radius}


\index[prmindex]{Outer radius}
\index[prmindexfull]{Geometry model!Spherical shell!Outer radius}
{\it Value:} 6336000


{\it Default:} 6336000


{\it Description:} Outer radius of the spherical shell. Units: m.


{\it Possible values:} [Double 0...1.79769e+308 (inclusive)]
\end{itemize}

\subsection{Parameters in section \tt Gravity model}
\label{parameters:Gravity_20model}

\begin{itemize}
\item {\it Parameter name:} {\tt Model name}
\phantomsection\label{parameters:Gravity model/Model name}


\index[prmindex]{Model name}
\index[prmindexfull]{Gravity model!Model name}
{\it Value:} vertical


{\it Default:} unspecified


{\it Description:} Select one of the following models:

`function': Gravity is given in terms of an explicit formula that is elaborated in the parameters in section ``Gravity model|Function''. The format of these functions follows the syntax understood by the muparser library, see Section~\ref{sec:muparser-format}.

`radial constant': A gravity model in which the gravity direction is radially inward and at constant magnitude. The magnitude is read from the parameter file in subsection 'Radial constant'.

`radial earth-like': A gravity model in which the gravity direction is radially inward and with a magnitude that matches that of the earth at the core-mantle boundary as well as at the surface and in between is physically correct under the assumption of a constant density.

`radial linear': A gravity model which is radially inward, where the magnitudedecreases linearly with depth down to zero at the maximal depth the geometry returns, as you would get with a constantdensity spherical domain. (Note that this would be for a full sphere, not a spherical shell.) The magnitude of gravity at the surface is read from the input file in a section ``Gravity model/Radial linear''.

`vertical': A gravity model in which the gravity direction is vertically downward and at a constant magnitude by default equal to one.


{\it Possible values:} [Selection function|radial constant|radial earth-like|radial linear|vertical|unspecified ]
\end{itemize}



\subsection{Parameters in section \tt Gravity model/Function}
\label{parameters:Gravity_20model/Function}

\begin{itemize}
\item {\it Parameter name:} {\tt Function constants}
\phantomsection\label{parameters:Gravity model/Function/Function constants}


\index[prmindex]{Function constants}
\index[prmindexfull]{Gravity model!Function!Function constants}
{\it Value:} 


{\it Default:} 


{\it Description:} Sometimes it is convenient to use symbolic constants in the expression that describes the function, rather than having to use its numeric value everywhere the constant appears. These values can be defined using this parameter, in the form `var1=value1, var2=value2, ...'.

A typical example would be to set this runtime parameter to `pi=3.1415926536' and then use `pi' in the expression of the actual formula. (That said, for convenience this class actually defines both `pi' and `Pi' by default, but you get the idea.)


{\it Possible values:} [Anything]
\item {\it Parameter name:} {\tt Function expression}
\phantomsection\label{parameters:Gravity model/Function/Function expression}


\index[prmindex]{Function expression}
\index[prmindexfull]{Gravity model!Function!Function expression}
{\it Value:} 0; 0


{\it Default:} 0; 0


{\it Description:} The formula that denotes the function you want to evaluate for particular values of the independent variables. This expression may contain any of the usual operations such as addition or multiplication, as well as all of the common functions such as `sin' or `cos'. In addition, it may contain expressions like `if(x>0, 1, -1)' where the expression evaluates to the second argument if the first argument is true, and to the third argument otherwise. For a full overview of possible expressions accepted see the documentation of the muparser library at http://muparser.beltoforion.de/.

If the function you are describing represents a vector-valued function with multiple components, then separate the expressions for individual components by a semicolon.


{\it Possible values:} [Anything]
\item {\it Parameter name:} {\tt Variable names}
\phantomsection\label{parameters:Gravity model/Function/Variable names}


\index[prmindex]{Variable names}
\index[prmindexfull]{Gravity model!Function!Variable names}
{\it Value:} x,y,t


{\it Default:} x,y,t


{\it Description:} The name of the variables as they will be used in the function, separated by commas. By default, the names of variables at which the function will be evaluated is `x' (in 1d), `x,y' (in 2d) or `x,y,z' (in 3d) for spatial coordinates and `t' for time. You can then use these variable names in your function expression and they will be replaced by the values of these variables at which the function is currently evaluated. However, you can also choose a different set of names for the independent variables at which to evaluate your function expression. For example, if you work in spherical coordinates, you may wish to set this input parameter to `r,phi,theta,t' and then use these variable names in your function expression.


{\it Possible values:} [Anything]
\end{itemize}

\subsection{Parameters in section \tt Gravity model/Radial constant}
\label{parameters:Gravity_20model/Radial_20constant}

\begin{itemize}
\item {\it Parameter name:} {\tt Magnitude}
\phantomsection\label{parameters:Gravity model/Radial constant/Magnitude}


\index[prmindex]{Magnitude}
\index[prmindexfull]{Gravity model!Radial constant!Magnitude}
{\it Value:} 9.81


{\it Default:} 9.81


{\it Description:} Magnitude of the gravity vector in $m/s^2$. The direction is always radially inward towards the center of the earth.


{\it Possible values:} [Double 0...1.79769e+308 (inclusive)]
\end{itemize}

\subsection{Parameters in section \tt Gravity model/Radial linear}
\label{parameters:Gravity_20model/Radial_20linear}

\begin{itemize}
\item {\it Parameter name:} {\tt Magnitude at surface}
\phantomsection\label{parameters:Gravity model/Radial linear/Magnitude at surface}


\index[prmindex]{Magnitude at surface}
\index[prmindexfull]{Gravity model!Radial linear!Magnitude at surface}
{\it Value:} 9.8


{\it Default:} 9.8


{\it Description:} Magnitude of the radial gravity vector at the surface of the domain. Units: $m/s^2$


{\it Possible values:} [Double 0...1.79769e+308 (inclusive)]
\end{itemize}

\subsection{Parameters in section \tt Gravity model/Vertical}
\label{parameters:Gravity_20model/Vertical}

\begin{itemize}
\item {\it Parameter name:} {\tt Magnitude}
\phantomsection\label{parameters:Gravity model/Vertical/Magnitude}


\index[prmindex]{Magnitude}
\index[prmindexfull]{Gravity model!Vertical!Magnitude}
{\it Value:} 1


{\it Default:} 1


{\it Description:} Value of the gravity vector in $m/s^2$ directed along negative y (2D) or z (3D) axis.


{\it Possible values:} [Double 0...1.79769e+308 (inclusive)]
\end{itemize}

\subsection{Parameters in section \tt Heating model}
\label{parameters:Heating_20model}

\begin{itemize}
\item {\it Parameter name:} {\tt List of model names}
\phantomsection\label{parameters:Heating model/List of model names}


\index[prmindex]{List of model names}
\index[prmindexfull]{Heating model!List of model names}
{\it Value:} 


{\it Default:} 


{\it Description:} A comma separated list of heating models that will be used to calculate the heating terms in the energyequation. The results of each of these criteria , i.e., the heating source terms and the latent heat terms for theleft hand side will be added.

The following heating models are available:

`adiabatic heating': Implementation of a standard and a simplified model ofadiabatic heating.

`constant heating': Implementation of a model in which the heating rate is constant.

`function': Implementation of a model in which the heating rate is given in terms of an explicit formula that is elaborated in the parameters in section ``Heating model|Function''. The format of these functions follows the syntax understood by the muparser library, see Section~\ref{sec:muparser-format}.

The formula is interpreted as having units W/kg.

Since the symbol $t$ indicating time may appear in the formulas for the heating rate, it is interpreted as having units seconds unless the global parameter ``Use years in output instead of seconds'' is set.

`latent heat': Implementation of a standard model for latent heat.

`radioactive decay': Implementation of a model in which the internal heating rate is radioactive decaying in the following rule:
\[(\text{initial concentration})\cdot 0.5^{\text{time}/(\text{half life})}\]
The crust and mantle can have different concentrations, and the crust can be defined either by depth or by a certain compositional field.
The formula is interpreted as having units W/kg.

`shear heating': Implementation of a standard model for shear heating.


{\it Possible values:} [MultipleSelection adiabatic heating|constant heating|function|latent heat|radioactive decay|shear heating ]
\item {\it Parameter name:} {\tt Model name}
\phantomsection\label{parameters:Heating model/Model name}


\index[prmindex]{Model name}
\index[prmindexfull]{Heating model!Model name}
{\it Value:} unspecified


{\it Default:} unspecified


{\it Description:} Select one of the following models:

Warning: This is the old formulation of specifying heating models and shouldn't be used. Please use 'List ofmodel names' instead.`adiabatic heating': Implementation of a standard and a simplified model ofadiabatic heating.

`constant heating': Implementation of a model in which the heating rate is constant.

`function': Implementation of a model in which the heating rate is given in terms of an explicit formula that is elaborated in the parameters in section ``Heating model|Function''. The format of these functions follows the syntax understood by the muparser library, see Section~\ref{sec:muparser-format}.

The formula is interpreted as having units W/kg.

Since the symbol $t$ indicating time may appear in the formulas for the heating rate, it is interpreted as having units seconds unless the global parameter ``Use years in output instead of seconds'' is set.

`latent heat': Implementation of a standard model for latent heat.

`radioactive decay': Implementation of a model in which the internal heating rate is radioactive decaying in the following rule:
\[(\text{initial concentration})\cdot 0.5^{\text{time}/(\text{half life})}\]
The crust and mantle can have different concentrations, and the crust can be defined either by depth or by a certain compositional field.
The formula is interpreted as having units W/kg.

`shear heating': Implementation of a standard model for shear heating.


{\it Possible values:} [Selection adiabatic heating|constant heating|function|latent heat|radioactive decay|shear heating|unspecified ]
\end{itemize}



\subsection{Parameters in section \tt Heating model/Adiabatic heating}
\label{parameters:Heating_20model/Adiabatic_20heating}

\begin{itemize}
\item {\it Parameter name:} {\tt Use simplified adiabatic heating}
\phantomsection\label{parameters:Heating model/Adiabatic heating/Use simplified adiabatic heating}


\index[prmindex]{Use simplified adiabatic heating}
\index[prmindexfull]{Heating model!Adiabatic heating!Use simplified adiabatic heating}
{\it Value:} false


{\it Default:} false


{\it Description:} A flag indicating whether the adiabatic heating should be simplified from $\alpha T (\mathbf u \cdot \nabla p)$ to $ \alpha \rho T (\mathbf u \cdot \mathbf g) $.


{\it Possible values:} [Bool]
\end{itemize}

\subsection{Parameters in section \tt Heating model/Constant heating}
\label{parameters:Heating_20model/Constant_20heating}

\begin{itemize}
\item {\it Parameter name:} {\tt Radiogenic heating rate}
\phantomsection\label{parameters:Heating model/Constant heating/Radiogenic heating rate}


\index[prmindex]{Radiogenic heating rate}
\index[prmindexfull]{Heating model!Constant heating!Radiogenic heating rate}
{\it Value:} 0e0


{\it Default:} 0e0


{\it Description:} The specific rate of heating due to radioactive decay (or other bulk sources you may want to describe). This parameter corresponds to the variable $H$ in the temperature equation stated in the manual, and the heating term is $
ho H$. Units: W/kg.


{\it Possible values:} [Double 0...1.79769e+308 (inclusive)]
\end{itemize}

\subsection{Parameters in section \tt Heating model/Function}
\label{parameters:Heating_20model/Function}

\begin{itemize}
\item {\it Parameter name:} {\tt Function constants}
\phantomsection\label{parameters:Heating model/Function/Function constants}


\index[prmindex]{Function constants}
\index[prmindexfull]{Heating model!Function!Function constants}
{\it Value:} 


{\it Default:} 


{\it Description:} Sometimes it is convenient to use symbolic constants in the expression that describes the function, rather than having to use its numeric value everywhere the constant appears. These values can be defined using this parameter, in the form `var1=value1, var2=value2, ...'.

A typical example would be to set this runtime parameter to `pi=3.1415926536' and then use `pi' in the expression of the actual formula. (That said, for convenience this class actually defines both `pi' and `Pi' by default, but you get the idea.)


{\it Possible values:} [Anything]
\item {\it Parameter name:} {\tt Function expression}
\phantomsection\label{parameters:Heating model/Function/Function expression}


\index[prmindex]{Function expression}
\index[prmindexfull]{Heating model!Function!Function expression}
{\it Value:} 0


{\it Default:} 0


{\it Description:} The formula that denotes the function you want to evaluate for particular values of the independent variables. This expression may contain any of the usual operations such as addition or multiplication, as well as all of the common functions such as `sin' or `cos'. In addition, it may contain expressions like `if(x>0, 1, -1)' where the expression evaluates to the second argument if the first argument is true, and to the third argument otherwise. For a full overview of possible expressions accepted see the documentation of the muparser library at http://muparser.beltoforion.de/.

If the function you are describing represents a vector-valued function with multiple components, then separate the expressions for individual components by a semicolon.


{\it Possible values:} [Anything]
\item {\it Parameter name:} {\tt Variable names}
\phantomsection\label{parameters:Heating model/Function/Variable names}


\index[prmindex]{Variable names}
\index[prmindexfull]{Heating model!Function!Variable names}
{\it Value:} x,y,t


{\it Default:} x,y,t


{\it Description:} The name of the variables as they will be used in the function, separated by commas. By default, the names of variables at which the function will be evaluated is `x' (in 1d), `x,y' (in 2d) or `x,y,z' (in 3d) for spatial coordinates and `t' for time. You can then use these variable names in your function expression and they will be replaced by the values of these variables at which the function is currently evaluated. However, you can also choose a different set of names for the independent variables at which to evaluate your function expression. For example, if you work in spherical coordinates, you may wish to set this input parameter to `r,phi,theta,t' and then use these variable names in your function expression.


{\it Possible values:} [Anything]
\end{itemize}

\subsection{Parameters in section \tt Heating model/Latent heat}
\label{parameters:Heating_20model/Latent_20heat}


\subsection{Parameters in section \tt Heating model/Radioactive decay}
\label{parameters:Heating_20model/Radioactive_20decay}

\begin{itemize}
\item {\it Parameter name:} {\tt Crust composition number}
\phantomsection\label{parameters:Heating model/Radioactive decay/Crust composition number}


\index[prmindex]{Crust composition number}
\index[prmindexfull]{Heating model!Radioactive decay!Crust composition number}
{\it Value:} 0


{\it Default:} 0


{\it Description:} Which composition field should be treated as crust


{\it Possible values:} [Integer range 0...2147483647 (inclusive)]
\item {\it Parameter name:} {\tt Crust defined by composition}
\phantomsection\label{parameters:Heating model/Radioactive decay/Crust defined by composition}


\index[prmindex]{Crust defined by composition}
\index[prmindexfull]{Heating model!Radioactive decay!Crust defined by composition}
{\it Value:} false


{\it Default:} false


{\it Description:} Whether crust defined by composition or depth


{\it Possible values:} [Bool]
\item {\it Parameter name:} {\tt Crust depth}
\phantomsection\label{parameters:Heating model/Radioactive decay/Crust depth}


\index[prmindex]{Crust depth}
\index[prmindexfull]{Heating model!Radioactive decay!Crust depth}
{\it Value:} 0


{\it Default:} 0


{\it Description:} Depth of the crust when crust if defined by depth. Units: m


{\it Possible values:} [Double -1.79769e+308...1.79769e+308 (inclusive)]
\item {\it Parameter name:} {\tt Half decay times}
\phantomsection\label{parameters:Heating model/Radioactive decay/Half decay times}


\index[prmindex]{Half decay times}
\index[prmindexfull]{Heating model!Radioactive decay!Half decay times}
{\it Value:} 


{\it Default:} 


{\it Description:} Half decay times. Units: (Seconds), or (Years) if set 'use years instead of seconds'.


{\it Possible values:} [List list of [Double 0...1.79769e+308 (inclusive)] of length 0...4294967295 (inclusive)]
\item {\it Parameter name:} {\tt Heating rates}
\phantomsection\label{parameters:Heating model/Radioactive decay/Heating rates}


\index[prmindex]{Heating rates}
\index[prmindexfull]{Heating model!Radioactive decay!Heating rates}
{\it Value:} 


{\it Default:} 


{\it Description:} Heating rates of different elements (W/kg)


{\it Possible values:} [List list of [Double -1.79769e+308...1.79769e+308 (inclusive)] of length 0...4294967295 (inclusive)]
\item {\it Parameter name:} {\tt Initial concentrations crust}
\phantomsection\label{parameters:Heating model/Radioactive decay/Initial concentrations crust}


\index[prmindex]{Initial concentrations crust}
\index[prmindexfull]{Heating model!Radioactive decay!Initial concentrations crust}
{\it Value:} 


{\it Default:} 


{\it Description:} Initial concentrations of different elements (ppm)


{\it Possible values:} [List list of [Double 0...1.79769e+308 (inclusive)] of length 0...4294967295 (inclusive)]
\item {\it Parameter name:} {\tt Initial concentrations mantle}
\phantomsection\label{parameters:Heating model/Radioactive decay/Initial concentrations mantle}


\index[prmindex]{Initial concentrations mantle}
\index[prmindexfull]{Heating model!Radioactive decay!Initial concentrations mantle}
{\it Value:} 


{\it Default:} 


{\it Description:} Initial concentrations of different elements (ppm)


{\it Possible values:} [List list of [Double 0...1.79769e+308 (inclusive)] of length 0...4294967295 (inclusive)]
\item {\it Parameter name:} {\tt Number of elements}
\phantomsection\label{parameters:Heating model/Radioactive decay/Number of elements}


\index[prmindex]{Number of elements}
\index[prmindexfull]{Heating model!Radioactive decay!Number of elements}
{\it Value:} 0


{\it Default:} 0


{\it Description:} Number of radioactive elements


{\it Possible values:} [Integer range 0...2147483647 (inclusive)]
\end{itemize}

\subsection{Parameters in section \tt Heating model/Shear heating}
\label{parameters:Heating_20model/Shear_20heating}


\subsection{Parameters in section \tt Initial conditions}
\label{parameters:Initial_20conditions}

\begin{itemize}
\item {\it Parameter name:} {\tt Model name}
\phantomsection\label{parameters:Initial conditions/Model name}


\index[prmindex]{Model name}
\index[prmindexfull]{Initial conditions!Model name}
{\it Value:} perturbed box


{\it Default:} unspecified


{\it Description:} Select one of the following models:

`S40RTS perturbation': An initial temperature field in which the temperature is perturbed following the S20RTS or S40RTS shear wave velocity model by Ritsema and others, which can be downloaded here \url{http://www.earth.lsa.umich.edu/~jritsema/research.html}. Information on the vs model can be found in Ritsema, J., Deuss, A., van Heijst, H.J. \& Woodhouse, J.H., 2011. S40RTS: a degree-40 shear-velocity model for the mantle from new Rayleigh wave dispersion, teleseismic traveltime and normal-mode splitting function measurements, Geophys. J. Int. 184, 1223-1236. The scaling between the shear wave perturbation and the temperature perturbation can be set by the user with the 'vs to density scaling' parameter and the 'Thermal expansion coefficient in initial temperature scaling' parameter. The scaling is as follows: $\delta ln \rho (r,\theta,\phi) = \xi \cdot \delta ln v_s(r,\theta, \phi)$ and $\delta T(r,\theta,\phi) = - \frac{1}{\alpha} \delta ln \rho(r,\theta,\phi)$. $\xi$ is the 'vs to density scaling' parameter and $\alpha$ is the 'Thermal expansion coefficient in initial temperature scaling' parameter. The temperature perturbation is added to an otherwise constant temperature (incompressible model) or adiabatic reference profile (compressible model). If a depth is specified in 'Remove temperature heterogeneity down to specified depth', there is no temperature perturbation prescribed down to that depth.

`SAVANI perturbation': An initial temperature field in which the temperature is perturbed following the SAVANI shear wave velocity model by Auer and others, which can be downloaded here \url{http://n.ethz.ch/~auerl/savani.tar.bz2}. Information on the vs model can be found in Auer, L., Boschi, L., Becker, T.W., Nissen-Meyer, T. \& Giardini, D., 2014. Savani:A variable resolution whole‐mantle model of anisotropic shear velocityvariations based on multiple data sets. Journal of GeophysicalResearch: Solid Earth 119.4 (2014): 3006-3034. The scaling between the shear wave perturbation and the temperature perturbation can be set by the user with the 'vs to density scaling' parameter and the 'Thermal expansion coefficient in initial temperature scaling' parameter. The scaling is as follows: $\delta ln \rho (r,\theta,\phi) = \xi \cdot \delta ln v_s(r,\theta, \phi)$ and $\delta T(r,\theta,\phi) = - \frac{1}{\alpha} \delta ln \rho(r,\theta,\phi)$. $\xi$ is the 'vs to density scaling' parameter and $\alpha$ is the 'Thermal expansion coefficient in initial temperature scaling' parameter. The temperature perturbation is added to an otherwise constant temperature (incompressible model) or adiabatic reference profile (compressible model).

`adiabatic': Temperature is prescribed as an adiabatic profile with upper and lower thermal boundary layers, whose ages are given as input parameters.

`ascii data': Implementation of a model in which the initial temperature is derived from files containing data in ascii format. Note the required format of the input data: The first lines may contain any number of commentsif they begin with '\#', but one of these lines needs tocontain the number of grid points in each dimension asfor example '\# POINTS: 3 3'.The order of the data columns has to be 'x', 'y', 'Temperature [K]' in a 2d model and  'x', 'y', 'z', 'Temperature [K]' in a 3d model, which means that there has to be a single column containing the temperature. Note that the data in the input files need to be sorted in a specific order: the first coordinate needs to ascend first, followed by the second and the third at last in order to assign the correct data to the prescribed coordinates.If you use a spherical model, then the data will still be handled as cartesian,however the assumed grid changes. 'x' will be replaced by the radial distance of the point to the bottom of the model, 'y' by the azimuth angle and 'z' by the polar angle measured positive from the north pole. The grid will be assumed to be a latitude-longitude grid. Note that the order of spherical coordinates is 'r', 'phi', 'theta' and not 'r', 'theta', 'phi', since this allows for dimension independent expressions. 

`function': Specify the initial temperature in terms of an explicit formula. The format of these functions follows the syntax understood by the muparser library, see Section~\ref{sec:muparser-format}.

`harmonic perturbation': An initial temperature field in which the temperature is perturbed following a harmonic function (spherical harmonic or sine depending on geometry and dimension) in lateral and radial direction from an otherwise constant temperature (incompressible model) or adiabatic reference profile (compressible model).

`inclusion shape perturbation': An initial temperature field in which there is an inclusion in a constant-temperature box field. The size, shape, gradient, position, and temperature of the inclusion are defined by parameters.

`mandelbox': Fractal-shaped temperature field.

`perturbed box': An initial temperature field in which the temperature is perturbed slightly from an otherwise constant value equal to one. The perturbation is chosen in such a way that the initial temperature is constant to one along the entire boundary.

`polar box': An initial temperature field in which the temperature is perturbed slightly from an otherwise constant value equal to one. The perturbation is such that there are two poles on opposing corners of the box. 

`solidus': This is a temperature initial condition that starts the model close to solidus, it also contains a user defined lithoshpere thickness and with perturbations  in both lithosphere thickness and temperature based on spherical harmonic functions. It was used as the initial condition of early Mars after the freezing of the magma ocean, using the solidus from Parmentier et al., Melt-solid segregation, Fractional magma ocean solidification, and implications for longterm planetary evolution. Luna and Planetary Science, 2007.

`spherical gaussian perturbation': An initial temperature field in which the temperature is perturbed by a single Gaussian added to an otherwise spherically symmetric state. Additional parameters are read from the parameter file in subsection 'Spherical gaussian perturbation'.

`spherical hexagonal perturbation': An initial temperature field in which the temperature is perturbed following an $N$-fold pattern in a specified direction from an otherwise spherically symmetric state. The class's name comes from previous versions when the only option was $N=6$.


{\it Possible values:} [Selection S40RTS perturbation|SAVANI perturbation|adiabatic|ascii data|function|harmonic perturbation|inclusion shape perturbation|mandelbox|perturbed box|polar box|solidus|spherical gaussian perturbation|spherical hexagonal perturbation|unspecified ]
\end{itemize}



\subsection{Parameters in section \tt Initial conditions/Adiabatic}
\label{parameters:Initial_20conditions/Adiabatic}

\begin{itemize}
\item {\it Parameter name:} {\tt Age bottom boundary layer}
\phantomsection\label{parameters:Initial conditions/Adiabatic/Age bottom boundary layer}


\index[prmindex]{Age bottom boundary layer}
\index[prmindexfull]{Initial conditions!Adiabatic!Age bottom boundary layer}
{\it Value:} 0e0


{\it Default:} 0e0


{\it Description:} The age of the lower thermal boundary layer, used for the calculation of the half-space cooling model temperature. Units: years if the 'Use years in output instead of seconds' parameter is set; seconds otherwise.


{\it Possible values:} [Double 0...1.79769e+308 (inclusive)]
\item {\it Parameter name:} {\tt Age top boundary layer}
\phantomsection\label{parameters:Initial conditions/Adiabatic/Age top boundary layer}


\index[prmindex]{Age top boundary layer}
\index[prmindexfull]{Initial conditions!Adiabatic!Age top boundary layer}
{\it Value:} 0e0


{\it Default:} 0e0


{\it Description:} The age of the upper thermal boundary layer, used for the calculation of the half-space cooling model temperature. Units: years if the 'Use years in output instead of seconds' parameter is set; seconds otherwise.


{\it Possible values:} [Double 0...1.79769e+308 (inclusive)]
\item {\it Parameter name:} {\tt Amplitude}
\phantomsection\label{parameters:Initial conditions/Adiabatic/Amplitude}


\index[prmindex]{Amplitude}
\index[prmindexfull]{Initial conditions!Adiabatic!Amplitude}
{\it Value:} 0e0


{\it Default:} 0e0


{\it Description:} The amplitude (in K) of the initial spherical temperature perturbation at the bottom of the model domain. This perturbation will be added to the adiabatic temperature profile, but not to the bottom thermal boundary layer. Instead, the maximum of the perturbation and the bottom boundary layer temperature will be used.


{\it Possible values:} [Double 0...1.79769e+308 (inclusive)]
\item {\it Parameter name:} {\tt Position}
\phantomsection\label{parameters:Initial conditions/Adiabatic/Position}


\index[prmindex]{Position}
\index[prmindexfull]{Initial conditions!Adiabatic!Position}
{\it Value:} center


{\it Default:} center


{\it Description:} Where the initial temperature perturbation should be placed. If 'center' is given, then the perturbation will be centered along a 'midpoint' of some sort of the bottom boundary. For example, in the case of a box geometry, this is the center of the bottom face; in the case of a spherical shell geometry, it is along the inner surface halfway between the bounding radial lines.


{\it Possible values:} [Selection center ]
\item {\it Parameter name:} {\tt Radius}
\phantomsection\label{parameters:Initial conditions/Adiabatic/Radius}


\index[prmindex]{Radius}
\index[prmindexfull]{Initial conditions!Adiabatic!Radius}
{\it Value:} 0e0


{\it Default:} 0e0


{\it Description:} The Radius (in m) of the initial spherical temperature perturbation at the bottom of the model domain.


{\it Possible values:} [Double 0...1.79769e+308 (inclusive)]
\item {\it Parameter name:} {\tt Subadiabaticity}
\phantomsection\label{parameters:Initial conditions/Adiabatic/Subadiabaticity}


\index[prmindex]{Subadiabaticity}
\index[prmindexfull]{Initial conditions!Adiabatic!Subadiabaticity}
{\it Value:} 0e0


{\it Default:} 0e0


{\it Description:} If this value is larger than 0, the initial temperature profile will not be adiabatic, but subadiabatic. This value gives the maximal deviation from adiabaticity. Set to 0 for an adiabatic temperature profile. Units: K.

The function object in the Function subsection represents the compositional fields that will be used as a reference profile for calculating the thermal diffusivity. This function is one-dimensional and depends only on depth. The format of this functions follows the syntax understood by the muparser library, see Section~\ref{sec:muparser-format}.


{\it Possible values:} [Double 0...1.79769e+308 (inclusive)]
\end{itemize}



\subsection{Parameters in section \tt Initial conditions/Adiabatic/Function}
\label{parameters:Initial_20conditions/Adiabatic/Function}

\begin{itemize}
\item {\it Parameter name:} {\tt Function constants}
\phantomsection\label{parameters:Initial conditions/Adiabatic/Function/Function constants}


\index[prmindex]{Function constants}
\index[prmindexfull]{Initial conditions!Adiabatic!Function/Function constants}
{\it Value:} 


{\it Default:} 


{\it Description:} Sometimes it is convenient to use symbolic constants in the expression that describes the function, rather than having to use its numeric value everywhere the constant appears. These values can be defined using this parameter, in the form `var1=value1, var2=value2, ...'.

A typical example would be to set this runtime parameter to `pi=3.1415926536' and then use `pi' in the expression of the actual formula. (That said, for convenience this class actually defines both `pi' and `Pi' by default, but you get the idea.)


{\it Possible values:} [Anything]
\item {\it Parameter name:} {\tt Function expression}
\phantomsection\label{parameters:Initial conditions/Adiabatic/Function/Function expression}


\index[prmindex]{Function expression}
\index[prmindexfull]{Initial conditions!Adiabatic!Function/Function expression}
{\it Value:} 0


{\it Default:} 0


{\it Description:} The formula that denotes the function you want to evaluate for particular values of the independent variables. This expression may contain any of the usual operations such as addition or multiplication, as well as all of the common functions such as `sin' or `cos'. In addition, it may contain expressions like `if(x>0, 1, -1)' where the expression evaluates to the second argument if the first argument is true, and to the third argument otherwise. For a full overview of possible expressions accepted see the documentation of the muparser library at http://muparser.beltoforion.de/.

If the function you are describing represents a vector-valued function with multiple components, then separate the expressions for individual components by a semicolon.


{\it Possible values:} [Anything]
\item {\it Parameter name:} {\tt Variable names}
\phantomsection\label{parameters:Initial conditions/Adiabatic/Function/Variable names}


\index[prmindex]{Variable names}
\index[prmindexfull]{Initial conditions!Adiabatic!Function/Variable names}
{\it Value:} x,t


{\it Default:} x,t


{\it Description:} The name of the variables as they will be used in the function, separated by commas. By default, the names of variables at which the function will be evaluated is `x' (in 1d), `x,y' (in 2d) or `x,y,z' (in 3d) for spatial coordinates and `t' for time. You can then use these variable names in your function expression and they will be replaced by the values of these variables at which the function is currently evaluated. However, you can also choose a different set of names for the independent variables at which to evaluate your function expression. For example, if you work in spherical coordinates, you may wish to set this input parameter to `r,phi,theta,t' and then use these variable names in your function expression.


{\it Possible values:} [Anything]
\end{itemize}

\subsection{Parameters in section \tt Initial conditions/Ascii data model}
\label{parameters:Initial_20conditions/Ascii_20data_20model}

\begin{itemize}
\item {\it Parameter name:} {\tt Data directory}
\phantomsection\label{parameters:Initial conditions/Ascii data model/Data directory}


\index[prmindex]{Data directory}
\index[prmindexfull]{Initial conditions!Ascii data model!Data directory}
{\it Value:} \$ASPECT\_SOURCE\_DIR/data/initial-conditions/ascii-data/test/


{\it Default:} \$ASPECT\_SOURCE\_DIR/data/initial-conditions/ascii-data/test/


{\it Description:} The name of a directory that contains the model data. This path may either be absolute (if starting with a '/') or relative to the current directory. The path may also include the special text '\$ASPECT\_SOURCE\_DIR' which will be interpreted as the path in which the ASPECT source files were located when ASPECT was compiled. This interpretation allows, for example, to reference files located in the 'data/' subdirectory of ASPECT. 


{\it Possible values:} [DirectoryName]
\item {\it Parameter name:} {\tt Data file name}
\phantomsection\label{parameters:Initial conditions/Ascii data model/Data file name}


\index[prmindex]{Data file name}
\index[prmindexfull]{Initial conditions!Ascii data model!Data file name}
{\it Value:} box\_2d.txt


{\it Default:} box\_2d.txt


{\it Description:} The file name of the material data. Provide file in format: (Velocity file name).\%s\%d where \%s is a string specifying the boundary of the model according to the names of the boundary indicators (of a box or a spherical shell).\%d is any sprintf integer qualifier, specifying the format of the current file number. 


{\it Possible values:} [Anything]
\item {\it Parameter name:} {\tt Scale factor}
\phantomsection\label{parameters:Initial conditions/Ascii data model/Scale factor}


\index[prmindex]{Scale factor}
\index[prmindexfull]{Initial conditions!Ascii data model!Scale factor}
{\it Value:} 1


{\it Default:} 1


{\it Description:} Scalar factor, which is applied to the boundary velocity. You might want to use this to scale the velocities to a reference model (e.g. with free-slip boundary) or another plate reconstruction. Another way to use this factor is to convert units of the input files. The unit is assumed to bem/s or m/yr depending on the 'Use years in output instead of seconds' flag. If you provide velocities in cm/yr set this factor to 0.01.


{\it Possible values:} [Double 0...1.79769e+308 (inclusive)]
\end{itemize}

\subsection{Parameters in section \tt Initial conditions/Function}
\label{parameters:Initial_20conditions/Function}

\begin{itemize}
\item {\it Parameter name:} {\tt Function constants}
\phantomsection\label{parameters:Initial conditions/Function/Function constants}


\index[prmindex]{Function constants}
\index[prmindexfull]{Initial conditions!Function!Function constants}
{\it Value:} 


{\it Default:} 


{\it Description:} Sometimes it is convenient to use symbolic constants in the expression that describes the function, rather than having to use its numeric value everywhere the constant appears. These values can be defined using this parameter, in the form `var1=value1, var2=value2, ...'.

A typical example would be to set this runtime parameter to `pi=3.1415926536' and then use `pi' in the expression of the actual formula. (That said, for convenience this class actually defines both `pi' and `Pi' by default, but you get the idea.)


{\it Possible values:} [Anything]
\item {\it Parameter name:} {\tt Function expression}
\phantomsection\label{parameters:Initial conditions/Function/Function expression}


\index[prmindex]{Function expression}
\index[prmindexfull]{Initial conditions!Function!Function expression}
{\it Value:} 0


{\it Default:} 0


{\it Description:} The formula that denotes the function you want to evaluate for particular values of the independent variables. This expression may contain any of the usual operations such as addition or multiplication, as well as all of the common functions such as `sin' or `cos'. In addition, it may contain expressions like `if(x>0, 1, -1)' where the expression evaluates to the second argument if the first argument is true, and to the third argument otherwise. For a full overview of possible expressions accepted see the documentation of the muparser library at http://muparser.beltoforion.de/.

If the function you are describing represents a vector-valued function with multiple components, then separate the expressions for individual components by a semicolon.


{\it Possible values:} [Anything]
\item {\it Parameter name:} {\tt Variable names}
\phantomsection\label{parameters:Initial conditions/Function/Variable names}


\index[prmindex]{Variable names}
\index[prmindexfull]{Initial conditions!Function!Variable names}
{\it Value:} x,y,t


{\it Default:} x,y,t


{\it Description:} The name of the variables as they will be used in the function, separated by commas. By default, the names of variables at which the function will be evaluated is `x' (in 1d), `x,y' (in 2d) or `x,y,z' (in 3d) for spatial coordinates and `t' for time. You can then use these variable names in your function expression and they will be replaced by the values of these variables at which the function is currently evaluated. However, you can also choose a different set of names for the independent variables at which to evaluate your function expression. For example, if you work in spherical coordinates, you may wish to set this input parameter to `r,phi,theta,t' and then use these variable names in your function expression.


{\it Possible values:} [Anything]
\end{itemize}

\subsection{Parameters in section \tt Initial conditions/Harmonic perturbation}
\label{parameters:Initial_20conditions/Harmonic_20perturbation}

\begin{itemize}
\item {\it Parameter name:} {\tt Lateral wave number one}
\phantomsection\label{parameters:Initial conditions/Harmonic perturbation/Lateral wave number one}


\index[prmindex]{Lateral wave number one}
\index[prmindexfull]{Initial conditions!Harmonic perturbation!Lateral wave number one}
{\it Value:} 3


{\it Default:} 3


{\it Description:} Doubled first lateral wave number of the harmonic perturbation. Equals the spherical harmonic degree in 3D spherical shells. In all other cases one equals half of a sine period over the model domain. This allows for single up-/downswings. Negative numbers reverse the sign of the perturbation but are not allowed for the spherical harmonic case.


{\it Possible values:} [Integer range -2147483648...2147483647 (inclusive)]
\item {\it Parameter name:} {\tt Lateral wave number two}
\phantomsection\label{parameters:Initial conditions/Harmonic perturbation/Lateral wave number two}


\index[prmindex]{Lateral wave number two}
\index[prmindexfull]{Initial conditions!Harmonic perturbation!Lateral wave number two}
{\it Value:} 2


{\it Default:} 2


{\it Description:} Doubled second lateral wave number of the harmonic perturbation. Equals the spherical harmonic order in 3D spherical shells. In all other cases one equals half of a sine period over the model domain. This allows for single up-/downswings. Negative numbers reverse the sign of the perturbation.


{\it Possible values:} [Integer range -2147483648...2147483647 (inclusive)]
\item {\it Parameter name:} {\tt Magnitude}
\phantomsection\label{parameters:Initial conditions/Harmonic perturbation/Magnitude}


\index[prmindex]{Magnitude}
\index[prmindexfull]{Initial conditions!Harmonic perturbation!Magnitude}
{\it Value:} 1.0


{\it Default:} 1.0


{\it Description:} The magnitude of the Harmonic perturbation.


{\it Possible values:} [Double 0...1.79769e+308 (inclusive)]
\item {\it Parameter name:} {\tt Reference temperature}
\phantomsection\label{parameters:Initial conditions/Harmonic perturbation/Reference temperature}


\index[prmindex]{Reference temperature}
\index[prmindexfull]{Initial conditions!Harmonic perturbation!Reference temperature}
{\it Value:} 1600.0


{\it Default:} 1600.0


{\it Description:} The reference temperature that is perturbed by theharmonic function. Only used in incompressible models.


{\it Possible values:} [Double 0...1.79769e+308 (inclusive)]
\item {\it Parameter name:} {\tt Vertical wave number}
\phantomsection\label{parameters:Initial conditions/Harmonic perturbation/Vertical wave number}


\index[prmindex]{Vertical wave number}
\index[prmindexfull]{Initial conditions!Harmonic perturbation!Vertical wave number}
{\it Value:} 1


{\it Default:} 1


{\it Description:} Doubled radial wave number of the harmonic perturbation.  One equals half of a sine period over the model domain.  This allows for single up-/downswings. Negative numbers  reverse the sign of the perturbation.


{\it Possible values:} [Integer range -2147483648...2147483647 (inclusive)]
\end{itemize}

\subsection{Parameters in section \tt Initial conditions/Inclusion shape perturbation}
\label{parameters:Initial_20conditions/Inclusion_20shape_20perturbation}

\begin{itemize}
\item {\it Parameter name:} {\tt Ambient temperature}
\phantomsection\label{parameters:Initial conditions/Inclusion shape perturbation/Ambient temperature}


\index[prmindex]{Ambient temperature}
\index[prmindexfull]{Initial conditions!Inclusion shape perturbation!Ambient temperature}
{\it Value:} 1.0


{\it Default:} 1.0


{\it Description:} The background temperature for the temperature field.


{\it Possible values:} [Double -1.79769e+308...1.79769e+308 (inclusive)]
\item {\it Parameter name:} {\tt Center X}
\phantomsection\label{parameters:Initial conditions/Inclusion shape perturbation/Center X}


\index[prmindex]{Center X}
\index[prmindexfull]{Initial conditions!Inclusion shape perturbation!Center X}
{\it Value:} 0.5


{\it Default:} 0.5


{\it Description:} The X coordinate for the center of the shape.


{\it Possible values:} [Double -1.79769e+308...1.79769e+308 (inclusive)]
\item {\it Parameter name:} {\tt Center Y}
\phantomsection\label{parameters:Initial conditions/Inclusion shape perturbation/Center Y}


\index[prmindex]{Center Y}
\index[prmindexfull]{Initial conditions!Inclusion shape perturbation!Center Y}
{\it Value:} 0.5


{\it Default:} 0.5


{\it Description:} The Y coordinate for the center of the shape.


{\it Possible values:} [Double -1.79769e+308...1.79769e+308 (inclusive)]
\item {\it Parameter name:} {\tt Center Z}
\phantomsection\label{parameters:Initial conditions/Inclusion shape perturbation/Center Z}


\index[prmindex]{Center Z}
\index[prmindexfull]{Initial conditions!Inclusion shape perturbation!Center Z}
{\it Value:} 0.5


{\it Default:} 0.5


{\it Description:} The Z coordinate for the center of the shape. This is only necessary for three-dimensional fields.


{\it Possible values:} [Double -1.79769e+308...1.79769e+308 (inclusive)]
\item {\it Parameter name:} {\tt Inclusion gradient}
\phantomsection\label{parameters:Initial conditions/Inclusion shape perturbation/Inclusion gradient}


\index[prmindex]{Inclusion gradient}
\index[prmindexfull]{Initial conditions!Inclusion shape perturbation!Inclusion gradient}
{\it Value:} constant


{\it Default:} constant


{\it Description:} The gradient of the inclusion to be generated.


{\it Possible values:} [Selection gaussian|linear|constant ]
\item {\it Parameter name:} {\tt Inclusion shape}
\phantomsection\label{parameters:Initial conditions/Inclusion shape perturbation/Inclusion shape}


\index[prmindex]{Inclusion shape}
\index[prmindexfull]{Initial conditions!Inclusion shape perturbation!Inclusion shape}
{\it Value:} circle


{\it Default:} circle


{\it Description:} The shape of the inclusion to be generated.


{\it Possible values:} [Selection square|circle ]
\item {\it Parameter name:} {\tt Inclusion temperature}
\phantomsection\label{parameters:Initial conditions/Inclusion shape perturbation/Inclusion temperature}


\index[prmindex]{Inclusion temperature}
\index[prmindexfull]{Initial conditions!Inclusion shape perturbation!Inclusion temperature}
{\it Value:} 0.0


{\it Default:} 0.0


{\it Description:} The temperature of the inclusion shape. This is only the true temperature in the case of the constant gradient. In all other cases, it gives one endpoint of the temperature gradient for the shape.


{\it Possible values:} [Double -1.79769e+308...1.79769e+308 (inclusive)]
\item {\it Parameter name:} {\tt Shape radius}
\phantomsection\label{parameters:Initial conditions/Inclusion shape perturbation/Shape radius}


\index[prmindex]{Shape radius}
\index[prmindexfull]{Initial conditions!Inclusion shape perturbation!Shape radius}
{\it Value:} 1.0


{\it Default:} 1.0


{\it Description:} The radius of the inclusion to be generated. For shapes with no radius (e.g. square), this will be the width, and for shapes with no width, this gives a general guideline for the size of the shape.


{\it Possible values:} [Double 0...1.79769e+308 (inclusive)]
\end{itemize}

\subsection{Parameters in section \tt Initial conditions/S40RTS perturbation}
\label{parameters:Initial_20conditions/S40RTS_20perturbation}

\begin{itemize}
\item {\it Parameter name:} {\tt Data directory}
\phantomsection\label{parameters:Initial conditions/S40RTS perturbation/Data directory}


\index[prmindex]{Data directory}
\index[prmindexfull]{Initial conditions!S40RTS perturbation!Data directory}
{\it Value:} \$ASPECT\_SOURCE\_DIR/data/initial-conditions/S40RTS/


{\it Default:} \$ASPECT\_SOURCE\_DIR/data/initial-conditions/S40RTS/


{\it Description:} The path to the model data. 


{\it Possible values:} [DirectoryName]
\item {\it Parameter name:} {\tt Initial condition file name}
\phantomsection\label{parameters:Initial conditions/S40RTS perturbation/Initial condition file name}


\index[prmindex]{Initial condition file name}
\index[prmindexfull]{Initial conditions!S40RTS perturbation!Initial condition file name}
{\it Value:} S40RTS.sph


{\it Default:} S40RTS.sph


{\it Description:} The file name of the spherical harmonics coefficients from Ritsema et al.


{\it Possible values:} [Anything]
\item {\it Parameter name:} {\tt Reference temperature}
\phantomsection\label{parameters:Initial conditions/S40RTS perturbation/Reference temperature}


\index[prmindex]{Reference temperature}
\index[prmindexfull]{Initial conditions!S40RTS perturbation!Reference temperature}
{\it Value:} 1600.0


{\it Default:} 1600.0


{\it Description:} The reference temperature that is perturbed by the spherical harmonic functions. Only used in incompressible models.


{\it Possible values:} [Double 0...1.79769e+308 (inclusive)]
\item {\it Parameter name:} {\tt Remove degree 0 from perturbation}
\phantomsection\label{parameters:Initial conditions/S40RTS perturbation/Remove degree 0 from perturbation}


\index[prmindex]{Remove degree 0 from perturbation}
\index[prmindexfull]{Initial conditions!S40RTS perturbation!Remove degree 0 from perturbation}
{\it Value:} true


{\it Default:} true


{\it Description:} Option to remove the degree zero component from the perturbation, which will ensure that the laterally averaged temperature for a fixed depth is equal to the background temperature.


{\it Possible values:} [Bool]
\item {\it Parameter name:} {\tt Remove temperature heterogeneity down to specified depth}
\phantomsection\label{parameters:Initial conditions/S40RTS perturbation/Remove temperature heterogeneity down to specified depth}


\index[prmindex]{Remove temperature heterogeneity down to specified depth}
\index[prmindexfull]{Initial conditions!S40RTS perturbation!Remove temperature heterogeneity down to specified depth}
{\it Value:} -1.7976931348623157e+308


{\it Default:} -1.7976931348623157e+308


{\it Description:} This will set the heterogeneity prescribed by S20RTS or S40RTS to zero down to the specified depth (in meters).Note that your resolution has to be adquate to capture this cutoff. For example if you specify a depth of 660km, but your closest spherical depth layers are only at 500km and 750km (due to a coarse resolution) it will only zero out heterogeneities down to 500km. Similar caution has to be taken when using adaptive meshing.


{\it Possible values:} [Double -1.79769e+308...1.79769e+308 (inclusive)]
\item {\it Parameter name:} {\tt Spline knots depth file name}
\phantomsection\label{parameters:Initial conditions/S40RTS perturbation/Spline knots depth file name}


\index[prmindex]{Spline knots depth file name}
\index[prmindexfull]{Initial conditions!S40RTS perturbation!Spline knots depth file name}
{\it Value:} Spline\_knots.txt


{\it Default:} Spline\_knots.txt


{\it Description:} The file name of the spline knot locations from Ritsema et al.


{\it Possible values:} [Anything]
\item {\it Parameter name:} {\tt Thermal expansion coefficient in initial temperature scaling}
\phantomsection\label{parameters:Initial conditions/S40RTS perturbation/Thermal expansion coefficient in initial temperature scaling}


\index[prmindex]{Thermal expansion coefficient in initial temperature scaling}
\index[prmindexfull]{Initial conditions!S40RTS perturbation!Thermal expansion coefficient in initial temperature scaling}
{\it Value:} 2e-5


{\it Default:} 2e-5


{\it Description:} The value of the thermal expansion coefficient $\beta$. Units: $1/K$.


{\it Possible values:} [Double 0...1.79769e+308 (inclusive)]
\item {\it Parameter name:} {\tt vs to density scaling}
\phantomsection\label{parameters:Initial conditions/S40RTS perturbation/vs to density scaling}


\index[prmindex]{vs to density scaling}
\index[prmindexfull]{Initial conditions!S40RTS perturbation!vs to density scaling}
{\it Value:} 0.25


{\it Default:} 0.25


{\it Description:} This parameter specifies how the perturbation in shear wave velocity as prescribed by S20RTS or S40RTS is scaled into a density perturbation. See the general description of this model for more detailed information.


{\it Possible values:} [Double 0...1.79769e+308 (inclusive)]
\end{itemize}

\subsection{Parameters in section \tt Initial conditions/SAVANI perturbation}
\label{parameters:Initial_20conditions/SAVANI_20perturbation}

\begin{itemize}
\item {\it Parameter name:} {\tt Data directory}
\phantomsection\label{parameters:Initial conditions/SAVANI perturbation/Data directory}


\index[prmindex]{Data directory}
\index[prmindexfull]{Initial conditions!SAVANI perturbation!Data directory}
{\it Value:} \$ASPECT\_SOURCE\_DIR/data/initial-conditions/SAVANI/


{\it Default:} \$ASPECT\_SOURCE\_DIR/data/initial-conditions/SAVANI/


{\it Description:} The path to the model data. 


{\it Possible values:} [DirectoryName]
\item {\it Parameter name:} {\tt Initial condition file name}
\phantomsection\label{parameters:Initial conditions/SAVANI perturbation/Initial condition file name}


\index[prmindex]{Initial condition file name}
\index[prmindexfull]{Initial conditions!SAVANI perturbation!Initial condition file name}
{\it Value:} savani.dlnvs.60.m.ab


{\it Default:} savani.dlnvs.60.m.ab


{\it Description:} The file name of the spherical harmonics coefficients from Auer et al.


{\it Possible values:} [Anything]
\item {\it Parameter name:} {\tt Reference temperature}
\phantomsection\label{parameters:Initial conditions/SAVANI perturbation/Reference temperature}


\index[prmindex]{Reference temperature}
\index[prmindexfull]{Initial conditions!SAVANI perturbation!Reference temperature}
{\it Value:} 1600.0


{\it Default:} 1600.0


{\it Description:} The reference temperature that is perturbed by the spherical harmonic functions. Only used in incompressible models.


{\it Possible values:} [Double 0...1.79769e+308 (inclusive)]
\item {\it Parameter name:} {\tt Remove degree 0 from perturbation}
\phantomsection\label{parameters:Initial conditions/SAVANI perturbation/Remove degree 0 from perturbation}


\index[prmindex]{Remove degree 0 from perturbation}
\index[prmindexfull]{Initial conditions!SAVANI perturbation!Remove degree 0 from perturbation}
{\it Value:} true


{\it Default:} true


{\it Description:} Option to remove the degree zero component from the perturbation, which will ensure that the laterally averaged temperature for a fixed depth is equal to the background temperature.


{\it Possible values:} [Bool]
\item {\it Parameter name:} {\tt Remove temperature heterogeneity down to specified depth}
\phantomsection\label{parameters:Initial conditions/SAVANI perturbation/Remove temperature heterogeneity down to specified depth}


\index[prmindex]{Remove temperature heterogeneity down to specified depth}
\index[prmindexfull]{Initial conditions!SAVANI perturbation!Remove temperature heterogeneity down to specified depth}
{\it Value:} -1.7976931348623157e+308


{\it Default:} -1.7976931348623157e+308


{\it Description:} This will set the heterogeneity prescribed by SAVANI to zero down to the specified depth (in meters). Note that your resolution has to be adquate to capture this cutoff. For example if you specify a depth of 660km, but your closest spherical depth layers are only at 500km and 750km (due to a coarse resolution) it will only zero out heterogeneities down to 500km. Similar caution has to be taken when using adaptive meshing.


{\it Possible values:} [Double -1.79769e+308...1.79769e+308 (inclusive)]
\item {\it Parameter name:} {\tt Spline knots depth file name}
\phantomsection\label{parameters:Initial conditions/SAVANI perturbation/Spline knots depth file name}


\index[prmindex]{Spline knots depth file name}
\index[prmindexfull]{Initial conditions!SAVANI perturbation!Spline knots depth file name}
{\it Value:} Spline\_knots.txt


{\it Default:} Spline\_knots.txt


{\it Description:} The file name of the spline knots taken from the 28 spherical layersof SAVANI tomography model.


{\it Possible values:} [Anything]
\item {\it Parameter name:} {\tt Thermal expansion coefficient in initial temperature scaling}
\phantomsection\label{parameters:Initial conditions/SAVANI perturbation/Thermal expansion coefficient in initial temperature scaling}


\index[prmindex]{Thermal expansion coefficient in initial temperature scaling}
\index[prmindexfull]{Initial conditions!SAVANI perturbation!Thermal expansion coefficient in initial temperature scaling}
{\it Value:} 2e-5


{\it Default:} 2e-5


{\it Description:} The value of the thermal expansion coefficient $\beta$. Units: $1/K$.


{\it Possible values:} [Double 0...1.79769e+308 (inclusive)]
\item {\it Parameter name:} {\tt vs to density scaling}
\phantomsection\label{parameters:Initial conditions/SAVANI perturbation/vs to density scaling}


\index[prmindex]{vs to density scaling}
\index[prmindexfull]{Initial conditions!SAVANI perturbation!vs to density scaling}
{\it Value:} 0.25


{\it Default:} 0.25


{\it Description:} This parameter specifies how the perturbation in shear wave velocity as prescribed by SAVANI is scaled into a density perturbation. See the general description of this model for more detailed information.


{\it Possible values:} [Double 0...1.79769e+308 (inclusive)]
\end{itemize}

\subsection{Parameters in section \tt Initial conditions/Solidus}
\label{parameters:Initial_20conditions/Solidus}

\begin{itemize}
\item {\it Parameter name:} {\tt Lithosphere thickness}
\phantomsection\label{parameters:Initial conditions/Solidus/Lithosphere thickness}


\index[prmindex]{Lithosphere thickness}
\index[prmindexfull]{Initial conditions!Solidus!Lithosphere thickness}
{\it Value:} 0


{\it Default:} 0


{\it Description:} The thickness of lithosphere thickness. Units: m


{\it Possible values:} [Double 0...1.79769e+308 (inclusive)]
\item {\it Parameter name:} {\tt Supersolidus}
\phantomsection\label{parameters:Initial conditions/Solidus/Supersolidus}


\index[prmindex]{Supersolidus}
\index[prmindexfull]{Initial conditions!Solidus!Supersolidus}
{\it Value:} 0e0


{\it Default:} 0e0


{\it Description:} The difference from solidus, use this number to generate initial conditions that close to solidus instead of exactly at solidus. Use small negative number in this parameter to prevent large melting generation at the beginning.   Units: K 


{\it Possible values:} [Double -1.79769e+308...1.79769e+308 (inclusive)]
\end{itemize}



\subsection{Parameters in section \tt Initial conditions/Solidus/Data}
\label{parameters:Initial_20conditions/Solidus/Data}

\begin{itemize}
\item {\it Parameter name:} {\tt Solidus filename}
\phantomsection\label{parameters:Initial conditions/Solidus/Data/Solidus filename}


\index[prmindex]{Solidus filename}
\index[prmindexfull]{Initial conditions!Solidus!Data/Solidus filename}
{\it Value:} 


{\it Default:} 


{\it Description:} The solidus data filename. It is a function of radius or pressure in the following format: 
Line 1:  Header 
Line 2:  Unit of temperature (C/K)        Unit of pressure (GPa/kbar) or radius (km/m) 
Line 3~: Column of solidus temperature    Column of radius/pressure 
See data/initial-temperature/solidus.Mars as an example.

In order to facilitate placing input files in locations relative to the ASPECT source directory, the file name may also include the special text '\$ASPECT\_SOURCE\_DIR' which will be interpreted as the path in which the ASPECT source files were located when ASPECT was compiled. This interpretation allows, for example, to reference files located in the 'data/' subdirectory of ASPECT.


{\it Possible values:} [Anything]
\end{itemize}

\subsection{Parameters in section \tt Initial conditions/Solidus/Perturbation}
\label{parameters:Initial_20conditions/Solidus/Perturbation}

\begin{itemize}
\item {\it Parameter name:} {\tt Lateral wave number one}
\phantomsection\label{parameters:Initial conditions/Solidus/Perturbation/Lateral wave number one}


\index[prmindex]{Lateral wave number one}
\index[prmindexfull]{Initial conditions!Solidus!Perturbation/Lateral wave number one}
{\it Value:} 3


{\it Default:} 3


{\it Description:} Doubled first lateral wave number of the harmonic perturbation. Equals the spherical harmonic degree in 3D spherical shells. In all other cases one equals half of a sine period over the model domain. This allows for single up-/downswings. Negative numbers reverse the sign of the perturbation but are not allowed for the spherical harmonic case.


{\it Possible values:} [Integer range -2147483648...2147483647 (inclusive)]
\item {\it Parameter name:} {\tt Lateral wave number two}
\phantomsection\label{parameters:Initial conditions/Solidus/Perturbation/Lateral wave number two}


\index[prmindex]{Lateral wave number two}
\index[prmindexfull]{Initial conditions!Solidus!Perturbation/Lateral wave number two}
{\it Value:} 2


{\it Default:} 2


{\it Description:} Doubled second lateral wave number of the harmonic perturbation. Equals the spherical harmonic order in 3D spherical shells. In all other cases one equals half of a sine period over the model domain. This allows for single up-/downswings. Negative numbers reverse the sign of the perturbation.


{\it Possible values:} [Integer range -2147483648...2147483647 (inclusive)]
\item {\it Parameter name:} {\tt Lithosphere thickness amplitude}
\phantomsection\label{parameters:Initial conditions/Solidus/Perturbation/Lithosphere thickness amplitude}


\index[prmindex]{Lithosphere thickness amplitude}
\index[prmindexfull]{Initial conditions!Solidus!Perturbation/Lithosphere thickness amplitude}
{\it Value:} 0e0


{\it Default:} 0e0


{\it Description:} The amplitude of the initial lithosphere thickness perturbation in (m)


{\it Possible values:} [Double -1.79769e+308...1.79769e+308 (inclusive)]
\item {\it Parameter name:} {\tt Temperature amplitude}
\phantomsection\label{parameters:Initial conditions/Solidus/Perturbation/Temperature amplitude}


\index[prmindex]{Temperature amplitude}
\index[prmindexfull]{Initial conditions!Solidus!Perturbation/Temperature amplitude}
{\it Value:} 0e0


{\it Default:} 0e0


{\it Description:} The amplitude of the initial spherical temperature perturbation in (K)


{\it Possible values:} [Double 0...1.79769e+308 (inclusive)]
\end{itemize}

\subsection{Parameters in section \tt Initial conditions/Spherical gaussian perturbation}
\label{parameters:Initial_20conditions/Spherical_20gaussian_20perturbation}

\begin{itemize}
\item {\it Parameter name:} {\tt Amplitude}
\phantomsection\label{parameters:Initial conditions/Spherical gaussian perturbation/Amplitude}


\index[prmindex]{Amplitude}
\index[prmindexfull]{Initial conditions!Spherical gaussian perturbation!Amplitude}
{\it Value:} 0.01


{\it Default:} 0.01


{\it Description:} The amplitude of the perturbation.


{\it Possible values:} [Double 0...1.79769e+308 (inclusive)]
\item {\it Parameter name:} {\tt Angle}
\phantomsection\label{parameters:Initial conditions/Spherical gaussian perturbation/Angle}


\index[prmindex]{Angle}
\index[prmindexfull]{Initial conditions!Spherical gaussian perturbation!Angle}
{\it Value:} 0e0


{\it Default:} 0e0


{\it Description:} The angle where the center of the perturbation is placed.


{\it Possible values:} [Double 0...1.79769e+308 (inclusive)]
\item {\it Parameter name:} {\tt Filename for initial geotherm table}
\phantomsection\label{parameters:Initial conditions/Spherical gaussian perturbation/Filename for initial geotherm table}


\index[prmindex]{Filename for initial geotherm table}
\index[prmindexfull]{Initial conditions!Spherical gaussian perturbation!Filename for initial geotherm table}
{\it Value:} initial-geotherm-table


{\it Default:} initial-geotherm-table


{\it Description:} The file from which the initial geotherm table is to be read. The format of the file is defined by what is read in source/initial\_conditions/spherical\_shell.cc.


{\it Possible values:} [FileName (Type: input)]
\item {\it Parameter name:} {\tt Non-dimensional depth}
\phantomsection\label{parameters:Initial conditions/Spherical gaussian perturbation/Non-dimensional depth}


\index[prmindex]{Non-dimensional depth}
\index[prmindexfull]{Initial conditions!Spherical gaussian perturbation!Non-dimensional depth}
{\it Value:} 0.7


{\it Default:} 0.7


{\it Description:} The non-dimensional radial distance where the center of the perturbation is placed.


{\it Possible values:} [Double 0...1.79769e+308 (inclusive)]
\item {\it Parameter name:} {\tt Sigma}
\phantomsection\label{parameters:Initial conditions/Spherical gaussian perturbation/Sigma}


\index[prmindex]{Sigma}
\index[prmindexfull]{Initial conditions!Spherical gaussian perturbation!Sigma}
{\it Value:} 0.2


{\it Default:} 0.2


{\it Description:} The standard deviation of the Gaussian perturbation.


{\it Possible values:} [Double 0...1.79769e+308 (inclusive)]
\item {\it Parameter name:} {\tt Sign}
\phantomsection\label{parameters:Initial conditions/Spherical gaussian perturbation/Sign}


\index[prmindex]{Sign}
\index[prmindexfull]{Initial conditions!Spherical gaussian perturbation!Sign}
{\it Value:} 1


{\it Default:} 1


{\it Description:} The sign of the perturbation.


{\it Possible values:} [Double -1.79769e+308...1.79769e+308 (inclusive)]
\end{itemize}

\subsection{Parameters in section \tt Initial conditions/Spherical hexagonal perturbation}
\label{parameters:Initial_20conditions/Spherical_20hexagonal_20perturbation}

\begin{itemize}
\item {\it Parameter name:} {\tt Angular mode}
\phantomsection\label{parameters:Initial conditions/Spherical hexagonal perturbation/Angular mode}


\index[prmindex]{Angular mode}
\index[prmindexfull]{Initial conditions!Spherical hexagonal perturbation!Angular mode}
{\it Value:} 6


{\it Default:} 6


{\it Description:} The number of convection cells to perturb the system with.


{\it Possible values:} [Integer range -2147483648...2147483647 (inclusive)]
\item {\it Parameter name:} {\tt Rotation offset}
\phantomsection\label{parameters:Initial conditions/Spherical hexagonal perturbation/Rotation offset}


\index[prmindex]{Rotation offset}
\index[prmindexfull]{Initial conditions!Spherical hexagonal perturbation!Rotation offset}
{\it Value:} -45


{\it Default:} -45


{\it Description:} Amount of clockwise rotation in degrees to apply to the perturbations. Default is set to -45 in order to provide backwards compatibility.


{\it Possible values:} [Double -1.79769e+308...1.79769e+308 (inclusive)]
\end{itemize}

\subsection{Parameters in section \tt Material model}
\label{parameters:Material_20model}

\begin{itemize}
\item {\it Parameter name:} {\tt Material averaging}
\phantomsection\label{parameters:Material model/Material averaging}


\index[prmindex]{Material averaging}
\index[prmindexfull]{Material model!Material averaging}
{\it Value:} none


{\it Default:} none


{\it Description:} Whether or not (and in the first case, how) to do any averaging of material model output data when constructing the linear systems for velocity/pressure, temperature, and compositions in each time step, as well as their corresponding preconditioners.

Possible choices: none|arithmetic average|harmonic average|geometric average|pick largest|project to Q1|log average

The process of averaging, and where it may be used, is discussed in more detail in Section~\ref{sec:sinker-with-averaging}.

More averaging schemes are available in the averaging material model. This material model is a ``compositing material model'' which can be used in combination with other material models.


{\it Possible values:} [Selection none|arithmetic average|harmonic average|geometric average|pick largest|project to Q1|log average ]
\item {\it Parameter name:} {\tt Model name}
\phantomsection\label{parameters:Material model/Model name}


\index[prmindex]{Model name}
\index[prmindexfull]{Material model!Model name}
{\it Value:} simple


{\it Default:} unspecified


{\it Description:} The name of the material model to be used in this simulation. There are many material models you can choose from, as listed below. They generally fall into two category: (i) models that implement a particular case of material behavior, (ii) models that modify other models in some way. We sometimes call the latter ``compositing models''. An example of a compositing model is the ``depth dependent'' model below in that it takes another, freely choosable model as its base and then modifies that model's output in some way.

You can select one of the following models:

`Morency and Doin': An implementation of the visco-plastic rheology described by (Morency and Doin, 2004). Compositional fields can each be assigned individual activation energies, reference densities, thermal expansivities, and stress exponents. The effective viscosity is defined as \[v_{eff} = \left(\frac{1}{v_{eff}^v}+\frac{1}{v_{eff}^p}\right)^{-1}\] where \[v_{eff}^v = B \left(\frac{\dot{\epsilon}}{\dot{\epsilon}_{ref}} \right)^{-1+1/n_v} exp\left(\frac{E_a +V_a \rho_m g z}{n_v R T}\right) \] \[v_{eff}^p = (\tau_0 + \gamma \rho_m g z) \left( \frac{\dot{\epsilon}^{-1+1/n_p}} {\dot{\epsilon}_{ref}^{1/n_p}} \right) \] where $B$ is a scaling constant; $\dot{\epsilon}$ is defined as the quadratic sum of the second invariant of the strain rate tensor and a minimum strain rate, $\dot{\epsilon}_0$; $\dot{\epsilon}_{ref}$ is a reference strain rate; $n_v$, and $n_p$ are stress exponents; $E_a$ is the activation energy; $V_a$ is the activation volume; $\rho_m$ is the mantle density; $R$ is the gas constant; $T$ is temperature; $\tau_0$ is the cohestive strength of rocks at the surface; $\gamma$ is a coefficient of yield stress increase with depth; and $z$ is depth. 

 Note: (Morency and Doin, 2004) defines the second invariant of the strain rate in a nonstandard way. The formulation in the paper is given as $\epsilon_{II} = \sqrt{\frac{1}{2} (\epsilon_{11}^2 + \epsilon_{12}^2)}$ where $\epsilon$ is the strain rate tensor. For consistency, that is also the formulation implemented in this material model. 

 Morency, C., and M‐P. Doin. "Numerical simulations of the mantle lithosphere delamination." Journal of Geophysical Research: Solid Earth (1978–2012) 109.B3 (2004). 

 The value for the components of this formula and additional parameters are read from the parameter file in subsection 'Material model/Morency and Doin'.

`Steinberger': This material model looks up the viscosity from the tables that correspond to the paper of Steinberger and Calderwood 2006 (``Models of large-scale viscous flow in the Earth's mantle with constraints from mineral physics and surface observations'', Geophys. J. Int., 167, 1461-1481, \url{http://dx.doi.org/10.1111/j.1365-246X.2006.03131.x}) and material data from a database generated by the thermodynamics code \texttt{Perplex}, see \url{http://www.perplex.ethz.ch/}. The default example data builds upon the thermodynamic database by Stixrude 2011 and assumes a pyrolitic composition by Ringwood 1988 but is easily replaceable by other data files. 

`averaging': The ``averaging'' Material model applies an averaging of the quadrature points within a cell. The values to average are supplied by any of the other available material models. In other words, it is a ``compositing material model''. Parameters related to the average model are read from a subsection ``Material model/Averaging''. 

The user must specify a ``Base model'' from which material properties are derived. Furthermore an averaging operation must be selected, where the Choice should be from the list none|arithmetic average|harmonic average|geometric average|pick largest|log average|NWD arithmetic average|NWD harmonic average|NWD geometric average. 

NWD stands for Normalized Weighed Distance. The models with this in front of their name work with a weighed average, which means each quadrature point requires an individual weight. The weight is determined by the distance, where the exact relation is determined by a bell shaped curve. A bell shaped curve is a continuous function which is one at it's maximum and exactly zero at and beyond it's limit. This bell shaped curve is spanned around each quadrature point to determine the weighting map for each quadrature point. The used bell shape comes from Lucy (1977). The distance is normalized so the largest distance becomes one. This means that if variable ''Bell shape limit'' is exactly one, the farthest quadrature point is just on the limit and it's weight will be exactly zero. In this plugin it is not implemented as larger and equal than the limit, but larger than, to ensure the the quadrature point at distance zero is always included.

`composition reaction': A material model that behaves in the same way as the simple material model, but includes two compositional fields and a reaction between them. Above a depth given in the input file, the first fields gets converted to the second field. 

`depth dependent': The ``depth dependent'' Material model applies a depth-dependent scaling to any of the other available material models. In other words, it is a ``compositing material model''.

Parameters related to the depth dependent model are read from a subsection ``Material model/Depth dependent model''. The user must specify a ``Base model'' from which material properties are derived. Currently the depth dependent model only allows depth dependence of viscosity - other material properties are taken from the ``Base model''. Viscosity $\eta$ at depth $z$ is calculated according to:\begin{equation}\eta(z,p,T,X,...) = \eta(z) \eta_b(p,T,X,..)/\eta_{rb}\end{equation}where $\eta(z)$ is the the depth-dependence specified by the depth dependent model, $\eta_b(p,T,X,...)$ is the viscosity calculated from the base model, and $\eta_{rb}$ is the reference viscosity of the ``Base model''. In addition to the specification of the ``Base model'', the user must specify the method to be used to calculate the depth-dependent viscosity $\eta(z)$ as ``Material model/Depth dependent model/Depth dependence method'', which can be chosen among ``None|Function|File|List''. Each method and the associated parameters are as follows:

``Function'': read a user-specified parsed function from the input file in a subsection ``Material model/Depth dependent model/Viscosity depth function''. By default, this function is uniformly equal to 1.0e21. Specifying a function that returns a value less than or equal to 0.0 anywhere in the model domain will produce an error. 

``File'': read a user-specified file containing viscosity values at specified depths. The file containing depth-dependent viscosities is read from a directory specified by the user as ``Material model/Depth dependent model/Data directory'', from a file with name specified as ``Material model/Depth dependent model/Viscosity depth file''. The format of this file is ascii text and contains two columns with one header line:

example Viscosity depth file:\\Depth (m)    Viscosity (Pa-s)\\0.0000000e+00     1.0000000e+21\\6.7000000e+05     1.0000000e+22\\

Viscosity is interpolated from this file using linear interpolation. ``None'': no depth-dependence. Viscosity is taken directly from ``Base model''

``List:'': read a comma-separated list of depth values corresponding to the maximum depths of layers having constant depth-dependence $\eta(z)$. The layers must be specified in order of increasing depth, and the last layer in the list must have a depth greater than or equal to the maximal depth of the model. The list of layer depths is specified as ``Material model/Depth dependent model/Depth list'' and the corresponding list of layer viscosities is specified as ``Material model/Depth dependent model/Viscosity list''

`diffusion dislocation':  An implementation of a viscous rheology including diffusion and dislocation creep. Compositional fields can each be assigned individual activation energies, reference densities, thermal expansivities, and stress exponents. The effective viscosity is defined as 

 \[v_\text{eff} = \left(\frac{1}{v_\text{eff}^\text{diff}}+ \frac{1}{v_\text{eff}^\text{dis}}\right)^{-1}\] where \[v_\text{i} = 0.5 * A^{-\frac{1}{n_i}} d^\frac{m_i}{n_i} \dot{\varepsilon_i}^{\frac{1-n_i}{n_i}} \exp\left(\frac{E_i^* + PV_i^*}{n_iRT}\right)\] 

 where $d$ is grain size, $i$ corresponds to diffusion or dislocation creep, $\dot{\varepsilon}$ is the square root of the second invariant of the strain rate tensor, $R$ is the gas constant, $T$ is temperature,  and $P$ is pressure. $A_i$ are prefactors, $n_i$ and $m_i$ are stress and grain size exponents $E_i$ are the activation energies and $V_i$ are the activation volumes. 

 The ratio of diffusion to dislocation strain rate is found by Newton's method, iterating to find the stress which satisfies the above equations. The value for the components of this formula and additional parameters are read from the parameter file in subsection 'Material model/DiffusionDislocation'.

`drucker prager': A material model that has constant values for all coefficients but the density and viscosity. The defaults for all coefficients are chosen to be similar to what is believed to be correct for Earth's mantle. All of the values that define this model are read from a section ``Material model/Drucker Prager'' in the input file, see Section~\ref{parameters:Material_model/Drucker Prager}.Note that the model does not take into account any dependencies of material properties on compositional fields. 

The viscosity is computed according to the Drucker Prager frictional plasticity criterion (non-associative) based on a user-defined internal friction angle $\phi$ and cohesion $C$. In 3D:  $\sigma_y = \frac{6 C \cos(\phi)}{\sqrt(3) (3+\sin(\phi))} + \frac{2 P \sin(\phi)}{\sqrt(3) (3+\sin(\phi))}$, where $P$ is the pressure. See for example Zienkiewicz, O. C., Humpheson, C. and Lewis, R. W. (1975), G\'{e}otechnique 25, No. 4, 671-689. With this formulation we circumscribe instead of inscribe the Mohr Coulomb yield surface. In 2D the Drucker Prager yield surface is the same as the Mohr Coulomb surface:  $\sigma_y = P \sin(\phi) + C \cos(\phi)$. Note that in 2D for $\phi=0$, these criteria revert to the von Mises criterion (no pressure dependence). See for example Thieulot, C. (2011), PEPI 188, 47-68. 

Note that we enforce the pressure to be positive to prevent negative yield strengths and viscosities. 

We then use the computed yield strength to scale back the viscosity on to the yield surface using the Viscosity Rescaling Method described in Kachanov, L. M. (2004), Fundamentals of the Theory of Plasticity, Dover Publications, Inc. (Not Radial Return.)A similar implementation can be found in GALE (https://geodynamics.org/cig/software/gale/gale-manual.pdf). 

To avoid numerically unfavourably large (or even negative) viscosity ranges, we cut off the viscosity with a user-defined minimum and maximum viscosity: $\eta_eff = \frac{1}{\frac{1}{\eta_min + \eta}+ \frac{1}{\eta_max}}$. 

Note that this model uses the formulation that assumes an incompressible medium despite the fact that the density follows the law $\rho(T)=\rho_0(1-\beta(T-T_{\text{ref}}))$. 

`latent heat': A material model that includes phase transitions and the possibility that latent heat is released or absorbed when material crosses one of the phase transitions of up to two different materials (compositional fields). This model implements a standard approximation of the latent heat terms following Christensen \& Yuen, 1985. The change of entropy is calculated as $Delta S = \gamma \frac{\Delta\rho}{\rho^2}$ with the Clapeyron slope $\gamma$ and the density change $\Delta\rho$ of the phase transition being input parameters. The model employs an analytic phase function in the form $X=0.5 \left( 1 + \tanh \left( \frac{\Delta p}{\Delta p_0} \right) \right)$ with $\Delta p = p - p_{transition} - \gamma \left( T - T_{transition} \right)$ and $\Delta p_0$ being the pressure difference over the width of the phase transition (specified as input parameter).

`latent heat melt': A material model that includes the latent heat of melting for two materials: peridotite and pyroxenite. The melting model for peridotite is taken from Katz et al., 2003 (A new parameterization of hydrous mantle melting) and the one for pyroxenite from Sobolev et al., 2011 (Linking mantle plumes, large igneous provinces and environmental catastrophes). The model assumes a constant entropy change for melting 100\% of the material, which can be specified in the input file. The partial derivatives of entropy with respect to temperature and pressure required for calculating the latent heat consumption are then calculated as product of this constant entropy change, and the respective derivative of the function the describes the melt fraction. This is linearly averaged with respect to the fractions of the two materials present. If no compositional fields are specified in the input file, the model assumes that the material is peridotite. If compositional fields are specified, the model assumes that the first compositional field is the fraction of pyroxenite and the rest of the material is peridotite. 

Otherwise, this material model has a temperature- and pressure-dependent density and viscosity and the density and thermal expansivity depend on the melt fraction present. It is possible to extent this model to include a melt fraction dependence of all the material parameters by calling the function melt\_fraction in the calculation of the respective parameter. However, melt and solid move with the same velocity and melt extraction is not taken into account (batch melting). 

`multicomponent': This model is for use with an arbitrary number of compositional fields, where each field represents a rock type which can have completely different properties from the others. However, each rock type itself has constant material properties.  The value of the  compositional field is interpreed as a volume fraction. If the sum of the fields is greater than one, they are renormalized.  If it is less than one, material properties  for ``background mantle'' make up the rest. When more than one field is present, the material properties are averaged arithmetically.  An exception is the viscosity, where the averaging should make more of a difference.  For this, the user selects between arithmetic, harmonic, geometric, or maximum composition averaging.

`simple': A material model that has constant values for all coefficients but the density and viscosity. The defaults for all coefficients are chosen to be similar to what is believed to be correct for Earth's mantle. All of the values that define this model are read from a section ``Material model/Simple model'' in the input file, see Section~\ref{parameters:Material_20model/Simple_20model}.

This model uses the following set of equations for the two coefficients that are non-constant: \begin{align}  \eta(p,T,\mathfrak c) &= \tau(T) \zeta(\mathfrak c) \eta_0, \\  \rho(p,T,\mathfrak c) &= \left(1-\alpha (T-T_0)\right)\rho_0 + \Delta\rho \; c_0,\end{align}where $c_0$ is the first component of the compositional vector $\mathfrak c$ if the model uses compositional fields, or zero otherwise. 

The temperature pre-factor for the viscosity formula above is defined as \begin{align}  \tau(T) &= H\left(e^{-\beta (T-T_0)/T_0}\right),  \qquad\qquad H(x) = \begin{cases}                            10^{-2} & \text{if}\; x<10^{-2}, \\                            x & \text{if}\; 10^{-2}\le x \le 10^2, \\                            10^{2} & \text{if}\; x>10^{2}, \\                         \end{cases}\end{align} where $\beta$ corresponds to the input parameter ``Thermal viscosity exponent'' and $T_0$ to the parameter ``Reference temperature''. If you set $T_0=0$ in the input file, the thermal pre-factor $\tau(T)=1$.

The compositional pre-factor for the viscosity is defined as \begin{align}  \zeta(\mathfrak c) &= \xi^{c_0}\end{align} if the model has compositional fields and equals one otherwise. $\xi$ corresponds to the parameter ``Composition viscosity prefactor'' in the input file.

Finally, in the formula for the density, $\alpha$ corresponds to the ``Thermal expansion coefficient'' and $\Delta\rho$ corresponds to the parameter ``Density differential for compositional field 1''.

Note that this model uses the formulation that assumes an incompressible medium despite the fact that the density follows the law $\rho(T)=\rho_0(1-\alpha(T-T_{\text{ref}}))$. 

\note{Despite its name, this material model is not exactly ``simple'', as indicated by the formulas above. While it was originally intended to be simple, it has over time acquired all sorts of temperature and compositional dependencies that weren't initially intended. Consequently, there is now a ``simpler'' material model that now fills the role the current model was originally intended to fill.}

`simple compressible': A material model that has constant values for all coefficients but the density. The defaults for all coefficients are chosen to be similar to what is believed to be correct for Earth's mantle. All of the values that define this model are read from a section ``Material model/Simple compressible model'' in the input file, see Section~\ref{parameters:Material_20model/Simple_20compressible_20model}.

This model uses the following equations for the density: \begin{align}  \rho(p,T) = \rho_0              \left(1-\alpha (T-T_a)\right)               \exp{\beta (P-P_0))}\end{align}

`simpler': A material model that has constant values except for density, which depends linearly on temperature: \begin{align}  \rho(p,T) &= \left(1-\alpha (T-T_0)\right)\rho_0.\end{align}

\note{This material model fills the role the ``simple'' material model was originally intended to fill, before the latter acquired all sorts of complicated temperature and compositional dependencies.}


{\it Possible values:} [Selection Morency and Doin|Steinberger|averaging|composition reaction|depth dependent|diffusion dislocation|drucker prager|latent heat|latent heat melt|multicomponent|simple|simple compressible|simpler|unspecified ]
\end{itemize}



\subsection{Parameters in section \tt Material model/Averaging}
\label{parameters:Material_20model/Averaging}

\begin{itemize}
\item {\it Parameter name:} {\tt Averaging operation}
\phantomsection\label{parameters:Material model/Averaging/Averaging operation}


\index[prmindex]{Averaging operation}
\index[prmindexfull]{Material model!Averaging!Averaging operation}
{\it Value:} none


{\it Default:} none


{\it Description:} Chose the averaging operation to use.


{\it Possible values:} [Selection none|arithmetic average|harmonic average|geometric average|pick largest|log average|nwd arithmetic average|nwd harmonic average|nwd geometric average ]
\item {\it Parameter name:} {\tt Base model}
\phantomsection\label{parameters:Material model/Averaging/Base model}


\index[prmindex]{Base model}
\index[prmindexfull]{Material model!Averaging!Base model}
{\it Value:} simple


{\it Default:} simple


{\it Description:} The name of a material model that will be modified by anaveraging operation. Valid values for this parameter are the names of models that are also valid for the ``Material models/Model name'' parameter. See the documentation for that for more information.


{\it Possible values:} [Selection Morency and Doin|Steinberger|averaging|composition reaction|depth dependent|diffusion dislocation|drucker prager|latent heat|latent heat melt|multicomponent|simple|simple compressible|simpler ]
\item {\it Parameter name:} {\tt Bell shape limit}
\phantomsection\label{parameters:Material model/Averaging/Bell shape limit}


\index[prmindex]{Bell shape limit}
\index[prmindexfull]{Material model!Averaging!Bell shape limit}
{\it Value:} 1


{\it Default:} 1


{\it Description:} The limit normalized distance between 0 and 1 where the bell shape becomes zero. See the manual for a more information.


{\it Possible values:} [Double 0...1.79769e+308 (inclusive)]
\end{itemize}

\subsection{Parameters in section \tt Material model/Composition reaction model}
\label{parameters:Material_20model/Composition_20reaction_20model}

\begin{itemize}
\item {\it Parameter name:} {\tt Composition viscosity prefactor 1}
\phantomsection\label{parameters:Material model/Composition reaction model/Composition viscosity prefactor 1}


\index[prmindex]{Composition viscosity prefactor 1}
\index[prmindexfull]{Material model!Composition reaction model!Composition viscosity prefactor 1}
{\it Value:} 1.0


{\it Default:} 1.0


{\it Description:} A linear dependency of viscosity on the first compositional field. Dimensionless prefactor. With a value of 1.0 (the default) the viscosity does not depend on the composition.


{\it Possible values:} [Double 0...1.79769e+308 (inclusive)]
\item {\it Parameter name:} {\tt Composition viscosity prefactor 2}
\phantomsection\label{parameters:Material model/Composition reaction model/Composition viscosity prefactor 2}


\index[prmindex]{Composition viscosity prefactor 2}
\index[prmindexfull]{Material model!Composition reaction model!Composition viscosity prefactor 2}
{\it Value:} 1.0


{\it Default:} 1.0


{\it Description:} A linear dependency of viscosity on the second compositional field. Dimensionless prefactor. With a value of 1.0 (the default) the viscosity does not depend on the composition.


{\it Possible values:} [Double 0...1.79769e+308 (inclusive)]
\item {\it Parameter name:} {\tt Density differential for compositional field 1}
\phantomsection\label{parameters:Material model/Composition reaction model/Density differential for compositional field 1}


\index[prmindex]{Density differential for compositional field 1}
\index[prmindexfull]{Material model!Composition reaction model!Density differential for compositional field 1}
{\it Value:} 0


{\it Default:} 0


{\it Description:} If compositional fields are used, then one would frequently want to make the density depend on these fields. In this simple material model, we make the following assumptions: if no compositional fields are used in the current simulation, then the density is simply the usual one with its linear dependence on the temperature. If there are compositional fields, then the density only depends on the first and the second one in such a way that the density has an additional term of the kind $+\Delta \rho \; c_1(\mathbf x)$. This parameter describes the value of $\Delta \rho$ for the first field. Units: $kg/m^3/\textrm{unit change in composition}$.


{\it Possible values:} [Double -1.79769e+308...1.79769e+308 (inclusive)]
\item {\it Parameter name:} {\tt Density differential for compositional field 2}
\phantomsection\label{parameters:Material model/Composition reaction model/Density differential for compositional field 2}


\index[prmindex]{Density differential for compositional field 2}
\index[prmindexfull]{Material model!Composition reaction model!Density differential for compositional field 2}
{\it Value:} 0


{\it Default:} 0


{\it Description:} If compositional fields are used, then one would frequently want to make the density depend on these fields. In this simple material model, we make the following assumptions: if no compositional fields are used in the current simulation, then the density is simply the usual one with its linear dependence on the temperature. If there are compositional fields, then the density only depends on the first and the second one in such a way that the density has an additional term of the kind $+\Delta \rho \; c_1(\mathbf x)$. This parameter describes the value of $\Delta \rho$ for the second field. Units: $kg/m^3/\textrm{unit change in composition}$.


{\it Possible values:} [Double -1.79769e+308...1.79769e+308 (inclusive)]
\item {\it Parameter name:} {\tt Reaction depth}
\phantomsection\label{parameters:Material model/Composition reaction model/Reaction depth}


\index[prmindex]{Reaction depth}
\index[prmindexfull]{Material model!Composition reaction model!Reaction depth}
{\it Value:} 0


{\it Default:} 0


{\it Description:} Above this depth the compositional fields react: The first field gets converted to the second field. Units: $m$.


{\it Possible values:} [Double 0...1.79769e+308 (inclusive)]
\item {\it Parameter name:} {\tt Reference density}
\phantomsection\label{parameters:Material model/Composition reaction model/Reference density}


\index[prmindex]{Reference density}
\index[prmindexfull]{Material model!Composition reaction model!Reference density}
{\it Value:} 3300


{\it Default:} 3300


{\it Description:} Reference density $\rho_0$. Units: $kg/m^3$.


{\it Possible values:} [Double 0...1.79769e+308 (inclusive)]
\item {\it Parameter name:} {\tt Reference specific heat}
\phantomsection\label{parameters:Material model/Composition reaction model/Reference specific heat}


\index[prmindex]{Reference specific heat}
\index[prmindexfull]{Material model!Composition reaction model!Reference specific heat}
{\it Value:} 1250


{\it Default:} 1250


{\it Description:} The value of the specific heat $cp$. Units: $J/kg/K$.


{\it Possible values:} [Double 0...1.79769e+308 (inclusive)]
\item {\it Parameter name:} {\tt Reference temperature}
\phantomsection\label{parameters:Material model/Composition reaction model/Reference temperature}


\index[prmindex]{Reference temperature}
\index[prmindexfull]{Material model!Composition reaction model!Reference temperature}
{\it Value:} 293


{\it Default:} 293


{\it Description:} The reference temperature $T_0$. Units: $K$.


{\it Possible values:} [Double 0...1.79769e+308 (inclusive)]
\item {\it Parameter name:} {\tt Thermal conductivity}
\phantomsection\label{parameters:Material model/Composition reaction model/Thermal conductivity}


\index[prmindex]{Thermal conductivity}
\index[prmindexfull]{Material model!Composition reaction model!Thermal conductivity}
{\it Value:} 4.7


{\it Default:} 4.7


{\it Description:} The value of the thermal conductivity $k$. Units: $W/m/K$.


{\it Possible values:} [Double 0...1.79769e+308 (inclusive)]
\item {\it Parameter name:} {\tt Thermal expansion coefficient}
\phantomsection\label{parameters:Material model/Composition reaction model/Thermal expansion coefficient}


\index[prmindex]{Thermal expansion coefficient}
\index[prmindexfull]{Material model!Composition reaction model!Thermal expansion coefficient}
{\it Value:} 2e-5


{\it Default:} 2e-5


{\it Description:} The value of the thermal expansion coefficient $\beta$. Units: $1/K$.


{\it Possible values:} [Double 0...1.79769e+308 (inclusive)]
\item {\it Parameter name:} {\tt Thermal viscosity exponent}
\phantomsection\label{parameters:Material model/Composition reaction model/Thermal viscosity exponent}


\index[prmindex]{Thermal viscosity exponent}
\index[prmindexfull]{Material model!Composition reaction model!Thermal viscosity exponent}
{\it Value:} 0.0


{\it Default:} 0.0


{\it Description:} The temperature dependence of viscosity. Dimensionless exponent.


{\it Possible values:} [Double 0...1.79769e+308 (inclusive)]
\item {\it Parameter name:} {\tt Viscosity}
\phantomsection\label{parameters:Material model/Composition reaction model/Viscosity}


\index[prmindex]{Viscosity}
\index[prmindexfull]{Material model!Composition reaction model!Viscosity}
{\it Value:} 5e24


{\it Default:} 5e24


{\it Description:} The value of the constant viscosity. Units: $kg/m/s$.


{\it Possible values:} [Double 0...1.79769e+308 (inclusive)]
\end{itemize}

\subsection{Parameters in section \tt Material model/Depth dependent model}
\label{parameters:Material_20model/Depth_20dependent_20model}

\begin{itemize}
\item {\it Parameter name:} {\tt Base model}
\phantomsection\label{parameters:Material model/Depth dependent model/Base model}


\index[prmindex]{Base model}
\index[prmindexfull]{Material model!Depth dependent model!Base model}
{\it Value:} simple


{\it Default:} simple


{\it Description:} The name of a material model that will be modified by a depth dependent viscosity. Valid values for this parameter are the names of models that are also valid for the ``Material models/Model name'' parameter. See the documentation for that for more information.


{\it Possible values:} [Selection Morency and Doin|Steinberger|averaging|composition reaction|depth dependent|diffusion dislocation|drucker prager|latent heat|latent heat melt|multicomponent|simple|simple compressible|simpler ]
\item {\it Parameter name:} {\tt Data directory}
\phantomsection\label{parameters:Material model/Depth dependent model/Data directory}


\index[prmindex]{Data directory}
\index[prmindexfull]{Material model!Depth dependent model!Data directory}
{\it Value:} ./


{\it Default:} ./


{\it Description:} The path to the model data. The path may also include the special text `\$ASPECT\_SOURCE\_DIR' which will be interpreted as the path in which the ASPECT source files were located when ASPECT was compiled. This interpretation allows, for example, to reference files located in the 'data/' subdirectory of ASPECT. 


{\it Possible values:} [DirectoryName]
\item {\it Parameter name:} {\tt Depth dependence method}
\phantomsection\label{parameters:Material model/Depth dependent model/Depth dependence method}


\index[prmindex]{Depth dependence method}
\index[prmindexfull]{Material model!Depth dependent model!Depth dependence method}
{\it Value:} None


{\it Default:} None


{\it Description:} Method that is used to specify how the viscosity should vary with depth. 


{\it Possible values:} [Selection Function|File|List|None ]
\item {\it Parameter name:} {\tt Depth list}
\phantomsection\label{parameters:Material model/Depth dependent model/Depth list}


\index[prmindex]{Depth list}
\index[prmindexfull]{Material model!Depth dependent model!Depth list}
{\it Value:} 


{\it Default:} 


{\it Description:} A comma-separated list of depth values for use with the ``List'' ``Depth dependence method''. The list must be provided in order ofincreasing depth, and the last value must be greater than or equal to the maximal depth of the model. The depth list is interpreted as a layered viscosity structure and the depth values specify the maximum depths of each layer. 


{\it Possible values:} [List list of [Double -1.79769e+308...1.79769e+308 (inclusive)] of length 0...4294967295 (inclusive)]
\item {\it Parameter name:} {\tt Viscosity depth file}
\phantomsection\label{parameters:Material model/Depth dependent model/Viscosity depth file}


\index[prmindex]{Viscosity depth file}
\index[prmindexfull]{Material model!Depth dependent model!Viscosity depth file}
{\it Value:} visc-depth.txt


{\it Default:} visc-depth.txt


{\it Description:} The name of the file containing depth-dependent viscosity data. 


{\it Possible values:} [Anything]
\item {\it Parameter name:} {\tt Viscosity list}
\phantomsection\label{parameters:Material model/Depth dependent model/Viscosity list}


\index[prmindex]{Viscosity list}
\index[prmindexfull]{Material model!Depth dependent model!Viscosity list}
{\it Value:} 


{\it Default:} 


{\it Description:} A comma-separated list of viscosity values, corresponding to the depth values provided in ``Depth list''. The number of viscosity values specified here must be the same as the number of depths provided in ``Depth list'' 


{\it Possible values:} [List list of [Double -1.79769e+308...1.79769e+308 (inclusive)] of length 0...4294967295 (inclusive)]
\end{itemize}



\subsection{Parameters in section \tt Material model/Depth dependent model/Viscosity depth function}
\label{parameters:Material_20model/Depth_20dependent_20model/Viscosity_20depth_20function}

\begin{itemize}
\item {\it Parameter name:} {\tt Function constants}
\phantomsection\label{parameters:Material model/Depth dependent model/Viscosity depth function/Function constants}


\index[prmindex]{Function constants}
\index[prmindexfull]{Material model!Depth dependent model!Viscosity depth function/Function constants}
{\it Value:} 


{\it Default:} 


{\it Description:} Sometimes it is convenient to use symbolic constants in the expression that describes the function, rather than having to use its numeric value everywhere the constant appears. These values can be defined using this parameter, in the form `var1=value1, var2=value2, ...'.

A typical example would be to set this runtime parameter to `pi=3.1415926536' and then use `pi' in the expression of the actual formula. (That said, for convenience this class actually defines both `pi' and `Pi' by default, but you get the idea.)


{\it Possible values:} [Anything]
\item {\it Parameter name:} {\tt Function expression}
\phantomsection\label{parameters:Material model/Depth dependent model/Viscosity depth function/Function expression}


\index[prmindex]{Function expression}
\index[prmindexfull]{Material model!Depth dependent model!Viscosity depth function/Function expression}
{\it Value:} 1.0e21


{\it Default:} 1.0e21


{\it Possible values:} [Anything]
\item {\it Parameter name:} {\tt Variable names}
\phantomsection\label{parameters:Material model/Depth dependent model/Viscosity depth function/Variable names}


\index[prmindex]{Variable names}
\index[prmindexfull]{Material model!Depth dependent model!Viscosity depth function/Variable names}
{\it Value:} x,t


{\it Default:} x,t


{\it Description:} The name of the variables as they will be used in the function, separated by commas. By default, the names of variables at which the function will be evaluated is `x' (in 1d), `x,y' (in 2d) or `x,y,z' (in 3d) for spatial coordinates and `t' for time. You can then use these variable names in your function expression and they will be replaced by the values of these variables at which the function is currently evaluated. However, you can also choose a different set of names for the independent variables at which to evaluate your function expression. For example, if you work in spherical coordinates, you may wish to set this input parameter to `r,phi,theta,t' and then use these variable names in your function expression.


{\it Possible values:} [Anything]
\end{itemize}

\subsection{Parameters in section \tt Material model/Diffusion dislocation}
\label{parameters:Material_20model/Diffusion_20dislocation}

\begin{itemize}
\item {\it Parameter name:} {\tt Activation energies for diffusion creep}
\phantomsection\label{parameters:Material model/Diffusion dislocation/Activation energies for diffusion creep}


\index[prmindex]{Activation energies for diffusion creep}
\index[prmindexfull]{Material model!Diffusion dislocation!Activation energies for diffusion creep}
{\it Value:} 375e3


{\it Default:} 375e3


{\it Description:} List of activation energies, $E_a$, for background mantle and compositional fields, for a total of N+1 values, where N is the number of compositional fields. If only one values is given, then all use the same value.  Units: $J / mol$


{\it Possible values:} [List list of [Double 0...1.79769e+308 (inclusive)] of length 0...4294967295 (inclusive)]
\item {\it Parameter name:} {\tt Activation energies for dislocation creep}
\phantomsection\label{parameters:Material model/Diffusion dislocation/Activation energies for dislocation creep}


\index[prmindex]{Activation energies for dislocation creep}
\index[prmindexfull]{Material model!Diffusion dislocation!Activation energies for dislocation creep}
{\it Value:} 530e3


{\it Default:} 530e3


{\it Description:} List of activation energies, $E_a$, for background mantle and compositional fields, for a total of N+1 values, where N is the number of compositional fields. If only one values is given, then all use the same value.  Units: $J / mol$


{\it Possible values:} [List list of [Double 0...1.79769e+308 (inclusive)] of length 0...4294967295 (inclusive)]
\item {\it Parameter name:} {\tt Activation volumes for diffusion creep}
\phantomsection\label{parameters:Material model/Diffusion dislocation/Activation volumes for diffusion creep}


\index[prmindex]{Activation volumes for diffusion creep}
\index[prmindexfull]{Material model!Diffusion dislocation!Activation volumes for diffusion creep}
{\it Value:} 6e-6


{\it Default:} 6e-6


{\it Description:} List of activation volumes, $V_a$, for background mantle and compositional fields, for a total of N+1 values, where N is the number of compositional fields. If only one value is given, then all use the same value.  Units: $m^3 / mol$


{\it Possible values:} [List list of [Double 0...1.79769e+308 (inclusive)] of length 0...4294967295 (inclusive)]
\item {\it Parameter name:} {\tt Activation volumes for dislocation creep}
\phantomsection\label{parameters:Material model/Diffusion dislocation/Activation volumes for dislocation creep}


\index[prmindex]{Activation volumes for dislocation creep}
\index[prmindexfull]{Material model!Diffusion dislocation!Activation volumes for dislocation creep}
{\it Value:} 1.4e-5


{\it Default:} 1.4e-5


{\it Description:} List of activation volumes, $V_a$, for background mantle and compositional fields, for a total of N+1 values, where N is the number of compositional fields. If only one value is given, then all use the same value.  Units: $m^3 / mol$


{\it Possible values:} [List list of [Double 0...1.79769e+308 (inclusive)] of length 0...4294967295 (inclusive)]
\item {\it Parameter name:} {\tt Densities}
\phantomsection\label{parameters:Material model/Diffusion dislocation/Densities}


\index[prmindex]{Densities}
\index[prmindexfull]{Material model!Diffusion dislocation!Densities}
{\it Value:} 3300.


{\it Default:} 3300.


{\it Description:} List of densities, $\rho$, for background mantle and compositional fields, for a total of N+1 values, where N is the number of compositional fields. If only one values is given, then all use the same value.  Units: $kg / m^3$


{\it Possible values:} [List list of [Double 0...1.79769e+308 (inclusive)] of length 0...4294967295 (inclusive)]
\item {\it Parameter name:} {\tt Effective viscosity coefficient}
\phantomsection\label{parameters:Material model/Diffusion dislocation/Effective viscosity coefficient}


\index[prmindex]{Effective viscosity coefficient}
\index[prmindexfull]{Material model!Diffusion dislocation!Effective viscosity coefficient}
{\it Value:} 1.0


{\it Default:} 1.0


{\it Description:} Scaling coefficient for effective viscosity.


{\it Possible values:} [Double 0...1.79769e+308 (inclusive)]
\item {\it Parameter name:} {\tt Grain size}
\phantomsection\label{parameters:Material model/Diffusion dislocation/Grain size}


\index[prmindex]{Grain size}
\index[prmindexfull]{Material model!Diffusion dislocation!Grain size}
{\it Value:} 1e-3


{\it Default:} 1e-3


{\it Description:} Units: $m$


{\it Possible values:} [Double 0...1.79769e+308 (inclusive)]
\item {\it Parameter name:} {\tt Grain size exponents for diffusion creep}
\phantomsection\label{parameters:Material model/Diffusion dislocation/Grain size exponents for diffusion creep}


\index[prmindex]{Grain size exponents for diffusion creep}
\index[prmindexfull]{Material model!Diffusion dislocation!Grain size exponents for diffusion creep}
{\it Value:} 3


{\it Default:} 3


{\it Description:} List of grain size exponents, $m_diffusion$, for background mantle and compositional fields, for a total of N+1 values, where N is the number of compositional fields. If only one values is given, then all use the same value.  Units: None


{\it Possible values:} [List list of [Double 0...1.79769e+308 (inclusive)] of length 0...4294967295 (inclusive)]
\item {\it Parameter name:} {\tt Heat capacity}
\phantomsection\label{parameters:Material model/Diffusion dislocation/Heat capacity}


\index[prmindex]{Heat capacity}
\index[prmindexfull]{Material model!Diffusion dislocation!Heat capacity}
{\it Value:} 1.25e3


{\it Default:} 1.25e3


{\it Description:} Units: $J / (K * kg)$


{\it Possible values:} [Double 0...1.79769e+308 (inclusive)]
\item {\it Parameter name:} {\tt Maximum strain rate ratio iterations}
\phantomsection\label{parameters:Material model/Diffusion dislocation/Maximum strain rate ratio iterations}


\index[prmindex]{Maximum strain rate ratio iterations}
\index[prmindexfull]{Material model!Diffusion dislocation!Maximum strain rate ratio iterations}
{\it Value:} 40


{\it Default:} 40


{\it Description:} Maximum number of iterations to find the correct diffusion/dislocation strain rate ratio.


{\it Possible values:} [Integer range 0...2147483647 (inclusive)]
\item {\it Parameter name:} {\tt Maximum viscosity}
\phantomsection\label{parameters:Material model/Diffusion dislocation/Maximum viscosity}


\index[prmindex]{Maximum viscosity}
\index[prmindexfull]{Material model!Diffusion dislocation!Maximum viscosity}
{\it Value:} 1e28


{\it Default:} 1e28


{\it Description:} Upper cutoff for effective viscosity. Units: $Pa s$


{\it Possible values:} [Double 0...1.79769e+308 (inclusive)]
\item {\it Parameter name:} {\tt Minimum strain rate}
\phantomsection\label{parameters:Material model/Diffusion dislocation/Minimum strain rate}


\index[prmindex]{Minimum strain rate}
\index[prmindexfull]{Material model!Diffusion dislocation!Minimum strain rate}
{\it Value:} 1.4e-20


{\it Default:} 1.4e-20


{\it Description:} Stabilizes strain dependent viscosity. Units: $1 / s$


{\it Possible values:} [Double 0...1.79769e+308 (inclusive)]
\item {\it Parameter name:} {\tt Minimum viscosity}
\phantomsection\label{parameters:Material model/Diffusion dislocation/Minimum viscosity}


\index[prmindex]{Minimum viscosity}
\index[prmindexfull]{Material model!Diffusion dislocation!Minimum viscosity}
{\it Value:} 1e17


{\it Default:} 1e17


{\it Description:} Lower cutoff for effective viscosity. Units: $Pa s$


{\it Possible values:} [Double 0...1.79769e+308 (inclusive)]
\item {\it Parameter name:} {\tt Prefactors for diffusion creep}
\phantomsection\label{parameters:Material model/Diffusion dislocation/Prefactors for diffusion creep}


\index[prmindex]{Prefactors for diffusion creep}
\index[prmindexfull]{Material model!Diffusion dislocation!Prefactors for diffusion creep}
{\it Value:} 1.5e-15


{\it Default:} 1.5e-15


{\it Description:} List of viscosity prefactors, $A$, for background mantle and compositional fields, for a total of N+1 values, where N is the number of compositional fields. If only one values is given, then all use the same value. Units: $Pa^{-n_{diffusion}} m^{n_{diffusion}/m_{diffusion}} s^{-1}$


{\it Possible values:} [List list of [Double 0...1.79769e+308 (inclusive)] of length 0...4294967295 (inclusive)]
\item {\it Parameter name:} {\tt Prefactors for dislocation creep}
\phantomsection\label{parameters:Material model/Diffusion dislocation/Prefactors for dislocation creep}


\index[prmindex]{Prefactors for dislocation creep}
\index[prmindexfull]{Material model!Diffusion dislocation!Prefactors for dislocation creep}
{\it Value:} 1.1e-16


{\it Default:} 1.1e-16


{\it Description:} List of viscosity prefactors, $A$, for background mantle and compositional fields, for a total of N+1 values, where N is the number of compositional fields. If only one values is given, then all use the same value. Units: $Pa^{-n_{dislocation}} m^{n_{dislocation}/m_{dislocation}} s^{-1}$


{\it Possible values:} [List list of [Double 0...1.79769e+308 (inclusive)] of length 0...4294967295 (inclusive)]
\item {\it Parameter name:} {\tt Reference temperature}
\phantomsection\label{parameters:Material model/Diffusion dislocation/Reference temperature}


\index[prmindex]{Reference temperature}
\index[prmindexfull]{Material model!Diffusion dislocation!Reference temperature}
{\it Value:} 293


{\it Default:} 293


{\it Description:} For calculating density by thermal expansivity. Units: $K$


{\it Possible values:} [Double 0...1.79769e+308 (inclusive)]
\item {\it Parameter name:} {\tt Reference viscosity}
\phantomsection\label{parameters:Material model/Diffusion dislocation/Reference viscosity}


\index[prmindex]{Reference viscosity}
\index[prmindexfull]{Material model!Diffusion dislocation!Reference viscosity}
{\it Value:} 1e22


{\it Default:} 1e22


{\it Description:} Reference viscosity for nondimensionalization. Units $Pa s$


{\it Possible values:} [Double 0...1.79769e+308 (inclusive)]
\item {\it Parameter name:} {\tt Strain rate residual tolerance}
\phantomsection\label{parameters:Material model/Diffusion dislocation/Strain rate residual tolerance}


\index[prmindex]{Strain rate residual tolerance}
\index[prmindexfull]{Material model!Diffusion dislocation!Strain rate residual tolerance}
{\it Value:} 1e-22


{\it Default:} 1e-22


{\it Description:} Tolerance for correct diffusion/dislocation strain rate ratio.


{\it Possible values:} [Double 0...1.79769e+308 (inclusive)]
\item {\it Parameter name:} {\tt Stress exponents for diffusion creep}
\phantomsection\label{parameters:Material model/Diffusion dislocation/Stress exponents for diffusion creep}


\index[prmindex]{Stress exponents for diffusion creep}
\index[prmindexfull]{Material model!Diffusion dislocation!Stress exponents for diffusion creep}
{\it Value:} 1


{\it Default:} 1


{\it Description:} List of stress exponents, $n_diffusion$, for background mantle and compositional fields, for a total of N+1 values, where N is the number of compositional fields. If only one values is given, then all use the same value.  Units: None


{\it Possible values:} [List list of [Double 0...1.79769e+308 (inclusive)] of length 0...4294967295 (inclusive)]
\item {\it Parameter name:} {\tt Stress exponents for dislocation creep}
\phantomsection\label{parameters:Material model/Diffusion dislocation/Stress exponents for dislocation creep}


\index[prmindex]{Stress exponents for dislocation creep}
\index[prmindexfull]{Material model!Diffusion dislocation!Stress exponents for dislocation creep}
{\it Value:} 3.5


{\it Default:} 3.5


{\it Description:} List of stress exponents, $n_dislocation$, for background mantle and compositional fields, for a total of N+1 values, where N is the number of compositional fields. If only one values is given, then all use the same value.  Units: None


{\it Possible values:} [List list of [Double 0...1.79769e+308 (inclusive)] of length 0...4294967295 (inclusive)]
\item {\it Parameter name:} {\tt Thermal diffusivity}
\phantomsection\label{parameters:Material model/Diffusion dislocation/Thermal diffusivity}


\index[prmindex]{Thermal diffusivity}
\index[prmindexfull]{Material model!Diffusion dislocation!Thermal diffusivity}
{\it Value:} 0.8e-6


{\it Default:} 0.8e-6


{\it Description:} Units: $m^2/s$


{\it Possible values:} [Double 0...1.79769e+308 (inclusive)]
\item {\it Parameter name:} {\tt Thermal expansivities}
\phantomsection\label{parameters:Material model/Diffusion dislocation/Thermal expansivities}


\index[prmindex]{Thermal expansivities}
\index[prmindexfull]{Material model!Diffusion dislocation!Thermal expansivities}
{\it Value:} 3.5e-5


{\it Default:} 3.5e-5


{\it Description:} List of thermal expansivities for background mantle and compositional fields, for a total of N+1 values, where N is the number of compositional fields. If only one values is given, then all use the same value.  Units: $1 / K$


{\it Possible values:} [List list of [Double 0...1.79769e+308 (inclusive)] of length 0...4294967295 (inclusive)]
\item {\it Parameter name:} {\tt Viscosity averaging scheme}
\phantomsection\label{parameters:Material model/Diffusion dislocation/Viscosity averaging scheme}


\index[prmindex]{Viscosity averaging scheme}
\index[prmindexfull]{Material model!Diffusion dislocation!Viscosity averaging scheme}
{\it Value:} harmonic


{\it Default:} harmonic


{\it Description:} When more than one compositional field is present at a point with different viscosities, we need to come up with an average viscosity at that point.  Select a weighted harmonic, arithmetic, geometric, or maximum composition.


{\it Possible values:} [Selection arithmetic|harmonic|geometric|maximum composition ]
\end{itemize}

\subsection{Parameters in section \tt Material model/Drucker Prager}
\label{parameters:Material_20model/Drucker_20Prager}

\begin{itemize}
\item {\it Parameter name:} {\tt Reference density}
\phantomsection\label{parameters:Material model/Drucker Prager/Reference density}


\index[prmindex]{Reference density}
\index[prmindexfull]{Material model!Drucker Prager!Reference density}
{\it Value:} 3300


{\it Default:} 3300


{\it Description:} The reference density $\rho_0$. Units: $kg/m^3$.


{\it Possible values:} [Double 0...1.79769e+308 (inclusive)]
\item {\it Parameter name:} {\tt Reference specific heat}
\phantomsection\label{parameters:Material model/Drucker Prager/Reference specific heat}


\index[prmindex]{Reference specific heat}
\index[prmindexfull]{Material model!Drucker Prager!Reference specific heat}
{\it Value:} 1250


{\it Default:} 1250


{\it Description:} The value of the specific heat $cp$. Units: $J/kg/K$.


{\it Possible values:} [Double 0...1.79769e+308 (inclusive)]
\item {\it Parameter name:} {\tt Reference temperature}
\phantomsection\label{parameters:Material model/Drucker Prager/Reference temperature}


\index[prmindex]{Reference temperature}
\index[prmindexfull]{Material model!Drucker Prager!Reference temperature}
{\it Value:} 293


{\it Default:} 293


{\it Description:} The reference temperature $T_0$. The reference temperature is used in the density calculation. Units: $K$.


{\it Possible values:} [Double 0...1.79769e+308 (inclusive)]
\item {\it Parameter name:} {\tt Reference viscosity}
\phantomsection\label{parameters:Material model/Drucker Prager/Reference viscosity}


\index[prmindex]{Reference viscosity}
\index[prmindexfull]{Material model!Drucker Prager!Reference viscosity}
{\it Value:} 1e22


{\it Default:} 1e22


{\it Description:} The value of the reference viscosity $\eta_0$. Units: $kg/m/s$.


{\it Possible values:} [Double 0...1.79769e+308 (inclusive)]
\item {\it Parameter name:} {\tt Thermal conductivity}
\phantomsection\label{parameters:Material model/Drucker Prager/Thermal conductivity}


\index[prmindex]{Thermal conductivity}
\index[prmindexfull]{Material model!Drucker Prager!Thermal conductivity}
{\it Value:} 4.7


{\it Default:} 4.7


{\it Description:} The value of the thermal conductivity $k$. Units: $W/m/K$.


{\it Possible values:} [Double 0...1.79769e+308 (inclusive)]
\item {\it Parameter name:} {\tt Thermal expansion coefficient}
\phantomsection\label{parameters:Material model/Drucker Prager/Thermal expansion coefficient}


\index[prmindex]{Thermal expansion coefficient}
\index[prmindexfull]{Material model!Drucker Prager!Thermal expansion coefficient}
{\it Value:} 2e-5


{\it Default:} 2e-5


{\it Description:} The value of the thermal expansion coefficient $\beta$. Units: $1/K$.


{\it Possible values:} [Double 0...1.79769e+308 (inclusive)]
\end{itemize}



\subsection{Parameters in section \tt Material model/Drucker Prager/Viscosity}
\label{parameters:Material_20model/Drucker_20Prager/Viscosity}

\begin{itemize}
\item {\it Parameter name:} {\tt Angle of internal friction}
\phantomsection\label{parameters:Material model/Drucker Prager/Viscosity/Angle of internal friction}


\index[prmindex]{Angle of internal friction}
\index[prmindexfull]{Material model!Drucker Prager!Viscosity/Angle of internal friction}
{\it Value:} 0


{\it Default:} 0


{\it Description:} The value of the angle of internal friction $\phi$. For a value of zero, in 2D the von Mises criterion is retrieved. Angles higher than 30 degrees are harder to solve numerically. Units: degrees.


{\it Possible values:} [Double 0...1.79769e+308 (inclusive)]
\item {\it Parameter name:} {\tt Cohesion}
\phantomsection\label{parameters:Material model/Drucker Prager/Viscosity/Cohesion}


\index[prmindex]{Cohesion}
\index[prmindexfull]{Material model!Drucker Prager!Viscosity/Cohesion}
{\it Value:} 2e7


{\it Default:} 2e7


{\it Description:} The value of the cohesion $C$. Units: $Pa$.


{\it Possible values:} [Double 0...1.79769e+308 (inclusive)]
\item {\it Parameter name:} {\tt Maximum viscosity}
\phantomsection\label{parameters:Material model/Drucker Prager/Viscosity/Maximum viscosity}


\index[prmindex]{Maximum viscosity}
\index[prmindexfull]{Material model!Drucker Prager!Viscosity/Maximum viscosity}
{\it Value:} 1e24


{\it Default:} 1e24


{\it Description:} The value of the maximum viscosity cutoff $\eta_max$. Units: $Pa\;s$.


{\it Possible values:} [Double 0...1.79769e+308 (inclusive)]
\item {\it Parameter name:} {\tt Minimum viscosity}
\phantomsection\label{parameters:Material model/Drucker Prager/Viscosity/Minimum viscosity}


\index[prmindex]{Minimum viscosity}
\index[prmindexfull]{Material model!Drucker Prager!Viscosity/Minimum viscosity}
{\it Value:} 1e19


{\it Default:} 1e19


{\it Description:} The value of the minimum viscosity cutoff $\eta_min$. Units: $Pa\;s$.


{\it Possible values:} [Double 0...1.79769e+308 (inclusive)]
\item {\it Parameter name:} {\tt Reference strain rate}
\phantomsection\label{parameters:Material model/Drucker Prager/Viscosity/Reference strain rate}


\index[prmindex]{Reference strain rate}
\index[prmindexfull]{Material model!Drucker Prager!Viscosity/Reference strain rate}
{\it Value:} 1e-15


{\it Default:} 1e-15


{\it Description:} The value of the initial strain rate prescribed during the first nonlinear iteration $\dot{\epsilon}_ref$. Units: $1/s$.


{\it Possible values:} [Double 0...1.79769e+308 (inclusive)]
\end{itemize}

\subsection{Parameters in section \tt Material model/Latent heat}
\label{parameters:Material_20model/Latent_20heat}

\begin{itemize}
\item {\it Parameter name:} {\tt Composition viscosity prefactor}
\phantomsection\label{parameters:Material model/Latent heat/Composition viscosity prefactor}


\index[prmindex]{Composition viscosity prefactor}
\index[prmindexfull]{Material model!Latent heat!Composition viscosity prefactor}
{\it Value:} 1.0


{\it Default:} 1.0


{\it Description:} A linear dependency of viscosity on composition. Dimensionless prefactor.


{\it Possible values:} [Double 0...1.79769e+308 (inclusive)]
\item {\it Parameter name:} {\tt Compressibility}
\phantomsection\label{parameters:Material model/Latent heat/Compressibility}


\index[prmindex]{Compressibility}
\index[prmindexfull]{Material model!Latent heat!Compressibility}
{\it Value:} 5.124e-12


{\it Default:} 5.124e-12


{\it Description:} The value of the compressibility $\kappa$. Units: $1/Pa$.


{\it Possible values:} [Double 0...1.79769e+308 (inclusive)]
\item {\it Parameter name:} {\tt Corresponding phase for density jump}
\phantomsection\label{parameters:Material model/Latent heat/Corresponding phase for density jump}


\index[prmindex]{Corresponding phase for density jump}
\index[prmindexfull]{Material model!Latent heat!Corresponding phase for density jump}
{\it Value:} 


{\it Default:} 


{\it Description:} A list of phases, which correspond to the Phase transition density jumps. The density jumps occur only in the phase that is given by this phase value. 0 stands for the 1st compositional fields, 1 for the second compositional field and -1 for none of them. List must have the same number of entries as Phase transition depths. Units: $Pa/K$.


{\it Possible values:} [List list of [Integer range 0...2147483647 (inclusive)] of length 0...4294967295 (inclusive)]
\item {\it Parameter name:} {\tt Define transition by depth instead of pressure}
\phantomsection\label{parameters:Material model/Latent heat/Define transition by depth instead of pressure}


\index[prmindex]{Define transition by depth instead of pressure}
\index[prmindexfull]{Material model!Latent heat!Define transition by depth instead of pressure}
{\it Value:} true


{\it Default:} true


{\it Description:} Whether to list phase transitions by depth or pressure. If this parameter is true,then the input file will use Phase transitions depths and Phase transition widthsto define the phase transition. If it is false, the parameter file will read inphase transition data from Phase transition pressures andPhase transition pressure widths.


{\it Possible values:} [Bool]
\item {\it Parameter name:} {\tt Density differential for compositional field 1}
\phantomsection\label{parameters:Material model/Latent heat/Density differential for compositional field 1}


\index[prmindex]{Density differential for compositional field 1}
\index[prmindexfull]{Material model!Latent heat!Density differential for compositional field 1}
{\it Value:} 0


{\it Default:} 0


{\it Description:} If compositional fields are used, then one would frequently want to make the density depend on these fields. In this simple material model, we make the following assumptions: if no compositional fields are used in the current simulation, then the density is simply the usual one with its linear dependence on the temperature. If there are compositional fields, then the density only depends on the first one in such a way that the density has an additional term of the kind $+\Delta \rho \; c_1(\mathbf x)$. This parameter describes the value of $\Delta \rho$. Units: $kg/m^3/\textrm{unit change in composition}$.


{\it Possible values:} [Double -1.79769e+308...1.79769e+308 (inclusive)]
\item {\it Parameter name:} {\tt Phase transition Clapeyron slopes}
\phantomsection\label{parameters:Material model/Latent heat/Phase transition Clapeyron slopes}


\index[prmindex]{Phase transition Clapeyron slopes}
\index[prmindexfull]{Material model!Latent heat!Phase transition Clapeyron slopes}
{\it Value:} 


{\it Default:} 


{\it Description:} A list of Clapeyron slopes for each phase transition. A positive Clapeyron slope indicates that the phase transition will occur in a greater depth, if the temperature is higher than the one given in Phase transition temperatures and in a smaller depth, if the temperature is smaller than the one given in Phase transition temperatures. For negative slopes the other way round. List must have the same number of entries as Phase transition depths. Units: $Pa/K$.


{\it Possible values:} [List list of [Double -1.79769e+308...1.79769e+308 (inclusive)] of length 0...4294967295 (inclusive)]
\item {\it Parameter name:} {\tt Phase transition density jumps}
\phantomsection\label{parameters:Material model/Latent heat/Phase transition density jumps}


\index[prmindex]{Phase transition density jumps}
\index[prmindexfull]{Material model!Latent heat!Phase transition density jumps}
{\it Value:} 


{\it Default:} 


{\it Description:} A list of density jumps at each phase transition. A positive value means that the density increases with depth. The corresponding entry in Corresponding phase for density jump determines if the density jump occurs in peridotite, eclogite or none of them.List must have the same number of entries as Phase transition depths. Units: $kg/m^3$.


{\it Possible values:} [List list of [Double 0...1.79769e+308 (inclusive)] of length 0...4294967295 (inclusive)]
\item {\it Parameter name:} {\tt Phase transition depths}
\phantomsection\label{parameters:Material model/Latent heat/Phase transition depths}


\index[prmindex]{Phase transition depths}
\index[prmindexfull]{Material model!Latent heat!Phase transition depths}
{\it Value:} 


{\it Default:} 


{\it Description:} A list of depths where phase transitions occur. Values must monotonically increase. Units: $m$.


{\it Possible values:} [List list of [Double 0...1.79769e+308 (inclusive)] of length 0...4294967295 (inclusive)]
\item {\it Parameter name:} {\tt Phase transition pressure widths}
\phantomsection\label{parameters:Material model/Latent heat/Phase transition pressure widths}


\index[prmindex]{Phase transition pressure widths}
\index[prmindexfull]{Material model!Latent heat!Phase transition pressure widths}
{\it Value:} 


{\it Default:} 


{\it Description:} A list of widths for each phase transition, in terms of pressure. The phase functions are scaled with these values, leading to a jump betwen phases for a value of zero and a gradual transition for larger values. List must have the same number of entries as Phase transition pressures. Define transition by depth instead of pressure must be set to falseto use this parameter.Units: $Pa$.


{\it Possible values:} [List list of [Double 0...1.79769e+308 (inclusive)] of length 0...4294967295 (inclusive)]
\item {\it Parameter name:} {\tt Phase transition pressures}
\phantomsection\label{parameters:Material model/Latent heat/Phase transition pressures}


\index[prmindex]{Phase transition pressures}
\index[prmindexfull]{Material model!Latent heat!Phase transition pressures}
{\it Value:} 


{\it Default:} 


{\it Description:} A list of pressures where phase transitions occur. Values must monotonically increase. Define transition by depth instead ofpressure must be set to false to use this parameter.Units: $Pa$.


{\it Possible values:} [List list of [Double 0...1.79769e+308 (inclusive)] of length 0...4294967295 (inclusive)]
\item {\it Parameter name:} {\tt Phase transition temperatures}
\phantomsection\label{parameters:Material model/Latent heat/Phase transition temperatures}


\index[prmindex]{Phase transition temperatures}
\index[prmindexfull]{Material model!Latent heat!Phase transition temperatures}
{\it Value:} 


{\it Default:} 


{\it Description:} A list of temperatures where phase transitions occur. Higher or lower temperatures lead to phase transition ocurring in smaller or greater depths than given in Phase transition depths, depending on the Clapeyron slope given in Phase transition Clapeyron slopes. List must have the same number of entries as Phase transition depths. Units: $K$.


{\it Possible values:} [List list of [Double 0...1.79769e+308 (inclusive)] of length 0...4294967295 (inclusive)]
\item {\it Parameter name:} {\tt Phase transition widths}
\phantomsection\label{parameters:Material model/Latent heat/Phase transition widths}


\index[prmindex]{Phase transition widths}
\index[prmindexfull]{Material model!Latent heat!Phase transition widths}
{\it Value:} 


{\it Default:} 


{\it Description:} A list of widths for each phase transition, in terms of depth. The phase functions are scaled with these values, leading to a jump betwen phases for a value of zero and a gradual transition for larger values. List must have the same number of entries as Phase transition depths. Units: $m$.


{\it Possible values:} [List list of [Double 0...1.79769e+308 (inclusive)] of length 0...4294967295 (inclusive)]
\item {\it Parameter name:} {\tt Reference density}
\phantomsection\label{parameters:Material model/Latent heat/Reference density}


\index[prmindex]{Reference density}
\index[prmindexfull]{Material model!Latent heat!Reference density}
{\it Value:} 3300


{\it Default:} 3300


{\it Description:} Reference density $\rho_0$. Units: $kg/m^3$.


{\it Possible values:} [Double 0...1.79769e+308 (inclusive)]
\item {\it Parameter name:} {\tt Reference specific heat}
\phantomsection\label{parameters:Material model/Latent heat/Reference specific heat}


\index[prmindex]{Reference specific heat}
\index[prmindexfull]{Material model!Latent heat!Reference specific heat}
{\it Value:} 1250


{\it Default:} 1250


{\it Description:} The value of the specific heat $cp$. Units: $J/kg/K$.


{\it Possible values:} [Double 0...1.79769e+308 (inclusive)]
\item {\it Parameter name:} {\tt Reference temperature}
\phantomsection\label{parameters:Material model/Latent heat/Reference temperature}


\index[prmindex]{Reference temperature}
\index[prmindexfull]{Material model!Latent heat!Reference temperature}
{\it Value:} 293


{\it Default:} 293


{\it Description:} The reference temperature $T_0$. Units: $K$.


{\it Possible values:} [Double 0...1.79769e+308 (inclusive)]
\item {\it Parameter name:} {\tt Thermal conductivity}
\phantomsection\label{parameters:Material model/Latent heat/Thermal conductivity}


\index[prmindex]{Thermal conductivity}
\index[prmindexfull]{Material model!Latent heat!Thermal conductivity}
{\it Value:} 2.38


{\it Default:} 2.38


{\it Description:} The value of the thermal conductivity $k$. Units: $W/m/K$.


{\it Possible values:} [Double 0...1.79769e+308 (inclusive)]
\item {\it Parameter name:} {\tt Thermal expansion coefficient}
\phantomsection\label{parameters:Material model/Latent heat/Thermal expansion coefficient}


\index[prmindex]{Thermal expansion coefficient}
\index[prmindexfull]{Material model!Latent heat!Thermal expansion coefficient}
{\it Value:} 4e-5


{\it Default:} 4e-5


{\it Description:} The value of the thermal expansion coefficient $\beta$. Units: $1/K$.


{\it Possible values:} [Double 0...1.79769e+308 (inclusive)]
\item {\it Parameter name:} {\tt Thermal viscosity exponent}
\phantomsection\label{parameters:Material model/Latent heat/Thermal viscosity exponent}


\index[prmindex]{Thermal viscosity exponent}
\index[prmindexfull]{Material model!Latent heat!Thermal viscosity exponent}
{\it Value:} 0.0


{\it Default:} 0.0


{\it Description:} The temperature dependence of viscosity. Dimensionless exponent.


{\it Possible values:} [Double 0...1.79769e+308 (inclusive)]
\item {\it Parameter name:} {\tt Viscosity}
\phantomsection\label{parameters:Material model/Latent heat/Viscosity}


\index[prmindex]{Viscosity}
\index[prmindexfull]{Material model!Latent heat!Viscosity}
{\it Value:} 5e24


{\it Default:} 5e24


{\it Description:} The value of the constant viscosity. Units: $kg/m/s$.


{\it Possible values:} [Double 0...1.79769e+308 (inclusive)]
\item {\it Parameter name:} {\tt Viscosity prefactors}
\phantomsection\label{parameters:Material model/Latent heat/Viscosity prefactors}


\index[prmindex]{Viscosity prefactors}
\index[prmindexfull]{Material model!Latent heat!Viscosity prefactors}
{\it Value:} 


{\it Default:} 


{\it Description:} A list of prefactors for the viscosity for each phase. The reference viscosity will be multiplied by this factor to get the corresponding viscosity for each phase. List must have one more entry than Phase transition depths. Units: non-dimensional.


{\it Possible values:} [List list of [Double 0...1.79769e+308 (inclusive)] of length 0...4294967295 (inclusive)]
\end{itemize}

\subsection{Parameters in section \tt Material model/Latent heat melt}
\label{parameters:Material_20model/Latent_20heat_20melt}

\begin{itemize}
\item {\it Parameter name:} {\tt A1}
\phantomsection\label{parameters:Material model/Latent heat melt/A1}


\index[prmindex]{A1}
\index[prmindexfull]{Material model!Latent heat melt!A1}
{\it Value:} 1085.7


{\it Default:} 1085.7


{\it Description:} Constant parameter in the quadratic function that approximates the solidus of peridotite. Units: $°C$.


{\it Possible values:} [Double -1.79769e+308...1.79769e+308 (inclusive)]
\item {\it Parameter name:} {\tt A2}
\phantomsection\label{parameters:Material model/Latent heat melt/A2}


\index[prmindex]{A2}
\index[prmindexfull]{Material model!Latent heat melt!A2}
{\it Value:} 1.329e-7


{\it Default:} 1.329e-7


{\it Description:} Prefactor of the linear pressure term in the quadratic function that approximates the solidus of peridotite. Units: $°C/Pa$.


{\it Possible values:} [Double -1.79769e+308...1.79769e+308 (inclusive)]
\item {\it Parameter name:} {\tt A3}
\phantomsection\label{parameters:Material model/Latent heat melt/A3}


\index[prmindex]{A3}
\index[prmindexfull]{Material model!Latent heat melt!A3}
{\it Value:} -5.1e-18


{\it Default:} -5.1e-18


{\it Description:} Prefactor of the quadratic pressure term in the quadratic function that approximates the solidus of peridotite. Units: $°C/(Pa^2)$.


{\it Possible values:} [Double -1.79769e+308...1.79769e+308 (inclusive)]
\item {\it Parameter name:} {\tt B1}
\phantomsection\label{parameters:Material model/Latent heat melt/B1}


\index[prmindex]{B1}
\index[prmindexfull]{Material model!Latent heat melt!B1}
{\it Value:} 1475.0


{\it Default:} 1475.0


{\it Description:} Constant parameter in the quadratic function that approximates the lherzolite liquidus used for calculating the fraction of peridotite-derived melt. Units: $°C$.


{\it Possible values:} [Double -1.79769e+308...1.79769e+308 (inclusive)]
\item {\it Parameter name:} {\tt B2}
\phantomsection\label{parameters:Material model/Latent heat melt/B2}


\index[prmindex]{B2}
\index[prmindexfull]{Material model!Latent heat melt!B2}
{\it Value:} 8.0e-8


{\it Default:} 8.0e-8


{\it Description:} Prefactor of the linear pressure term in the quadratic function that approximates the  lherzolite liquidus used for calculating the fraction of peridotite-derived melt. Units: $°C/Pa$.


{\it Possible values:} [Double -1.79769e+308...1.79769e+308 (inclusive)]
\item {\it Parameter name:} {\tt B3}
\phantomsection\label{parameters:Material model/Latent heat melt/B3}


\index[prmindex]{B3}
\index[prmindexfull]{Material model!Latent heat melt!B3}
{\it Value:} -3.2e-18


{\it Default:} -3.2e-18


{\it Description:} Prefactor of the quadratic pressure term in the quadratic function that approximates the  lherzolite liquidus used for calculating the fraction of peridotite-derived melt. Units: $°C/(Pa^2)$.


{\it Possible values:} [Double -1.79769e+308...1.79769e+308 (inclusive)]
\item {\it Parameter name:} {\tt C1}
\phantomsection\label{parameters:Material model/Latent heat melt/C1}


\index[prmindex]{C1}
\index[prmindexfull]{Material model!Latent heat melt!C1}
{\it Value:} 1780.0


{\it Default:} 1780.0


{\it Description:} Constant parameter in the quadratic function that approximates the liquidus of peridotite. Units: $°C$.


{\it Possible values:} [Double -1.79769e+308...1.79769e+308 (inclusive)]
\item {\it Parameter name:} {\tt C2}
\phantomsection\label{parameters:Material model/Latent heat melt/C2}


\index[prmindex]{C2}
\index[prmindexfull]{Material model!Latent heat melt!C2}
{\it Value:} 4.50e-8


{\it Default:} 4.50e-8


{\it Description:} Prefactor of the linear pressure term in the quadratic function that approximates the liquidus of peridotite. Units: $°C/Pa$.


{\it Possible values:} [Double -1.79769e+308...1.79769e+308 (inclusive)]
\item {\it Parameter name:} {\tt C3}
\phantomsection\label{parameters:Material model/Latent heat melt/C3}


\index[prmindex]{C3}
\index[prmindexfull]{Material model!Latent heat melt!C3}
{\it Value:} -2.0e-18


{\it Default:} -2.0e-18


{\it Description:} Prefactor of the quadratic pressure term in the quadratic function that approximates the liquidus of peridotite. Units: $°C/(Pa^2)$.


{\it Possible values:} [Double -1.79769e+308...1.79769e+308 (inclusive)]
\item {\it Parameter name:} {\tt Composition viscosity prefactor}
\phantomsection\label{parameters:Material model/Latent heat melt/Composition viscosity prefactor}


\index[prmindex]{Composition viscosity prefactor}
\index[prmindexfull]{Material model!Latent heat melt!Composition viscosity prefactor}
{\it Value:} 1.0


{\it Default:} 1.0


{\it Description:} A linear dependency of viscosity on composition. Dimensionless prefactor.


{\it Possible values:} [Double 0...1.79769e+308 (inclusive)]
\item {\it Parameter name:} {\tt Compressibility}
\phantomsection\label{parameters:Material model/Latent heat melt/Compressibility}


\index[prmindex]{Compressibility}
\index[prmindexfull]{Material model!Latent heat melt!Compressibility}
{\it Value:} 5.124e-12


{\it Default:} 5.124e-12


{\it Description:} The value of the compressibility $\kappa$. Units: $1/Pa$.


{\it Possible values:} [Double 0...1.79769e+308 (inclusive)]
\item {\it Parameter name:} {\tt D1}
\phantomsection\label{parameters:Material model/Latent heat melt/D1}


\index[prmindex]{D1}
\index[prmindexfull]{Material model!Latent heat melt!D1}
{\it Value:} 976.0


{\it Default:} 976.0


{\it Description:} Constant parameter in the quadratic function that approximates the solidus of pyroxenite. Units: $°C$.


{\it Possible values:} [Double -1.79769e+308...1.79769e+308 (inclusive)]
\item {\it Parameter name:} {\tt D2}
\phantomsection\label{parameters:Material model/Latent heat melt/D2}


\index[prmindex]{D2}
\index[prmindexfull]{Material model!Latent heat melt!D2}
{\it Value:} 1.329e-7


{\it Default:} 1.329e-7


{\it Description:} Prefactor of the linear pressure term in the quadratic function that approximates the solidus of pyroxenite. Note that this factor is different from the value given in Sobolev, 2011, because they use the potential temperature whereas we use the absolute temperature. Units: $°C/Pa$.


{\it Possible values:} [Double -1.79769e+308...1.79769e+308 (inclusive)]
\item {\it Parameter name:} {\tt D3}
\phantomsection\label{parameters:Material model/Latent heat melt/D3}


\index[prmindex]{D3}
\index[prmindexfull]{Material model!Latent heat melt!D3}
{\it Value:} -5.1e-18


{\it Default:} -5.1e-18


{\it Description:} Prefactor of the quadratic pressure term in the quadratic function that approximates the solidus of pyroxenite. Units: $°C/(Pa^2)$.


{\it Possible values:} [Double -1.79769e+308...1.79769e+308 (inclusive)]
\item {\it Parameter name:} {\tt Density differential for compositional field 1}
\phantomsection\label{parameters:Material model/Latent heat melt/Density differential for compositional field 1}


\index[prmindex]{Density differential for compositional field 1}
\index[prmindexfull]{Material model!Latent heat melt!Density differential for compositional field 1}
{\it Value:} 0


{\it Default:} 0


{\it Description:} If compositional fields are used, then one would frequently want to make the density depend on these fields. In this simple material model, we make the following assumptions: if no compositional fields are used in the current simulation, then the density is simply the usual one with its linear dependence on the temperature. If there are compositional fields, then the density only depends on the first one in such a way that the density has an additional term of the kind $+\Delta \rho \; c_1(\mathbf x)$. This parameter describes the value of $\Delta \rho$. Units: $kg/m^3/\textrm{unit change in composition}$.


{\it Possible values:} [Double -1.79769e+308...1.79769e+308 (inclusive)]
\item {\it Parameter name:} {\tt E1}
\phantomsection\label{parameters:Material model/Latent heat melt/E1}


\index[prmindex]{E1}
\index[prmindexfull]{Material model!Latent heat melt!E1}
{\it Value:} 663.8


{\it Default:} 663.8


{\it Description:} Prefactor of the linear depletion term in the quadratic function that approximates the melt fraction of pyroxenite. Units: $°C/Pa$.


{\it Possible values:} [Double -1.79769e+308...1.79769e+308 (inclusive)]
\item {\it Parameter name:} {\tt E2}
\phantomsection\label{parameters:Material model/Latent heat melt/E2}


\index[prmindex]{E2}
\index[prmindexfull]{Material model!Latent heat melt!E2}
{\it Value:} -611.4


{\it Default:} -611.4


{\it Description:} Prefactor of the quadratic depletion term in the quadratic function that approximates the melt fraction of pyroxenite. Units: $°C/(Pa^2)$.


{\it Possible values:} [Double -1.79769e+308...1.79769e+308 (inclusive)]
\item {\it Parameter name:} {\tt Mass fraction cpx}
\phantomsection\label{parameters:Material model/Latent heat melt/Mass fraction cpx}


\index[prmindex]{Mass fraction cpx}
\index[prmindexfull]{Material model!Latent heat melt!Mass fraction cpx}
{\it Value:} 0.15


{\it Default:} 0.15


{\it Description:} Mass fraction of clinopyroxene in the peridotite to be molten. Units: non-dimensional.


{\it Possible values:} [Double -1.79769e+308...1.79769e+308 (inclusive)]
\item {\it Parameter name:} {\tt Maximum pyroxenite melt fraction}
\phantomsection\label{parameters:Material model/Latent heat melt/Maximum pyroxenite melt fraction}


\index[prmindex]{Maximum pyroxenite melt fraction}
\index[prmindexfull]{Material model!Latent heat melt!Maximum pyroxenite melt fraction}
{\it Value:} 0.5429


{\it Default:} 0.5429


{\it Description:} Maximum melt fraction of pyroxenite in this parameterization. At higher temperatures peridotite begins to melt.


{\it Possible values:} [Double -1.79769e+308...1.79769e+308 (inclusive)]
\item {\it Parameter name:} {\tt Peridotite melting entropy change}
\phantomsection\label{parameters:Material model/Latent heat melt/Peridotite melting entropy change}


\index[prmindex]{Peridotite melting entropy change}
\index[prmindexfull]{Material model!Latent heat melt!Peridotite melting entropy change}
{\it Value:} -300


{\it Default:} -300


{\it Description:} The entropy change for the phase transition from solid to melt of peridotite. Units: $J/(kg K)$.


{\it Possible values:} [Double -1.79769e+308...1.79769e+308 (inclusive)]
\item {\it Parameter name:} {\tt Pyroxenite melting entropy change}
\phantomsection\label{parameters:Material model/Latent heat melt/Pyroxenite melting entropy change}


\index[prmindex]{Pyroxenite melting entropy change}
\index[prmindexfull]{Material model!Latent heat melt!Pyroxenite melting entropy change}
{\it Value:} -400


{\it Default:} -400


{\it Description:} The entropy change for the phase transition from solid to melt of pyroxenite. Units: $J/(kg K)$.


{\it Possible values:} [Double -1.79769e+308...1.79769e+308 (inclusive)]
\item {\it Parameter name:} {\tt Reference density}
\phantomsection\label{parameters:Material model/Latent heat melt/Reference density}


\index[prmindex]{Reference density}
\index[prmindexfull]{Material model!Latent heat melt!Reference density}
{\it Value:} 3300


{\it Default:} 3300


{\it Description:} Reference density $\rho_0$. Units: $kg/m^3$.


{\it Possible values:} [Double 0...1.79769e+308 (inclusive)]
\item {\it Parameter name:} {\tt Reference specific heat}
\phantomsection\label{parameters:Material model/Latent heat melt/Reference specific heat}


\index[prmindex]{Reference specific heat}
\index[prmindexfull]{Material model!Latent heat melt!Reference specific heat}
{\it Value:} 1250


{\it Default:} 1250


{\it Description:} The value of the specific heat $cp$. Units: $J/kg/K$.


{\it Possible values:} [Double 0...1.79769e+308 (inclusive)]
\item {\it Parameter name:} {\tt Reference temperature}
\phantomsection\label{parameters:Material model/Latent heat melt/Reference temperature}


\index[prmindex]{Reference temperature}
\index[prmindexfull]{Material model!Latent heat melt!Reference temperature}
{\it Value:} 293


{\it Default:} 293


{\it Description:} The reference temperature $T_0$. Units: $K$.


{\it Possible values:} [Double 0...1.79769e+308 (inclusive)]
\item {\it Parameter name:} {\tt Relative density of melt}
\phantomsection\label{parameters:Material model/Latent heat melt/Relative density of melt}


\index[prmindex]{Relative density of melt}
\index[prmindexfull]{Material model!Latent heat melt!Relative density of melt}
{\it Value:} 0.9


{\it Default:} 0.9


{\it Description:} The relative density of melt compared to the solid material. This means, the density change upon melting is this parameter times the density of solid material.Units: non-dimensional.


{\it Possible values:} [Double -1.79769e+308...1.79769e+308 (inclusive)]
\item {\it Parameter name:} {\tt Thermal conductivity}
\phantomsection\label{parameters:Material model/Latent heat melt/Thermal conductivity}


\index[prmindex]{Thermal conductivity}
\index[prmindexfull]{Material model!Latent heat melt!Thermal conductivity}
{\it Value:} 2.38


{\it Default:} 2.38


{\it Description:} The value of the thermal conductivity $k$. Units: $W/m/K$.


{\it Possible values:} [Double 0...1.79769e+308 (inclusive)]
\item {\it Parameter name:} {\tt Thermal expansion coefficient}
\phantomsection\label{parameters:Material model/Latent heat melt/Thermal expansion coefficient}


\index[prmindex]{Thermal expansion coefficient}
\index[prmindexfull]{Material model!Latent heat melt!Thermal expansion coefficient}
{\it Value:} 4e-5


{\it Default:} 4e-5


{\it Description:} The value of the thermal expansion coefficient $\alpha_s$. Units: $1/K$.


{\it Possible values:} [Double 0...1.79769e+308 (inclusive)]
\item {\it Parameter name:} {\tt Thermal expansion coefficient of melt}
\phantomsection\label{parameters:Material model/Latent heat melt/Thermal expansion coefficient of melt}


\index[prmindex]{Thermal expansion coefficient of melt}
\index[prmindexfull]{Material model!Latent heat melt!Thermal expansion coefficient of melt}
{\it Value:} 6.8e-5


{\it Default:} 6.8e-5


{\it Description:} The value of the thermal expansion coefficient $\alpha_f$. Units: $1/K$.


{\it Possible values:} [Double 0...1.79769e+308 (inclusive)]
\item {\it Parameter name:} {\tt Thermal viscosity exponent}
\phantomsection\label{parameters:Material model/Latent heat melt/Thermal viscosity exponent}


\index[prmindex]{Thermal viscosity exponent}
\index[prmindexfull]{Material model!Latent heat melt!Thermal viscosity exponent}
{\it Value:} 0.0


{\it Default:} 0.0


{\it Description:} The temperature dependence of viscosity. Dimensionless exponent.


{\it Possible values:} [Double 0...1.79769e+308 (inclusive)]
\item {\it Parameter name:} {\tt Viscosity}
\phantomsection\label{parameters:Material model/Latent heat melt/Viscosity}


\index[prmindex]{Viscosity}
\index[prmindexfull]{Material model!Latent heat melt!Viscosity}
{\it Value:} 5e24


{\it Default:} 5e24


{\it Description:} The value of the constant viscosity. Units: $kg/m/s$.


{\it Possible values:} [Double 0...1.79769e+308 (inclusive)]
\item {\it Parameter name:} {\tt beta}
\phantomsection\label{parameters:Material model/Latent heat melt/beta}


\index[prmindex]{beta}
\index[prmindexfull]{Material model!Latent heat melt!beta}
{\it Value:} 1.5


{\it Default:} 1.5


{\it Description:} Exponent of the melting temperature in the melt fraction calculation. Units: non-dimensional.


{\it Possible values:} [Double -1.79769e+308...1.79769e+308 (inclusive)]
\item {\it Parameter name:} {\tt r1}
\phantomsection\label{parameters:Material model/Latent heat melt/r1}


\index[prmindex]{r1}
\index[prmindexfull]{Material model!Latent heat melt!r1}
{\it Value:} 0.5


{\it Default:} 0.5


{\it Description:} Constant in the linear function that approximates the clinopyroxene reaction coefficient. Units: non-dimensional.


{\it Possible values:} [Double -1.79769e+308...1.79769e+308 (inclusive)]
\item {\it Parameter name:} {\tt r2}
\phantomsection\label{parameters:Material model/Latent heat melt/r2}


\index[prmindex]{r2}
\index[prmindexfull]{Material model!Latent heat melt!r2}
{\it Value:} 8e-11


{\it Default:} 8e-11


{\it Description:} Prefactor of the linear pressure term in the linear function that approximates the clinopyroxene reaction coefficient. Units: $1/Pa$.


{\it Possible values:} [Double -1.79769e+308...1.79769e+308 (inclusive)]
\end{itemize}

\subsection{Parameters in section \tt Material model/Morency and Doin}
\label{parameters:Material_20model/Morency_20and_20Doin}

\begin{itemize}
\item {\it Parameter name:} {\tt Activation energies}
\phantomsection\label{parameters:Material model/Morency and Doin/Activation energies}


\index[prmindex]{Activation energies}
\index[prmindexfull]{Material model!Morency and Doin!Activation energies}
{\it Value:} 500


{\it Default:} 500


{\it Description:} List of activation energies, $E_a$, for background mantle and compositional fields,for a total of N+1 values, where N is the number of compositional fields.If only one values is given, then all use the same value.  Units: $kJ / mol$


{\it Possible values:} [List list of [Double 0...1.79769e+308 (inclusive)] of length 0...4294967295 (inclusive)]
\item {\it Parameter name:} {\tt Activation volume}
\phantomsection\label{parameters:Material model/Morency and Doin/Activation volume}


\index[prmindex]{Activation volume}
\index[prmindexfull]{Material model!Morency and Doin!Activation volume}
{\it Value:} 6.4e-6


{\it Default:} 6.4e-6


{\it Description:} ($V_a$). Units: $m^3 / mol$


{\it Possible values:} [Double 0...1.79769e+308 (inclusive)]
\item {\it Parameter name:} {\tt Coefficient of yield stress increase with depth}
\phantomsection\label{parameters:Material model/Morency and Doin/Coefficient of yield stress increase with depth}


\index[prmindex]{Coefficient of yield stress increase with depth}
\index[prmindexfull]{Material model!Morency and Doin!Coefficient of yield stress increase with depth}
{\it Value:} 0.25


{\it Default:} 0.25


{\it Description:} ($\gamma$). Units: None


{\it Possible values:} [Double 0...1.79769e+308 (inclusive)]
\item {\it Parameter name:} {\tt Cohesive strength of rocks at the surface}
\phantomsection\label{parameters:Material model/Morency and Doin/Cohesive strength of rocks at the surface}


\index[prmindex]{Cohesive strength of rocks at the surface}
\index[prmindexfull]{Material model!Morency and Doin!Cohesive strength of rocks at the surface}
{\it Value:} 117


{\it Default:} 117


{\it Description:} ($\tau_0$). Units: $Pa$


{\it Possible values:} [Double 0...1.79769e+308 (inclusive)]
\item {\it Parameter name:} {\tt Densities}
\phantomsection\label{parameters:Material model/Morency and Doin/Densities}


\index[prmindex]{Densities}
\index[prmindexfull]{Material model!Morency and Doin!Densities}
{\it Value:} 3300.


{\it Default:} 3300.


{\it Description:} List of densities, $\rho$, for background mantle and compositional fields,for a total of N+1 values, where N is the number of compositional fields.If only one values is given, then all use the same value.  Units: $kg / m^3$


{\it Possible values:} [List list of [Double 0...1.79769e+308 (inclusive)] of length 0...4294967295 (inclusive)]
\item {\it Parameter name:} {\tt Heat capacity}
\phantomsection\label{parameters:Material model/Morency and Doin/Heat capacity}


\index[prmindex]{Heat capacity}
\index[prmindexfull]{Material model!Morency and Doin!Heat capacity}
{\it Value:} 1.25e3


{\it Default:} 1.25e3


{\it Description:} Units: $J / (K * kg)$


{\it Possible values:} [Double 0...1.79769e+308 (inclusive)]
\item {\it Parameter name:} {\tt Minimum strain rate}
\phantomsection\label{parameters:Material model/Morency and Doin/Minimum strain rate}


\index[prmindex]{Minimum strain rate}
\index[prmindexfull]{Material model!Morency and Doin!Minimum strain rate}
{\it Value:} 1.4e-20


{\it Default:} 1.4e-20


{\it Description:} Stabilizes strain dependent viscosity. Units: $1 / s$


{\it Possible values:} [Double 0...1.79769e+308 (inclusive)]
\item {\it Parameter name:} {\tt Preexponential constant for viscous rheology law}
\phantomsection\label{parameters:Material model/Morency and Doin/Preexponential constant for viscous rheology law}


\index[prmindex]{Preexponential constant for viscous rheology law}
\index[prmindexfull]{Material model!Morency and Doin!Preexponential constant for viscous rheology law}
{\it Value:} 1.24e14


{\it Default:} 1.24e14


{\it Description:} ($B$). Units: None


{\it Possible values:} [Double 0...1.79769e+308 (inclusive)]
\item {\it Parameter name:} {\tt Reference strain rate}
\phantomsection\label{parameters:Material model/Morency and Doin/Reference strain rate}


\index[prmindex]{Reference strain rate}
\index[prmindexfull]{Material model!Morency and Doin!Reference strain rate}
{\it Value:} 6.4e-16


{\it Default:} 6.4e-16


{\it Description:} ($\dot{\epsilon}_{ref}$). Units: $1 / s$


{\it Possible values:} [Double 0...1.79769e+308 (inclusive)]
\item {\it Parameter name:} {\tt Reference temperature}
\phantomsection\label{parameters:Material model/Morency and Doin/Reference temperature}


\index[prmindex]{Reference temperature}
\index[prmindexfull]{Material model!Morency and Doin!Reference temperature}
{\it Value:} 293


{\it Default:} 293


{\it Description:} For calculating density by thermal expansivity. Units: $K$


{\it Possible values:} [Double 0...1.79769e+308 (inclusive)]
\item {\it Parameter name:} {\tt Reference viscosity}
\phantomsection\label{parameters:Material model/Morency and Doin/Reference viscosity}


\index[prmindex]{Reference viscosity}
\index[prmindexfull]{Material model!Morency and Doin!Reference viscosity}
{\it Value:} 1e22


{\it Default:} 1e22


{\it Description:} Reference viscosity for nondimensionalization.


{\it Possible values:} [Double 0...1.79769e+308 (inclusive)]
\item {\it Parameter name:} {\tt Stress exponents for plastic rheology}
\phantomsection\label{parameters:Material model/Morency and Doin/Stress exponents for plastic rheology}


\index[prmindex]{Stress exponents for plastic rheology}
\index[prmindexfull]{Material model!Morency and Doin!Stress exponents for plastic rheology}
{\it Value:} 30


{\it Default:} 30


{\it Description:} List of stress exponents, $n_p$, for background mantle and compositional fields,for a total of N+1 values, where N is the number of compositional fields.If only one values is given, then all use the same value.  Units: None


{\it Possible values:} [List list of [Double 0...1.79769e+308 (inclusive)] of length 0...4294967295 (inclusive)]
\item {\it Parameter name:} {\tt Stress exponents for viscous rheology}
\phantomsection\label{parameters:Material model/Morency and Doin/Stress exponents for viscous rheology}


\index[prmindex]{Stress exponents for viscous rheology}
\index[prmindexfull]{Material model!Morency and Doin!Stress exponents for viscous rheology}
{\it Value:} 3


{\it Default:} 3


{\it Description:} List of stress exponents, $n_v$, for background mantle and compositional fields,for a total of N+1 values, where N is the number of compositional fields.If only one values is given, then all use the same value.  Units: None


{\it Possible values:} [List list of [Double 0...1.79769e+308 (inclusive)] of length 0...4294967295 (inclusive)]
\item {\it Parameter name:} {\tt Thermal diffusivity}
\phantomsection\label{parameters:Material model/Morency and Doin/Thermal diffusivity}


\index[prmindex]{Thermal diffusivity}
\index[prmindexfull]{Material model!Morency and Doin!Thermal diffusivity}
{\it Value:} 0.8e-6


{\it Default:} 0.8e-6


{\it Description:} Units: $m^2/s$


{\it Possible values:} [Double 0...1.79769e+308 (inclusive)]
\item {\it Parameter name:} {\tt Thermal expansivities}
\phantomsection\label{parameters:Material model/Morency and Doin/Thermal expansivities}


\index[prmindex]{Thermal expansivities}
\index[prmindexfull]{Material model!Morency and Doin!Thermal expansivities}
{\it Value:} 3.5e-5


{\it Default:} 3.5e-5


{\it Description:} List of thermal expansivities for background mantle and compositional fields,for a total of N+1 values, where N is the number of compositional fields.If only one values is given, then all use the same value.  Units: $1 / K$


{\it Possible values:} [List list of [Double 0...1.79769e+308 (inclusive)] of length 0...4294967295 (inclusive)]
\end{itemize}

\subsection{Parameters in section \tt Material model/Multicomponent}
\label{parameters:Material_20model/Multicomponent}

\begin{itemize}
\item {\it Parameter name:} {\tt Densities}
\phantomsection\label{parameters:Material model/Multicomponent/Densities}


\index[prmindex]{Densities}
\index[prmindexfull]{Material model!Multicomponent!Densities}
{\it Value:} 3300.


{\it Default:} 3300.


{\it Description:} List of densities for background mantle and compositional fields,for a total of N+1 values, where N is the number of compositional fields.If only one value is given, then all use the same value.  Units: $kg / m^3$


{\it Possible values:} [List list of [Double 0...1.79769e+308 (inclusive)] of length 0...4294967295 (inclusive)]
\item {\it Parameter name:} {\tt Reference temperature}
\phantomsection\label{parameters:Material model/Multicomponent/Reference temperature}


\index[prmindex]{Reference temperature}
\index[prmindexfull]{Material model!Multicomponent!Reference temperature}
{\it Value:} 293


{\it Default:} 293


{\it Description:} The reference temperature $T_0$. Units: $K$.


{\it Possible values:} [Double 0...1.79769e+308 (inclusive)]
\item {\it Parameter name:} {\tt Specific heats}
\phantomsection\label{parameters:Material model/Multicomponent/Specific heats}


\index[prmindex]{Specific heats}
\index[prmindexfull]{Material model!Multicomponent!Specific heats}
{\it Value:} 1250.


{\it Default:} 1250.


{\it Description:} List of specific heats for background mantle and compositional fields,for a total of N+1 values, where N is the number of compositional fields.If only one value is given, then all use the same value. Units: $J /kg /K$


{\it Possible values:} [List list of [Double 0...1.79769e+308 (inclusive)] of length 0...4294967295 (inclusive)]
\item {\it Parameter name:} {\tt Thermal conductivities}
\phantomsection\label{parameters:Material model/Multicomponent/Thermal conductivities}


\index[prmindex]{Thermal conductivities}
\index[prmindexfull]{Material model!Multicomponent!Thermal conductivities}
{\it Value:} 4.7


{\it Default:} 4.7


{\it Description:} List of thermal conductivities for background mantle and compositional fields,for a total of N+1 values, where N is the number of compositional fields.If only one value is given, then all use the same value. Units: $W/m/K$ 


{\it Possible values:} [List list of [Double 0...1.79769e+308 (inclusive)] of length 0...4294967295 (inclusive)]
\item {\it Parameter name:} {\tt Thermal expansivities}
\phantomsection\label{parameters:Material model/Multicomponent/Thermal expansivities}


\index[prmindex]{Thermal expansivities}
\index[prmindexfull]{Material model!Multicomponent!Thermal expansivities}
{\it Value:} 4.e-5


{\it Default:} 4.e-5


{\it Description:} List of thermal expansivities for background mantle and compositional fields,for a total of N+1 values, where N is the number of compositional fields.If only one value is given, then all use the same value. Units: $1/K$


{\it Possible values:} [List list of [Double 0...1.79769e+308 (inclusive)] of length 0...4294967295 (inclusive)]
\item {\it Parameter name:} {\tt Viscosities}
\phantomsection\label{parameters:Material model/Multicomponent/Viscosities}


\index[prmindex]{Viscosities}
\index[prmindexfull]{Material model!Multicomponent!Viscosities}
{\it Value:} 1.e21


{\it Default:} 1.e21


{\it Description:} List of viscosities for background mantle and compositional fields,for a total of N+1 values, where N is the number of compositional fields.If only one value is given, then all use the same value. Units: $Pa s$


{\it Possible values:} [List list of [Double 0...1.79769e+308 (inclusive)] of length 0...4294967295 (inclusive)]
\item {\it Parameter name:} {\tt Viscosity averaging scheme}
\phantomsection\label{parameters:Material model/Multicomponent/Viscosity averaging scheme}


\index[prmindex]{Viscosity averaging scheme}
\index[prmindexfull]{Material model!Multicomponent!Viscosity averaging scheme}
{\it Value:} harmonic


{\it Default:} harmonic


{\it Description:} When more than one compositional field is present at a point with different viscosities, we need to come up with an average viscosity at that point.  Select a weighted harmonic, arithmetic, geometric, or maximum composition.


{\it Possible values:} [Selection arithmetic|harmonic|geometric|maximum composition ]
\end{itemize}

\subsection{Parameters in section \tt Material model/Simple compressible model}
\label{parameters:Material_20model/Simple_20compressible_20model}

\begin{itemize}
\item {\it Parameter name:} {\tt Reference compressibility}
\phantomsection\label{parameters:Material model/Simple compressible model/Reference compressibility}


\index[prmindex]{Reference compressibility}
\index[prmindexfull]{Material model!Simple compressible model!Reference compressibility}
{\it Value:} 4e-12


{\it Default:} 4e-12


{\it Description:} The value of the reference compressibility. Units: $1/Pa$.


{\it Possible values:} [Double 0...1.79769e+308 (inclusive)]
\item {\it Parameter name:} {\tt Reference density}
\phantomsection\label{parameters:Material model/Simple compressible model/Reference density}


\index[prmindex]{Reference density}
\index[prmindexfull]{Material model!Simple compressible model!Reference density}
{\it Value:} 3300


{\it Default:} 3300


{\it Description:} Reference density $\rho_0$. Units: $kg/m^3$.


{\it Possible values:} [Double 0...1.79769e+308 (inclusive)]
\item {\it Parameter name:} {\tt Reference specific heat}
\phantomsection\label{parameters:Material model/Simple compressible model/Reference specific heat}


\index[prmindex]{Reference specific heat}
\index[prmindexfull]{Material model!Simple compressible model!Reference specific heat}
{\it Value:} 1250


{\it Default:} 1250


{\it Description:} The value of the specific heat $cp$. Units: $J/kg/K$.


{\it Possible values:} [Double 0...1.79769e+308 (inclusive)]
\item {\it Parameter name:} {\tt Thermal conductivity}
\phantomsection\label{parameters:Material model/Simple compressible model/Thermal conductivity}


\index[prmindex]{Thermal conductivity}
\index[prmindexfull]{Material model!Simple compressible model!Thermal conductivity}
{\it Value:} 4.7


{\it Default:} 4.7


{\it Description:} The value of the thermal conductivity $k$. Units: $W/m/K$.


{\it Possible values:} [Double 0...1.79769e+308 (inclusive)]
\item {\it Parameter name:} {\tt Thermal expansion coefficient}
\phantomsection\label{parameters:Material model/Simple compressible model/Thermal expansion coefficient}


\index[prmindex]{Thermal expansion coefficient}
\index[prmindexfull]{Material model!Simple compressible model!Thermal expansion coefficient}
{\it Value:} 2e-5


{\it Default:} 2e-5


{\it Description:} The value of the thermal expansion coefficient $\alpha$. Units: $1/K$.


{\it Possible values:} [Double 0...1.79769e+308 (inclusive)]
\item {\it Parameter name:} {\tt Viscosity}
\phantomsection\label{parameters:Material model/Simple compressible model/Viscosity}


\index[prmindex]{Viscosity}
\index[prmindexfull]{Material model!Simple compressible model!Viscosity}
{\it Value:} 1e21


{\it Default:} 1e21


{\it Description:} The value of the constant viscosity $\eta_0$. Units: $kg/m/s$.


{\it Possible values:} [Double 0...1.79769e+308 (inclusive)]
\end{itemize}

\subsection{Parameters in section \tt Material model/Simple model}
\label{parameters:Material_20model/Simple_20model}

\begin{itemize}
\item {\it Parameter name:} {\tt Composition viscosity prefactor}
\phantomsection\label{parameters:Material model/Simple model/Composition viscosity prefactor}


\index[prmindex]{Composition viscosity prefactor}
\index[prmindexfull]{Material model!Simple model!Composition viscosity prefactor}
{\it Value:} 1.0


{\it Default:} 1.0


{\it Description:} A linear dependency of viscosity on the first compositional field. Dimensionless prefactor. With a value of 1.0 (the default) the viscosity does not depend on the composition. See the general documentation of this model for a formula that states the dependence of the viscosity on this factor, which is called $\xi$ there.


{\it Possible values:} [Double 0...1.79769e+308 (inclusive)]
\item {\it Parameter name:} {\tt Density differential for compositional field 1}
\phantomsection\label{parameters:Material model/Simple model/Density differential for compositional field 1}


\index[prmindex]{Density differential for compositional field 1}
\index[prmindexfull]{Material model!Simple model!Density differential for compositional field 1}
{\it Value:} 0


{\it Default:} 0


{\it Description:} If compositional fields are used, then one would frequently want to make the density depend on these fields. In this simple material model, we make the following assumptions: if no compositional fields are used in the current simulation, then the density is simply the usual one with its linear dependence on the temperature. If there are compositional fields, then the density only depends on the first one in such a way that the density has an additional term of the kind $+\Delta \rho \; c_1(\mathbf x)$. This parameter describes the value of $\Delta \rho$. Units: $kg/m^3/\textrm{unit change in composition}$.


{\it Possible values:} [Double -1.79769e+308...1.79769e+308 (inclusive)]
\item {\it Parameter name:} {\tt Reference density}
\phantomsection\label{parameters:Material model/Simple model/Reference density}


\index[prmindex]{Reference density}
\index[prmindexfull]{Material model!Simple model!Reference density}
{\it Value:} 3300


{\it Default:} 3300


{\it Description:} Reference density $\rho_0$. Units: $kg/m^3$.


{\it Possible values:} [Double 0...1.79769e+308 (inclusive)]
\item {\it Parameter name:} {\tt Reference specific heat}
\phantomsection\label{parameters:Material model/Simple model/Reference specific heat}


\index[prmindex]{Reference specific heat}
\index[prmindexfull]{Material model!Simple model!Reference specific heat}
{\it Value:} 1250


{\it Default:} 1250


{\it Description:} The value of the specific heat $cp$. Units: $J/kg/K$.


{\it Possible values:} [Double 0...1.79769e+308 (inclusive)]
\item {\it Parameter name:} {\tt Reference temperature}
\phantomsection\label{parameters:Material model/Simple model/Reference temperature}


\index[prmindex]{Reference temperature}
\index[prmindexfull]{Material model!Simple model!Reference temperature}
{\it Value:} 293


{\it Default:} 293


{\it Description:} The reference temperature $T_0$. The reference temperature is used in both the density and viscosity formulas. Units: $K$.


{\it Possible values:} [Double 0...1.79769e+308 (inclusive)]
\item {\it Parameter name:} {\tt Thermal conductivity}
\phantomsection\label{parameters:Material model/Simple model/Thermal conductivity}


\index[prmindex]{Thermal conductivity}
\index[prmindexfull]{Material model!Simple model!Thermal conductivity}
{\it Value:} 4.7


{\it Default:} 4.7


{\it Description:} The value of the thermal conductivity $k$. Units: $W/m/K$.


{\it Possible values:} [Double 0...1.79769e+308 (inclusive)]
\item {\it Parameter name:} {\tt Thermal expansion coefficient}
\phantomsection\label{parameters:Material model/Simple model/Thermal expansion coefficient}


\index[prmindex]{Thermal expansion coefficient}
\index[prmindexfull]{Material model!Simple model!Thermal expansion coefficient}
{\it Value:} 2e-5


{\it Default:} 2e-5


{\it Description:} The value of the thermal expansion coefficient $\alpha$. Units: $1/K$.


{\it Possible values:} [Double 0...1.79769e+308 (inclusive)]
\item {\it Parameter name:} {\tt Thermal viscosity exponent}
\phantomsection\label{parameters:Material model/Simple model/Thermal viscosity exponent}


\index[prmindex]{Thermal viscosity exponent}
\index[prmindexfull]{Material model!Simple model!Thermal viscosity exponent}
{\it Value:} 0.0


{\it Default:} 0.0


{\it Description:} The temperature dependence of viscosity. Dimensionless exponent. See the general documentation of this model for a formula that states the dependence of the viscosity on this factor, which is called $\beta$ there.


{\it Possible values:} [Double 0...1.79769e+308 (inclusive)]
\item {\it Parameter name:} {\tt Viscosity}
\phantomsection\label{parameters:Material model/Simple model/Viscosity}


\index[prmindex]{Viscosity}
\index[prmindexfull]{Material model!Simple model!Viscosity}
{\it Value:} 5e24


{\it Default:} 5e24


{\it Description:} The value of the constant viscosity $\eta_0$. This viscosity may be modified by both temperature and compositional dependencies. Units: $kg/m/s$.


{\it Possible values:} [Double 0...1.79769e+308 (inclusive)]
\end{itemize}

\subsection{Parameters in section \tt Material model/Simpler model}
\label{parameters:Material_20model/Simpler_20model}

\begin{itemize}
\item {\it Parameter name:} {\tt Reference density}
\phantomsection\label{parameters:Material model/Simpler model/Reference density}


\index[prmindex]{Reference density}
\index[prmindexfull]{Material model!Simpler model!Reference density}
{\it Value:} 3300


{\it Default:} 3300


{\it Description:} Reference density $\rho_0$. Units: $kg/m^3$.


{\it Possible values:} [Double 0...1.79769e+308 (inclusive)]
\item {\it Parameter name:} {\tt Reference specific heat}
\phantomsection\label{parameters:Material model/Simpler model/Reference specific heat}


\index[prmindex]{Reference specific heat}
\index[prmindexfull]{Material model!Simpler model!Reference specific heat}
{\it Value:} 1250


{\it Default:} 1250


{\it Description:} The value of the specific heat $cp$. Units: $J/kg/K$.


{\it Possible values:} [Double 0...1.79769e+308 (inclusive)]
\item {\it Parameter name:} {\tt Reference temperature}
\phantomsection\label{parameters:Material model/Simpler model/Reference temperature}


\index[prmindex]{Reference temperature}
\index[prmindexfull]{Material model!Simpler model!Reference temperature}
{\it Value:} 293


{\it Default:} 293


{\it Description:} The reference temperature $T_0$. The reference temperature is used in the density formula. Units: $K$.


{\it Possible values:} [Double 0...1.79769e+308 (inclusive)]
\item {\it Parameter name:} {\tt Thermal conductivity}
\phantomsection\label{parameters:Material model/Simpler model/Thermal conductivity}


\index[prmindex]{Thermal conductivity}
\index[prmindexfull]{Material model!Simpler model!Thermal conductivity}
{\it Value:} 4.7


{\it Default:} 4.7


{\it Description:} The value of the thermal conductivity $k$. Units: $W/m/K$.


{\it Possible values:} [Double 0...1.79769e+308 (inclusive)]
\item {\it Parameter name:} {\tt Thermal expansion coefficient}
\phantomsection\label{parameters:Material model/Simpler model/Thermal expansion coefficient}


\index[prmindex]{Thermal expansion coefficient}
\index[prmindexfull]{Material model!Simpler model!Thermal expansion coefficient}
{\it Value:} 2e-5


{\it Default:} 2e-5


{\it Description:} The value of the thermal expansion coefficient $\beta$. Units: $1/K$.


{\it Possible values:} [Double 0...1.79769e+308 (inclusive)]
\item {\it Parameter name:} {\tt Viscosity}
\phantomsection\label{parameters:Material model/Simpler model/Viscosity}


\index[prmindex]{Viscosity}
\index[prmindexfull]{Material model!Simpler model!Viscosity}
{\it Value:} 5e24


{\it Default:} 5e24


{\it Description:} The value of the viscosity $\eta$. Units: $kg/m/s$.


{\it Possible values:} [Double 0...1.79769e+308 (inclusive)]
\end{itemize}

\subsection{Parameters in section \tt Material model/Steinberger model}
\label{parameters:Material_20model/Steinberger_20model}

\begin{itemize}
\item {\it Parameter name:} {\tt Bilinear interpolation}
\phantomsection\label{parameters:Material model/Steinberger model/Bilinear interpolation}


\index[prmindex]{Bilinear interpolation}
\index[prmindexfull]{Material model!Steinberger model!Bilinear interpolation}
{\it Value:} true


{\it Default:} true


{\it Description:} Whether to use bilinear interpolation to compute material properties (slower but more accurate). 


{\it Possible values:} [Bool]
\item {\it Parameter name:} {\tt Compressible}
\phantomsection\label{parameters:Material model/Steinberger model/Compressible}


\index[prmindex]{Compressible}
\index[prmindexfull]{Material model!Steinberger model!Compressible}
{\it Value:} false


{\it Default:} false


{\it Description:} Whether to include a compressible material description.For a description see the manual section. 


{\it Possible values:} [Bool]
\item {\it Parameter name:} {\tt Data directory}
\phantomsection\label{parameters:Material model/Steinberger model/Data directory}


\index[prmindex]{Data directory}
\index[prmindexfull]{Material model!Steinberger model!Data directory}
{\it Value:} \$ASPECT\_SOURCE\_DIR/data/material-model/steinberger/


{\it Default:} \$ASPECT\_SOURCE\_DIR/data/material-model/steinberger/


{\it Description:} The path to the model data. The path may also include the special text '\$ASPECT\_SOURCE\_DIR' which will be interpreted as the path in which the ASPECT source files were located when ASPECT was compiled. This interpretation allows, for example, to reference files located in the 'data/' subdirectory of ASPECT. 


{\it Possible values:} [DirectoryName]
\item {\it Parameter name:} {\tt Latent heat}
\phantomsection\label{parameters:Material model/Steinberger model/Latent heat}


\index[prmindex]{Latent heat}
\index[prmindexfull]{Material model!Steinberger model!Latent heat}
{\it Value:} false


{\it Default:} false


{\it Description:} Whether to include latent heat effects in the calculation of thermal expansivity and specific heat. Following the approach of Nakagawa et al. 2009. 


{\it Possible values:} [Bool]
\item {\it Parameter name:} {\tt Lateral viscosity file name}
\phantomsection\label{parameters:Material model/Steinberger model/Lateral viscosity file name}


\index[prmindex]{Lateral viscosity file name}
\index[prmindexfull]{Material model!Steinberger model!Lateral viscosity file name}
{\it Value:} temp-viscosity-prefactor.txt


{\it Default:} temp-viscosity-prefactor.txt


{\it Description:} The file name of the lateral viscosity data. 


{\it Possible values:} [Anything]
\item {\it Parameter name:} {\tt Material file names}
\phantomsection\label{parameters:Material model/Steinberger model/Material file names}


\index[prmindex]{Material file names}
\index[prmindexfull]{Material model!Steinberger model!Material file names}
{\it Value:} pyr-ringwood88.txt


{\it Default:} pyr-ringwood88.txt


{\it Description:} The file names of the material data. List with as many components as active compositional fields (material data is assumed to be in order with the ordering of the fields). 


{\it Possible values:} [List list of [Anything] of length 0...4294967295 (inclusive)]
\item {\it Parameter name:} {\tt Maximum lateral viscosity variation}
\phantomsection\label{parameters:Material model/Steinberger model/Maximum lateral viscosity variation}


\index[prmindex]{Maximum lateral viscosity variation}
\index[prmindexfull]{Material model!Steinberger model!Maximum lateral viscosity variation}
{\it Value:} 1e2


{\it Default:} 1e2


{\it Description:} The relative cutoff value for lateral viscosity variations caused by temperature deviations. The viscosity may vary laterally by this factor squared.


{\it Possible values:} [Double 0...1.79769e+308 (inclusive)]
\item {\it Parameter name:} {\tt Maximum viscosity}
\phantomsection\label{parameters:Material model/Steinberger model/Maximum viscosity}


\index[prmindex]{Maximum viscosity}
\index[prmindexfull]{Material model!Steinberger model!Maximum viscosity}
{\it Value:} 1e23


{\it Default:} 1e23


{\it Description:} The maximum viscosity that is allowed in the viscosity calculation. Larger values will be cut off.


{\it Possible values:} [Double 0...1.79769e+308 (inclusive)]
\item {\it Parameter name:} {\tt Minimum viscosity}
\phantomsection\label{parameters:Material model/Steinberger model/Minimum viscosity}


\index[prmindex]{Minimum viscosity}
\index[prmindexfull]{Material model!Steinberger model!Minimum viscosity}
{\it Value:} 1e19


{\it Default:} 1e19


{\it Description:} The minimum viscosity that is allowed in the viscosity calculation. Smaller values will be cut off.


{\it Possible values:} [Double 0...1.79769e+308 (inclusive)]
\item {\it Parameter name:} {\tt Radial viscosity file name}
\phantomsection\label{parameters:Material model/Steinberger model/Radial viscosity file name}


\index[prmindex]{Radial viscosity file name}
\index[prmindexfull]{Material model!Steinberger model!Radial viscosity file name}
{\it Value:} radial-visc.txt


{\it Default:} radial-visc.txt


{\it Description:} The file name of the radial viscosity data. 


{\it Possible values:} [Anything]
\item {\it Parameter name:} {\tt Reference viscosity}
\phantomsection\label{parameters:Material model/Steinberger model/Reference viscosity}


\index[prmindex]{Reference viscosity}
\index[prmindexfull]{Material model!Steinberger model!Reference viscosity}
{\it Value:} 1e23


{\it Default:} 1e23


{\it Description:} The reference viscosity that is used for pressure scaling. 


{\it Possible values:} [Double 0...1.79769e+308 (inclusive)]
\item {\it Parameter name:} {\tt Use lateral average temperature for viscosity}
\phantomsection\label{parameters:Material model/Steinberger model/Use lateral average temperature for viscosity}


\index[prmindex]{Use lateral average temperature for viscosity}
\index[prmindexfull]{Material model!Steinberger model!Use lateral average temperature for viscosity}
{\it Value:} true


{\it Default:} true


{\it Description:} Whether to use to use the laterally averaged temperature instead of the adiabatic temperature for the viscosity calculation. This ensures that the laterally averaged viscosities remain more or less constant over the model runtime. This behaviour might or might not be desired.


{\it Possible values:} [Bool]
\end{itemize}

\subsection{Parameters in section \tt Mesh refinement}
\label{parameters:Mesh_20refinement}

\begin{itemize}
\item {\it Parameter name:} {\tt Additional refinement times}
\phantomsection\label{parameters:Mesh refinement/Additional refinement times}


\index[prmindex]{Additional refinement times}
\index[prmindexfull]{Mesh refinement!Additional refinement times}
{\it Value:} 


{\it Default:} 


{\it Description:} A list of times so that if the end time of a time step is beyond this time, an additional round of mesh refinement is triggered. This is mostly useful to make sure we can get through the initial transient phase of a simulation on a relatively coarse mesh, and then refine again when we are in a time range that we are interested in and where we would like to use a finer mesh. Units: Each element of the list has units years if the 'Use years in output instead of seconds' parameter is set; seconds otherwise.


{\it Possible values:} [List list of [Double 0...1.79769e+308 (inclusive)] of length 0...4294967295 (inclusive)]
\item {\it Parameter name:} {\tt Coarsening fraction}
\phantomsection\label{parameters:Mesh refinement/Coarsening fraction}


\index[prmindex]{Coarsening fraction}
\index[prmindexfull]{Mesh refinement!Coarsening fraction}
{\it Value:} 0.05


{\it Default:} 0.05


{\it Description:} The fraction of cells with the smallest error that should be flagged for coarsening.


{\it Possible values:} [Double 0...1 (inclusive)]
\item {\it Parameter name:} {\tt Initial adaptive refinement}
\phantomsection\label{parameters:Mesh refinement/Initial adaptive refinement}


\index[prmindex]{Initial adaptive refinement}
\index[prmindexfull]{Mesh refinement!Initial adaptive refinement}
{\it Value:} 2


{\it Default:} 2


{\it Description:} The number of adaptive refinement steps performed after initial global refinement but while still within the first time step.


{\it Possible values:} [Integer range 0...2147483647 (inclusive)]
\item {\it Parameter name:} {\tt Initial global refinement}
\phantomsection\label{parameters:Mesh refinement/Initial global refinement}


\index[prmindex]{Initial global refinement}
\index[prmindexfull]{Mesh refinement!Initial global refinement}
{\it Value:} 2


{\it Default:} 2


{\it Description:} The number of global refinement steps performed on the initial coarse mesh, before the problem is first solved there.


{\it Possible values:} [Integer range 0...2147483647 (inclusive)]
\item {\it Parameter name:} {\tt Minimum refinement level}
\phantomsection\label{parameters:Mesh refinement/Minimum refinement level}


\index[prmindex]{Minimum refinement level}
\index[prmindexfull]{Mesh refinement!Minimum refinement level}
{\it Value:} 0


{\it Default:} 0


{\it Description:} The minimum refinement level each cell should have, and that can not be exceeded by coarsening. Should not be higher than the 'Initial global refinement' parameter.


{\it Possible values:} [Integer range 0...2147483647 (inclusive)]
\item {\it Parameter name:} {\tt Normalize individual refinement criteria}
\phantomsection\label{parameters:Mesh refinement/Normalize individual refinement criteria}


\index[prmindex]{Normalize individual refinement criteria}
\index[prmindexfull]{Mesh refinement!Normalize individual refinement criteria}
{\it Value:} true


{\it Default:} true


{\it Description:} If multiple refinement criteria are specified in the ``Strategy'' parameter, then they need to be combined somehow to form the final refinement indicators. This is done using the method described by the ``Refinement criteria merge operation'' parameter which can either operate on the raw refinement indicators returned by each strategy (i.e., dimensional quantities) or using normalized values where the indicators of each strategy are first normalized to the interval $[0,1]$ (which also makes them non-dimensional). This parameter determines whether this normalization will happen.


{\it Possible values:} [Bool]
\item {\it Parameter name:} {\tt Refinement criteria merge operation}
\phantomsection\label{parameters:Mesh refinement/Refinement criteria merge operation}


\index[prmindex]{Refinement criteria merge operation}
\index[prmindexfull]{Mesh refinement!Refinement criteria merge operation}
{\it Value:} max


{\it Default:} max


{\it Description:} If multiple mesh refinement criteria are computed for each cell (by passing a list of more than element to the \texttt{Strategy} parameter in this section of the input file) then one will have to decide which one should win when deciding which cell to refine. The operation that selects from these competing criteria is the one that is selected here. The options are:

\begin{itemize}
\item \texttt{plus}: Add the various error indicators together and refine those cells on which the sum of indicators is largest.
\item \texttt{max}: Take the maximum of the various error indicators and refine those cells on which the maximal indicators is largest.
\end{itemize}The refinement indicators computed by each strategy are modified by the ``Normalize individual refinement criteria'' and ``Refinement criteria scale factors'' parameters.


{\it Possible values:} [Selection plus|max ]
\item {\it Parameter name:} {\tt Refinement criteria scaling factors}
\phantomsection\label{parameters:Mesh refinement/Refinement criteria scaling factors}


\index[prmindex]{Refinement criteria scaling factors}
\index[prmindexfull]{Mesh refinement!Refinement criteria scaling factors}
{\it Value:} 


{\it Default:} 


{\it Description:} A list of scaling factors by which every individual refinement criterion will be multiplied by. If only a single refinement criterion is selected (using the ``Strategy'' parameter, then this parameter has no particular meaning. On the other hand, if multiple criteria are chosen, then these factors are used to weigh the various indicators relative to each other. 

If ``Normalize individual refinement criteria'' is set to true, then the criteria will first be normalized to the interval $[0,1]$ and then multiplied by the factors specified here. You will likely want to choose the factors to be not too far from 1 in that case, say between 1 and 10, to avoid essentially disabling those criteria with small weights. On the other hand, if the criteria are not normalized to $[0,1]$ using the parameter mentioned above, then the factors you specify here need to take into account the relative numerical size of refinement indicators (which in that case carry physical units).

You can experimentally play with these scaling factors by choosing to output the refinement indicators into the graphical output of a run.

If the list of indicators given in this parameter is empty, then this indicates that they should all be chosen equal to one. If the list is not empty then it needs to have as many entries as there are indicators chosen in the ``Strategy'' parameter.


{\it Possible values:} [List list of [Double 0...1.79769e+308 (inclusive)] of length 0...4294967295 (inclusive)]
\item {\it Parameter name:} {\tt Refinement fraction}
\phantomsection\label{parameters:Mesh refinement/Refinement fraction}


\index[prmindex]{Refinement fraction}
\index[prmindexfull]{Mesh refinement!Refinement fraction}
{\it Value:} 0.3


{\it Default:} 0.3


{\it Description:} The fraction of cells with the largest error that should be flagged for refinement.


{\it Possible values:} [Double 0...1 (inclusive)]
\item {\it Parameter name:} {\tt Run postprocessors on initial refinement}
\phantomsection\label{parameters:Mesh refinement/Run postprocessors on initial refinement}


\index[prmindex]{Run postprocessors on initial refinement}
\index[prmindexfull]{Mesh refinement!Run postprocessors on initial refinement}
{\it Value:} false


{\it Default:} false


{\it Description:} Whether or not the postproccessors should be run at the end of each of ths initial adaptive refinement cycles at the of the simulation start.


{\it Possible values:} [Bool]
\item {\it Parameter name:} {\tt Strategy}
\phantomsection\label{parameters:Mesh refinement/Strategy}


\index[prmindex]{Strategy}
\index[prmindexfull]{Mesh refinement!Strategy}
{\it Value:} thermal energy density


{\it Default:} thermal energy density


{\it Description:} A comma separated list of mesh refinement criteria that will be run whenever mesh refinement is required. The results of each of these criteria, i.e., the refinement indicators they produce for all the cells of the mesh will then be normalized to a range between zero and one and the results of different criteria will then be merged through the operation selected in this section.

The following criteria are available:

`boundary': A class that implements a mesh refinement criterion which always flags all cells on specified boundaries for refinement. This is useful to provide high accuracy for processes at or close to the edge of the model domain.

To use this refinement criterion, you may want to combine it with other refinement criteria, setting the 'Normalize individual refinement criteria' flag and using the 'max' setting for 'Refinement criteria merge operation'.

`composition': A mesh refinement criterion that computes refinement indicators from the compositional fields. If there is more than one compositional field, then it simply takes the sum of the indicators computed from each of the compositional field.

`density': A mesh refinement criterion that computes refinement indicators from a field that describes the spatial variability of the density, $\rho$. Because this quantity may not be a continuous function ($\rho$ and $C_p$ may be discontinuous functions along discontinuities in the medium, for example due to phase changes), we approximate the gradient of this quantity to refine the mesh. The error indicator defined here takes the magnitude of the approximate gradient and scales it by $h_K^{1+d/2}$ where $h_K$ is the diameter of each cell and $d$ is the dimension. This scaling ensures that the error indicators converge to zero as $h_K\rightarrow 0$ even if the energy density is discontinuous, since the gradient of a discontinuous function grows like $1/h_K$.

`maximum refinement function': A mesh refinement criterion that ensures a maximum refinement level described by an explicit formula with the depth or position as argument. Which coordinate representation is used is determined by an input parameter. Whatever the coordinate system chosen, the function you provide in the input file will by default depend on variables 'x', 'y' and 'z' (if in 3d). However, the meaning of these symbols depends on the coordinate system. In the Cartesian coordinate system, they simply refer to their natural meaning. If you have selected 'depth' for the coordinate system, then 'x' refers to the depth variable and 'y' and 'z' will simply always be zero. If you have selected a spherical coordinate system, then 'x' will refer to the radial distance of the point to the origin, 'y' to the azimuth angle and 'z' to the polar angle measured positive from the north pole. Note that the order of spherical coordinates is r,phi,theta and not r,theta,phi, since this allows for dimension independent expressions. After evaluating the function, its values are rounded to the nearest integer.

The format of these functions follows the syntax understood by the muparser library, see Section~\ref{sec:muparser-format}.

`minimum refinement function': A mesh refinement criterion that ensures a minimum refinement level described by an explicit formula with the depth or position as argument. Which coordinate representation is used is determined by an input parameter. Whatever the coordinate system chosen, the function you provide in the input file will by default depend on variables 'x', 'y' and 'z' (if in 3d). However, the meaning of these symbols depends on the coordinate system. In the Cartesian coordinate system, they simply refer to their natural meaning. If you have selected 'depth' for the coordinate system, then 'x' refers to the depth variable and 'y' and 'z' will simply always be zero. If you have selected a spherical coordinate system, then 'x' will refer to the radial distance of the point to the origin, 'y' to the azimuth angle and 'z' to the polar angle measured positive from the north pole. Note that the order of spherical coordinates is r,phi,theta and not r,theta,phi, since this allows for dimension independent expressions. After evaluating the function, its values are rounded to the nearest integer.

The format of these functions follows the syntax understood by the muparser library, see Section~\ref{sec:muparser-format}.

`nonadiabatic temperature': A mesh refinement criterion that computes refinement indicators from the excess temperature(difference between temperature and adiabatic temperature.

`particle density': A mesh refinement criterion that computes refinement indicators that equal the areal (in 2d) or volumetric (in 3d) density of particles in this cell. This plugin is useful for models with inhomogeneous particle density, e.g. when tracking an initial interface with a high particle density, or when the spatial particle density denotes the region of interest. Additionally, this plugin tends to balance the computational load between processes in parallel computations, because the particle number per cell is more similar.

`slope': A class that implements a mesh refinement criterion intended for use with a free surface. It calculates a local slope based on the angle between the surface normal and the local gravity vector. Cells with larger angles are marked for refinement.

To use this refinement criterion, you may want to combine it with other refinement criteria, setting the 'Normalize individual refinement criteria' flag and using the 'max' setting for 'Refinement criteria merge operation'.

`strain rate': A mesh refinement criterion that computes therefinement indicators equal to the strain rate norm computed at the center of the elements.

`temperature': A mesh refinement criterion that computes refinement indicators from the temperature field.

`thermal energy density': A mesh refinement criterion that computes refinement indicators from a field that describes the spatial variability of the thermal energy density, $\rho C_p T$. Because this quantity may not be a continuous function ($\rho$ and $C_p$ may be discontinuous functions along discontinuities in the medium, for example due to phase changes), we approximate the gradient of this quantity to refine the mesh. The error indicator defined here takes the magnitude of the approximate gradient and scales it by $h_K^{1.5}$ where $h_K$ is the diameter of each cell. This scaling ensures that the error indicators converge to zero as $h_K\rightarrow 0$ even if the energy density is discontinuous, since the gradient of a discontinuous function grows like $1/h_K$.

`topography': A class that implements a mesh refinement criterion, which always flags all cells in the uppermost layer for refinement. This is useful to provide high accuracy for processes at or close to the surface.

To use this refinement criterion, you may want to combine it with other refinement criteria, setting the 'Normalize individual refinement criteria' flag and using the 'max' setting for 'Refinement criteria merge operation'.

`velocity': A mesh refinement criterion that computes refinement indicators from the velocity field.

`viscosity': A mesh refinement criterion that computes refinement indicators from a field that describes the spatial variability of the logarithm of the viscosity, $\log\eta$. (We choose the logarithm of the viscosity because it can vary by orders of magnitude.)Because this quantity may not be a continuous function ($\eta$ may be a discontinuous function along discontinuities in the medium, for example due to phase changes), we approximate the gradient of this quantity to refine the mesh. The error indicator defined here takes the magnitude of the approximate gradient and scales it by $h_K^{1+d/2}$ where $h_K$ is the diameter of each cell and $d$ is the dimension. This scaling ensures that the error indicators converge to zero as $h_K\rightarrow 0$ even if the energy density is discontinuous, since the gradient of a discontinuous function grows like $1/h_K$.


{\it Possible values:} [MultipleSelection boundary|composition|density|maximum refinement function|minimum refinement function|nonadiabatic temperature|particle density|slope|strain rate|temperature|thermal energy density|topography|velocity|viscosity ]
\item {\it Parameter name:} {\tt Time steps between mesh refinement}
\phantomsection\label{parameters:Mesh refinement/Time steps between mesh refinement}


\index[prmindex]{Time steps between mesh refinement}
\index[prmindexfull]{Mesh refinement!Time steps between mesh refinement}
{\it Value:} 10


{\it Default:} 10


{\it Description:} The number of time steps after which the mesh is to be adapted again based on computed error indicators. If 0 then the mesh will never be changed.


{\it Possible values:} [Integer range 0...2147483647 (inclusive)]
\end{itemize}



\subsection{Parameters in section \tt Mesh refinement/Boundary}
\label{parameters:Mesh_20refinement/Boundary}

\begin{itemize}
\item {\it Parameter name:} {\tt Boundary refinement indicators}
\phantomsection\label{parameters:Mesh refinement/Boundary/Boundary refinement indicators}


\index[prmindex]{Boundary refinement indicators}
\index[prmindexfull]{Mesh refinement!Boundary!Boundary refinement indicators}
{\it Value:} 


{\it Default:} 


{\it Description:} A comma separated list of names denoting those boundaries where there should be mesh refinement.

The names of the boundaries listed here can either be numbers (in which case they correspond to the numerical boundary indicators assigned by the geometry object), or they can correspond to any of the symbolic names the geometry object may have provided for each part of the boundary. You may want to compare this with the documentation of the geometry model you use in your model.


{\it Possible values:} [List list of [Anything] of length 0...4294967295 (inclusive)]
\end{itemize}

\subsection{Parameters in section \tt Mesh refinement/Composition}
\label{parameters:Mesh_20refinement/Composition}

\begin{itemize}
\item {\it Parameter name:} {\tt Compositional field scaling factors}
\phantomsection\label{parameters:Mesh refinement/Composition/Compositional field scaling factors}


\index[prmindex]{Compositional field scaling factors}
\index[prmindexfull]{Mesh refinement!Composition!Compositional field scaling factors}
{\it Value:} 


{\it Default:} 


{\it Description:} A list of scaling factors by which every individual compositional field will be multiplied by. If only a single compositional field exists, then this parameter has no particular meaning. On the other hand, if multiple criteria are chosen, then these factors are used to weigh the various indicators relative to each other. 

If the list of scaling factors given in this parameter is empty, then this indicates that they should all be chosen equal to one. If the list is not empty then it needs to have as many entries as there are compositional fields.


{\it Possible values:} [List list of [Double 0...1.79769e+308 (inclusive)] of length 0...4294967295 (inclusive)]
\end{itemize}

\subsection{Parameters in section \tt Mesh refinement/Maximum refinement function}
\label{parameters:Mesh_20refinement/Maximum_20refinement_20function}

\begin{itemize}
\item {\it Parameter name:} {\tt Coordinate system}
\phantomsection\label{parameters:Mesh refinement/Maximum refinement function/Coordinate system}


\index[prmindex]{Coordinate system}
\index[prmindexfull]{Mesh refinement!Maximum refinement function!Coordinate system}
{\it Value:} depth


{\it Default:} depth


{\it Description:} A selection that determines the assumed coordinate system for the function variables. Allowed values are 'depth', 'cartesian' and 'spherical'. 'depth' will create a function, in which only the first variable is non-zero, which is interpreted to be the depth of the point. 'spherical' coordinates are interpreted as r,phi or r,phi,theta in 2D/3D respectively with theta being the polar angle.


{\it Possible values:} [Selection depth|cartesian|spherical ]
\item {\it Parameter name:} {\tt Function constants}
\phantomsection\label{parameters:Mesh refinement/Maximum refinement function/Function constants}


\index[prmindex]{Function constants}
\index[prmindexfull]{Mesh refinement!Maximum refinement function!Function constants}
{\it Value:} 


{\it Default:} 


{\it Description:} Sometimes it is convenient to use symbolic constants in the expression that describes the function, rather than having to use its numeric value everywhere the constant appears. These values can be defined using this parameter, in the form `var1=value1, var2=value2, ...'.

A typical example would be to set this runtime parameter to `pi=3.1415926536' and then use `pi' in the expression of the actual formula. (That said, for convenience this class actually defines both `pi' and `Pi' by default, but you get the idea.)


{\it Possible values:} [Anything]
\item {\it Parameter name:} {\tt Function expression}
\phantomsection\label{parameters:Mesh refinement/Maximum refinement function/Function expression}


\index[prmindex]{Function expression}
\index[prmindexfull]{Mesh refinement!Maximum refinement function!Function expression}
{\it Value:} 0


{\it Default:} 0


{\it Description:} The formula that denotes the function you want to evaluate for particular values of the independent variables. This expression may contain any of the usual operations such as addition or multiplication, as well as all of the common functions such as `sin' or `cos'. In addition, it may contain expressions like `if(x>0, 1, -1)' where the expression evaluates to the second argument if the first argument is true, and to the third argument otherwise. For a full overview of possible expressions accepted see the documentation of the muparser library at http://muparser.beltoforion.de/.

If the function you are describing represents a vector-valued function with multiple components, then separate the expressions for individual components by a semicolon.


{\it Possible values:} [Anything]
\item {\it Parameter name:} {\tt Variable names}
\phantomsection\label{parameters:Mesh refinement/Maximum refinement function/Variable names}


\index[prmindex]{Variable names}
\index[prmindexfull]{Mesh refinement!Maximum refinement function!Variable names}
{\it Value:} x,y,t


{\it Default:} x,y,t


{\it Description:} The name of the variables as they will be used in the function, separated by commas. By default, the names of variables at which the function will be evaluated is `x' (in 1d), `x,y' (in 2d) or `x,y,z' (in 3d) for spatial coordinates and `t' for time. You can then use these variable names in your function expression and they will be replaced by the values of these variables at which the function is currently evaluated. However, you can also choose a different set of names for the independent variables at which to evaluate your function expression. For example, if you work in spherical coordinates, you may wish to set this input parameter to `r,phi,theta,t' and then use these variable names in your function expression.


{\it Possible values:} [Anything]
\end{itemize}

\subsection{Parameters in section \tt Mesh refinement/Minimum refinement function}
\label{parameters:Mesh_20refinement/Minimum_20refinement_20function}

\begin{itemize}
\item {\it Parameter name:} {\tt Coordinate system}
\phantomsection\label{parameters:Mesh refinement/Minimum refinement function/Coordinate system}


\index[prmindex]{Coordinate system}
\index[prmindexfull]{Mesh refinement!Minimum refinement function!Coordinate system}
{\it Value:} depth


{\it Default:} depth


{\it Description:} A selection that determines the assumed coordinate system for the function variables. Allowed values are 'depth', 'cartesian' and 'spherical'. 'depth' will create a function, in which only the first variable is non-zero, which is interpreted to be the depth of the point. 'spherical' coordinates are interpreted as r,phi or r,phi,theta in 2D/3D respectively with theta being the polar angle.


{\it Possible values:} [Selection depth|cartesian|spherical ]
\item {\it Parameter name:} {\tt Function constants}
\phantomsection\label{parameters:Mesh refinement/Minimum refinement function/Function constants}


\index[prmindex]{Function constants}
\index[prmindexfull]{Mesh refinement!Minimum refinement function!Function constants}
{\it Value:} 


{\it Default:} 


{\it Description:} Sometimes it is convenient to use symbolic constants in the expression that describes the function, rather than having to use its numeric value everywhere the constant appears. These values can be defined using this parameter, in the form `var1=value1, var2=value2, ...'.

A typical example would be to set this runtime parameter to `pi=3.1415926536' and then use `pi' in the expression of the actual formula. (That said, for convenience this class actually defines both `pi' and `Pi' by default, but you get the idea.)


{\it Possible values:} [Anything]
\item {\it Parameter name:} {\tt Function expression}
\phantomsection\label{parameters:Mesh refinement/Minimum refinement function/Function expression}


\index[prmindex]{Function expression}
\index[prmindexfull]{Mesh refinement!Minimum refinement function!Function expression}
{\it Value:} 0


{\it Default:} 0


{\it Description:} The formula that denotes the function you want to evaluate for particular values of the independent variables. This expression may contain any of the usual operations such as addition or multiplication, as well as all of the common functions such as `sin' or `cos'. In addition, it may contain expressions like `if(x>0, 1, -1)' where the expression evaluates to the second argument if the first argument is true, and to the third argument otherwise. For a full overview of possible expressions accepted see the documentation of the muparser library at http://muparser.beltoforion.de/.

If the function you are describing represents a vector-valued function with multiple components, then separate the expressions for individual components by a semicolon.


{\it Possible values:} [Anything]
\item {\it Parameter name:} {\tt Variable names}
\phantomsection\label{parameters:Mesh refinement/Minimum refinement function/Variable names}


\index[prmindex]{Variable names}
\index[prmindexfull]{Mesh refinement!Minimum refinement function!Variable names}
{\it Value:} x,y,t


{\it Default:} x,y,t


{\it Description:} The name of the variables as they will be used in the function, separated by commas. By default, the names of variables at which the function will be evaluated is `x' (in 1d), `x,y' (in 2d) or `x,y,z' (in 3d) for spatial coordinates and `t' for time. You can then use these variable names in your function expression and they will be replaced by the values of these variables at which the function is currently evaluated. However, you can also choose a different set of names for the independent variables at which to evaluate your function expression. For example, if you work in spherical coordinates, you may wish to set this input parameter to `r,phi,theta,t' and then use these variable names in your function expression.


{\it Possible values:} [Anything]
\end{itemize}

\subsection{Parameters in section \tt Model settings}
\label{parameters:Model_20settings}

\begin{itemize}
\item {\it Parameter name:} {\tt Fixed composition boundary indicators}
\phantomsection\label{parameters:Model settings/Fixed composition boundary indicators}


\index[prmindex]{Fixed composition boundary indicators}
\index[prmindexfull]{Model settings!Fixed composition boundary indicators}
{\it Value:} 


{\it Default:} 


{\it Description:} A comma separated list of names denoting those boundaries on which the composition is fixed and described by the boundary composition object selected in its own section of this input file. All boundary indicators used by the geometry but not explicitly listed here will end up with no-flux (insulating) boundary conditions.

The names of the boundaries listed here can either by numbers (in which case they correspond to the numerical boundary indicators assigned by the geometry object), or they can correspond to any of the symbolic names the geometry object may have provided for each part of the boundary. You may want to compare this with the documentation of the geometry model you use in your model.

This parameter only describes which boundaries have a fixed composition, but not what composition should hold on these boundaries. The latter piece of information needs to be implemented in a plugin in the BoundaryComposition group, unless an existing implementation in this group already provides what you want.


{\it Possible values:} [List list of [Anything] of length 0...4294967295 (inclusive)]
\item {\it Parameter name:} {\tt Fixed temperature boundary indicators}
\phantomsection\label{parameters:Model settings/Fixed temperature boundary indicators}


\index[prmindex]{Fixed temperature boundary indicators}
\index[prmindexfull]{Model settings!Fixed temperature boundary indicators}
{\it Value:} 


{\it Default:} 


{\it Description:} A comma separated list of names denoting those boundaries on which the temperature is fixed and described by the boundary temperature object selected in its own section of this input file. All boundary indicators used by the geometry but not explicitly listed here will end up with no-flux (insulating) boundary conditions.

The names of the boundaries listed here can either by numbers (in which case they correspond to the numerical boundary indicators assigned by the geometry object), or they can correspond to any of the symbolic names the geometry object may have provided for each part of the boundary. You may want to compare this with the documentation of the geometry model you use in your model.

This parameter only describes which boundaries have a fixed temperature, but not what temperature should hold on these boundaries. The latter piece of information needs to be implemented in a plugin in the BoundaryTemperature group, unless an existing implementation in this group already provides what you want.


{\it Possible values:} [List list of [Anything] of length 0...4294967295 (inclusive)]
\item {\it Parameter name:} {\tt Free surface boundary indicators}
\phantomsection\label{parameters:Model settings/Free surface boundary indicators}


\index[prmindex]{Free surface boundary indicators}
\index[prmindexfull]{Model settings!Free surface boundary indicators}
{\it Value:} 


{\it Default:} 


{\it Description:} A comma separated list of names denoting those boundaries where there is a free surface. Set to nothing to disable all free surface computations.

The names of the boundaries listed here can either by numbers (in which case they correspond to the numerical boundary indicators assigned by the geometry object), or they can correspond to any of the symbolic names the geometry object may have provided for each part of the boundary. You may want to compare this with the documentation of the geometry model you use in your model.


{\it Possible values:} [List list of [Anything] of length 0...4294967295 (inclusive)]
\item {\it Parameter name:} {\tt Include adiabatic heating}
\phantomsection\label{parameters:Model settings/Include adiabatic heating}


\index[prmindex]{Include adiabatic heating}
\index[prmindexfull]{Model settings!Include adiabatic heating}
{\it Value:} false


{\it Default:} false


{\it Description:} Whether to include adiabatic heating into the model or not. From a physical viewpoint, adiabatic heating should always be used but may be undesirable when comparing results with known benchmarks that do not include this term in the temperature equation.Warning: deprecated! Add 'adiabatic heating' to the 'List of model names' instead.


{\it Possible values:} [Bool]
\item {\it Parameter name:} {\tt Include latent heat}
\phantomsection\label{parameters:Model settings/Include latent heat}


\index[prmindex]{Include latent heat}
\index[prmindexfull]{Model settings!Include latent heat}
{\it Value:} false


{\it Default:} false


{\it Description:} Whether to include the generation of latent heat at phase transitions into the model or not. From a physical viewpoint, latent heat should always be used but may be undesirable when comparing results with known benchmarks that do not include this term in the temperature equation or when dealing with a model without phase transitions.Warning: deprecated! Add 'latent heat' to the 'List of model names' instead.


{\it Possible values:} [Bool]
\item {\it Parameter name:} {\tt Include shear heating}
\phantomsection\label{parameters:Model settings/Include shear heating}


\index[prmindex]{Include shear heating}
\index[prmindexfull]{Model settings!Include shear heating}
{\it Value:} false


{\it Default:} false


{\it Description:} Whether to include shear heating into the model or not. From a physical viewpoint, shear heating should always be used but may be undesirable when comparing results with known benchmarks that do not include this term in the temperature equation.Warning: deprecated! Add 'shear heating' to the 'List of model names' instead.


{\it Possible values:} [Bool]
\item {\it Parameter name:} {\tt Prescribed traction boundary indicators}
\phantomsection\label{parameters:Model settings/Prescribed traction boundary indicators}


\index[prmindex]{Prescribed traction boundary indicators}
\index[prmindexfull]{Model settings!Prescribed traction boundary indicators}
{\it Value:} 


{\it Default:} 


{\it Description:} A comma separated list denoting those boundaries on which a traction force is prescribed, i.e., where known external forces act, resulting in an unknown velocity. This is often used to model ``open'' boundaries where we only know the pressure. This pressure then produces a force that is normal to the boundary and proportional to the pressure.

The format of valid entries for this parameter is that of a map given as ``key1 [selector]: value1, key2 [selector]: value2, key3: value3, ...'' where each key must be a valid boundary indicator (which is either an integer or the symbolic name the geometry model in use may have provided for this part of the boundary) and each value must be one of the currently implemented boundary traction models. ``selector'' is an optional string given as a subset of the letters 'xyz' that allows you to apply the boundary conditions only to the components listed. As an example, '1 y: function' applies the type 'function' to the y component on boundary 1. Without a selector it will affect all components of the traction.


{\it Possible values:} [Map map of [Anything]:[Selection function|zero traction ] of length 0...4294967295 (inclusive)]
\item {\it Parameter name:} {\tt Prescribed velocity boundary indicators}
\phantomsection\label{parameters:Model settings/Prescribed velocity boundary indicators}


\index[prmindex]{Prescribed velocity boundary indicators}
\index[prmindexfull]{Model settings!Prescribed velocity boundary indicators}
{\it Value:} 


{\it Default:} 


{\it Description:} A comma separated list denoting those boundaries on which the velocity is prescribed, i.e., where unknown external forces act to prescribe a particular velocity. This is often used to prescribe a velocity that equals that of overlying plates.

The format of valid entries for this parameter is that of a map given as ``key1 [selector]: value1, key2 [selector]: value2, key3: value3, ...'' where each key must be a valid boundary indicator (which is either an integer or the symbolic name the geometry model in use may have provided for this part of the boundary) and each value must be one of the currently implemented boundary velocity models. ``selector'' is an optional string given as a subset of the letters 'xyz' that allows you to apply the boundary conditions only to the components listed. As an example, '1 y: function' applies the type 'function' to the y component on boundary 1. Without a selector it will affect all components of the velocity.

Note that the no-slip boundary condition is a special case of the current one where the prescribed velocity happens to be zero. It can thus be implemented by indicating that a particular boundary is part of the ones selected using the current parameter and using ``zero velocity'' as the boundary values. Alternatively, you can simply list the part of the boundary on which the velocity is to be zero with the parameter ``Zero velocity boundary indicator'' in the current parameter section.

Note that when ``Use years in output instead of seconds'' is set to true, velocity should be given in m/yr. 


{\it Possible values:} [Map map of [Anything]:[Selection ascii data|function|gplates|zero velocity ] of length 0...4294967295 (inclusive)]
\item {\it Parameter name:} {\tt Remove nullspace}
\phantomsection\label{parameters:Model settings/Remove nullspace}


\index[prmindex]{Remove nullspace}
\index[prmindexfull]{Model settings!Remove nullspace}
{\it Value:} 


{\it Default:} 


{\it Description:} Choose none, one or several from 

\begin{itemize} \item net rotation \item angular momentum \item net translation \item linear momentum \item net x translation \item net y translation \item net z translation \item linear x momentum \item linear y momentum \item linear z momentum \end{itemize}

These are a selection of operations to remove certain parts of the nullspace from the velocity after solving. For some geometries and certain boundary conditions the velocity field is not uniquely determined but contains free translations and/or rotations. Depending on what you specify here, these non-determined modes will be removed from the velocity field at the end of the Stokes solve step.


The ``angular momentum'' option removes a rotation such that the net angular momentum is zero. The ``linear * momentum'' options remove translations such that the net momentum in the relevant direction is zero.  The ``net rotation'' option removes the net rotation of the domain, and the ``net * translation'' options remove the net translations in the relevant directions.  For most problems there should not be a significant difference between the momentum and rotation/translation versions of nullspace removal, although the momentum versions are more physically motivated. They are equivalent for constant density simulations, and approximately equivalent when the density variations are small.

Note that while more than one operation can be selected it only makes sense to pick one rotational and one translational operation.


{\it Possible values:} [MultipleSelection net rotation|angular momentum|net translation|linear momentum|net x translation|net y translation|net z translation|linear x momentum|linear y momentum|linear z momentum ]
\item {\it Parameter name:} {\tt Tangential velocity boundary indicators}
\phantomsection\label{parameters:Model settings/Tangential velocity boundary indicators}


\index[prmindex]{Tangential velocity boundary indicators}
\index[prmindexfull]{Model settings!Tangential velocity boundary indicators}
{\it Value:} 


{\it Default:} 


{\it Description:} A comma separated list of names denoting those boundaries on which the velocity is tangential and unrestrained, i.e., free-slip where no external forces act to prescribe a particular tangential velocity (although there is a force that requires the flow to be tangential).

The names of the boundaries listed here can either by numbers (in which case they correspond to the numerical boundary indicators assigned by the geometry object), or they can correspond to any of the symbolic names the geometry object may have provided for each part of the boundary. You may want to compare this with the documentation of the geometry model you use in your model.


{\it Possible values:} [List list of [Anything] of length 0...4294967295 (inclusive)]
\item {\it Parameter name:} {\tt Zero velocity boundary indicators}
\phantomsection\label{parameters:Model settings/Zero velocity boundary indicators}


\index[prmindex]{Zero velocity boundary indicators}
\index[prmindexfull]{Model settings!Zero velocity boundary indicators}
{\it Value:} 


{\it Default:} 


{\it Description:} A comma separated list of names denoting those boundaries on which the velocity is zero.

The names of the boundaries listed here can either by numbers (in which case they correspond to the numerical boundary indicators assigned by the geometry object), or they can correspond to any of the symbolic names the geometry object may have provided for each part of the boundary. You may want to compare this with the documentation of the geometry model you use in your model.


{\it Possible values:} [List list of [Anything] of length 0...4294967295 (inclusive)]
\end{itemize}

\subsection{Parameters in section \tt Postprocess}
\label{parameters:Postprocess}

\begin{itemize}
\item {\it Parameter name:} {\tt List of postprocessors}
\phantomsection\label{parameters:Postprocess/List of postprocessors}


\index[prmindex]{List of postprocessors}
\index[prmindexfull]{Postprocess!List of postprocessors}
{\it Value:} 


{\it Default:} 


{\it Description:} A comma separated list of postprocessor objects that should be run at the end of each time step. Some of these postprocessors will declare their own parameters which may, for example, include that they will actually do something only every so many time steps or years. Alternatively, the text 'all' indicates that all available postprocessors should be run after each time step.

The following postprocessors are available:

`Stokes residual': A postprocessor that outputs the Stokes residuals during the iterative solver algorithm into a file stokes\_residuals.txt in the output directory.

`basic statistics': A postprocessor that computes some simplified statistics like the Rayleigh number and other quantities that only make sense in certain model setups.

`boundary densities': A postprocessor that computes the laterally averaged density at the top and bottom of the domain.

`boundary pressures': A postprocessor that computes the laterally averaged pressure at the top and bottom of the domain.

`composition statistics': A postprocessor that computes some statistics about the compositional fields, if present in this simulation. In particular, it computes maximal and minimal values of each field, as well as the total mass contained in this field as defined by the integral $m_i(t) = \int_\Omega c_i(\mathbf x,t) \; dx$.

`depth average': A postprocessor that computes depth averaged quantities and writes them into a file $<$depth\_average.ext$>$ in the output directory, where the extension of the file is determined by the output format you select. In addition to the output format, a number of other parameters also influence this postprocessor, and they can be set in the section \texttt{Postprocess/Depth average} in the input file.

In the output files, the $x$-value of each data point corresponds to the depth, whereas the $y$-value corresponds to the simulation time. The time is provided in seconds or, if the global ``Use years in output instead of seconds'' parameter is set, in years.

`dynamic topography': A postprocessor that computes a measure of dynamic topography based on the stress at the surface. The data is written into text files named 'dynamic\_topography.NNNNN' in the output directory, where NNNNN is the number of the time step.

The exact approach works as follows: At the centers of all cells that sit along the top surface, we evaluate the stress and evaluate the component of it in the direction in which gravity acts. In other words, we compute $\sigma_{rr}={\hat g}^T(2 \eta \varepsilon(\mathbf u)- \frac 13 (\textrm{div}\;\mathbf u)I)\hat g - p_d$ where $\hat g = \mathbf g/\|\mathbf g\|$ is the direction of the gravity vector $\mathbf g$ and $p_d=p-p_a$ is the dynamic pressure computed by subtracting the adiabatic pressure $p_a$ from the total pressure $p$ computed as part of the Stokes solve. From this, the dynamic topography is computed using the formula $h=\frac{\sigma_{rr}}{\|\mathbf g\| \rho}$ where $\rho$ is the density at the cell center.
The file format then consists of lines with Euclidiean coordinates followed by the corresponding topography value.

(As a side note, the postprocessor chooses the cell center instead of the center of the cell face at the surface, where we really are interested in the quantity, since this often gives better accuracy. The results should in essence be the same, though.)

`heat flux statistics': A postprocessor that computes some statistics about the (conductive) heat flux across boundaries. For each boundary indicator (see your geometry description for which boundary indicators are used), the heat flux is computed in outward direction, i.e., from the domain to the outside, using the formula $\int_{\Gamma_i} k \nabla T \cdot \mathbf n$ where $\Gamma_i$ is the part of the boundary with indicator $i$, $k$ is the thermal conductivity as reported by the material model, $T$ is the temperature, and $\mathbf n$ is the outward normal. Note that the quantity so computed does not include any energy transported across the boundary by material transport in cases where $\mathbf u \cdot \mathbf n \neq 0$.

As stated, this postprocessor computes the \textit{outbound} heat flux. If you are interested in the opposite direction, for example from the core into the mantle when the domain describes the mantle, then you need to multiply the result by -1.

`heating statistics': A postprocessor that computes some statistics about heating, averaged by volume. 

`mass flux statistics': A postprocessor that computes some statistics about the mass flux across boundaries. For each boundary indicator (see your geometry description for which boundary indicators are used), the mass flux is computed in outward direction, i.e., from the domain to the outside, using the formula $\int_{\Gamma_i} \rho \mathbf v \cdot \mathbf n$ where $\Gamma_i$ is the part of the boundary with indicator $i$, $\rho$ is the density as reported by the material model, $\mathbf v$ is the velocity, and $\mathbf n$ is the outward normal. 

As stated, this postprocessor computes the \textit{outbound} mass flux. If you are interested in the opposite direction, for example from the core into the mantle when the domain describes the mantle, then you need to multiply the result by -1.

`pressure statistics': A postprocessor that computes some statistics about the pressure field.

`spherical velocity statistics': A postprocessor that computes radial, tangential and total RMS velocity.

`temperature statistics': A postprocessor that computes some statistics about the temperature field.

`topography': A postprocessor intended for use with a free surface.  After every step, it loops over all the vertices on the top surface and determines the maximum and minimum topography relative to a reference datum (initial box height for a box geometry model or initial radius for a sphere/spherical shell geometry model).  Outputs topography in meters

`tracers': A Postprocessor that creates tracer particles that follow the velocity field of the simulation. The particles can be generated and propagated in various ways and they can carry a number of constant or time-varying properties. The postprocessor can write output positions and properties of all tracers at chosen intervals, although this is not mandatory. It also allows other parts of the code to query the tracers for information.

`velocity boundary statistics': A postprocessor that computes some statistics about the velocity along the boundaries. For each boundary indicator (see your geometry description for which boundary indicators are used), the min and max velocity magnitude is computed.

`velocity statistics': A postprocessor that computes some statistics about the velocity field.

`viscous dissipation statistics': A postprocessor that computes the viscous dissipationfor the whole domain as: $\frac{1}{2} \int_{V} \sigma : \dot{\epsilon}dV$ = $\int_{V} (-p\nabla \cdot u+2\mu\dot{\epsilon}:\dot{\epsilon} - \frac{2\mu}{3}(\nabla\cdot u)^{2}) dV$.

The information produced by this postprocessor is a subset of what is generated by the 'heating statistics' postprocessor.

`visualization': A postprocessor that takes the solution and writes it into files that can be read by a graphical visualization program. Additional run time parameters are read from the parameter subsection 'Visualization'.


{\it Possible values:} [MultipleSelection Stokes residual|basic statistics|boundary densities|boundary pressures|composition statistics|depth average|dynamic topography|heat flux statistics|heating statistics|mass flux statistics|pressure statistics|spherical velocity statistics|temperature statistics|topography|tracers|velocity boundary statistics|velocity statistics|viscous dissipation statistics|visualization ]
\end{itemize}



\subsection{Parameters in section \tt Postprocess/Depth average}
\label{parameters:Postprocess/Depth_20average}

\begin{itemize}
\item {\it Parameter name:} {\tt List of output variables}
\phantomsection\label{parameters:Postprocess/Depth average/List of output variables}


\index[prmindex]{List of output variables}
\index[prmindexfull]{Postprocess!Depth average!List of output variables}
{\it Value:} all


{\it Default:} all


{\it Description:} A comma separated list which specifies which quantites to average in each depth slice. It defaults to averaging all availabe quantities, but this can be an expensive operation, so you may want to select only a few.


{\it Possible values:} [MultipleSelection all|temperature|composition|adiabatic temperature|velocity magnitude|sinking velocity|Vs|Vp|viscosity|vertical heat flux ]
\item {\it Parameter name:} {\tt Number of zones}
\phantomsection\label{parameters:Postprocess/Depth average/Number of zones}


\index[prmindex]{Number of zones}
\index[prmindexfull]{Postprocess!Depth average!Number of zones}
{\it Value:} 10


{\it Default:} 10


{\it Description:} The number of zones in depth direction within which we are to compute averages. By default, we subdivide the entire domain into 10 depth zones and compute temperature and other averages within each of these zones. However, if you have a very coarse mesh, it may not make much sense to subdivide the domain into so many zones and you may wish to choose less than this default. It may also make computations slightly faster. On the other hand, if you have an extremely highly resolved mesh, choosing more zones might also make sense.


{\it Possible values:} [Integer range 1...2147483647 (inclusive)]
\item {\it Parameter name:} {\tt Output format}
\phantomsection\label{parameters:Postprocess/Depth average/Output format}


\index[prmindex]{Output format}
\index[prmindexfull]{Postprocess!Depth average!Output format}
{\it Value:} gnuplot


{\it Default:} gnuplot


{\it Description:} The format in which the output shall be produced. The format in which the output is generated also determines the extension of the file into which data is written.


{\it Possible values:} [Selection none|dx|ucd|gnuplot|povray|eps|gmv|tecplot|tecplot\_binary|vtk|vtu|hdf5|svg|deal.II intermediate|txt ]
\item {\it Parameter name:} {\tt Time between graphical output}
\phantomsection\label{parameters:Postprocess/Depth average/Time between graphical output}


\index[prmindex]{Time between graphical output}
\index[prmindexfull]{Postprocess!Depth average!Time between graphical output}
{\it Value:} 1e8


{\it Default:} 1e8


{\it Description:} The time interval between each generation of graphical output files. A value of zero indicates that output should be generated in each time step. Units: years if the 'Use years in output instead of seconds' parameter is set; seconds otherwise.


{\it Possible values:} [Double 0...1.79769e+308 (inclusive)]
\end{itemize}

\subsection{Parameters in section \tt Postprocess/Dynamic Topography}
\label{parameters:Postprocess/Dynamic_20Topography}

\begin{itemize}
\item {\it Parameter name:} {\tt Subtract mean of dynamic topography}
\phantomsection\label{parameters:Postprocess/Dynamic Topography/Subtract mean of dynamic topography}


\index[prmindex]{Subtract mean of dynamic topography}
\index[prmindexfull]{Postprocess!Dynamic Topography!Subtract mean of dynamic topography}
{\it Value:} false


{\it Default:} false


{\it Description:} Option to remove the mean dynamic topography in the outputted data file (not visualization). 'true' subtracts the mean, 'false' leaves the calculated dynamic topography as is. 


{\it Possible values:} [Bool]
\end{itemize}

\subsection{Parameters in section \tt Postprocess/Tracers}
\label{parameters:Postprocess/Tracers}

\begin{itemize}
\item {\it Parameter name:} {\tt Data output format}
\phantomsection\label{parameters:Postprocess/Tracers/Data output format}


\index[prmindex]{Data output format}
\index[prmindexfull]{Postprocess!Tracers!Data output format}
{\it Value:} vtu


{\it Default:} vtu


{\it Description:} File format to output raw particle data in. If you select 'none' no output will be written.Select one of the following models:

`ascii': This particle output plugin writes particle positions and properties into space separated ascii files.

`hdf5': This particle output plugin writes particle positions and properties into hdf5 files.

`vtu': This particle output plugin writes particle positions and properties into vtu files.


{\it Possible values:} [Selection ascii|hdf5|vtu|none ]
\item {\it Parameter name:} {\tt Integration scheme}
\phantomsection\label{parameters:Postprocess/Tracers/Integration scheme}


\index[prmindex]{Integration scheme}
\index[prmindexfull]{Postprocess!Tracers!Integration scheme}
{\it Value:} rk2


{\it Default:} rk2


{\it Description:} Select one of the following models:

`euler': Explicit Euler scheme integrator, where $y_{n+1} = y_n + dt * v(y_n)$. This requires only one integration substep per timestep.

`rk2': Runge Kutta second order integrator, where $y_{n+1} = y_n + dt*v(0.5*k_1)$, $k_1 = dt*v(y_n)$.

`rk4': Runge Kutta fourth order integrator, where $y_{n+1} = y_n + (1/6)*k_1 + (1/3)*k_2 + (1/3)*k_3 + (1/6)*k_4$ and $k_1$, $k_2$, $k_3$, $k_4$ are defined as usual.


{\it Possible values:} [Selection euler|rk2|rk4 ]
\item {\it Parameter name:} {\tt Interpolation scheme}
\phantomsection\label{parameters:Postprocess/Tracers/Interpolation scheme}


\index[prmindex]{Interpolation scheme}
\index[prmindexfull]{Postprocess!Tracers!Interpolation scheme}
{\it Value:} first particle


{\it Default:} first particle


{\it Description:} Select one of the following models:

`first particle': Return the properties of the first tracer in the given cell.


{\it Possible values:} [Selection first particle ]
\item {\it Parameter name:} {\tt List of tracer properties}
\phantomsection\label{parameters:Postprocess/Tracers/List of tracer properties}


\index[prmindex]{List of tracer properties}
\index[prmindexfull]{Postprocess!Tracers!List of tracer properties}
{\it Value:} 


{\it Default:} 


{\it Description:} A comma separated list of tracer properties that should be tracked. By default none is selected, which means only position, velocity and id of the tracers are output. 

The following properties are available:

`function': Implementation of a model in which the tracer property is set by evaluating an explicit function at the initial position of each particle. The function is defined in the parameters in section ``Tracers|Function''. The format of these functions follows the syntax understood by the muparser library, see Section~\ref{sec:muparser-format}.



`initial composition': Implementation of a plugin in which the tracer property is given as the initial composition at the particle's initial position. The tracer gets as many properties as there are compositional fields.



`initial position': Implementation of a plugin in which the tracer property is given as the initial position of the tracer.



`pT path': Implementation of a plugin in which the tracer property is defined as the current pressure and temperature at this position. This can be used to generate pressure-temperature paths of material points over time.



`position': Implementation of a plugin in which the tracer property is defined as the current position. 



`velocity': Implementation of a plugin in which the tracer property is defined as the recent velocity at this position.




{\it Possible values:} [MultipleSelection function|initial composition|initial position|pT path|position|velocity ]
\item {\it Parameter name:} {\tt Load balancing strategy}
\phantomsection\label{parameters:Postprocess/Tracers/Load balancing strategy}


\index[prmindex]{Load balancing strategy}
\index[prmindexfull]{Postprocess!Tracers!Load balancing strategy}
{\it Value:} none


{\it Default:} none


{\it Description:} Strategy that is used to balance the computationalload across processors for adaptive meshes.


{\it Possible values:} [Selection none|remove particles|remove and add particles|repartition ]
\item {\it Parameter name:} {\tt Maximum tracers per cell}
\phantomsection\label{parameters:Postprocess/Tracers/Maximum tracers per cell}


\index[prmindex]{Maximum tracers per cell}
\index[prmindexfull]{Postprocess!Tracers!Maximum tracers per cell}
{\it Value:} 100


{\it Default:} 100


{\it Description:} Limit for how many particles are allowed per cell. This limit is useful to prevent coarse cells in adaptive meshes from slowing down the whole model. It will only be checked and enforced during mesh refinement and MPI transfer of tracers.


{\it Possible values:} [Integer range 0...2147483647 (inclusive)]
\item {\it Parameter name:} {\tt Minimum tracers per cell}
\phantomsection\label{parameters:Postprocess/Tracers/Minimum tracers per cell}


\index[prmindex]{Minimum tracers per cell}
\index[prmindexfull]{Postprocess!Tracers!Minimum tracers per cell}
{\it Value:} 100


{\it Default:} 100


{\it Description:} Limit for how many particles are allowed per cell. This limit is useful to prevent coarse cells in adaptive meshes from slowing down the whole model. It will only be checked and enforced during mesh refinement and MPI transfer of tracers.


{\it Possible values:} [Integer range 0...2147483647 (inclusive)]
\item {\it Parameter name:} {\tt Number of tracers}
\phantomsection\label{parameters:Postprocess/Tracers/Number of tracers}


\index[prmindex]{Number of tracers}
\index[prmindexfull]{Postprocess!Tracers!Number of tracers}
{\it Value:} 1000


{\it Default:} 1000


{\it Description:} Total number of tracers to create (not per processor or per element). The number is parsed as a floating point number (so that one can specify, for example, '1e4' particles) but it is interpreted as an integer, of course.


{\it Possible values:} [Double 0...1.79769e+308 (inclusive)]
\item {\it Parameter name:} {\tt Particle generator name}
\phantomsection\label{parameters:Postprocess/Tracers/Particle generator name}


\index[prmindex]{Particle generator name}
\index[prmindexfull]{Postprocess!Tracers!Particle generator name}
{\it Value:} random uniform


{\it Default:} random uniform


{\it Description:} Select one of the following models:

`ascii file': Generates a distribution of tracers from coordinates specified in an Ascii data file. The file format is a simple text file, with as many columns as spatial dimensions and as many lines as tracers to be generated. Initial comment lines starting with `\#' will be discarded.All of the values that define this generator are read from a section ``Particle generator/Ascii file'' in the input file, see Section~\ref{parameters:Particle_20generator/Ascii_20file}.

`probability density function': Generate a random distribution of particles over the entire simulation domain. The probability density is prescribed in the form of a user-prescribed function. The format of this function follows the syntax understood by the muparser library, see Section~\ref{sec:muparser-format}. The return value of the function is always checked to be a non-negative probability density but it can be zero inparts of the domain.

`random uniform': Generates a random uniform distribution of particles over the entire simulation domain.

`uniform box': Generate a uniform distribution of particles over a rectangular domain in 2D or 3D. Uniform here means the particles will be generated with an equal spacing in each spatial dimension. Note that in order to produce a regular distribution the number of generated tracers might not exactly match the one specified in the input file.

`uniform radial': Generate a uniform distribution of particlesover a spherical domain in 2D or 3D. Uniform here means the particles will be generated with an equal spacing in each spherical spatial dimension, i.e., the particles are created at positions that increase linearly with equal spacing in radius, colatitude and longitude around a certain center point. Note that in order to produce a regular distribution the number of generated tracers might not exactly match the one specified in the input file.


{\it Possible values:} [Selection ascii file|probability density function|random uniform|uniform box|uniform radial ]
\item {\it Parameter name:} {\tt Time between data output}
\phantomsection\label{parameters:Postprocess/Tracers/Time between data output}


\index[prmindex]{Time between data output}
\index[prmindexfull]{Postprocess!Tracers!Time between data output}
{\it Value:} 1e8


{\it Default:} 1e8


{\it Description:} The time interval between each generation of output files. A value of zero indicates that output should be generated every time step.

Units: years if the 'Use years in output instead of seconds' parameter is set; seconds otherwise.


{\it Possible values:} [Double 0...1.79769e+308 (inclusive)]
\item {\it Parameter name:} {\tt Tracer weight}
\phantomsection\label{parameters:Postprocess/Tracers/Tracer weight}


\index[prmindex]{Tracer weight}
\index[prmindexfull]{Postprocess!Tracers!Tracer weight}
{\it Value:} 10


{\it Default:} 10


{\it Description:} Weight that is associated with the computational load of a single particle. The sum of tracer weights will be added to the sum of cell weights to determine the partitioning of the mesh. Every cell without tracers is associated with a weight of 1000.


{\it Possible values:} [Integer range 0...2147483647 (inclusive)]
\end{itemize}



\subsection{Parameters in section \tt Postprocess/Tracers/Function}
\label{parameters:Postprocess/Tracers/Function}

\begin{itemize}
\item {\it Parameter name:} {\tt Function constants}
\phantomsection\label{parameters:Postprocess/Tracers/Function/Function constants}


\index[prmindex]{Function constants}
\index[prmindexfull]{Postprocess!Tracers!Function/Function constants}
{\it Value:} 


{\it Default:} 


{\it Description:} Sometimes it is convenient to use symbolic constants in the expression that describes the function, rather than having to use its numeric value everywhere the constant appears. These values can be defined using this parameter, in the form `var1=value1, var2=value2, ...'.

A typical example would be to set this runtime parameter to `pi=3.1415926536' and then use `pi' in the expression of the actual formula. (That said, for convenience this class actually defines both `pi' and `Pi' by default, but you get the idea.)


{\it Possible values:} [Anything]
\item {\it Parameter name:} {\tt Function expression}
\phantomsection\label{parameters:Postprocess/Tracers/Function/Function expression}


\index[prmindex]{Function expression}
\index[prmindexfull]{Postprocess!Tracers!Function/Function expression}
{\it Value:} 0


{\it Default:} 0


{\it Description:} The formula that denotes the function you want to evaluate for particular values of the independent variables. This expression may contain any of the usual operations such as addition or multiplication, as well as all of the common functions such as `sin' or `cos'. In addition, it may contain expressions like `if(x>0, 1, -1)' where the expression evaluates to the second argument if the first argument is true, and to the third argument otherwise. For a full overview of possible expressions accepted see the documentation of the muparser library at http://muparser.beltoforion.de/.

If the function you are describing represents a vector-valued function with multiple components, then separate the expressions for individual components by a semicolon.


{\it Possible values:} [Anything]
\item {\it Parameter name:} {\tt Variable names}
\phantomsection\label{parameters:Postprocess/Tracers/Function/Variable names}


\index[prmindex]{Variable names}
\index[prmindexfull]{Postprocess!Tracers!Function/Variable names}
{\it Value:} x,y,t


{\it Default:} x,y,t


{\it Description:} The name of the variables as they will be used in the function, separated by commas. By default, the names of variables at which the function will be evaluated is `x' (in 1d), `x,y' (in 2d) or `x,y,z' (in 3d) for spatial coordinates and `t' for time. You can then use these variable names in your function expression and they will be replaced by the values of these variables at which the function is currently evaluated. However, you can also choose a different set of names for the independent variables at which to evaluate your function expression. For example, if you work in spherical coordinates, you may wish to set this input parameter to `r,phi,theta,t' and then use these variable names in your function expression.


{\it Possible values:} [Anything]
\end{itemize}

\subsection{Parameters in section \tt Postprocess/Tracers/Generator}
\label{parameters:Postprocess/Tracers/Generator}


\subsection{Parameters in section \tt Postprocess/Tracers/Generator/Ascii file}
\label{parameters:Postprocess/Tracers/Generator/Ascii_20file}

\begin{itemize}
\item {\it Parameter name:} {\tt Data directory}
\phantomsection\label{parameters:Postprocess/Tracers/Generator/Ascii file/Data directory}


\index[prmindex]{Data directory}
\index[prmindexfull]{Postprocess!Tracers!Generator/Ascii file/Data directory}
{\it Value:} \$ASPECT\_SOURCE\_DIR/data/particle/generator/ascii/


{\it Default:} \$ASPECT\_SOURCE\_DIR/data/particle/generator/ascii/


{\it Description:} The name of a directory that contains the tracer data. This path may either be absolute (if starting with a '/') or relative to the current directory. The path may also include the special text '\$ASPECT\_SOURCE\_DIR' which will be interpreted as the path in which the ASPECT source files were located when ASPECT was compiled. This interpretation allows, for example, to reference files located in the 'data/' subdirectory of ASPECT. 


{\it Possible values:} [DirectoryName]
\item {\it Parameter name:} {\tt Data file name}
\phantomsection\label{parameters:Postprocess/Tracers/Generator/Ascii file/Data file name}


\index[prmindex]{Data file name}
\index[prmindexfull]{Postprocess!Tracers!Generator/Ascii file/Data file name}
{\it Value:} tracer.dat


{\it Default:} tracer.dat


{\it Description:} The name of the tracer file.


{\it Possible values:} [Anything]
\end{itemize}

\subsection{Parameters in section \tt Postprocess/Tracers/Generator/Probability density function}
\label{parameters:Postprocess/Tracers/Generator/Probability_20density_20function}

\begin{itemize}
\item {\it Parameter name:} {\tt Function constants}
\phantomsection\label{parameters:Postprocess/Tracers/Generator/Probability density function/Function constants}


\index[prmindex]{Function constants}
\index[prmindexfull]{Postprocess!Tracers!Generator/Probability density function/Function constants}
{\it Value:} 


{\it Default:} 


{\it Description:} Sometimes it is convenient to use symbolic constants in the expression that describes the function, rather than having to use its numeric value everywhere the constant appears. These values can be defined using this parameter, in the form `var1=value1, var2=value2, ...'.

A typical example would be to set this runtime parameter to `pi=3.1415926536' and then use `pi' in the expression of the actual formula. (That said, for convenience this class actually defines both `pi' and `Pi' by default, but you get the idea.)


{\it Possible values:} [Anything]
\item {\it Parameter name:} {\tt Function expression}
\phantomsection\label{parameters:Postprocess/Tracers/Generator/Probability density function/Function expression}


\index[prmindex]{Function expression}
\index[prmindexfull]{Postprocess!Tracers!Generator/Probability density function/Function expression}
{\it Value:} 0


{\it Default:} 0


{\it Description:} The formula that denotes the function you want to evaluate for particular values of the independent variables. This expression may contain any of the usual operations such as addition or multiplication, as well as all of the common functions such as `sin' or `cos'. In addition, it may contain expressions like `if(x>0, 1, -1)' where the expression evaluates to the second argument if the first argument is true, and to the third argument otherwise. For a full overview of possible expressions accepted see the documentation of the muparser library at http://muparser.beltoforion.de/.

If the function you are describing represents a vector-valued function with multiple components, then separate the expressions for individual components by a semicolon.


{\it Possible values:} [Anything]
\item {\it Parameter name:} {\tt Variable names}
\phantomsection\label{parameters:Postprocess/Tracers/Generator/Probability density function/Variable names}


\index[prmindex]{Variable names}
\index[prmindexfull]{Postprocess!Tracers!Generator/Probability density function/Variable names}
{\it Value:} x,y,t


{\it Default:} x,y,t


{\it Description:} The name of the variables as they will be used in the function, separated by commas. By default, the names of variables at which the function will be evaluated is `x' (in 1d), `x,y' (in 2d) or `x,y,z' (in 3d) for spatial coordinates and `t' for time. You can then use these variable names in your function expression and they will be replaced by the values of these variables at which the function is currently evaluated. However, you can also choose a different set of names for the independent variables at which to evaluate your function expression. For example, if you work in spherical coordinates, you may wish to set this input parameter to `r,phi,theta,t' and then use these variable names in your function expression.


{\it Possible values:} [Anything]
\end{itemize}

\subsection{Parameters in section \tt Postprocess/Tracers/Generator/Uniform box}
\label{parameters:Postprocess/Tracers/Generator/Uniform_20box}

\begin{itemize}
\item {\it Parameter name:} {\tt Maximal x}
\phantomsection\label{parameters:Postprocess/Tracers/Generator/Uniform box/Maximal x}


\index[prmindex]{Maximal x}
\index[prmindexfull]{Postprocess!Tracers!Generator/Uniform box/Maximal x}
{\it Value:} 1


{\it Default:} 1


{\it Description:} Maximal x coordinate for the region of tracers.


{\it Possible values:} [Double -1.79769e+308...1.79769e+308 (inclusive)]
\item {\it Parameter name:} {\tt Maximal y}
\phantomsection\label{parameters:Postprocess/Tracers/Generator/Uniform box/Maximal y}


\index[prmindex]{Maximal y}
\index[prmindexfull]{Postprocess!Tracers!Generator/Uniform box/Maximal y}
{\it Value:} 1


{\it Default:} 1


{\it Description:} Maximal y coordinate for the region of tracers.


{\it Possible values:} [Double -1.79769e+308...1.79769e+308 (inclusive)]
\item {\it Parameter name:} {\tt Maximal z}
\phantomsection\label{parameters:Postprocess/Tracers/Generator/Uniform box/Maximal z}


\index[prmindex]{Maximal z}
\index[prmindexfull]{Postprocess!Tracers!Generator/Uniform box/Maximal z}
{\it Value:} 1


{\it Default:} 1


{\it Description:} Maximal z coordinate for the region of tracers.


{\it Possible values:} [Double -1.79769e+308...1.79769e+308 (inclusive)]
\item {\it Parameter name:} {\tt Minimal x}
\phantomsection\label{parameters:Postprocess/Tracers/Generator/Uniform box/Minimal x}


\index[prmindex]{Minimal x}
\index[prmindexfull]{Postprocess!Tracers!Generator/Uniform box/Minimal x}
{\it Value:} 0


{\it Default:} 0


{\it Description:} Minimal x coordinate for the region of tracers.


{\it Possible values:} [Double -1.79769e+308...1.79769e+308 (inclusive)]
\item {\it Parameter name:} {\tt Minimal y}
\phantomsection\label{parameters:Postprocess/Tracers/Generator/Uniform box/Minimal y}


\index[prmindex]{Minimal y}
\index[prmindexfull]{Postprocess!Tracers!Generator/Uniform box/Minimal y}
{\it Value:} 0


{\it Default:} 0


{\it Description:} Minimal y coordinate for the region of tracers.


{\it Possible values:} [Double -1.79769e+308...1.79769e+308 (inclusive)]
\item {\it Parameter name:} {\tt Minimal z}
\phantomsection\label{parameters:Postprocess/Tracers/Generator/Uniform box/Minimal z}


\index[prmindex]{Minimal z}
\index[prmindexfull]{Postprocess!Tracers!Generator/Uniform box/Minimal z}
{\it Value:} 0


{\it Default:} 0


{\it Description:} Minimal z coordinate for the region of tracers.


{\it Possible values:} [Double -1.79769e+308...1.79769e+308 (inclusive)]
\end{itemize}

\subsection{Parameters in section \tt Postprocess/Tracers/Generator/Uniform radial}
\label{parameters:Postprocess/Tracers/Generator/Uniform_20radial}

\begin{itemize}
\item {\it Parameter name:} {\tt Center x}
\phantomsection\label{parameters:Postprocess/Tracers/Generator/Uniform radial/Center x}


\index[prmindex]{Center x}
\index[prmindexfull]{Postprocess!Tracers!Generator/Uniform radial/Center x}
{\it Value:} 0


{\it Default:} 0


{\it Description:} x coordinate for the center of the spherical region, where tracers are generated.


{\it Possible values:} [Double -1.79769e+308...1.79769e+308 (inclusive)]
\item {\it Parameter name:} {\tt Center y}
\phantomsection\label{parameters:Postprocess/Tracers/Generator/Uniform radial/Center y}


\index[prmindex]{Center y}
\index[prmindexfull]{Postprocess!Tracers!Generator/Uniform radial/Center y}
{\it Value:} 0


{\it Default:} 0


{\it Description:} y coordinate for the center of the spherical region, where tracers are generated.


{\it Possible values:} [Double -1.79769e+308...1.79769e+308 (inclusive)]
\item {\it Parameter name:} {\tt Center z}
\phantomsection\label{parameters:Postprocess/Tracers/Generator/Uniform radial/Center z}


\index[prmindex]{Center z}
\index[prmindexfull]{Postprocess!Tracers!Generator/Uniform radial/Center z}
{\it Value:} 0


{\it Default:} 0


{\it Description:} z coordinate for the center of the spherical region, where tracers are generated.


{\it Possible values:} [Double -1.79769e+308...1.79769e+308 (inclusive)]
\item {\it Parameter name:} {\tt Maximum latitude}
\phantomsection\label{parameters:Postprocess/Tracers/Generator/Uniform radial/Maximum latitude}


\index[prmindex]{Maximum latitude}
\index[prmindexfull]{Postprocess!Tracers!Generator/Uniform radial/Maximum latitude}
{\it Value:} 3.1415


{\it Default:} 3.1415


{\it Description:} Maximum latitude coordinate for the region of tracers in degrees. Measured from the center position.


{\it Possible values:} [Double 0...180 (inclusive)]
\item {\it Parameter name:} {\tt Maximum longitude}
\phantomsection\label{parameters:Postprocess/Tracers/Generator/Uniform radial/Maximum longitude}


\index[prmindex]{Maximum longitude}
\index[prmindexfull]{Postprocess!Tracers!Generator/Uniform radial/Maximum longitude}
{\it Value:} 3.1415


{\it Default:} 3.1415


{\it Description:} Maximum longitude coordinate for the region of tracers in degrees. Measured from the center position.


{\it Possible values:} [Double 0...360 (inclusive)]
\item {\it Parameter name:} {\tt Maximum radius}
\phantomsection\label{parameters:Postprocess/Tracers/Generator/Uniform radial/Maximum radius}


\index[prmindex]{Maximum radius}
\index[prmindexfull]{Postprocess!Tracers!Generator/Uniform radial/Maximum radius}
{\it Value:} 1


{\it Default:} 1


{\it Description:} Maximum radial coordinate for the region of tracers. Measured from the center position.


{\it Possible values:} [Double -1.79769e+308...1.79769e+308 (inclusive)]
\item {\it Parameter name:} {\tt Minimum latitude}
\phantomsection\label{parameters:Postprocess/Tracers/Generator/Uniform radial/Minimum latitude}


\index[prmindex]{Minimum latitude}
\index[prmindexfull]{Postprocess!Tracers!Generator/Uniform radial/Minimum latitude}
{\it Value:} 0


{\it Default:} 0


{\it Description:} Minimum latitude coordinate for the region of tracers in degrees. Measured from the center position.


{\it Possible values:} [Double 0...180 (inclusive)]
\item {\it Parameter name:} {\tt Minimum longitude}
\phantomsection\label{parameters:Postprocess/Tracers/Generator/Uniform radial/Minimum longitude}


\index[prmindex]{Minimum longitude}
\index[prmindexfull]{Postprocess!Tracers!Generator/Uniform radial/Minimum longitude}
{\it Value:} 0


{\it Default:} 0


{\it Description:} Minimum longitude coordinate for the region of tracers in degrees. Measured from the center position.


{\it Possible values:} [Double 0...360 (inclusive)]
\item {\it Parameter name:} {\tt Minimum radius}
\phantomsection\label{parameters:Postprocess/Tracers/Generator/Uniform radial/Minimum radius}


\index[prmindex]{Minimum radius}
\index[prmindexfull]{Postprocess!Tracers!Generator/Uniform radial/Minimum radius}
{\it Value:} 0


{\it Default:} 0


{\it Description:} Minimum radial coordinate for the region of tracers. Measured from the center position.


{\it Possible values:} [Double 0...1.79769e+308 (inclusive)]
\item {\it Parameter name:} {\tt Radial layers}
\phantomsection\label{parameters:Postprocess/Tracers/Generator/Uniform radial/Radial layers}


\index[prmindex]{Radial layers}
\index[prmindexfull]{Postprocess!Tracers!Generator/Uniform radial/Radial layers}
{\it Value:} 1


{\it Default:} 1


{\it Description:} The number of radial shells of particles that will be generatedaround the central point.


{\it Possible values:} [Integer range 1...2147483647 (inclusive)]
\end{itemize}

\subsection{Parameters in section \tt Postprocess/Visualization}
\label{parameters:Postprocess/Visualization}

\begin{itemize}
\item {\it Parameter name:} {\tt Interpolate output}
\phantomsection\label{parameters:Postprocess/Visualization/Interpolate output}


\index[prmindex]{Interpolate output}
\index[prmindexfull]{Postprocess!Visualization!Interpolate output}
{\it Value:} false


{\it Default:} false


{\it Description:} deal.II offers the possibility to linearly interpolate output fields of higher order elements to a finer resolution. This somewhat compensates the fact that most visualization software only offers linear interpolation between grid points and therefore the output file is a very coarse representation of the actual solution field. Activating this option increases the spatial resolution in each dimension by a factor equal to the polynomial degree used for the velocity finite element (usually 2). In other words, instead of showing one quadrilateral or hexahedron in the visualization per cell on which \aspect{} computes, it shows multiple (for quadratic elements, it will describe each cell of the mesh on which we compute as $2\times 2$ or $2\times 2\times 2$ cells in 2d and 3d, respectively; correspondingly more subdivisions are used if you use cubic, quartic, or even higher order elements for the velocity).

The effect of using this option can be seen in the following picture showing a variation of the output produced with the input files from Section~\ref{sec:shell-simple-2d}:

\begin{center}  \includegraphics[width=0.5\textwidth]{viz/parameters/build-patches}\end{center}Here, the left picture shows one visualization cell per computational cell (i.e., the option is switch off, as is the default), and the right picture shows the same simulation with the option switched on. The images show the same data, demonstrating that interpolating the solution onto bilinear shape functions as is commonly done in visualizing data loses information.

Of course, activating this option also greatly increases the amount of data \aspect{} will write to disk: approximately by a factor of 4 in 2d, and a factor of 8 in 3d, when using quadratic elements for the velocity, and correspondingly more for even higher order elements.


{\it Possible values:} [Bool]
\item {\it Parameter name:} {\tt List of output variables}
\phantomsection\label{parameters:Postprocess/Visualization/List of output variables}


\index[prmindex]{List of output variables}
\index[prmindexfull]{Postprocess!Visualization!List of output variables}
{\it Value:} 


{\it Default:} 


{\it Description:} A comma separated list of visualization objects that should be run whenever writing graphical output. By default, the graphical output files will always contain the primary variables velocity, pressure, and temperature. However, one frequently wants to also visualize derived quantities, such as the thermodynamic phase that corresponds to a given temperature-pressure value, or the corresponding seismic wave speeds. The visualization objects do exactly this: they compute such derived quantities and place them into the output file. The current parameter is the place where you decide which of these additional output variables you want to have in your output file.

The following postprocessors are available:

`Vp anomaly': A visualization output object that generates output showing the anomaly in the seismic compression wave speed $V_p$ as a spatially variable function with one value per cell. This anomaly is shown as a percentage change relative to the average value of $V_p$ at the depth of this cell.

`Vs anomaly': A visualization output object that generates output showing the anomaly in the seismic shear wave speed $V_s$ as a spatially variable function with one value per cell. This anomaly is shown as a percentage change relative to the average value of $V_s$ at the depth of this cell.

`artificial viscosity': A visualization output object that generates output showing the value of the artificial viscosity on each cell.

`boundary indicators': A visualization output object that generates output about the used boundary indicators. In a loop over the active cells, if a cell lies at a domain boundary, the boundary indicator of the face along the boundary is requested. In case the cell does not lie along any domain boundary, the cell is assigned the value of the largest used boundary indicator plus one. When a cell is situated in one of the corners of the domain, multiple faces will have a boundary indicator. This postprocessor returns the value of the first face along a boundary that is encountered in a loop over all the faces. 

`compositional vector': A visualization output object that outputs vectors whose components are derived from compositional fields. Input parameters for this postprocessor are defined in section Postprocess/Visualization/Compositional fields as vectors

`density': A visualization output object that generates output for the density.

`depth': A visualization output postprocessor that outputs the depth for all points inside the domain, as determined by the geometry model.

`dynamic topography': A visualization output object that generates output for the dynamic topography. The approach to determine the dynamic topography requires us to compute the stress tensor and evaluate the component of it in the direction in which gravity acts. In other words, we compute $\sigma_{rr}={\hat g}^T(2 \eta \varepsilon(\mathbf u)-\frac 13 (\textrm{div}\;\mathbf u)I)\hat g - p_d$ where $\hat g = \mathbf g/\|\mathbf g\|$ is the direction of the gravity vector $\mathbf g$ and $p_d=p-p_a$ is the dynamic pressure computed by subtracting the adiabatic pressure $p_a$ from the total pressure $p$ computed as part of the Stokes solve. From this, the dynamic topography is computed using the formula $h=\frac{\sigma_{rr}}{\|\mathbf g\| \rho}$ where $\rho$ is the density at the cell center.

Strictly speaking, the dynamic topography is of course a quantity that is only of interest at the surface. However, we compute it everywhere to make things fit into the framework within which we produce data for visualization. You probably only want to visualize whatever data this postprocessor generates at the surface of your domain and simply ignore the rest of the data generated.

`error indicator': A visualization output object that generates output showing the estimated error or other mesh refinement indicator as a spatially variable function with one value per cell.

`friction heating': A visualization output object that generates output for the amount of friction heating often referred to as $\tau:\epsilon$. More concisely, in the incompressible case, the quantity that is output is defined as $\eta \varepsilon(\mathbf u):\varepsilon(\mathbf u)$ where $\eta$ is itself a function of temperature, pressure and strain rate. In the compressible case, the quantity that's computed is $\eta [\varepsilon(\mathbf u)-\tfrac 13(\textrm{tr}\;\varepsilon(\mathbf u))\mathbf I]:[\varepsilon(\mathbf u)-\tfrac 13(\textrm{tr}\;\varepsilon(\mathbf u))\mathbf I]$.

`gravity': A visualization output object that outputs the gravity vector.

`heating': A visualization output object that generates output for all the heating terms used in the energy equation.

`material properties': A visualization output object that generates output for the material properties given by the material model.There are a number of other visualization postprocessors that offer to write individual material properties. However, they all individually have to evaluate the material model. This is inefficient if one wants to output more than just one or two of the fields provided by the material model. The current postprocessor allows to output a (potentially large) subset of all of the information provided by material models at once, with just a single material model evaluation per output point.

`melt fraction': A visualization output object that generates output for the melt fraction at the temperature and pressure of the current point (batch melting). Does not take into account latent heat. If there are no compositional fields, this postprocessor will visualize the melt fraction of peridotite (calculated using the anhydrous model of Katz, 2003). If there is at least one compositional field, the postprocessor assumes that the  first compositional field is the content of pyroxenite, and will visualize the melt fraction for a mixture of peridotite and pyroxenite (using the melting model of Sobolev, 2011 for pyroxenite). All the parameters that were used in these calculations can be changed in the input file, the most relevant maybe being the mass fraction of Cpx in peridotite in the Katz melting model (Mass fraction cpx), which right now has a default of 15\%. The corresponding p-T-diagrams can be generated by running the tests melt\_postprocessor\_peridotite and melt\_postprocessor\_pyroxenite.

`nonadiabatic pressure': A visualization output object that generates output for the non-adiabatic component of the pressure.

`nonadiabatic temperature': A visualization output object that generates output for the non-adiabatic component of the pressure.

`partition': A visualization output object that generates output for the parallel partition that every cell of the mesh is associated with.

`seismic vp': A visualization output object that generates output for the seismic P-wave speed.

`seismic vs': A visualization output object that generates output for the seismic S-wave speed.

`shear stress': A visualization output object that generates output for the 3 (in 2d) or 6 (in 3d) components of the shear stress tensor, i.e., for the components of the tensor $2\eta\varepsilon(\mathbf u)$ in the incompressible case and $2\eta\left[\varepsilon(\mathbf u)-\tfrac 13(\textrm{tr}\;\varepsilon(\mathbf u))\mathbf I\right]$ in the compressible case. The shear stress differs from the full stress tensor by the absence of the pressure.

`specific heat': A visualization output object that generates output for the specific heat $C_p$.

`strain rate': A visualization output object that generates output for the norm of the strain rate, i.e., for the quantity $\sqrt{\varepsilon(\mathbf u):\varepsilon(\mathbf u)}$ in the incompressible case and $\sqrt{[\varepsilon(\mathbf u)-\tfrac 13(\textrm{tr}\;\varepsilon(\mathbf u))\mathbf I]:[\varepsilon(\mathbf u)-\tfrac 13(\textrm{tr}\;\varepsilon(\mathbf u))\mathbf I]}$ in the compressible case.

`stress': A visualization output object that generates output for the 3 (in 2d) or 6 (in 3d) components of the stress tensor, i.e., for the components of the tensor $2\eta\varepsilon(\mathbf u)+pI$ in the incompressible case and $2\eta\left[\varepsilon(\mathbf u)-\tfrac 13(\textrm{tr}\;\varepsilon(\mathbf u))\mathbf I\right]+pI$ in the compressible case.

`thermal conductivity': A visualization output object that generates output for the thermal conductivity $k$.

`thermal diffusivity': A visualization output object that generates output for the thermal diffusivity $\kappa$=$\frac{k}{\rho c_p}$, with $k$ the thermal conductivity.

`thermal expansivity': A visualization output object that generates output for the thermal expansivity.

`thermodynamic phase': A visualization output object that generates output for the integer number of the phase that is thermodynamically stable at the temperature and pressure of the current point.

`vertical heat flux': A visualization output object that generates output for the heat flux in the vertical direction, which is the sum of the advective and the conductive heat flux, with the sign convention of positive flux upwards.

`viscosity': A visualization output object that generates output for the viscosity.

`viscosity ratio': A visualization output object that generates output for the ratio between dislocation viscosity and diffusion viscosity.


{\it Possible values:} [MultipleSelection Vp anomaly|Vs anomaly|artificial viscosity|boundary indicators|compositional vector|density|depth|dynamic topography|error indicator|friction heating|gravity|heating|material properties|melt fraction|nonadiabatic pressure|nonadiabatic temperature|partition|seismic vp|seismic vs|shear stress|specific heat|strain rate|stress|thermal conductivity|thermal diffusivity|thermal expansivity|thermodynamic phase|vertical heat flux|viscosity|viscosity ratio ]
\item {\it Parameter name:} {\tt Number of grouped files}
\phantomsection\label{parameters:Postprocess/Visualization/Number of grouped files}


\index[prmindex]{Number of grouped files}
\index[prmindexfull]{Postprocess!Visualization!Number of grouped files}
{\it Value:} 0


{\it Default:} 0


{\it Description:} VTU file output supports grouping files from several CPUs into a given number of files using MPI I/O when writing on a parallel filesystem. Select 0 for no grouping. This will disable parallel file output and instead write one file per processor in a background thread. A value of 1 will generate one big file containing the whole solution, while a larger value will create that many files (at most as many as there are mpi ranks).


{\it Possible values:} [Integer range 0...2147483647 (inclusive)]
\item {\it Parameter name:} {\tt Output format}
\phantomsection\label{parameters:Postprocess/Visualization/Output format}


\index[prmindex]{Output format}
\index[prmindexfull]{Postprocess!Visualization!Output format}
{\it Value:} vtu


{\it Default:} vtu


{\it Description:} The file format to be used for graphical output.


{\it Possible values:} [Selection none|dx|ucd|gnuplot|povray|eps|gmv|tecplot|tecplot\_binary|vtk|vtu|hdf5|svg|deal.II intermediate ]
\item {\it Parameter name:} {\tt Output mesh velocity}
\phantomsection\label{parameters:Postprocess/Visualization/Output mesh velocity}


\index[prmindex]{Output mesh velocity}
\index[prmindexfull]{Postprocess!Visualization!Output mesh velocity}
{\it Value:} false


{\it Default:} false


{\it Description:} For free surface computations Aspect uses an Arbitrary-Lagrangian-Eulerian formulation to handle deforming the domain, so the mesh has its own velocity field.  This may be written as an output field by setting this parameter to true.


{\it Possible values:} [Bool]
\item {\it Parameter name:} {\tt Time between graphical output}
\phantomsection\label{parameters:Postprocess/Visualization/Time between graphical output}


\index[prmindex]{Time between graphical output}
\index[prmindexfull]{Postprocess!Visualization!Time between graphical output}
{\it Value:} 1e8


{\it Default:} 1e8


{\it Description:} The time interval between each generation of graphical output files. A value of zero indicates that output should be generated in each time step. Units: years if the 'Use years in output instead of seconds' parameter is set; seconds otherwise.


{\it Possible values:} [Double 0...1.79769e+308 (inclusive)]
\end{itemize}



\subsection{Parameters in section \tt Postprocess/Visualization/Compositional fields as vectors}
\label{parameters:Postprocess/Visualization/Compositional_20fields_20as_20vectors}

\begin{itemize}
\item {\it Parameter name:} {\tt Names of fields}
\phantomsection\label{parameters:Postprocess/Visualization/Compositional fields as vectors/Names of fields}


\index[prmindex]{Names of fields}
\index[prmindexfull]{Postprocess!Visualization!Compositional fields as vectors/Names of fields}
{\it Value:} 


{\it Default:} 


{\it Description:} A list of sets of compositional fields which should be output as vectors. Sets are separated from each other by semicolons and vector components within each set are separated by commas (e.g. $vec1_x$, $vec1_y$ ; $vec2_x$, $vec2_y$) where each name must be a defined named compositional field. If only one name is given in a set, it is interpreted as the first in a sequence of dim consecutive compositional fields.


{\it Possible values:} [Anything]
\item {\it Parameter name:} {\tt Names of vectors}
\phantomsection\label{parameters:Postprocess/Visualization/Compositional fields as vectors/Names of vectors}


\index[prmindex]{Names of vectors}
\index[prmindexfull]{Postprocess!Visualization!Compositional fields as vectors/Names of vectors}
{\it Value:} 


{\it Default:} 


{\it Description:} Names of vectors as they will appear in the output.


{\it Possible values:} [List list of [Anything] of length 0...4294967295 (inclusive)]
\end{itemize}

\subsection{Parameters in section \tt Postprocess/Visualization/Dynamic Topography}
\label{parameters:Postprocess/Visualization/Dynamic_20Topography}

\begin{itemize}
\item {\it Parameter name:} {\tt Subtract mean of dynamic topography}
\phantomsection\label{parameters:Postprocess/Visualization/Dynamic Topography/Subtract mean of dynamic topography}


\index[prmindex]{Subtract mean of dynamic topography}
\index[prmindexfull]{Postprocess!Visualization!Dynamic Topography/Subtract mean of dynamic topography}
{\it Value:} false


{\it Default:} false


{\it Description:} Option to remove the mean dynamic topography in the outputted data file (not visualization). 'true' subtracts the mean, 'false' leaves the calculated dynamic topography as is. 


{\it Possible values:} [Bool]
\end{itemize}

\subsection{Parameters in section \tt Postprocess/Visualization/Material properties}
\label{parameters:Postprocess/Visualization/Material_20properties}

\begin{itemize}
\item {\it Parameter name:} {\tt List of material properties}
\phantomsection\label{parameters:Postprocess/Visualization/Material properties/List of material properties}


\index[prmindex]{List of material properties}
\index[prmindexfull]{Postprocess!Visualization!Material properties/List of material properties}
{\it Value:} density,thermal expansivity,specific heat,viscosity


{\it Default:} density,thermal expansivity,specific heat,viscosity


{\it Description:} A comma separated list of material properties that should be written whenever writing graphical output. By default, the material properties will always contain the density, thermal expansivity, specific heat and viscosity. The following material properties are available:

viscosity|density|thermal expansivity|specific heat|thermal conductivity|thermal diffusivity|compressibility|entropy derivative temperature|entropy derivative pressure|reaction terms


{\it Possible values:} [MultipleSelection viscosity|density|thermal expansivity|specific heat|thermal conductivity|thermal diffusivity|compressibility|entropy derivative temperature|entropy derivative pressure|reaction terms ]
\end{itemize}

\subsection{Parameters in section \tt Postprocess/Visualization/Melt fraction}
\label{parameters:Postprocess/Visualization/Melt_20fraction}

\begin{itemize}
\item {\it Parameter name:} {\tt A1}
\phantomsection\label{parameters:Postprocess/Visualization/Melt fraction/A1}


\index[prmindex]{A1}
\index[prmindexfull]{Postprocess!Visualization!Melt fraction/A1}
{\it Value:} 1085.7


{\it Default:} 1085.7


{\it Description:} Constant parameter in the quadratic function that approximates the solidus of peridotite. Units: $°C$.


{\it Possible values:} [Double -1.79769e+308...1.79769e+308 (inclusive)]
\item {\it Parameter name:} {\tt A2}
\phantomsection\label{parameters:Postprocess/Visualization/Melt fraction/A2}


\index[prmindex]{A2}
\index[prmindexfull]{Postprocess!Visualization!Melt fraction/A2}
{\it Value:} 1.329e-7


{\it Default:} 1.329e-7


{\it Description:} Prefactor of the linear pressure term in the quadratic function that approximates the solidus of peridotite. Units: $°C/Pa$.


{\it Possible values:} [Double -1.79769e+308...1.79769e+308 (inclusive)]
\item {\it Parameter name:} {\tt A3}
\phantomsection\label{parameters:Postprocess/Visualization/Melt fraction/A3}


\index[prmindex]{A3}
\index[prmindexfull]{Postprocess!Visualization!Melt fraction/A3}
{\it Value:} -5.1e-18


{\it Default:} -5.1e-18


{\it Description:} Prefactor of the quadratic pressure term in the quadratic function that approximates the solidus of peridotite. Units: $°C/(Pa^2)$.


{\it Possible values:} [Double -1.79769e+308...1.79769e+308 (inclusive)]
\item {\it Parameter name:} {\tt B1}
\phantomsection\label{parameters:Postprocess/Visualization/Melt fraction/B1}


\index[prmindex]{B1}
\index[prmindexfull]{Postprocess!Visualization!Melt fraction/B1}
{\it Value:} 1475.0


{\it Default:} 1475.0


{\it Description:} Constant parameter in the quadratic function that approximates the lherzolite liquidus used for calculating the fraction of peridotite-derived melt. Units: $°C$.


{\it Possible values:} [Double -1.79769e+308...1.79769e+308 (inclusive)]
\item {\it Parameter name:} {\tt B2}
\phantomsection\label{parameters:Postprocess/Visualization/Melt fraction/B2}


\index[prmindex]{B2}
\index[prmindexfull]{Postprocess!Visualization!Melt fraction/B2}
{\it Value:} 8.0e-8


{\it Default:} 8.0e-8


{\it Description:} Prefactor of the linear pressure term in the quadratic function that approximates the  lherzolite liquidus used for calculating the fraction of peridotite-derived melt. Units: $°C/Pa$.


{\it Possible values:} [Double -1.79769e+308...1.79769e+308 (inclusive)]
\item {\it Parameter name:} {\tt B3}
\phantomsection\label{parameters:Postprocess/Visualization/Melt fraction/B3}


\index[prmindex]{B3}
\index[prmindexfull]{Postprocess!Visualization!Melt fraction/B3}
{\it Value:} -3.2e-18


{\it Default:} -3.2e-18


{\it Description:} Prefactor of the quadratic pressure term in the quadratic function that approximates the  lherzolite liquidus used for calculating the fraction of peridotite-derived melt. Units: $°C/(Pa^2)$.


{\it Possible values:} [Double -1.79769e+308...1.79769e+308 (inclusive)]
\item {\it Parameter name:} {\tt C1}
\phantomsection\label{parameters:Postprocess/Visualization/Melt fraction/C1}


\index[prmindex]{C1}
\index[prmindexfull]{Postprocess!Visualization!Melt fraction/C1}
{\it Value:} 1780.0


{\it Default:} 1780.0


{\it Description:} Constant parameter in the quadratic function that approximates the liquidus of peridotite. Units: $°C$.


{\it Possible values:} [Double -1.79769e+308...1.79769e+308 (inclusive)]
\item {\it Parameter name:} {\tt C2}
\phantomsection\label{parameters:Postprocess/Visualization/Melt fraction/C2}


\index[prmindex]{C2}
\index[prmindexfull]{Postprocess!Visualization!Melt fraction/C2}
{\it Value:} 4.50e-8


{\it Default:} 4.50e-8


{\it Description:} Prefactor of the linear pressure term in the quadratic function that approximates the liquidus of peridotite. Units: $°C/Pa$.


{\it Possible values:} [Double -1.79769e+308...1.79769e+308 (inclusive)]
\item {\it Parameter name:} {\tt C3}
\phantomsection\label{parameters:Postprocess/Visualization/Melt fraction/C3}


\index[prmindex]{C3}
\index[prmindexfull]{Postprocess!Visualization!Melt fraction/C3}
{\it Value:} -2.0e-18


{\it Default:} -2.0e-18


{\it Description:} Prefactor of the quadratic pressure term in the quadratic function that approximates the liquidus of peridotite. Units: $°C/(Pa^2)$.


{\it Possible values:} [Double -1.79769e+308...1.79769e+308 (inclusive)]
\item {\it Parameter name:} {\tt D1}
\phantomsection\label{parameters:Postprocess/Visualization/Melt fraction/D1}


\index[prmindex]{D1}
\index[prmindexfull]{Postprocess!Visualization!Melt fraction/D1}
{\it Value:} 976.0


{\it Default:} 976.0


{\it Description:} Constant parameter in the quadratic function that approximates the solidus of pyroxenite. Units: $°C$.


{\it Possible values:} [Double -1.79769e+308...1.79769e+308 (inclusive)]
\item {\it Parameter name:} {\tt D2}
\phantomsection\label{parameters:Postprocess/Visualization/Melt fraction/D2}


\index[prmindex]{D2}
\index[prmindexfull]{Postprocess!Visualization!Melt fraction/D2}
{\it Value:} 1.329e-7


{\it Default:} 1.329e-7


{\it Description:} Prefactor of the linear pressure term in the quadratic function that approximates the solidus of pyroxenite. Note that this factor is different from the value given in Sobolev, 2011, because they use the potential temperature whereas we use the absolute temperature. Units: $°C/Pa$.


{\it Possible values:} [Double -1.79769e+308...1.79769e+308 (inclusive)]
\item {\it Parameter name:} {\tt D3}
\phantomsection\label{parameters:Postprocess/Visualization/Melt fraction/D3}


\index[prmindex]{D3}
\index[prmindexfull]{Postprocess!Visualization!Melt fraction/D3}
{\it Value:} -5.1e-18


{\it Default:} -5.1e-18


{\it Description:} Prefactor of the quadratic pressure term in the quadratic function that approximates the solidus of pyroxenite. Units: $°C/(Pa^2)$.


{\it Possible values:} [Double -1.79769e+308...1.79769e+308 (inclusive)]
\item {\it Parameter name:} {\tt E1}
\phantomsection\label{parameters:Postprocess/Visualization/Melt fraction/E1}


\index[prmindex]{E1}
\index[prmindexfull]{Postprocess!Visualization!Melt fraction/E1}
{\it Value:} 663.8


{\it Default:} 663.8


{\it Description:} Prefactor of the linear depletion term in the quadratic function that approximates the melt fraction of pyroxenite. Units: $°C/Pa$.


{\it Possible values:} [Double -1.79769e+308...1.79769e+308 (inclusive)]
\item {\it Parameter name:} {\tt E2}
\phantomsection\label{parameters:Postprocess/Visualization/Melt fraction/E2}


\index[prmindex]{E2}
\index[prmindexfull]{Postprocess!Visualization!Melt fraction/E2}
{\it Value:} -611.4


{\it Default:} -611.4


{\it Description:} Prefactor of the quadratic depletion term in the quadratic function that approximates the melt fraction of pyroxenite. Units: $°C/(Pa^2)$.


{\it Possible values:} [Double -1.79769e+308...1.79769e+308 (inclusive)]
\item {\it Parameter name:} {\tt Mass fraction cpx}
\phantomsection\label{parameters:Postprocess/Visualization/Melt fraction/Mass fraction cpx}


\index[prmindex]{Mass fraction cpx}
\index[prmindexfull]{Postprocess!Visualization!Melt fraction/Mass fraction cpx}
{\it Value:} 0.15


{\it Default:} 0.15


{\it Description:} Mass fraction of clinopyroxene in the peridotite to be molten. Units: non-dimensional.


{\it Possible values:} [Double -1.79769e+308...1.79769e+308 (inclusive)]
\item {\it Parameter name:} {\tt beta}
\phantomsection\label{parameters:Postprocess/Visualization/Melt fraction/beta}


\index[prmindex]{beta}
\index[prmindexfull]{Postprocess!Visualization!Melt fraction/beta}
{\it Value:} 1.5


{\it Default:} 1.5


{\it Description:} Exponent of the melting temperature in the melt fraction calculation. Units: non-dimensional.


{\it Possible values:} [Double -1.79769e+308...1.79769e+308 (inclusive)]
\item {\it Parameter name:} {\tt r1}
\phantomsection\label{parameters:Postprocess/Visualization/Melt fraction/r1}


\index[prmindex]{r1}
\index[prmindexfull]{Postprocess!Visualization!Melt fraction/r1}
{\it Value:} 0.5


{\it Default:} 0.5


{\it Description:} Constant in the linear function that approximates the clinopyroxene reaction coefficient. Units: non-dimensional.


{\it Possible values:} [Double -1.79769e+308...1.79769e+308 (inclusive)]
\item {\it Parameter name:} {\tt r2}
\phantomsection\label{parameters:Postprocess/Visualization/Melt fraction/r2}


\index[prmindex]{r2}
\index[prmindexfull]{Postprocess!Visualization!Melt fraction/r2}
{\it Value:} 8e-11


{\it Default:} 8e-11


{\it Description:} Prefactor of the linear pressure term in the linear function that approximates the clinopyroxene reaction coefficient. Units: $1/Pa$.


{\it Possible values:} [Double -1.79769e+308...1.79769e+308 (inclusive)]
\end{itemize}

\subsection{Parameters in section \tt Prescribed Stokes solution}
\label{parameters:Prescribed_20Stokes_20solution}

\begin{itemize}
\item {\it Parameter name:} {\tt Model name}
\phantomsection\label{parameters:Prescribed Stokes solution/Model name}


\index[prmindex]{Model name}
\index[prmindexfull]{Prescribed Stokes solution!Model name}
{\it Value:} unspecified


{\it Default:} unspecified


{\it Description:} Select one of the following models:

`ascii data': Implementation of a model in which the velocityis derived from files containing data in ascii format. Note the required format of the input data: The first lines may contain any number of commentsif they begin with '\#', but one of these lines needs tocontain the number of grid points in each dimension asfor example '\# POINTS: 3 3'.The order of the data columns has to be 'x', 'y', 'v$_x$' , 'v$_y$' in a 2d model and  'x', 'y', 'z', 'v$_x$' , 'v$_y$' , 'v$_z$' in a 3d model.Note that the data in the input files need to be sorted in a specific order: the first coordinate needs to ascend first, followed by the second and the third at last in order to assign the correct data to the prescribed coordinates.If you use a spherical model, then the data will still be handled as cartesian,however the assumed grid changes. 'x' will be replaced by the radial distance of the point to the bottom of the model, 'y' by the azimuth angle and 'z' by the polar angle measured positive from the north pole. The grid will be assumed to be a latitude-longitude grid. Note that the order of spherical coordinates is 'r', 'phi', 'theta' and not 'r', 'theta', 'phi', since this allows for dimension independent expressions. 

`circle': This value describes a vector field that rotates around the z-axis with constant angular velocity (i.e., with a velocity that increases with distance from the axis). The pressure is set to zero.

`function': This plugin allows to prescribe the Stokes solution for the velocity and pressure field in terms of an explicit formula. The format of these functions follows the syntax understood by the muparser library, see Section~\ref{sec:muparser-format}.


{\it Possible values:} [Selection ascii data|circle|function|unspecified ]
\end{itemize}



\subsection{Parameters in section \tt Prescribed Stokes solution/Ascii data model}
\label{parameters:Prescribed_20Stokes_20solution/Ascii_20data_20model}

\begin{itemize}
\item {\it Parameter name:} {\tt Data directory}
\phantomsection\label{parameters:Prescribed Stokes solution/Ascii data model/Data directory}


\index[prmindex]{Data directory}
\index[prmindexfull]{Prescribed Stokes solution!Ascii data model!Data directory}
{\it Value:} \$ASPECT\_SOURCE\_DIR/data/prescribed-stokes-solution/


{\it Default:} \$ASPECT\_SOURCE\_DIR/data/prescribed-stokes-solution/


{\it Description:} The name of a directory that contains the model data. This path may either be absolute (if starting with a '/') or relative to the current directory. The path may also include the special text '\$ASPECT\_SOURCE\_DIR' which will be interpreted as the path in which the ASPECT source files were located when ASPECT was compiled. This interpretation allows, for example, to reference files located in the 'data/' subdirectory of ASPECT. 


{\it Possible values:} [DirectoryName]
\item {\it Parameter name:} {\tt Data file name}
\phantomsection\label{parameters:Prescribed Stokes solution/Ascii data model/Data file name}


\index[prmindex]{Data file name}
\index[prmindexfull]{Prescribed Stokes solution!Ascii data model!Data file name}
{\it Value:} box\_2d.txt


{\it Default:} box\_2d.txt


{\it Description:} The file name of the material data. Provide file in format: (Velocity file name).\%s\%d where \%s is a string specifying the boundary of the model according to the names of the boundary indicators (of a box or a spherical shell).\%d is any sprintf integer qualifier, specifying the format of the current file number. 


{\it Possible values:} [Anything]
\item {\it Parameter name:} {\tt Scale factor}
\phantomsection\label{parameters:Prescribed Stokes solution/Ascii data model/Scale factor}


\index[prmindex]{Scale factor}
\index[prmindexfull]{Prescribed Stokes solution!Ascii data model!Scale factor}
{\it Value:} 1


{\it Default:} 1


{\it Description:} Scalar factor, which is applied to the boundary velocity. You might want to use this to scale the velocities to a reference model (e.g. with free-slip boundary) or another plate reconstruction. Another way to use this factor is to convert units of the input files. The unit is assumed to bem/s or m/yr depending on the 'Use years in output instead of seconds' flag. If you provide velocities in cm/yr set this factor to 0.01.


{\it Possible values:} [Double 0...1.79769e+308 (inclusive)]
\end{itemize}

\subsection{Parameters in section \tt Prescribed Stokes solution/Pressure function}
\label{parameters:Prescribed_20Stokes_20solution/Pressure_20function}

\begin{itemize}
\item {\it Parameter name:} {\tt Function constants}
\phantomsection\label{parameters:Prescribed Stokes solution/Pressure function/Function constants}


\index[prmindex]{Function constants}
\index[prmindexfull]{Prescribed Stokes solution!Pressure function!Function constants}
{\it Value:} 


{\it Default:} 


{\it Description:} Sometimes it is convenient to use symbolic constants in the expression that describes the function, rather than having to use its numeric value everywhere the constant appears. These values can be defined using this parameter, in the form `var1=value1, var2=value2, ...'.

A typical example would be to set this runtime parameter to `pi=3.1415926536' and then use `pi' in the expression of the actual formula. (That said, for convenience this class actually defines both `pi' and `Pi' by default, but you get the idea.)


{\it Possible values:} [Anything]
\item {\it Parameter name:} {\tt Function expression}
\phantomsection\label{parameters:Prescribed Stokes solution/Pressure function/Function expression}


\index[prmindex]{Function expression}
\index[prmindexfull]{Prescribed Stokes solution!Pressure function!Function expression}
{\it Value:} 0


{\it Default:} 0


{\it Description:} The formula that denotes the function you want to evaluate for particular values of the independent variables. This expression may contain any of the usual operations such as addition or multiplication, as well as all of the common functions such as `sin' or `cos'. In addition, it may contain expressions like `if(x>0, 1, -1)' where the expression evaluates to the second argument if the first argument is true, and to the third argument otherwise. For a full overview of possible expressions accepted see the documentation of the muparser library at http://muparser.beltoforion.de/.

If the function you are describing represents a vector-valued function with multiple components, then separate the expressions for individual components by a semicolon.


{\it Possible values:} [Anything]
\item {\it Parameter name:} {\tt Variable names}
\phantomsection\label{parameters:Prescribed Stokes solution/Pressure function/Variable names}


\index[prmindex]{Variable names}
\index[prmindexfull]{Prescribed Stokes solution!Pressure function!Variable names}
{\it Value:} x,y,t


{\it Default:} x,y,t


{\it Description:} The name of the variables as they will be used in the function, separated by commas. By default, the names of variables at which the function will be evaluated is `x' (in 1d), `x,y' (in 2d) or `x,y,z' (in 3d) for spatial coordinates and `t' for time. You can then use these variable names in your function expression and they will be replaced by the values of these variables at which the function is currently evaluated. However, you can also choose a different set of names for the independent variables at which to evaluate your function expression. For example, if you work in spherical coordinates, you may wish to set this input parameter to `r,phi,theta,t' and then use these variable names in your function expression.


{\it Possible values:} [Anything]
\end{itemize}

\subsection{Parameters in section \tt Prescribed Stokes solution/Velocity function}
\label{parameters:Prescribed_20Stokes_20solution/Velocity_20function}

\begin{itemize}
\item {\it Parameter name:} {\tt Function constants}
\phantomsection\label{parameters:Prescribed Stokes solution/Velocity function/Function constants}


\index[prmindex]{Function constants}
\index[prmindexfull]{Prescribed Stokes solution!Velocity function!Function constants}
{\it Value:} 


{\it Default:} 


{\it Description:} Sometimes it is convenient to use symbolic constants in the expression that describes the function, rather than having to use its numeric value everywhere the constant appears. These values can be defined using this parameter, in the form `var1=value1, var2=value2, ...'.

A typical example would be to set this runtime parameter to `pi=3.1415926536' and then use `pi' in the expression of the actual formula. (That said, for convenience this class actually defines both `pi' and `Pi' by default, but you get the idea.)


{\it Possible values:} [Anything]
\item {\it Parameter name:} {\tt Function expression}
\phantomsection\label{parameters:Prescribed Stokes solution/Velocity function/Function expression}


\index[prmindex]{Function expression}
\index[prmindexfull]{Prescribed Stokes solution!Velocity function!Function expression}
{\it Value:} 0; 0


{\it Default:} 0; 0


{\it Description:} The formula that denotes the function you want to evaluate for particular values of the independent variables. This expression may contain any of the usual operations such as addition or multiplication, as well as all of the common functions such as `sin' or `cos'. In addition, it may contain expressions like `if(x>0, 1, -1)' where the expression evaluates to the second argument if the first argument is true, and to the third argument otherwise. For a full overview of possible expressions accepted see the documentation of the muparser library at http://muparser.beltoforion.de/.

If the function you are describing represents a vector-valued function with multiple components, then separate the expressions for individual components by a semicolon.


{\it Possible values:} [Anything]
\item {\it Parameter name:} {\tt Variable names}
\phantomsection\label{parameters:Prescribed Stokes solution/Velocity function/Variable names}


\index[prmindex]{Variable names}
\index[prmindexfull]{Prescribed Stokes solution!Velocity function!Variable names}
{\it Value:} x,y,t


{\it Default:} x,y,t


{\it Description:} The name of the variables as they will be used in the function, separated by commas. By default, the names of variables at which the function will be evaluated is `x' (in 1d), `x,y' (in 2d) or `x,y,z' (in 3d) for spatial coordinates and `t' for time. You can then use these variable names in your function expression and they will be replaced by the values of these variables at which the function is currently evaluated. However, you can also choose a different set of names for the independent variables at which to evaluate your function expression. For example, if you work in spherical coordinates, you may wish to set this input parameter to `r,phi,theta,t' and then use these variable names in your function expression.


{\it Possible values:} [Anything]
\end{itemize}

\subsection{Parameters in section \tt Termination criteria}
\label{parameters:Termination_20criteria}

\begin{itemize}
\item {\it Parameter name:} {\tt Checkpoint on termination}
\phantomsection\label{parameters:Termination criteria/Checkpoint on termination}


\index[prmindex]{Checkpoint on termination}
\index[prmindexfull]{Termination criteria!Checkpoint on termination}
{\it Value:} false


{\it Default:} false


{\it Description:} Whether to checkpoint the simulation right before termination.


{\it Possible values:} [Bool]
\item {\it Parameter name:} {\tt End step}
\phantomsection\label{parameters:Termination criteria/End step}


\index[prmindex]{End step}
\index[prmindexfull]{Termination criteria!End step}
{\it Value:} 100


{\it Default:} 100


{\it Description:} Terminate the simulation once the specified timestep has been reached.


{\it Possible values:} [Integer range 0...2147483647 (inclusive)]
\item {\it Parameter name:} {\tt Termination criteria}
\phantomsection\label{parameters:Termination criteria/Termination criteria}


\index[prmindex]{Termination criteria}
\index[prmindexfull]{Termination criteria!Termination criteria}
{\it Value:} end time


{\it Default:} end time


{\it Description:} A comma separated list of termination criteria that will determine when the simulation should end. Whether explicitly stated or not, the ``end time'' termination criterion will always be used.The following termination criteria are available:

`end step': Terminate the simulation once the specified timestep has been reached. 

`end time': Terminate the simulation once the end time specified in the input file has been reached. Unlike all other termination criteria, this criterion is \textit{always} active, whether it has been explicitly selected or not in the input file (this is done to preserve historical behavior of \aspect{}, but it also likely does not inconvenience anyone since it is what would be selected in most cases anyway).

`steady state velocity': A criterion that terminates the simulation when the RMS of the velocity field stays within a certain range for a specified period of time.

`user request': Terminate the simulation gracefully when a file with a specified name appears in the output directory. This allows the user to gracefully exit the simulation at any time by simply creating such a file using, for example, \texttt{touch output/terminate}. The file's location is chosen to be in the output directory, rather than in a generic location such as the Aspect directory, so that one can run multiple simulations at the same time (which presumably write to different output directories) and can selectively terminate a particular one.


{\it Possible values:} [MultipleSelection end step|end time|steady state velocity|user request ]
\end{itemize}



\subsection{Parameters in section \tt Termination criteria/Steady state velocity}
\label{parameters:Termination_20criteria/Steady_20state_20velocity}

\begin{itemize}
\item {\it Parameter name:} {\tt Maximum relative deviation}
\phantomsection\label{parameters:Termination criteria/Steady state velocity/Maximum relative deviation}


\index[prmindex]{Maximum relative deviation}
\index[prmindexfull]{Termination criteria!Steady state velocity!Maximum relative deviation}
{\it Value:} 0.05


{\it Default:} 0.05


{\it Description:} The maximum relative deviation of the RMS in recent simulation time for the system to be considered in steady state. If the actual deviation is smaller than this number, then the simulation will be terminated.


{\it Possible values:} [Double 0...1.79769e+308 (inclusive)]
\item {\it Parameter name:} {\tt Time in steady state}
\phantomsection\label{parameters:Termination criteria/Steady state velocity/Time in steady state}


\index[prmindex]{Time in steady state}
\index[prmindexfull]{Termination criteria!Steady state velocity!Time in steady state}
{\it Value:} 1e7


{\it Default:} 1e7


{\it Description:} The minimum length of simulation time that the system should be in steady state before termination.Units: years if the 'Use years in output instead of seconds' parameter is set; seconds otherwise.


{\it Possible values:} [Double 0...1.79769e+308 (inclusive)]
\end{itemize}

\subsection{Parameters in section \tt Termination criteria/User request}
\label{parameters:Termination_20criteria/User_20request}

\begin{itemize}
\item {\it Parameter name:} {\tt File name}
\phantomsection\label{parameters:Termination criteria/User request/File name}


\index[prmindex]{File name}
\index[prmindexfull]{Termination criteria!User request!File name}
{\it Value:} terminate-aspect


{\it Default:} terminate-aspect


{\it Description:} The name of a file that, if it exists in the output directory (whose name is also specified in the input file) will lead to termination of the simulation. The file's location is chosen to be in the output directory, rather than in a generic location such as the Aspect directory, so that one can run multiple simulations at the same time (which presumably write to different output directories) and can selectively terminate a particular one.


{\it Possible values:} [FileName (Type: input)]
\end{itemize}



\pagebreak

% print the list of references. make sure the page number in the index is
% correct by putting the \addcontentsline inside the command that prints the
% title of the page, see http://www.dfki.de/~loeckelt/latexbib.html

\let\myRefname\refname
\renewcommand\refname{%
  \addcontentsline{toc}{section}{\numberline{}References}
  \myRefname
}
\bibliographystyle{alpha}
\bibliography{manual}


\pagebreak

% print the index. note that we put the \label and \addcontentsline into the
% text printed at the top to make sure they get processed when latex is
% already on the page where this shows up (otherwise we end up with wrong page
% labels)
\indexprologue{The following is a listing of all run-time parameters that can
  be set in the input parameter file. They are all described in
  Section~\ref{sec:parameters} and the listed page numbers are where their
  detailed documentation can be found. A listing of all parameters sorted by
  the section name in which they are declared is given in the index on
  page~\pageref{sec:runtime-parameter-index-full} below.
  \addcontentsline{toc}{section}{\numberline{}Index of run-time parameter
    entries}
  \label{sec:runtime-parameter-index}
}
\printindex[prmindex]

\indexprologue{The following is a listing of all run-time parameters, sorted
  by the section in which they appear. To find entries sorted by their name,
  rather than their section, see the index on
  page~\pageref{sec:runtime-parameter-index} above.
  \addcontentsline{toc}{section}{\numberline{}Index of run-time parameters with
    section names}
  \label{sec:runtime-parameter-index-full}
}
\printindex[prmindexfull]

\end{document}
